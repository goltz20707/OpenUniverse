%%
%% $Id: charact.tex,v 1.1.1.1 2004/10/04 19:39:56 goltz20707 Exp $
%%
\externaldocument[AG-]{adventure_guide}

\chapter{Character Action}
\label{cha:character-action}

Once the players have generated characters and the GM has created at
least one star system (or has studied the star system in
\emph{Adventure Guide} Chapter
\ref{AG-cha:advent-lost-laidl}), an adventure may be played. The GM
must provide 
some method for the characters to meet and, once gathered, must
provide them with a common purpose, enemy, or goal, so that they will
be enticed to work together and enter the worlds created by the GM. A
group of characters that are setting out together on an adventure are
called a party, and this term is used in these rules in reference to
all the players' characters.

An adventure usually consists of a series of \emph{encounters}, or
unusual situations that the GM places the characters in. An encounter
provides the party with a source of information, mystery, personal
gain, amusement, or straightforward combat against creatures or
non-player characters. Through the imaginative use of encounters, the
GM weaves an ever-expanding story, written with the help of the
characters' actions in those encounters. The adventure in
\emph{Adventure Guide} Chapter
\ref{AG-cha:advent-lost-laidl} serves as an example of how an
adventure may be 
created and played.


\section{Travel and Record Keeping }
\label{sec:trav-record-keep}



As the adventures the GM has created are played, he will have to keep
track of the players' actions and positions at all times. This is done
to keep an accurate accounting of time passage and to place and
resolve encounters.

The GM will need to keep track on his own map of the characters'
progress while travelling and record the passage of time in whatever
method he finds most convenient. For this purpose, the GM will need to
create maps of various scales as need arises. Often a specific site
will have to be mapped out at an extremely small scale to facilitate
detailed exploration and interaction.


\subsection[Tracking Location]{The GM keeps track of the party's
  location on the World Log, on  an Environ Hex Map, or on a map of
  his own devising showing part of  the environ in more detail.} 
\label{sec:gm-track-location}



A map or log should always be available for the players to refer to,
although it may contain much less information than the GM's version of
the map.

When the party is travelling through an environ that is not of
particular importance to the adventure, the GM need not provide a hex
map; the terrain and contour features are considered the same
throughout the environ (except for the shape of any land or liquid
masses). Remember, each environ represents a square area \textbf{4,000
  km} to a side.

When the party is travelling through an environ with specific points
of interest and varying terrains and contours, the GM provides the
players with an Environ Hex Map (created in accordance with
\ref{sec:environ-hex-map}). Each hex on the map represents an area
approximately \textbf{100 km} across. When the party is in a
particular hex of the map, the contour and terrain features of that
hex are used for all game purposes, instead of the overall features of
the environ.

If the GM wishes to provide the players with a map that shows a
specific hex or site within a hex in great detail, he may use a blank
Environ Hex Map. Often an incident within an adventure will involve a
building, campsite, spacestation, cavern, spaceship, or other
small-scale location. The map of this location should be drawn at a
small enough scale to allow easy transfer to an Action Display should
any combat need to be resolved. Buildings and other rectilinear
structures may be drawn easier on four squares to the inch graph paper
and then transferred to the Action Display.

\subsection[Keeping Time]{The GM is responsible for keeping time
  during play.}
\label{sec:keeping-time}



The passage of game time while the characters are on an adventure is
measured in three different scales, depending on the needs of the
situation.

\begin{description}
\item[\index{Action Rounds}Action Rounds.] Each Action Round represents about 15 seconds.
  Often a specific task during a combat situation will take a certain
  amount of time (a 60-second fuse on a time-bomb, for instance), and
  the GM must keep track of the number of Action Rounds elapsed to
  measure time and apply these effects.
\item[Hours.] Travel overland or in planetary space will often require
  keeping track of time in hours. This measurement allows calculating
  movement as well as air supply, repair time, healing time, and other
  game functions.
\item[Days.] Interplanetary travel, supply expenditure, power
  availability (robot batteries, for instance) all require keeping
  track of time in days elapsed.
\end{description}

The GM must inform the players at the beginning of an adventure how
much time has elapsed since the resolution of the last one. This does
not apply if the preceding adventure was left ``frozen'' to be picked up
later. Letting the players know this allows them to calculate any
expenses they might have incurred, any healing which needed to be
done, and whether any equipment or robots they might have ordered are
ready. It is recommended the GM allow at least one week (of game
time) to pass between adventures for these purposes.


\subsection[Vehicle Travel]{The party will usually travel on the
  surface of a world in a vehicle.} 
\label{sec:vehicle-travel}



The vast distances and hostile environments on most worlds make long
distance travel on foot nearly impossible, especially when scientific
or military equipment must be carried. If the party is not provided
with a vehicle by their sponsor for an adventure, they should be given
the means to purchase, lease, or borrow one by the GM.

The rate at which a vehicle travels is listed on the appropriate
vehicle chart, in kilometers per hour. The speed of a ground vehicle
is affected by the \emph{Terrain Value} of the environ it is
travelling through (see the Terrain Effects Chart, \ref{tab:terrain-effects}).
Many vehicles are prohibited from entering heavily vegetated and/or
mountainous areas. Often while travelling, the party will come to an
area they wish to investigate that their vehicle cannot traverse. The
vehicle then becomes a base from which the characters venture,
exploring and adventuring, returning for supplies.


\subsection[Travelling By Foot]{When the party is travelling by foot,
  they may traverse 50 kilometers in one day's travel under ideal
  conditions.} 
\label{sec:foot-travel}



One day's travel is roughly equal to 12 hours of marching, interrupted
by a short break, followed by 12 hours of relaxation and sleep. If the
party wishes to march for a longer period without rest, or wishes to
march at a faster rate, the GM may allow this. However, he should
penalize the party's subsequent actions. For example, an exhausted
party would be much more likely to be \emph{unaware} of a creature or
NPC that they might encounter.

The party's daily movement rate is reduced by the Terrain Value of the
environ they are travelling through and the gravity of the world.
Refer to the Terrain Effects Chart (\ref{tab:terrain-effects}) 
to find the Terrain Value of the
environ. Multiply the Terrain Value by the gravity (in G) of the
world.  Divide the ideal movement rate by this product to determine
the party's actual movement rate (rounded to the nearest five
kilometers, for ease of play). Thus, a party that is travelling
through lightly vegetated mountains on a Size 7 world (1.7 G) may
traverse about 15 km in one day.

A character wearing an expedition suit or body armor may be slowed
down by its Encumbrance Rating (or he may move faster if in augmented
body armor). The suit/armor movement modifiers in
\ref{sec:action-round-movement} 
apply.

\textbf{Note:} For purposes of calculating daily movement rates, a
Size \textbf{1} or \textbf{2} world is considered to have a gravity of
\textbf{0.4 G} (like that of a Size 3 world). Thus, movement through
the same environ on a Size \textbf{1}, \textbf{2} or \textbf{3} world
is conducted at an identical rate.


\subsection[Carrying Equipment]{The number of kilograms of equipment
  and accoutrements that a character can carry is limited by his
  Strength Rating.}
\label{sec:carrying-equipment}



A character may move unhindered when carrying a number of kilograms
\emph{equal to or less than} his Strength Rating. The movement of such a
character is not affected during long treks or during an Action Round.

A character may move, but is \emph{hindered}, when carrying a number
of kilograms equal to or less than \textbf{4 $\times$} his Strength
Rating. The daily movement rate of such a character is \emph{divided
  by two}. He may move a maximum of \emph{one hex} on the Action
Display during a single Action Round.

A character may lift, but \emph{may not move with}, a number of kilograms
equal to or less than \textbf{10 $\times$} his Strength Rating.

All the preceding limits are divided by the gravity (in G) of the
world the characters are on. There is no limit \emph{per se} to the number of
kilograms a character in a weightless environment or on a size 1 world
may carry. However, the GM should consider the bulk of the items the
character wishes to carry; would the character actually be able to
hold all those things?

The weight of a character's expedition suit or body armor \emph{is}
considered when determining how much he may carry. \textbf{Exception:}
If the character's Body Armor or EVA Skill Level is \emph{equal to or
  greater than} the Encumbrance Rating of the suit or armor (see
Protective Attire Chart), the weight is \emph{not} considered. If a
character wearing \emph{augmented} body armor possesses a Body Armor
Skill Level that is \emph{greater} than the Encumbrance Rating of the
armor, the \emph{difference is added} to his Strength Rating when
determining how much he can carry.



\subsection[Action Display]{The positions of the characters relative
  to each other and to any creatures, NPC's, or landmarks they
  encounter are shown on the Action Display.} 
\label{sec:action-display}



Any large Hex grid map may be used as the Action Display; one is not
provided with the game (although the spaceship combat maps will
suffice if nothing else is available). A sheet with 19mm or 25mm hexes
is recommended. The Action Display has a scale of 5 \emph{meters per hex}.

Any available cardboard counters or miniature figures may be used to
represent the characters and other individuals in the adventure.
Those pieces representing the characters may be deployed in the hexes
of the display at the beginning of the adventure to show their march
order; i.e., if the party is on foot, the order in which they will be
walking. Each time the party stops to rest or changes their march
order, the players alter the relative position of their characters to
show their new deployment. When the characters encounter a creature or
NPC, the GM places a piece or pieces on the display to show their
position in relation to the characters.

If an encounter results in the conduct of an \emph{Action Round}, the
involved characters, creatures, and NPC's are moved through the hexes
of the Action Display by the players and the GM according to
\ref{sec:creature-npc-interaction} and \ref{sec:action-rounds}.

The GM may show specific features (such as a crater edge, building, or
dense patch of vegetation) on the Action Display by placing additional
pieces on it or by drawing on it.


\section{Creating Encounters}
\label{sec:creating-encounters}


As the characters travel through the GM's worlds, they will encounter
creatures, non-player characters, and other interesting or dangerous
situations. The Encounter Table provides the GM with a wide source of
encounters he may place the characters in, depending on their current
location. Results from the table provide the GM with a type of
encounter (spaceship, federal, accident, etc.).  The specific nature
of the encounter is then determined in accordance with this Section
and/or \emph{Adventure Guide} Chapter \ref{AG-cha:advent-lost-laidl}.


\subsection[Encounters]{The GM uses the Encounter Table three
  times per game day (once every eight hours).} 
\label{sec:encounters}



\textbf{Exception:} When the characters are travelling in a spaceship, the
Encounter Table is used \emph{once per day} (every 24 hours).

The GM secretly rolls percentile dice and locates the result in the
column of the Table corresponding to the current location of the
characters. The type of encounter corresponding to the dice result is
then carried out. Certain encounters are treated as \emph{no
  encounter} if the conditions detailed on the table are fulfilled.

Each creature encounter listed on the table is assigned a
\emph{Creature Value} ranging from \textbf{1} to \textbf{6}, which
affects the chance of the encounter occurring in each environ type. If
the Creature Value listed with the encounter is \emph{equal to or
  greater than} the Creature Value of the environ (see the
\textbf{Terrain Effects Chart}, \ref{tab:terrain-effects}), a creature is
encountered. If the encounter result Creature Value is \emph{less
  than} the Creature Value of the environ, no encounter occurs. Thus,
the \emph{lower} the Creature Value of an environ, the more chance of
a creature encounter in that environ. The Creature Value of a given
environ is \emph{increased by} \textbf{1} if the party is travelling
faster than \textbf{20 km} per hour.

The GM may reduce the Creature Value of an environ by assigning the
entire world a \emph{Danger Level} of \textbf{1} or \textbf{2}. The
Danger Level of a world is subtracted from the Creature Value in every
environ on the world. A world without an assigned Danger Level has a
Danger Level of \textbf{0}.

\textbf{Example:} The party is travelling through a woods/hill environ
(Creature Value of 3) in an ATV at a speed of 25 km/hour. The Danger
Level of the world is 2; thus, the Creature Value is considered 2 (3 +
1 -- 2). If the GM rolls a creature encounter result with a Creature
Value of 2 or higher, an encounter will occur.

The Creature Value and Danger Level have no affect on any types of
encounters other than creature encounters.

An encounter need not be put into play the moment it is rolled. The GM
may wait until an opportune or logical situation arises within a few
game hours of the roll. In fact, if the GM makes his use of the
Encounter Table too predictable, the players will always be ready for
a possible encounter.


\subsection[Encounter Table]{The Encounter Table is used to determine
  if the party has an encounter.} 
\label{sec:encounter-table}

See table \ref{tab:encounter} \vpageref{tab:encounter}.

\begin{table}[htbp]
  \small
  \centering
  \fbox{%
    \begin{minipage}{5.25in}
  \renewcommand{\thefootnote}{\arabic{footnote}}
      \centering
      \caption{Encounter Table}
      \label{tab:encounter}
      
      \medskip
      
      \begin{tabular}{lrcccccccc}
        && \multicolumn{8}{c}{ENVIRON/HUMAN POPULTION}\\
        ENCOUNTER & &
        \rotate{DEEP SPACE} &
        \rotate{PLANET SPACE} &
        \rotate{SPACEPORT} &
        \rotate{URBAN} &
        \rotate{\parbox{1in}{SUBURBAN OR\\TOWN}} &
        \rotate{\parbox{1in}{10 MILLION\\OR MORE}} &
        \rotate{\parbox{1in}{100 THOUSAND\\TO 10 MILLION}} &
        \rotate{\parbox{1in}{100 THOUSAND\\OR FEWER}}\\
\hline
\rowcolor{grey}
Spacecraft\footnotemark[1] & C & 1--3 & 1--12 & - & - & - & - & - & -\\
Spacecraft & R & 4--5 & 13--18 & - & - & - & - & - & -\\
\rowcolor{grey}
Spacecraft & U & 6 & 19--20 & - & - & - & - & - & -\\
\hline
Federation\footnotemark[2] & - & - & 1--15 & 1--5 & 1 & - & - & -\\
\hline
\rowcolor{grey}
NPC & C\footnotemark[3] & - & - & 16--47 & 6--41 & 2--31 & 1--21 & 1--12 & 1--3\\
NPC & R & - & - & 48--64 & 42--59 & 32--46 & 22--32 & 13--18 & 4--5\\
\rowcolor{grey}
NPC & U & - & - & 65--70 & 60--65 & 47--51 & 33--35 & 19--29 & 6\\
\hline
Creature & C\footnotemark[4]1 & - & - & - & - & - & 36--37 & 21--24 & 7--12\\
\rowcolor{grey}
& 2 & - & - & - & - & 52 & 38--39 & 25--28 & 13--19\\
& 3 & - & - & - & - & - & 40--41 & 29--32 & 20--25\\
\rowcolor{grey}
& 4 & - & - & - & - & 53 & 42--43 & 33--36 & 26--32\\
& 5 & - & - & - & - & - & 44--45 & 37--40 & 33--38\\
\rowcolor{grey}
& 6 & - & - & - & - & 54 & 46--47 & 41--44 & 39--45\\
\hline
Creature & R 1 & - & - & - & - & - & 48 & 45--46 & 46--48\\
\rowcolor{grey}
& 2 & - & - & - & - & 55 & 49 & 47--48 & 49--51\\
& 3 & - & - & - & - & - & 50 & 49--50 & 52--54\\
\rowcolor{grey}
& 4 & - & - & - & - & - & 51 & 51--52 & 55--58\\
& 5 & - & - & - & - & 56 & 52 & 53--54 & 59--61\\
\rowcolor{grey}
& 6 & - & - & - & - & - & 53 & 55--56 & 62--64\\
\hline
Creature & U 2 & - & - & - & - & - & 54 & 57 & 65--66\\
\rowcolor{grey}
& 4 & - & - & - & - & 57 & - & 58--59 & 67--69\\
& 6 & - & - & - & - & - & 55 & 60 & 70--71\\
\rowcolor{grey}
\hline
Accident\footnotemark[5]& C & 7--9 & 21--26 & 71--73 & 66--71 & 58--63 & 56--61 & 61--66 & 72--77\\
Accident & R & 10--11 & 27--29 & 74--75 & 72--74 & 64--66 & 62--64 & 67--69 & 78--80\\
\rowcolor{grey}
Accident & U & 12 & 30 & 76 & 75 & 67 & 65 & 70 & 81\\
\hline
No Encounter && 13+ & 31+ & 77+ & 76+ & 68+ & 66+ & 71+ & 82+\\
      \end{tabular}

      \medskip

      \parbox{\textwidth}{\textbf{NOTES:}

        \textbf{C:} Common. \\
        \textbf{R:} Rare. \\
        \textbf{U:} Unique. 
        
        \textbf{Note:} Creatures are sub-divided into Creature Value
        categories, which affects their likelihood of occurrence (see
        \ref{sec:encounters}).
        
        \textbf{1.} Treat as no encounter if the characters are not
        involved in controlling or maintaining the spacecraft, of if
        within the planet space of a world that has no human
        population.
        
        \textbf{2.} Treat as no encounter if the Law Level is 0 or 1.
        
        \textbf{3.} Treat as no encounter if the party is in an
        environ with no known human population, or if the party is
        travelling faster than 20 km/h and have declared that they are
        not interested in passers-by.
        
        \textbf{4.} Treat as no encounter if the party has declared no
        interest in the wild life of the area, or if there is no known
        life on the world.
        
        \textbf{5.} Treat as no encounter if the characters are
        travelling in a vehicle which they do not control.}
    \end{minipage}}
\end{table}

\subsection[Spaceship Encounters]{Spaceship encounters may occur in
  planetary space or in deep space.} 
\label{sec:spaceship-encounters}



\emph{Planetary space} is defined as all space within \textbf{300,000
  km} of any world (this equals 15 hexes on the \emph{Delta Vee} hex
maps). \emph{Deep space} is all space outside planetary space. While
hyperjumping, a spaceship travels across no space at all, so no
encounters will occur while a ship moves in this manner.

When a spaceship encounter occurs, the GM refers to \emph{Adventure
  Guide} section \ref{AG-sec:spaceship-encounters}
and chooses a common, rare, or unique spaceship encounter (as
indicated by the Encounter Table) appropriate to the situation. The
few examples given in no manner approach the varied number of
different ships which ply the space lanes, and the GM will need to
create many others.

The GM should rationalize every ship, which the characters would
encounter; space travel is still not an easy task, and any ship
encountered will have \emph{some} good reason for being there. This
reason need not be obvious to the player characters; indeed, many ship
captains will be hesitant to let every passer-by in on his business.
Most of the time the ships the characters encounter have simple,
logical reasons for going where they are going (resource trade,
scientific missions, etc.). Occasionally a ship will have business to
conduct that it wishes to keep secret (whether or not this business is
illegal) and the GM might utilize this kind of encounter to rouse the
players' curiosity.


\subsection[Non-player Characters]{Non-player character encounters may
  occur wherever the player characters find themselves.} 
\label{sec:non-player-character}

When an NPC encounter occurs, the GM refers to \emph{Adventure Guide}
\ref{AG-sec:non-play-char} and 
chooses a common, rare, or unique NPC (as indicated by the Encounter
Table) appropriate to the situation. The examples given cannot
possibly cover all the varied types of people the characters would
encounter while living in a futuristic society, and the GM should
create NPC's of his own.

The GM does not have to create all NPC's in the detail of those in
\emph{Adventure Guide} \ref{AG-sec:non-play-char}; most people the characters will meet become mere
memories soon thereafter and are never again heard from. This type
would include most officials, men-on-the-street, etc., and these
should be created as the need arises on the spur of the moment and
discarded when the encounter is finished. Others, however, will have a
lasting effect, be constantly encountered, or used as a character's
companion, and these should be fully fleshed out. The GM may create
these using the character generation procedure or dream them up out of
his head, whatever he wishes. NPC's should be varied in ability and
background, and form a group of both enemies and friends the player
characters come to know.

NPC's need not follow the skills, professions, or areas of study
player characters must; their development should fill in the areas of
society the player characters do not. There are and will be many more
non-adventuring individuals with mundane skills in society and the
NPC's should reflect this fact.

It is not necessary for the GM to fill out a Character Record for
every NPC he fleshes out. This file of NPC's will grow and is often
better handled by using 5''$\times$7'' index cards and a card file,
filling out the NPC's characteristics, skills, possessions, and any
other pertinent information on the card.

\subsection[Federal Encounters]{A federal encounter may occur in a
  spaceport, urban area, or suburban area.} 
\label{sec:federal-encounters}



Federal encounters are the major means by which the GM may trouble
characters that have committed crimes or that are carrying illegal
weapons. A federal encounter may take the form of a customs agent in a
spaceport, a civil inspector and entourage on the surface of an
underdeveloped world, or a detachment of rangers or Space Troopers
operating in a restricted or dangerous area. If the characters possess
any objects that are two Civ Levels (or more) in sophistication above
that of the world, a civil inspector may question them as to the
source of the items. If the characters cannot account for their
possession, the items may be confiscated.

The Law Level of a world affects the nature of federal encounters. If
the Law Level is \textbf{0} or \textbf{1}, no federal presence exists
and the encounter is considered "no encounter" If the Law Level is
\textbf{2} or higher, the characters will be badgered with an
increasing degree of determination. Illegal weapons that are
discovered by a federal agent will always be confiscated if official
cause cannot be given for their possession. Fines and/or incarceration
will vary depending on the Law Level.

If a character is foolish enough to kill a federal agent (a capital
crime), any future federal encounters should be geared toward bringing
the character to justice. Federal encounters do not include planetary
authorities, which are considered non-player characters.

\subsection[Creature Encounters]{A creature encounter may occur in any
  environ without a high concentration of human population.}
\label{sec:creature-encounters}


When a creature encounter occurs, the GM consults the \emph{Creature
  Location} section of the Terrain Effects Chart (\ref{tab:terrain-effects}).
Cross-referencing the type of encounter (common, rare, or unique) with
the environ the party is in will yield a group of numbers. Each number
corresponds to one of the creature descriptions in \emph{Adventure Guide} \ref{AG-sec:creatures}.
The GM chooses one of the listed creatures to use as the encounter.
Additional restrictions may be placed on a creature (such as
temperature or gravity ranges) by its description, and these should be
considered when choosing a creature. The meaning of all the creature
descriptions and explanations of any special powers they may possess
are detailed in \emph{Adventure Guide} \ref{AG-sec:creatures}. Once a creature is chosen, the
encounter is played out in accordance with \ref{sec:encounter-procedure}.

The creature list in \emph{Adventure Guide} \ref{AG-sec:creatures} is by no means intended as a
comprehensive bestiary of known space. The manner in which the
creatures are described, and the way that their powers may be mixed
and matched, makes it easy for the GM to modify the listed creatures
and to design his own creatures. When the GM has a good grasp of how
the creatures work in play, he may create creatures in the same
format. The general explanation of the creature descriptions should be
used as a guideline; it details the meaning of each part of the
creature description and how each is used in play.

\subsection[Terrain Effects Chart]{The Terrain Effects Chart
  summarizes the effect that each environ type has on movement,
  combat, and creature encounters.}
\label{sec:terrain-effects-chart}

See Table \vref{tab:terrain-effects}.


\begin{table}[htbp]
  \centering
  \fbox{%
    \begin{minipage}{0.95\textwidth}
      \centering
      \caption{Terrain Effects Chart}
      \label{tab:terrain-effects}
      
      \medskip
      
      \begin{tabular}{lccclll}
        ENVIRON TYPE & 
        \rotate{TERRAIN VALUE} &
        \rotate{\parbox{1.6in}{ACTION~ROUND\\MOVEMENT~MODIFIER}} & 
        \rotate{CREATURE VALUE} &
        COMMON & RARE & UNIQUE\\
\rowcolor{grey}
\hline
Volcanic/Hills & 2 & --4 & 6 & 9,13,15,17,18 & 21,27,29,31 & 38,39,40\\
Volcanic/Mountains & 3 & --6 & 6 & 9,13,15,17,18 & 21,27,29,31 & 38,39,40\\
\rowcolor{grey}
Volcanic/Peaks & 4 & --8 & 6 & 9,13,15,17,18 & 24,27,29,30 & 38,39,40\\
\hline
Crater/Flat & 1.5 & --2 & 6 & 9,11,13,14,17,18 & 23,27,29,30 & 38,39,40\\
\rowcolor{grey}
Crater/Hills & 2 & --4 & 6 & 9,11,13,14,17,18 & 21,27,29,31 & 38,39,40\\
Crater/Mountains & 3 & --6 & 6 & 9,11,13,14,17,18 & 21,27,29,31 & 38,39,40\\
\rowcolor{grey}
Crater/Peaks & 4 & --8 & 6 & 9,11,13,14,17,18 & 27,29,30,31 & 38,39,40\\
\hline
Barren/Flat & 1 & 0 & 5 & 9,11,14,15,17,18 & 21,23,25,27 & 37,39,40\\
\rowcolor{grey}
Barren/Hills & 1.5 & --2 & 5 & 9,11,14,15,17,18 & 21,23,25,27 & 37,39,40\\
Barren/Mountains & 2 & --4 & 5 & 9,11,14,15,17,18 & 21,25,27,29 & 37,39,40\\
\rowcolor{grey}
Barren/Peaks & 3 & --6 & 5 & 9,11,14,15,17,18 & 24,27,29,30 & 37,39,40\\
\hline
Light Veg/Flat & 1 & 0 & 4 & 1,4,5,6,7,13 & 23,24,29,30 & 32,34,37\\
\rowcolor{grey}
Light Veg/Hills & 1.5 & --2 & 4 & 1,4,5,6,13,16 & 23,24,29,30 & 32,34,37\\
Light Veg/Mountains & 2 & --4 & 4 & 1,4,5,6,7,13 & 24,29,30,31 & 32,34,37\\
\rowcolor{grey}
Light Veg/Peaks & 3 & --6 & 4 & 1,4,5,6,9,11 & 24,29,30,31 & 32,34,37\\
\hline
Woods/Flat & 1.5 & --2 & 3 & 1,2,4,7,8,16 & 19,20,24,26 & 32,33,36\\
\rowcolor{grey}
Woods/Hills & 2 & --4 & 3 & 2,4,5,7,8,16 & 19,20,24,26 & 32,33,36\\
Woods/Mountains & 3 & --6 & 3 & 1,2,4,5,6,7 & 19,20,24,26 & 32,33,36\\
\rowcolor{grey}
Woods/Peaks & 4 & --8 & 3 & 1,2,4,5,6,8 & 19,20,24,26 & 32,33,36\\
\hline
Forest/Flat & 2 & --4 & 2 & 1,2,4,5,7,8 & 19,20,24,29 & 34,35,37\\
\rowcolor{grey}
Forest/Hills & 3 & --6 & 2 & 1,2,4,5,8,17 & 19,20,24,29 & 34,35,37\\
Forest/Mountains & 4 & --8 & 2 & 2,4,5,7,8,18 & 19,20,24,29 & 34,35,37\\
\rowcolor{grey}
Forest/Peaks & 5 & --10 & 2 & 1,2,4,5,7,8 & 19,20,24,29 & 34,35,37\\
\hline
Jungle/Flat & 3 & --6 & 1 & 3,4,5,7,17,18 & 28,29,30,31 & 37,38,39\\
\rowcolor{grey}
Jungle/Hills & 4 & --8 & 1 & 3,4,5,7,17,18 & 28,29,30,31 & 37,38,39\\
Jungle/Mountains & 5 & --10 & 1 & 3,4,5,7,17,18 & 28,29,30,31 & 37,38,39\\
\rowcolor{grey}
\hline
Marsh/Flat & 1.5 & --4 & 2 & 2,3,4,5,7,17 & 19,24,29,30 & 38,39,40\\
Marsh/Hills & 2 & --6 & 2 & 2,3,4,5,7,17 & 19,24,29,30 & 38,39,40\\
\rowcolor{grey}
\hline
Ice/Flat & 1 & --4 & 5 & 13,15,16,17,18 & 24,29,30,31 & 37,38,40\\
Ice/Hills & 3 & --6 & 5 & 13,15,16,17,18 & 24,29,30,31 & 37,38,40\\
\rowcolor{grey}
\hline
Water & A & --10 & 2 & 10,12,18 & 22,24,31 & 32,33 
      \end{tabular}

      \medskip

      \parbox{\textwidth}{\textbf{NOTES:}
        
        \textbf{Terrain Value:} Affects daily and hourly movement
        rates of characters and vehicles (see \ref{sec:foot-travel}),
        Action Round movement rates of vehicles (see
        \ref{sec:vehicles}), and Hit Chance of weapon fire
        (~\ref{sec:firing-weapons}).
        
        \textbf{Action Round Movement Modifier:} Affects movement rate
        of characters in an Action Round (see \ref{sec:action-round-movement}).
        
        \textbf{Creature Value:} Affects likelihood of a creature
        encounter (see \ref{sec:encounters}).
        
        \textbf{Creature Location:} Identifies creatures (by identity
        number) which may be encounter (see \ref{sec:creature-encounters}).
        
        \textbf{A:} Terrain on water surface ranges from \textbf{1} to
        \textbf{3}, depending on choppiness of surface.
        
        Terrain Value when submerged ranges from \textbf{1.5} to
        \textbf{5}, depending on depth and opacity of liquid. A
        Terrain Value of \textbf{1} is considered \textbf{0} when
        calculating Hit Chance of weapon fire.}
    \end{minipage}}
\end{table}


\subsection[Accidents]{An accident may occur in any location listed on the
        Encounter Table.}
\label{sec:accidents}


When the GM rolls a \emph{common}, \emph{rare}, or \emph{unique}
accident encounter, he refers to the list of accidents in
\emph{Adventure Guide} \ref{AG-sec:accidents} and chooses one that
fits the character's current 
situation. The list presents two types of accidents (either of which
may be chosen). An \emph{avoidable} accident is actually a dangerous
situation that the characters may attempt to overcome by using their
skills. An \emph{unavoidable} accident is an occurrence that the
characters are powerless to prevent.

An avoidable vehicle accident is resolved as described in
\ref{sec:vehaccidents}. That procedure may also be used to resolve a
spaceship accident (using pilot skill). However, any damage incurred
by a spaceship would be applied using the \emph{Delta Vee} hit system
(see \emph{Delta Vee} \ref{DV-sec:laser-particle-fire}).

An avoidable expedition suit, body armor, or jetpack accident is
resolved as follows:

\begin{enumerate}
\item The base chance to avoid the accident is determined (see
  \emph{Adventure Guide} \ref{AG-sec:accidents}). If the character is
  skilled, his \index{agility!and avoiding an accident}Agility Rating
  and the square of his Skill Level is added to the chance.
\item The endangered character rolls percentile dice. If the dice
  result is less than or equal to the modified chance, no
  accident occurs. If the result is greater than the modified chance,
  proceed to Step \ref{item:accident-3}.
\item The GM subtracts the modified chance from the dice result and
  locates the difference on the Hit Table (~\ref{tab:hit}) to
  determine both the damage incurred by the involved equipment and the
  hits incurred by the character. If armor is involved, its Projectile
  Defence Rating may protect itself and the character, as described in
  \ref{sec:armor-defense}.\label{item:accident-3}
\end{enumerate}

When an \emph{unavoidable} accident indicates that damage has occurred
to equipment, the extent of damage is determined by rolling \emph{two
  dice} and referring to the Hit Table, adding \textbf{3} to the dice
result for a \emph{common} accident, \textbf{11} to the result for a
\emph{rare} accident, or \textbf{20} to the result for a \emph{unique}
accident.

The GM need not limit accident checks to encounters. If a character
declares that he wishes to attempt something especially risky, the GM
may use these procedures, applying the appropriate skill, to see if
the character comes through safely.


\subsection[``No Encounter'']{The GM may replace a ``no encounter''
  result from the Encounter Table with a event of his own devising.} 
\label{sec:no-encounter}



Any event not covered by the Encounter Table results that the GM
wishes to use in an adventure may be assigned a space on the Table
occupied by a ``no encounter'' result. Such events might include the
discovery of an alien artefact or structure, the discovery of
abandoned equipment, news of an occurrence that may or may not affect
the characters, or some event that may only occur in a specific
location or at a specific time. See \ref{AG-sec:laidley-encounters} of
the Adventure 
Guide for an example. If the GM wishes that such an event definitely
occur, he should forego use of the Encounter Table and automatically
implement his event instead.


\section{Creature and NPC Interaction }
\label{sec:creature-npc-interaction}



Once the GM has determined that the party has encountered a creature
or NPC, the encounter is played out, either verbally or on the Action
Display. The appropriate encounter procedure (there are two; one for
creatures and one for NPC's) is undertaken to determine how the two
sides become aware of each other, how they react to each other and, if
called for, how hostilities are initiated.

The GM is not required to use these procedures if he has provided the
creature or NPC with a specific purpose and strategy. However, the
step-by-step structure of the procedures will prove helpful in
resolving all types of creature and NPC encounters.

\subsection[Encounter Procedure]{When the party encounters a creature,
  the GM undertakes the following procedure:} 
\label{sec:encounter-procedure}

\begin{enumerate}
\item Secretly determine all the attributes of the creature (if not
  already done).
\item Make sure the characters are properly deployed on the Action
  Display.
\item Secretly roll percentile dice to determine the \emph{awareness}
  of the creature and the party and place the creature on the Action
  Display in accordance with the awareness result. If the creature is
  \emph{unaware} of the party, read its \emph{sight} description to
  the characters and skip step \ref{item:encounter-4}. If both are aware of each
  other, read the \emph{warning} description to the party. If the
  party is unaware of the creature, read nothing.
\item Use the Creature Reaction Table to secretly determine the
  creature's reaction to the party. If the party is unaware of the
  creature, and the creature chooses to attack or flee, conduct the
  Interaction indicated on the Creature Interaction
  Matrix.\label{item:encounter-4}
\item If the sight description of the creature has already been read
  to the characters, allow them to attempt \emph{perception} of the
%%Original:
%%  creature.  
%%New:
  creature (see the bilolgy skill, Section
  \ref{sec:scientific-skills}).
%%End change
Then ask the characters to declare their strategy (see
  \ref{sec:creature-reaction}). Cross-reference the character's
  strategy with the creature's reaction on the Creature Interaction
  Matrix and carry out the indicated Interaction.\label{item:encounter-5}
\end{enumerate}

Step \ref{item:encounter-5} is repeated after each Interaction to determine the
next Interaction. The character's are read the \emph{sight}
description of a creature at the beginning of the second conduct of
step \ref{item:encounter-5} in a given encounter (if they have not heard it
before). The GM may change a creature's reaction to a party at the end
of any Interaction or Action Round. If he chooses to do so, the
Interaction Matrix is referred to again.

\subsection[Awareness]{The awareness of the party and the creature
  they encounter determines the initial deployment of the creature and
  the initial information received by the party.} 
\label{sec:awareness}



The creature's \emph{Initiative Percentage} is \emph{multiplied} by
the \emph{Terrain Value} of the environ to determine the base
\emph{awareness chance}. The attributes of the party modify this
chance as follows:

\begin{itemize}
\item Subtract \emph{twice} the highest \emph{Environ} Skill Level in
  the party.
\item Subtract the highest \emph{Battlefield} Skill Level in the
  party. If the encounter is with an NPC, subtract the \emph{square}
  of the highest Battlefield Skill Level.
\item Subtract \emph{ten} if the party possesses an operating
  Neuroscanner and the creature is protein-based or more Terran-like.
\item Subtract \textbf{20} if the party possesses a robot with a bio
  system and the creature is protein based or more Terran-like. This
  subtraction may not be combined with the Neuroscanner subtraction.
\item Subtract the \emph{square} of the highest \emph{Life Sense}
  Skill Level in the party.
\item Subtract the \emph{square} of the highest \emph{Mental Power}
  Rating in the party if the creature is \emph{intelligent}.
\item Add \textbf{20} if the party is resting without a watch.
\item Add a variable amount if the party has exceeded the movement
  rate recommended in \ref{sec:foot-travel}.
\end{itemize}

The GM rolls percentile dice. If the result is equal to or less than
the modified chance, the party is \emph{unaware} of the creature. If
the creature's \index{aggression!and creature awareness}Aggression Rating is \textbf{7} or higher, place a
creature in a hex of the Action Display from which it could attack one
character (determined at random) without moving (often this will be in
the same hex as a character). The creature is not committed to an
attack at this point. If the Aggression Rating is \textbf{6} or less,
place the creature in a hex at the edge of the party's sight (see
\ref{sec:maximum-range}).

If the result is greater than the modified chance by \textbf{30} or less, the
party and the creature are aware of each other. Place the creature in
a hex at the edge of the party's sight and read the creature's \emph{warning}
description to the party (see \emph{Adventure Guide} \ref{AG-sec:creatures}).

If the result is greater than the modified chance by more than
\textbf{30}, the creature is \emph{unaware} of the party. Place the
creature in a hex at the edge of the party's sight and read the
creature's \emph{sight} description to the party.

With some alterations, the GM may use this procedure to check for
character NPC awareness, if an encounter occurs in which one side
might wish to ambush or elude the other. The NPC's Ambush and Environ
Skill Levels should be taken into account (instead of the Initiative
Percentage). If the party is unaware of an NPC, place him in the same
hex as a character (if the NPC is unarmed) or four hexes away with a
clear line of fire (if the NPC is armed).

\subsection[Maximum Range]{The maximum range at which a being may be
  seen (and thus fired at) is 200 meters or 40 hexes.} 
\label{sec:maximum-range}

The maximum range in any given encounter is \emph{divided by} the
\emph{Terrain Value} of the encounter area (see the Terrain Effects
Chart). Thus, in a Forest Mountain area (Terrain Value of \textbf{4})
the maximum range is \textbf{50} meters or \textbf{10} hexes. When the
GM is placing a creature or NPC on the Action Display for an
encounter, it should be placed in a hex at the maximum range from one
of the characters (unless the characters are unaware, see
\ref{sec:awareness}). When a creature or NPC has moved beyond the maximum
range from all characters the GM should remove it from the Display and
conduct its actions secretly, keeping track of its position until at
least one character has re- established sight (by moving back within
maximum range).

If the encounter is taking place in darkness, add three to the Terrain
Value up to a maximum of six.

\textbf{Exception:} A creature with heightened vision or a character
with night glasses is not affected by darkness when seeing in the
infrared, heat residue may be contained in haze, dust fog and liquid
thermals which may affect infrared sight. Light intensification is
dependent on available light to work no matter how faint.

The GM may impose other additions to the Terrain Value when
determining maximum range in the case of dust, fog or other
non-terrain visual obstructions.

$$\mathrm{Visible~distance} =
\frac{200\mathrm{~m~(40~Hexes)}}{\mathrm{Terrain~Value}}$$

\noindent\textbf{\emph{Modifiers:}}
 
\noindent\textbf{Moonlit Night} = Terrain Value +3.\\
\textbf{Moonless Night} = Terrain Value +4.\\
\textbf{Light Dust, Haze or Fog:} Terrain Value +1.\\
\textbf{Medium Dust, Haze or Fog:} Terrain Value +2.\\
\textbf{Heavy Dust, Haze or Fog:} Terrain Value +3.

\subsection[Creature Reactions]{A creature's reaction to the party,
  and the party's 
  strategy determines how an encounter will be resolved.}
\label{sec:creature-reaction}



Unless the GM has chosen a reaction for the creature, he secretly
rolls one die and \emph{adds} the creature's
\emph{\index{aggression!and creature reaction}Aggression Rating}
to the result. He locates the modified die result on the Creature
Reaction Table to determine one of the following creature reactions:

\begin{description}
\item[Flee.] Terrified of the party; will leave the area quickly as
  possible.
\item[Leave Slowly.] Bothered by the party; will amble off.
\item[Watch Warily.] Suspicious of the party but has not yet decided
  what to do about it.
\item[Ignore.] Totally unconcerned about the party and its actions.
\item[Protect.] Will attack to defend itself if provoked.
\item[Attack.] Angered or hungry; will attempt to kill or capture
  (depending on the creature type).
\item[Communicate.] Intelligent; will attempt to establish contact
  with the party.
\end{description}

If the party is aware of the creature, the GM asks them to choose one
of the following strategies:

\begin{description}
\item[Attack.] The party plans to fire weapons, strike the creature,
  or take any other hostile action against the creature.
\item[Maneuver.]The party plans to move towards and around the
  creature without taking hostile action.
\item[Watch Warily.] The party will hold its ground, watching and
  recording the creature's actions.
\item[Ignore.] The party will disregard the creature.
\item[Flee.] The party will move away from the creature as quickly as
  possible.
\item[Communicate.]The party will attempt to establish contact with
  the creature.
\end{description}

The GM secretly cross-references the creature's reaction with the
character's strategy on the Creature Interaction Matrix and carries
out the indicated \emph{Interaction}. Each possible Interaction is
explained with the table and may call for the conduct of an Action
Round, may allow maneuver by one side or the other, or may indicate
some other activity. The GM should return to Step \ref{item:encounter-5} of the Creature
Encounter Procedure after conducting an Interaction, unless otherwise
stated in the Interaction description. When it becomes obvious that
the party will not be changing its strategy and the creature will not
be changing its reaction (especially when combat has been initiated by
one or both sides), the GM should stop using Interactions. Instead,
Action Rounds are conducted until the encounter is resolved.

\textbf{Note:} Any reaction result calling for movement by an immobile
creature is considered a Watch Warily reaction. However, the creature
will ``close up'' if possible.


\subsection[Creature Reaction Table]{The GM uses the Creature Reaction
  Table to determine a 
  creature's attitude toward the party.}
\label{sec:creature-reaction-table}

%See Table \vref{tab:creature-interaction}.


\begin{table}[htbp]
  \centering
  \fbox{%
    \begin{minipage}{5in}
      \centering
      \caption{Creature Interaction Table}
      \label{tab:creature-interaction}
      
      \medskip
      
      \parbox{2.25in}{\begin{tabular}{cl}
          \rowcolor{grey}
          ONE DIE PLUS & \\
          \rowcolor{grey}
          CREATURE'S & \\
          \rowcolor{grey}
          AGGRESSION & REACTION \\
          2--8 & Flee (J) \\
          \rowcolor{grey}
          9--10 & Leave Slowly \\
          11 & Watch Warily*\\ 
          \rowcolor{grey}
          12 & Ignore \\
          13 & Watch Warily* \\
          \rowcolor{grey}
          14,15 & Protect* \\
          16--22 & Attack (D)\\ 
        \end{tabular}}
      \hspace{.25in}
      \parbox{2.25in}{\textbf{NOTES:}
        \begin{description}
        \item[*:] If the creature possesses an 
          Intelligence Rating, its reaction is to 
          communicate (instead of the listed result).
        \item[J:] If 
          the party is unaware of the creature, immediately 
          conduct Interaction J. 
        \item[D:] If the party is unaware 
          of the creature, immediately conduct Interaction 
          D. 
        \end{description}
        
        This table is not used if the creature is 
        unaware of the party. 
        See \ref{sec:creature-reaction} for explanation of use.}
    \end{minipage}}
\end{table}


\subsection[Creature Interaction Matrix]{The GM uses the Creature
        Interaction Matrix to determine the interaction between a
        creature and the party.} 
\label{sec:creature-interaction-matrix}


\begin{table}[htbp]
  \centering
  \fbox{%
    \begin{minipage}{4in}
      \centering
      \caption{Creature Interaction Matrix}
      \label{tab:creature-interaction-matrix}
      
      \medskip
      
      \begin{tabular}{lcccccc}
        CHARACTER STRATEGY &&&&&&\\
        CREATURE REACTION &
        \rotate{ATTACK} &
        \rotate{MANEUVER} &
        \rotate{\parbox{1in}{WATCH\\WARILY}} &
        \rotate{\parbox{1in}{IGNORE/\\UNAWARE}} &
        \rotate{FLEE} &
        \rotate{COMMUNICATE} \\
        \rowcolor{grey}
        Attack & A & B & C & D & E & F\\
        Protect & A & G & H & I & J & K\\
        \rowcolor{grey}
        Watch Warily & L & M & N & N & J & P\\
        Ignore/Unaware & Q & M & N & N & J & R\\
        \rowcolor{grey}
        Leave Slowly & S & T & U & J & J & V\\
        Flee & W & X & Y & J & J & Y\\
        \rowcolor{grey}
        Communicate & Z & AA & BB & N & J & CC\\
      \end{tabular}

      \medskip

      \parbox{\textwidth}{See notes on pages
      \pageref{tab:creature-interaction-notes-1}--\pageref{tab:creature-interaction-notes-2}.

      See \ref{sec:creature-reaction} for explanation of use.}
    \end{minipage}}
\end{table}


\begin{table}[htbp]
  \centering
  \fbox{%
    \begin{minipage}{0.95\textwidth}
      \caption{Creature Interaction Matrix (notes)}
      \label{tab:creature-interaction-notes-1}
      
      \medskip\centering
      
      \begin{minipage}{0.95\textwidth}
        \begin{multicols}{2}
          \textbf{A.} Conduct Action Rounds until one side or the other
          is
          dead, captured, or escaped (thus concluding the encounter).\\
          \textbf{B.} Conduct an Action Round. The characters have the
          initiative but may not attack the creature in any manner.\\
          \textbf{C.} Conduct an Action Round, adding \emph{five} to the
          creature's
          initiative die roll.\\
          \textbf{D.} Conduct an Action Round. The creature has the
          initiative. Skip Step \ref{item:ar-4} of the Action Round.\\
          \textbf{E.} Each character may move as far as would be allowed
          in one Action Round (see \ref{sec:action-round-movement}).
          Then, conduct an Action Round,
          adding \emph{eight} to the creature's initiative die roll.\\
          \textbf{F.} Conduct an Action Round. The creature has the
          initiative.\\
          \textbf{G.} Each character may move as far as would be allowed
          in
          one Action Round (see \ref{sec:action-round-movement}) Then
          conduct interaction H.\\
          \textbf{H.} If the creature is able to attack without moving,
          conduct an Action Round, adding \emph{five} to the creature's
          initiative die roll. Otherwise, return to Encounter Step
          \ref{item:encounter-5}.\\
          \textbf{I.} If the creature is able to attack without moving,
          conduct interaction F.  Otherwise, return to Encounter Step
          \ref{item:encounter-5}.\\
          \textbf{J.} The encounter is over (the creature, the
          characters,
          or both have left the area).\\
          \textbf{K.} The GM rolls one die and adds the creature's
          \emph{Intelligence Rating} (if any) to the result. If the
          modified die result is \emph{less than} the creature's
          \emph{Aggression Rating}, conduct interaction F (the creature
          has mistaken the character's actions for an attack).
          Otherwise, after time
          passes without incident, return to Encounter Step
          \ref{item:encounter-5}.\\ 
          \textbf{L.} Conduct an Action Round, adding \emph{five} to the
          party's
          initiative die roll.\\
          \textbf{M.} Each character may move as far as would be allowed
          in
          one Action Round (see \ref{sec:action-round-movement}).\\
          \textbf{N.} Time passes as neither side takes any action.\\
          \textbf{P.} The characters' communication efforts seem to have
          no effect. If the creature has no \emph{Intelligence Rating},
          or the following check is not successful, time passes without
          incident; return to Encounter Step \ref{item:encounter-5}. The
          GM rolls percentile dice. If the result is less than the
          creature's Intelligence Rating plus the highest \emph{Empathy
            Rating} of those characters attempting communication, roll
          one die and refer to the Creature Reaction Table to see if the
          creature changes its reaction (the party retains the
          communicate strategy). If the creature possesses psionic
          powers, the \emph{square} of one character's \emph{Psionic
            Communication Skill Level} may be used
          \emph{instead of an Empathy Rating}.\\
          \textbf{Q.} Conduct an Action Round. The party has the
          initiative.
          Skip Step \ref{item:ar-4} of the Action Round.\\
          \textbf{R.} The creature shows no interest as time passes;
          return to Encounter Step \ref{item:encounter-5}.
          \emph{Exception:} If the characters are attempting to
          communicate with a creature that is unaware of them, and the
          GM feels that the creature has been made aware by the
          character's actions, he may roll one die and refer to the
          Creature Reaction Table to determine the creature's
          reaction (the party retains the communicate strategy).\\
          \textbf{S.} Move the creature two hexes away from the
          characters.  Then conduct an Action Round, giving the
          characters the
          initiative.\\
          \textbf{T.} Each character may move as far as would be allowed
          in one Action Round (see \ref{sec:action-round-movement}). At
          some point during the character's movement, move the creature
          two hexes away from
          them.\\
          \textbf{U.} Move the creature twohexes away from the party.\\
          \textbf{V.} The party's communication efforts have no effect.
          If the creature has no Intelligence Rating, or if the
          following check is not successful, it wanders away and the
          encounter is over. The GM rolls percentile dice. If the result
          is \emph{less} than the creature's \emph{Intelligence Rating}
          plus the highest \emph{Empathy Rating} of those characters
          attempting communication, roll one die and refer to the
          Creature Reaction Table again to see if the creature changes
          its reaction (the party retains the communicate strategy). If
          the creature possesses psionic powers, the \emph{square} of
          one character's \emph{Psionic Communication
            Skill Level} may be used instead of an Empathy Rating.\\
          \textbf{W.} Move the creature away from the characters a
          number of hexes equal to its Agility Rating. Then conduct an
          Action
          Round, giving the characters the initiative.\\
          \textbf{X.} Move the creature away from the characters a
          number of hexes equal to its Agility Rating. Then each
          character may move as far as would be allowed in one Action
          Round (see
          \ref{sec:action-round-movement}). Finally, move the creature
          again (as above).
      \end{multicols}
    \end{minipage}

    \medskip
  \end{minipage}}
\end{table}


\begin{table}[htbp]
  \centering
  \fbox{%
    \begin{minipage}{0.95\textwidth}
      \caption{Creature Interaction Matrix (notes cont.)}
      \label{tab:creature-interaction-notes-2}
      
      \medskip\centering

      \begin{minipage}{0.95\textwidth}
        \begin{multicols}{2}
          \textbf{Y.} Move the creature away from the characters a
          number of
          hexes equal to its Agility Rating.\\
          \textbf{Z.} Conduct Interaction Q. Before returning to
          Encounter Step \ref{item:encounter-5}, roll percentile dice.
          If the result is equal 
          to or less than the highest \emph{Intelligence} or
          \emph{Empathy Rating} among the characters, inform them that
          the creature has been attempting to communicate. If the
          creature possesses psionic powers, the \emph{square} of one
          character's \emph{Mental Power Rating} may be used
          instead of an Intelligence or Empathy Rating.\\
          \textbf{AA.} Each character may move as far as would be allow
          ed in one Action Round (see \ref{sec:action-round-movement}).
          Then check to see if the 
          characters become aware of the creature's communication
          attempt in accordance with Interaction Z, before returning to
          Encounter Step \ref{item:encounter-5}.\\
          \textbf{BB.} Time passes as neither side moves. Before
          returning to Encounter Step \ref{item:encounter-5}, roll two dice. If
          the result is 
          \emph{less} than the highest \emph{Intelligence} or
          \emph{Empathy Rating} 
          among the characters, inform them that the creature is
          attempting to communicate. If the creature possesses psionic
          powers, the \emph{square} of one character's \emph{Mental Power
            Rating} may be used
          instead of an Intelligence or Empathy Rating.\\
          \textbf{CC.} The party and the creature are attempting to
          communicate.  \emph{Multiply} the creature's \emph{Intelligence
            Rating} by the highest \emph{Linguistics Skill Level} among the
          characters (minimum of one) \emph{or}, if the creature has psionic
          powers, multiply its \emph{Intelligence Rating} by the highest
          \emph{Psionic Communication Skill Level} plus the highest
          \emph{Mental Power Rating} possessed by one character. Roll
          percentile dice. If the result is equal to or less than the
          product calculated above, a successful means of communication
          has been established; the GM should play the creature as an
          NPC. If the dice result is greater than the product, return to
          Encounter Step \ref{item:encounter-5}. Only one dice roll is
          allowed per attempt, 
          but as long as both sides choose to continue communicating,
          one dice roll may be made each time. However, the chance of
          success is \emph{reduced} by \emph{10} percentage points for
          each additional 
          attempt (this is cumulative).  The passage of about three
          hours should be noted for each attempt.
        \end{multicols}
      \end{minipage}

      \medskip
    \end{minipage}}
\end{table}


\subsection[NPC Encounters]{When the party encounters an NPC, the GM
  undertakes one of two procedures, depending on the type of encounter
  he envisions.} 
\label{sec:npc-encounters}



The GM secretly determines all the attributes of the NPC (if not
already done) and assesses the party's situation. If he feels that
combat would precede any vocal interaction (in say, an ambush or
battlefield situation), the following steps are conducted.

\begin{enumerate}
\item Make sure the characters are properly deployed on the Action Display. 
\item The GM secretly rolls percentile dice to determine the
  \emph{awareness} of the NPC and the party (see \ref{sec:awareness}).
  Place the NPC(s) on the Action Display in accordance with the
  awareness result.
\item Conduct an Action Round. If one side is unaware of the other,
  the aware side has the automatic initiative; and step \ref{item:ar-4} of the
  Action Round is skipped. Continue conducting Action Rounds until the
  combat is resolved or both sides choose to cease hostilities. After
  the first Action Round, both sides are considered aware of each
  other.
\end{enumerate}

If the GM does not feel that combat would be immediately initiated and
wants to allow interaction between the party and the NPC (conversation
and reaction) the following steps are conducted (unless the GM has
chosen a distinct attitude for the NPC).

\begin{enumerate}
\item The GM secretly rolls percentile dice and compares the dice
  result to the NPC's \emph{\index{aggression!and NPC reaction}Aggression} Rating \emph{times ten}. This
  comparison will yield a positive or negative \emph{reaction number}.
  \textbf{Example:} An NPC's Aggression Rating of 4 multiplied by 10
  equals 40. Percentile dice are rolled and the result is 76.
  Comparing the multiplied Aggression of 40 with the roll of 76 yields
  a reaction number of +36.
\item The GM locates the reaction number on the NPC Reaction Table
  (~\ref{tab:npc-reaction}) and makes a mental note of the indicated NPC
  reaction.  The party may now choose one character among them as
  their spokesman. If no spokesman is chosen, skip Steps \ref{item:npc-interact-6}
  and \ref{item:npc-interact-7} and proceed to Step \ref{item:npc-interact-8}.
\item The spokesman rolls one die. If the result is \emph{less than}
  his \emph{Empathy Rating}, the GM reads the \emph{Key Word} of the
  NPC reaction to the party, as an indication of the NPC's apparent
  intentions. If the die result is \emph{equal to} or \emph{greater
    than} the spokesman's Empathy Rating, the Key Word is not
  revealed.\label{item:npc-interact-6}
\item The spokesman may perform the \emph{communication task}; see the
  streetwise and diplomacy skill descriptions
  (\ref{sec:interpersonal-skills}). For every \textbf{10} (or
  fraction) \emph{below} the modified chance for the task the dice
  result indicates, the party receives a \emph{friendly shift} of
  \emph{one line towards 0} on the Reaction Table. For every
  \textbf{10} \emph{above} the modified chance, the party receives a
  \emph{hostile shift} of one line \emph{away} from 0.\label{item:npc-interact-7}
\item The GM checks the list of Reaction Shifts (listed with the
  Reaction Table) to determine whether any shifts are applied (in
  addition to any applied as stated in Step \ref{item:npc-interact-7}). Any verbal
  interaction by the players may also be considered. Apply the shifts
  in the direction indicated.\label{item:npc-interact-8}
\item After all shifts are applied, the GM locates the new line on the
  NPC Reaction Table. He reads or paraphrases the Key Word and the NPC
  reaction to the players (if he thinks it would be obvious) and the
  result is enacted as it applies to the situation.
\end{enumerate}



\subsection[NPC Reaction Table]{The GM uses the NPC Reaction Table to
  determine the attitude of an NPC towards the party in an encounter.}
\label{sec:npc-reaction-table}

See Table \vref{tab:npc-reaction}.

\begin{table}[htbp]
  \small
  \centering
  \fbox{%
    \begin{minipage}{0.75\textwidth}
      \centering
      \caption{NPC Reaction Table}
      \label{tab:npc-reaction}
      
      \medskip
      
      \begin{tabular}{ccl}
        ROLL DIFFERENCE & KEY WORD & NPC REACTION\\
        \rowcolor{grey}
        --111 to --120 & Attack & Viciously tries to kill the party.\\
        --101 to --110 & Attack & Tries to grievously hurt the party.\\
        \rowcolor{grey}
        --91 to --100 & Attack & Attacks party to stop them.\\
        --81 to --90 & Attack & Takes the offensive to warn the party.\\
        \rowcolor{grey}
        --71 to --80 & Attack & Tries to stop the party without bloodshed.\\
        --61 to --70 & Attack & Aims weapons at the party.\\
        \rowcolor{grey}
        --51 to --60 & Attack & Draws weapons on party.\\
        --41 to --50 & Hesitant & Prepares to take offensive action.\\
        \rowcolor{grey}
        --31 to --40 & Cautious & Distrustful and will wait and see.\\
        --21 to --30 & Wary & Doubts party's word, but remains patient.\\
        \rowcolor{grey}
        --11 to --20 & Suspicious & Needs more information to act.\\
        --1 to --10 & Suggestible & Will listen to party's story.\\
        \rowcolor{grey}
        0 & Friendly & Will aid party if possible.\\
        +1 to +10 & Suggestible & Will hear the party out.\\
        \rowcolor{grey}
        +11 to +20 & Suspicious & Thinks party is here to make trouble.\\
        +21 to +30 & Wary & Nervous because party could cause harm.\\
        \rowcolor{grey}
        +31 to +40 & Cautious & Party is intimidating and fear is
        growing.\\
        +41 to +50 & Hesitant & Party causing great fear.\\
        \rowcolor{grey}
        +51 to +60 & Flee & Backs away from the party.\\
        +61 to +70 & Flee & Tries to hide from the party.\\
        \rowcolor{grey}
        +71 to +80 & Flee & Moves quickly away from the party.\\
        +81 to +90 & Flee & Runs frantically away from the party.
      \end{tabular} 

      \medskip
      
      \parbox{\textwidth}{If the actions of the party cause a shift
        into Flee or Attack reaction from a less extreme reaction, the
        NPC receives the Initiative in the first Action Round. Upon
        reaching \textbf{0} (Friendly), no further shifting can occur
        for the party's benefit. If the die roll matches the NPC's
        Aggression \textbf{$\times$10} exactly, interpret it to mean
        extreme interest, and sexual attraction if possible. A party
        can ruin this reaction by taking harmful actions.  See
        \ref{sec:npc-encounters} for explanation of use.}

      \medskip

      \begin{tabular}{ll}
        & \textbf{Friendly Shifts (TOWARDS 0)}\\
        \rowcolor{grey}
        Shift \textbf{2} & The party has no weapons visible.\\
        Shift \textbf{1} & Character has military rank and NPC was/is
        in the military.\\ 
        \rowcolor{grey}
        Shift \textbf{1} & Party spokesman's social standing is
        +/-- \textbf{1} of NPC's.$^\mathrm{A}$\\ 
        Shift \textbf{2} & Characters are disguised as allies or are allies.\\
        \rowcolor{grey}
        Shift \textbf{1} & Party has correct ID or papers, which pass
        inspection.\\ 
        Note$^\mathrm{B}$ & NPC Party outnumbers characters.\\
        \rowcolor{grey}
        Shift \textbf{1} & Characters adopt friendly attitude toward
        NPC's.$^\mathrm{C}$\\ 
        Shift \textbf{2} & Characters adopt helpful attitude toward
        NPC's.$^\mathrm{C}$\\ 
        & \textbf{Hostile Shifts (AWAY FROM 0)}\\
        \rowcolor{grey}
        Shift \textbf{1} & The party has weapons visible.\\
        Shift \textbf{2} & The party has weapons at the ready.\\
        \rowcolor{grey}
        Shift \textbf{4} & The party has no weapons aimed.\\
        Shift \textbf{1} & Party spokesman's social standing beyond
        +/-- \textbf{1} of NPC's.$^\mathrm{A}$\\ 
        \rowcolor{grey}
        Shift \textbf{2} & Characters disguises are seen through.\\
        Shift \textbf{3} & Party's ID or papers, did not pass inspection.\\
        \rowcolor{grey}
        Shift \textbf{4} & Characters are disguised as the enemy or
        are the enemy.\\ 
        Note$^\mathrm{B}$ & Character party outnumbers the NPC's.\\
        \rowcolor{grey}
        Shift \textbf{2} & No one in the party can speak the NPC's language.\\
        Shift \textbf{1} & Characters adopt an angry or disdainful
        attitude.$^\mathrm{C}$\\ 
        \rowcolor{grey}
        Shift \textbf{2} & Character(s) revealed as psionic; no NPC psionic.\\
        Shift \textbf{2} & Characters actually threaten NPC's.$^\mathrm{C}$\\
      \end{tabular}
      
      \medskip

      \parbox{\textwidth}{\textbf{A.} If the characters have not
        designated a spokesman, these shifts are ignored. \\
        \textbf{B.} The GM should shift 1 for every two characters or
        NPC's, rounding fractions up.  \\
        \textbf{C.} These actions are verbally enacted by the players
        in their interplay with the GM.}
    \end{minipage}}
\end{table}

\section{\index{Action Rounds}Action Rounds}
\label{sec:action-rounds}



\index{Action Rounds}Action Rounds are used to resolve combat between characters. NPC's and
creatures. When a possible combat situation arises, the GM and the
players prepare the Action Display as described in
\ref{sec:action-display}, the Creature Encounter Procedure
(~\ref{sec:encounter-procedure}) and the NPC Encounter Procedure
(~\ref{sec:npc-encounters}).

During a creature encounter, an Action Round is undertaken when called
for by an Interaction (see \ref{sec:creature-reaction}). Some
Interactions call for the conduct of one Action Round, and then
another Interaction is determined. Others call for the repeated
conduct of Action Rounds until combat is resolved (one side or the
other is incapacitated or escapes).

During an \textbf{NPC} encounter, an Action Round is undertaken if
either the characters or NPC's wish to initiate combat. Action Rounds
are repeated until combat is resolved or both sides choose to cease
hostilities.


\paragraph{ACTION ROUND PROCEDURE:}
\label{sec:action-round-procedure}

\begin{enumerate}
\item If not stated in the Encounter Procedure or Interaction,
  determine which side has the \emph{initiative} (see
  \ref{sec:initiative}).\label{item:ar-1}
\item Every individual that is not stunned, passed out or restrained
  on the side \emph{with initiative} may move, fire and/or perform other
  actions. Each individual conducts his actions one at a time; that
  is, one performs actions, than the next, then the next, and so on.
  The order in which individuals perform their actions is up to the
  players (if characters) or the GM (if NPC's or creatures). If an
  enemy individual is attacked or fired upon, the effects are
  implemented immediately.\label{item:ar-2}
\item Every individual on the side \emph{without} initiative that was
  fired at in Step \ref{item:ar-2} must conduct a \emph{willpower check}. Those
  individuals that fail the check must now perform a \emph{rash} or
  \emph{protective} action.\label{item:ar-3}
\item Every individual on the side \emph{without initiative} may
  perform actions as described in Step \ref{item:ar-2}.
  \textbf{Exception:} An individual that was attacked in \emph{close
    combat} during Step \ref{item:ar-2} or that \emph{failed} a willpower
  check in Step \ref{item:ar-3} may not perform any actions \emph{at
    all}. This step is skipped if the side without initiative is
  unaware or is ignoring the side with initiative.\label{item:ar-4}
\item The GM may check the NPC's or creatures to determine if their
  reaction to the characters changes. The characters may also change
  their strategy. If one or both sides wishes to continue combat,
  return to Step \ref{item:ar-1} of this procedure. However, if one or both sides
  changes their strategy during a creature encounter, check the
  appropriate interaction to see how the next Action Round will be
  conducted (if at all).
\end{enumerate}

Throughout this section, any references to abilities and options of
the characters also applies to NPC's.


\subsection[Initiative]{The players and the GM determine which side has the
  initiative at the beginning of each Action Round.}
\label{sec:initiative}



The players use the one character in their party that has the highest
\emph{Initiative Value}, determined as follows: Add together the
character's \emph{Leadership} Rating (if the enemy is a creature,
\emph{Intelligence} may be used instead), his \emph{Environ} Skill
Level, and his \emph{Battlefield} Skill Level (if the characters are
fighting NPC's, \emph{double} the Battlefield Skill Level). The
highest sum obtained by one character is the party's Initiative Value
and that character is considered the party's leader.

The GM secretly determines the Initiative Value of the enemy force. If
a creature, its initiative percentage is used as its Initiative Value.
If NPC's, the Value is determined in the same way as for characters.
One side's Initiative Value may be increased in accordance with a
creature interaction (see \ref{sec:creature-interaction-matrix}).

The player controlling the character being used for initiative rolls
one die and adds the result to his Initiative Value. The GM secretly
does the same for the enemy force. The side that achieves the higher
sum receives the initiative this round, and performs actions first.

If one side is unaware in the first Action Round of a combat, they are
considered aware in all the following Rounds. However, until the
initially unaware side actually gains the initiative in an Action
Round, 3 is subtracted from all their initiative die rolls.

A character or NPC that is currently stunned, unconscious, or that
failed a will power check in the previous Action Round may not be used
to determine the Initiative Value.

\textbf{Close Combat Initiative.} If a character is engaged in close
combat with an NPC or creature (see \ref{sec:close-combat}), the GM should
have the character and his enemy check for initiative in relation to
each other separate from the rest of the combatants. Their rolls will
have no effect on the other individuals. Thus it is possible that a
character will have the initiative in his close combat situation while
the rest of the party will not, and vice versa.


\subsection[Action Round Movement]{A character may move a number of
  hexes equal to his \index{agility!and Action Round movement}Agility Rating in a single Action Round, with the
  following modifiers:}
\label{sec:action-round-movement}

\begin{itemize}
\item Add the Terrain Movement Modifier for the environ (see the
  Terrain Effects Chart, \ref{tab:terrain-effects}). 
\item Add the Gravity Movement Modifier for the world (see the world
  log). 
\item Add the character's appropriate Gravity Skill Level. 
\end{itemize}

\textbf{Example:} A character with an Agility Rating of 7 in a
woods/hill environ (-4) on a size 4 world (+2) who has a Light Gravity
Skill Level of +1, would be able to move up to 6 hexes in a single
Action Round.

The number of hexes a character may move may be increased or decreased
if he is wearing an expedition suit or body armor.  \emph{Subtract}
the character's EVA or Body Armor Skill Level (whichever is higher)
from the \emph{Encumbrance Rating} of the suit or armor. If the result
is greater than one, \emph{divide} the character's movement rate by
the result (rounding to the nearest whole number). If the result of
the subtraction is one or less, the character's movement is not
affected.

\textbf{Exception:} If a character using his Body Armor Skill is
wearing \emph{augmented} armor, and the subtraction result is --2 or
less, \emph{multiply} the character's movement rate by the
\emph{absolute value} of the result.

\textbf{Example:} Assuming the character in the above example is
wearing Civ Level 7 Impact Armor (encumbrance Rating of 3, augmented)
and does not have the EVA or Body Armor Skill, he would be able to
move only two hexes per Action Round (6 + 3). If he had a Skill
Level from 2 to 4 he could move the full six hexes. If he had a Body
Armor Skill Level of 5 (or 6) he could take advantage of the armor's
augmentation and move 12 (or 18) hexes.

Unless a character's Agility Rating is \textbf{0} (which indicates
that he may not move at all) or he is restrained, he may always move
at least one hex, even if his calculated movement is \textbf{0} or
less.

The movement rate of a robot is calculated in the same manner as that
of a character (robots have no Gravity Skill Levels). However, if a
robot's calculated movement rate is 0 or less it may not move \emph{at
  all} (unless it has a creative thought system).

The movement rate of a creature is equal to its \index{agility!and
  creature movement}Agility Rating only.
The effects of gravity and the environ are already included in its
Agility Rating.


\begin{figure}[htbp]
  \centering
  \fbox{%
    \begin{minipage}{5in}
      \centering
      \caption{Movement Rate Calculation Summary}
      \label{fig:movement-rate-calculation-summary}
      
      \medskip
      
      \parbox{\textwidth}{\noindent\emph{\textbf{Character Action
      Round Movement Rate\ldots}}\\
      Character \index{agility}Agility Rating \\
      \emph{plus} Gravity Skill Level \\
      \emph{plus} Gravity Movement Modifier \\
      \emph{plus} Terrain Movement Modifier.

      \begin{center}
        \textbf{Character AR Move Rate = Chr AY +
          GSk + GMv mod + TMv mod}
      \end{center}
      
      If the character is wearing protective attire with an
      Encumbrance Rating, \emph{subtract} the EVA or Body Armor Skill
      Level from the Rating. If the result is \textbf{2} or greater,
      \emph{divide} the sum of the preceding calculation by the
      subtraction result. If the result is \textbf{--2} or less and
      the character is wearing augmented (powered) body armor,
      \emph{multiply} the sum of the preceding calculation by the
      absolute value of the subtraction result. If the result is
      \textbf{--1,} \textbf{0} or \textbf{+1}, the character's
      movement is not affected by the attire.

      \medskip

      \noindent\emph{\textbf{Ground Vehicle Movement Rate\ldots}}\\
      Vehicle Speed (in hexes per Action Round or kilometers per hour;
      see \ref{sec:land-vehicles})  \emph{divided by} (Terrain Value,
      \ref{sec:terrain-effects-chart} + Terrain Value Modifier,
      \ref{sec:land-vehicles}) 

      \begin{center}
        \textbf{Vehicle AR Move Rate = VSpd/(TV + TV mod)}
      \end{center}
      
      A Ground Vehicle may not traverse terrain with a Terrain Value
      that is greater than the vehicle's Terrain Value Limit
      (\ref{sec:land-vehicles}).

      \medskip

      See \ref{sec:action-round-movement} for additional information.}

    \medskip
    \end{minipage}}
\end{figure}

\subsection[Other Actions]{When a character is eligible to move, he
  may perform other actions before, after or instead of moving.} 
\label{sec:other-actions}

An action a character performs during an Action Round other than
movement may reduce the number of hexes he may move in that Round or
may prevent him from moving altogether. Unless otherwise specified,
actions may be performed before moving, after moving, or instead of
moving. They may not be performed before and after moving and may not
interrupt movement. Thus a character may perform actions and then
move, or he may move and then perform actions, or he may perform
actions only. If a single action would reduce a character's movement
rate below \textbf{0}, he may still perform that action but may not do
anything else in the Action Round.

\begin{description}
\item[Fall prone.] No reduction. May be performed after move only.
\item[Get up.] -1 hex. Performed before move if prone.
\item[Open or close door.] -1 hex.
\item[Pick up object.] -2 hexes.
\item[Pass through narrow opening.] -2 hexes.
\item[Load weapon.] One Action Round.
\item[Fire weapon.] No reduction (however, see
  \ref{sec:fire-modifier-chart}).
\item[Draw Weapon.] -1 hex.
\item[Exit enemy-occupied hex.] -2 hexes. May not be performed if
  engaged in close combat. May be performed before move only.
\item[Attack in Close Combat.] -3 hexes. May be performed after move
  only. A character may conduct no more than one close attack in a
  single Action Round.
\item[Break off from Close Combat.] Movement rate reduced to one hex.
  Character must have the initiative and may not break off if
  restrained. May be performed before move only.
\item[Pressurize or Depressurise in Airlock.] Two Action Rounds.
\item[Emplace Machine Gun.] Two Action Rounds (a machine gun mounted on
  a structure or vehicle in considered emplaced).
\item[Jump over object.] -3 hexes. May be performed after move only,
  although the jump may include an additional forward momentum of one
  hex. A character may jump a number of feet equal to one half of his
  total movement rate for the Action Round.
\item[Perceive Creature without Bioscanner.] No reduction.
\item[Perceive Creature with Bioscanner.] One Action Round.
\item[Jet Pack Movement.] A skilled character wearing a jetpack may
  move (fly) a number of hexes equal to twice his \index{agility!and
  jet pack movement}Agility Rating, plus
  his Gravity Skill Level, plus the Gravity Movement Modifier, plus
  the \emph{square} of his Jetpack Skill Level. No modifier for
  terrain is considered.  An unskilled character may move a number of
  hexes equal to his Agility Rating plus his Gravity Skill Level plus
  the Gravity Movement Modifier (the GM may wish to check for an
  accident when an unskilled character is using a jetpack). Jetpack
  movement is halved (rounded up) if a character is taking off or
  landing before or after the move. A character using a jetpack cannot
  take off and land in 1 Action Round.
\end{description}

The movement rates of vehicles (in hexes per Action Round) are listed
on the appropriate vehicle chart. As long as a vehicle is being
driven, it may move at its listed movement rate (as modified by the
Terrain Value of the environ (see \ref{sec:land-vehicles}). A
character that wishes to switch from driving a vehicle to moving on
foot (or vice versa) must spend one Action Round shutting down (or
starting) the vehicle.

A character that is controlling a robot's actions in an Action Round
may not himself move, fire a weapon or perform any other actions in
that Round (unless the robot has a self-activation system).



\subsection[Sizes Of Individuals]{The size of an individual or object
  determines how much space it occupies in a hex.} 
\label{sec:sizes-of-individuals}

The size classifications are: \emph{Minuscule} (smaller than a coin),
\emph{Very Small} (the size of a book), \emph{Small} (the size of a
small child), \emph{Man-size}, \emph{Large} (the size of a large
horse), \emph{One-hex}, \emph{Two-hex}, and so on.

There is no limit to the number of minuscule or very small objects or
individuals. That may occupy a hex. Twenty small, 10 Man-size, or
three large objects or individuals may occupy a hex. The size of an
object or individual affects weapon fire in the form of a modifier
applied to the Hit Chance (see the Fire Modifier Chart).


\subsection[Firing Weapons]{When eligible to move, a character may
  fire a loaded 
  weapon he possesses at any target within his sight.}
\label{sec:firing-weapons}

He may fire the weapon up to a number of times equal to the weapon's
\emph{Fire Rate} (listed on the Weapon Chart, Table
\vref{tab:weapon}). However, if he fires a number of times
\emph{greater than half} the weapon's Fire Rate, the weapon becomes
\emph{unloaded} and may not be fired again until loaded (an action
listed in \ref{sec:other-actions}). Thus, a weapon with a Fire Rate of
1 must be reloaded after each fire. As long as a weapon is fired a
number of times equal to or less than half its Fire Rate, the
character need not pause to reload. A character that is \emph{not}
skilled with a weapon may only fire it once in an Action Round.

For each fire he wishes to conduct, the character declares his
intended target and conducts the following steps:

\begin{enumerate}
\item Referring to the Weapon Chart, cross-reference the type of
  weapon being fired with the column corresponding to the number of
  hexes away the target lies to find the Base Hit Chance. This range
  is counted by including the target hex but not the firing
  character's hex.
\item Multiply the \emph{Terrain Value} in the environ by the
  \emph{Terrain Multiplier} (listed with the range on the Weapon
  Chart) and subtract this product from the Base Hit Chance. If the
  target is \emph{prone}, \emph{double} the subtraction.
\item If the firing character is skilled with the weapon, add his
  Dexterity Rating and the square of his Skill Level to the Hit
  Chance.
\item Consult the Fire Modifier Chart to see if any other additions or
  subtractions are applied to the Hit Chance, depending on the
  situation. Certain Hit Chances for manual weapons indicate that the
  character's \emph{Strength} Rating is also added to the Hit Chance
  (see the Weapon Chart).
\item The character rolls percentile dice. If the dice result is equal
  to or less than the modified Hit Chance, he hits the target; refer
  to the Hit Table to determine the damage incurred by target in
  accordance with \ref{sec:hits-and-damage}. (\textbf{Exception:} If a
  target is fired at more than once, damage is determined after all
  fires are conducted). If the dice result is greater than the chance,
  he misses the target; the GM may check to see if the fire hits a
  \emph{likely target} (see \ref{sec:misses}).
\end{enumerate}

A character may fire at any number of different targets in a single
Action Round (within the limit of his weapon's fire rate). However, a
reduction (listed on the Fire Modifier Chart) is applied to \emph{all}
fires if he does so. A reduction is also applied to all fires when
firing a weapon with recoil more than once in an Action Round (even if
firing at the same target).

A character in a hex occupied by an enemy capable of movement may only
fire at a target in that hex.

\subsection[Fire Modifier Chart]{The Fire Modifier Chart lists all the
  modifiers that may 
  be applied to the Hit Chance when firing a weapon.}
\label{sec:fire-modifier-chart}

See Table \vref{fig:fire-modifier}.


\begin{figure}[htbp]
  \centering
  \fbox{%
    \begin{minipage}{0.95\textwidth}
      \begin{center}
        \caption{Fire Modifier Summary}
        \label{fig:fire-modifier}
      \end{center}
      
      \textbf{\emph{Chance to Hit the Target\ldots}}\\
      Base Hit Chance \emph{minus} \\
      (Terrain Value \texttimes\ Terrain Multiplier) \emph{plus} \\
      character's Dexterity Rating (if skilled) \emph{plus} \\
      square of character's skill level \\
      Any of the following modifiers that apply are also considered
      and are cumulative.  

      \medskip

      \textbf{\emph{(Modifier) if the Firing Character\ldots}}\\
      \textbf{(-20)} \ldots moves on foot in the same Action Round. \\
      \textbf{(-30)} \ldots is driving a vehicle. \\
      \textbf{( -5)} \ldots is in a moving vehicle. \\
      \textbf{(-10)} \ldots is firing at more than one target in the same
      Action Round. Apply modifier to all fires for each  \\
      target beyond one (e.g. if 3 different targets are fired at, all
      hit chances are reduced by 20).  \\
      \textbf{( -5)} \ldots is firing more than one shot with a recoil weapon
      in the same Action Round. Apply modifier to \\ 
      all fires for each fire beyond one (e.g. if 4 fires are made,
      reduce all hit chances by 15) 

      \medskip

      \textbf{\emph{(Modifier) if the Target is\ldots}}\\
      \textbf{(Hit Impossible)} \ldots Miniscule \\
      \textbf{(-45)} \ldots Very Small \\
      \textbf{(-30)} \ldots Small \\
      \textbf{(+20)} \ldots Large \\
      \textbf{(+40)} \ldots One Hex \\
      \textbf{(+20)} \ldots Immobile\\
      \textbf{(Double Terrain Value)} \ldots Prone 

      \medskip
      
      The GM may apply further subtraction if the target is partially
      obstructed by a distinct object (remember natural terrain
      objects are accounted for by the Terrain Value and Multiplier).
      The Base Hit Chance and the Terrain Multiplier are listed on the
      Weapon Chart. The Terrain Value is listed on the Terrain Effects
      Chart.

      See \ref{sec:sizes-of-individuals} for additional weapon fire
      restrictions.  

      \medskip
    \end{minipage}}
\end{figure}

\subsection[Misses]{If a fire misses its intended target, the GM may check to
  see if a likely target is hit instead.}
\label{sec:misses}



The GM locates the difference between the dice result and the modified
Hit Chance on one of the following likely target results. If one
applies, that object or individual suffers the effects of the fire.

\textbf{1--30.} No other object hit. \textbf{Exception:} If the target
is engaged in close combat, the fire hits the target's adversary.\\
\textbf{31-50.} Fire hits an object or individual other than the
intended target in the target hex or in a hex that the fire passed
through (including
hexes beyond the target). \\
\textbf{51 or more.} Fire hits an item or individual in a hex adjacent
to any hex in the above result.

The GM determines which likely target is hit if more than one is
eligible. Any fire that does not hit its target or a likely target
always hits a \emph{wall} (if present) behind the intended target
(this could be very dangerous if a vacuum or a hostile atmosphere
exists beyond a pressurized chamber). \textbf{Exception:} A skilled
character firing an \emph{arc gun} will never hit an object,
individual or wall behind his intended target.

If the dice result is greater than the modified chance when throwing a
\emph{grenade}, the grenade strikes one hex away from the target hex
for every \textbf{10} percentage points (or fraction thereof) over the
chance the dice result indicates. The GM randomly determines which
direction from the target the grenade goes.

\subsection[Will Power Check]{A character without initiative must
  conduct a will power check if he is fired upon and does not incur
  damage.} 
\label{sec:will-power-check}



The character rolls one die; if the result is \emph{equal} to his
\emph{\index{aggression!and will power check}Aggression} Rating, he passes the check. The character may add
or subtract a number \emph{up to} his \emph{Mental Power} Rating or
his \emph{Battlefield} Skill Level (his choice) to or from the die
result. Thus, a character with an Aggression of \textbf{7} and a
Mental Power of \textbf{2} must roll a \textbf{5,6,7,8} or \textbf{9}
to pass a willpower check. If the character has a Battlefield Skill
Level of \textbf{3}, he must roll a \textbf{4} through \textbf{10} to
pass the check.

A character that passes a will power check may participate in Step
\ref{item:ar-4} of the Action Round. A character that fails a will
power check may not do anything in Step \ref{item:ar-4}; instead, he
must immediately perform one of the following:

\begin{itemize}
\item A character that fails a willpower check by rolling \emph{too
    high} must immediately fall prone in the hex he occupies or any
  adjacent hex that is further away from the source of fire (his
  choice). A character that is already prone does nothing.
\item A character that fails a will power check by rolling \emph{too
    low} must perform a rash action. He must immediately move into the
  hex occupied by the individual that attacked him (or as close to
  this hex as possible if his movement rate is insufficient). If the
  character is able to conduct a close attack or fire his weapon after
  moving he must do so (his choice if able to do both). If a rash
  character is firing his weapon, he must conduct as many fires as
  possible. He may not fall prone after his rash action.
\end{itemize}

Creatures never conduct willpower checks. A character on the side with
initiative does not conduct a willpower check when fired upon.

A character does not conduct a willpower check when attacked in close
combat.


\subsection[Close Combat]{A character or creature may attack an enemy
  in the same hex by using close combat, instead of firing a weapon.} 
\label{sec:close-combat}



Once close combat is initiated, the two participants are considered
\emph{engaged} until one or the other is stunned, passes out, dies or
breaks off. An individual \emph{without} initiative that is engaged in
close combat may not perform any actions \emph{at all} in step
\ref{item:ar-4} of the Action Round. An individual \emph{with}
initiative that is engaged may only conduct a close attack or attempt
to break off (see \ref{sec:other-actions}) during Step \ref{item:ar-2}
of the Action Round. A character is not required to initiate close
combat when in an enemy-occupied hex; if he is not engaged, he may
fire a weapon at a target in the hex instead.

The close combat strengths of the attacker and defender depend on how
each individual declares he will fight; \emph{unarmed}, with a
\emph{blade} or in \emph{body armor}. An individual may declare any
one of the three types for which he is eligible.

\begin{description}
\item[Attacking or defending unarmed.] The character's Dexterity,
  Strength \emph{or} \index{agility!and unarmed combat}Agility Rating (his choice) is added to the
  square of his Unarmed Combat Skill Level. If the character is not
  skilled, he uses one half (rounded up) of one of his Ratings only.
\item[Attacking or defending with a melee weapon.] The character's
  Dexterity or Agility Rating (his choice) is added to the \emph{Hit
    Strength} of the melee weapon (see the Weapon Chart, Table
  \vref{tab:weapon}) and the square of the Blade or Melee Weapon Skill
  Leve l. If the character is not skilled, he uses the Hit Strength of
  his Melee Weapon only.
\item[Attacking or defending in body armor.] The character's Strength
  Rating is added to the \emph{Hit Strength} of his armor and the
  square of his Body Armor Skill Level. If the character is not
  skilled, he uses the Hit Strength of his armor \emph{or} his
  Strength Rating only.

  \textbf{Note:} A character in body armor may declare that he is
  attacking or defending unarmed if he wishes. 
\item[Creature (attacking or defending):] The creature's Combat Rating
  is multiplied by one half (rounded down) of his \index{agility!and
  creature combat}Agility Rating.
  Thus, a creature with a Combat Rating of 6 and an Agility Rating of
  7 would have a close combat strength of 18.
\end{description}

After determining the close combat strength of the attacker and
defender, resolve the combat by \emph{subtracting} the defender's
close combat strength from the attacker's close combat strength to
determine the differential (a negative number is possible). The
attacker rolls \emph{one die} and \emph{adds} the die result to the
differential.

\begin{itemize}
\item If the sum of the differential and the die result is 4 \emph{or
    greater}, \emph{divide the sum by two} (rounding down) and locate
  the halved sum on the Hit Table to determine hits incurred by the
  \emph{defender} (see \ref{sec:hit-table}).
\item If the sum of the differential and the dice result ranges from
  -3 to 3 the attack has no effect. 
\item If the sum of the differential and the die result is -4 \emph{or
    less}, divide the \emph{absolute value} of the sum by two
  (rounding down) and locate the halved sum on the Hit Table to
  determine hits incurred by the \emph{attacker} (see
  \ref{sec:hit-table}).
\end{itemize}

Before resolving a close combat, the attacker (only) may declare that
he is attempting to \emph{restrain} his enemy. If this is declared,
reduce any hits received in the combat by his enemy by three. If the
adjusted number is 0 or higher, his enemy is considered restrained
(suffering the adjusted number of hits) and remains restrained until
the attacker releases him, or is stunned, passes out, or dies. A
restrained individual may not perform any actions at all. An
individual that is restraining an individual may not perform any
actions other than movement (however, see
\ref{sec:carrying-equipment}), unless it is a creature that possesses
the \emph{multiple attack} power or that has restrained its adversary
in \emph{webs}.

\section{Hits and Damage}
\label{sec:hits-and-damage}



A character, creature, or NPC may suffer hits; and a vehicle, machine,
or other piece of equipment may suffer damage, as a result of weapon
fire, close, combat, or an accident. Hits against an individual reduce
his Physical Characteristic Ratings. Points lost from Physical
Characteristics are regained by healing (see \ref{sec:rate-of-healing}
and the Diagnosis and Treatment skill descriptions). Damage to
equipment may puncture the object and/or render it inoperable until
repaired (see \ref{sec:repair} and the tech skill descriptions). The
Hit Table is used to determine all types of hits and damage, although
its use changes depending on the situation.

\subsection[Weapon Fire]{When an individual is hit by weapon fire, the
  character 
  who fired the weapon uses the Hit Table to determine the number of
  hits his target receives.}
\label{sec:hit-table}

He rolls \emph{one die} and \emph{adds the Hit Strength} of the weapon
to the die result. The modified die result is located on the Table to
find which physical characteristic is affected and how many points are
subtracted from that characteristics.

If an individual is hit by more than one fire from the same weapon in
a single Action Round, the Hit Strength of the weapon is
\emph{multiplied} by the number of times the target was hit. One die
is rolled and the result is added to this product to determine the one
result that will be applied to the Hit Table. Thus, the Hit Strength
of a paint gun that hits one target three times in one Action Round is
24 (3$\times$8).

The die is not rolled when referring to the Hit Table after a close
combat. The numerical result derived from a close combat (see
\ref{sec:close-combat}) is applied directly to the Table to determine
hits received.

When a creature is hit, the GM should secretly conduct the Hit Table
die roll and apply any results unannounced.

\subsection[Hits]{Hits suffered by an individual are applied as reductions
  to his Physical Characteristics.}
\label{sec:hits}

When a character incurs a hit result, the listed number of points are
immediately subtracted from the listed Characteristic Rating. This is
done by recording the reduced Characteristic Rating next to the
original rating on the Character Record (do not erase the original
rating). Until healed, the character uses the reduced rating for all
game purposes. When a character's \emph{Endurance} Rating is reduced
to \textbf{0}, he immediately \emph{passes out} and will not come to
until healing increases his Endurance to \textbf{1}. When a
character's \emph{\index{agility!reduced to zero}Agility} is reduced to \textbf{0}, he may not move
at all (he may use his hands to operate a small device or weapon). If
a character's \emph{Dexterity} is reduced to \textbf{0}, he may hold
nothing and fire no weapon (he may still move). There is no immediate
effect of a character's Strength being reduced to 0. However, if a
character's Strength \emph{and} Endurance are reduced to \textbf{0},
he dies.

A Characteristic Rating may never be reduced below 0. If a hit result
calls for a greater reduction to a characteristic than is possible to
apply, the excess reduction is applied to the next characteristic
listed down on the Hit Table (the number of hits received is not
increased). \textbf{Example:} A character with an Endurance Rating of
6 receives a hit result of 8 Endurance (result 16); his Endurance
Rating is reduced to 0 and his Strength Rating (the next
characteristic listed on the table) is reduced by \textbf{2}.
Exception: If the 38 or more result is incurred on the Hit Table, use
the next characteristic \emph{up} (Strength) after applying reductions
to endurance.

When applying hits to a creature, treat its Combat Rating as its
Strength and Endurance Rating (for purposes of using the Hit Table)
and treat its \index{agility!and hits to creature}Agility Rating as its Agility and Dexterity Ratings.
When a creature's Combat Rating is reduced to \textbf{0}, it passes
out. When its Combat and Agility Ratings are reduced to \textbf{0}, it
dies.


\subsection[Shock Check]{Each time a character or NPC suffers one or
  more hits, he 
  must immediately conduct a shock check.}
\label{sec:shock-check}

The hit character rolls one die. If the die result is equal to or less
than his \emph{Mental Power Rating}, he is unaffected. If the die
result is greater than his Mental Power Rating, the character
immediately drops whatever he may be holding and falls down; he may
perform no actions at all for the remainder of the current Action
Round and the entirety of the following Action Round. After the
following Action Round, he is no longer affected. Note that a
character with a Mental Power Rating of \textbf{0} will always fail a
shock check. A character that passes out or dies when hit does not
conduct a shock check. Creatures never perform shock checks.

\subsection[Stun Pistols]{An individual that is hit with a stun pistol
  pulse does not suffer any hits but may black out briefly.} 
\label{sec:stun-pistol}

The Hit Table is not used. Instead, the GM rolls \emph{one die} and
\emph{adds} the Stun Strength of the weapon (either \textbf{8} or
\textbf{16}) to the die result.  This modified result is compared to
the \emph{sum} of the target's \emph{Endurance} and \emph{Mental
  Power} Ratings. For every point above the sum the modified result
indicates, the individual is stunned for one Action Round. Thus, a
character with a combined Endurance and Mental Power Rating of
\textbf{12} that suffers a modified stun result of \textbf{17} is
stunned for five Action Rounds. A stunned individual immediately drops
whatever he is holding and falls down; he may perform no actions at
all until the requisite number of Action Rounds has passed. The count
of Action Rounds for stun duration does not include the Action Round
in which the individual is hit. The GM may wish to conceal the
duration of a stun result from the characters, informing them only
when the affected individual comes to.

The strength of a stun weapon is \emph{halved} (before adding a die
roll) if the target is wearing any type of full body armor or an
expedition suit. A robot is not affected by stun weapons.

When checking for the result of a stun pulse against a creature, use
its Combat Rating (only). Certain creature powers may render a stun
pulse ineffective or alter its effects (see \emph{Adventure Guide} \ref{AG-sec:creatures}). In
addition, the GM may choose to make creatures that have no central
nervous system (in the GM's opinion) immune to stun weapons.

\subsection[Equipment Damage]{Damage to a robot, vehicle, or other
  piece of equipment hit by fire or involved in an accident ranges
  from superficial damage to complete destruction of the object.} 
\label{sec:equipment-damage}

The Damage column of the Hit Table is used to determine whether the
object suffers superficial, light, or heavy damage, or becomes
partially or totally destroyed. An object that suffers more than
superficial damage is rendered inoperable until repaired (see
\ref{sec:technical-skills}; the exact nature of the problem is up to
the GM). An object that is totally destroyed may not be repaired.

When an object is hit by weapon fire, the damage result is determined
as described in \ref{sec:hit-table}, using the Damage column instead
of the Physical Characteristic column. A vehicle or robot that is hit
by fire may be protected by its armor Defense Rating (see
\ref{sec:armor-defense}).

Damage to a vehicle or other object in an accident is assessed in
accordance with \ref{sec:vehaccidents} and \ref{sec:accidents}. If a
vehicle has a Projectile Armor Rating, the armor may protect the
vehicle from damage. Note that any character in a vehicle that incurs
damage may suffer hits as a result.

Damage suffered by body armor and other protective attire is not
determined using the Damage column of the Hit Table. Instead, the
Armor columns are used, as explained in \ref{sec:armor-defense}.


\subsection[Protection from Armor]{A character wearing body armor or
  any other attire with a 
  Projectile and/or Beam Defense Rating receives protection from
  hits.}
\label{sec:armor-defense}



The Protective Attire Chart (\ref{tab:protective-attire}) lists a
\emph{Projectile Defense Rating} (use for protection from projectile weapons,
close combat, and accidents) and a \emph{Beam Defense Rating} (used for
protection from laser pistols, paint guns, arc guns, and other beam
weapons).

When a character incurs a result on the Hit Table due to any type of
combat or mishap, he checks the \emph{Armor Result} listed with the
hit result.

\begin{itemize}
\item If the Armor Result is less than the appropriate Defense Rating
  of the character's armor, the character and the armor are not harmed
  at all.
\item If the Armor Result is equal to the Defense Rating of the armor,
  the character is not harmed, but the armor suffers \emph{superficial
    damage} and \emph{both} of its Defense Ratings are reduced by 1.
\item If the Armor Result is greater than the Defense Rating of the
  armor, the armor is \emph{pierced} and the character receives hits.
  The Defense Rating of the armor (before the current hit) is
  subtracted from the number of hits the character receives, and the
  appropriate Physical Characteristic is reduced by this adjusted
  amount. If the Armor Result is greater than the Defense Rating by
  \textbf{1}, the armor suffers \emph{light damage} and \emph{both}
  Defense Ratings are reduced by \textbf{1}. If the Armor Result is
  greater than the Defense Rating by \textbf{2}, \emph{heavy damage}
  is suffered, and both Defense Ratings are reduced by \textbf{2}. If
  greater by \textbf{3}, the armor is \emph{partially destroyed}, and
  both Ratings are reduced by \textbf{3}. If greater by \textbf{4} or
  more, the armor is \emph{totally destroyed}. An Armor Defense Rating
  may never be reduced below \textbf{0}.
\end{itemize}

There are four columns of Armor Results on the Hit Table. When hit by
weapon fire, use the column corresponding to the total number of fires
that the target was hit by. When involved in a close combat or an
accident, use the \textbf{2 Fires} column of the Table.

All vehicles possess Armor Ratings, which are used in the same way as
personal armor to protect the machine from damage.  Certain creature
powers provide a creature with natural armor (with specific ratings
given in the power descriptions), used in the same way as character
armor.

\subsection[Toxin Effects]{The Toxin Effects Matrix is used to
  determine how an 
  individual hit by a treated projectile is affected.}
\label{sec:toxin-effects-matrix}

When an individual is hit by a needle from a needle pistol or rifle,
or is exposed, to toxic gas from a grenade, the GM cross-references
the type of toxin used with the composition of the target to determine
the effects of poison. All character and NPC targets use the
\textbf{Human} row of the matrix. A creature target uses the row that
corresponds to its composition (only known by the GM, unless the
characters have examined the creature, see \emph{Adventure Guide} \ref{AG-sec:creatures}).

A number from the matrix is treated as a Hit Strength; roll one die,
adding the number to the die result, and determine the hits suffered
by the individual as explained in \ref{sec:hit-table} and
\ref{sec:hits}. A letter result obtained from the matrix affects the
target as explained (see matrix). In addition to its toxic qualities,
a needle has a Hit Strength of 1. This applied to the target before
determining the effects of the toxin. If the needle does no harm to
the target, the toxin has no effect either.

\subsection[Toxin Effects Matrix]{The Toxin Effects Matrix explains
  the effects that various 
  toxins have on beings of various compositions.}
\label{sec:toxin-effects-matrix-1}

See Table \vref{tab:toxin-effects}.

\begin{table}[htbp]
  \centering
  \fbox{%
    \begin{minipage}{5.2in}
      \centering
      \caption{Toxin Effects Matrix}
      \label{tab:toxin-effects}
      
      \medskip
      
      \begin{tabular}{lccccc}
        COMPOSITION & NERVE & POISON & KNOCK-OUT & ACETIC & ALKALOID \\
        \rowcolor{grey}
        Human/Humanoid & 20t & P & 15s & Ne & Ne\\
        Mammalian & 10t & 15c & 5s & Ne & R\\
        \rowcolor{grey}
        Terran-Like & 15s & P & 10c & R & Ne\\
        Protein & P & D & Ne & 15s & 10t\\
        \rowcolor{grey}
        Carbon & D & Ne & 15t & P & 5c\\
        Non-Carbon & Ne & 10s & R & 10c & P\\
      \end{tabular}

      \medskip

      \parbox{\textwidth}{Number results indicate Hit Strength (see
        \ref{sec:hit-table} and \ref{sec:hits}).
        
        \textbf{Ne:} No Effect. \\
        \textbf{R:} Creature raging; will attack for remainder of
        current Action Round with its Combat Rating doubled.  \\
        \textbf{D:} Creature dazed; blacks out for a number of Action
        Rounds equal to the roll of one die. When the creature regains
        consciousness, it will be raging (result \textbf{R}).\\
        \textbf{P:} The individual loses one point from the Endurance
        Rating (Combat Rating, if creature) each Action Round. When
        the rating reaches 0, the individual loses one point from the
        Strength Rating (Agility if creature) each Action Round. When
        that rating reaches 0, the individual is dead.  \\
        \textbf{c:} Apply any hits received to the creatures' Combat
        Rating only.  \\
        \textbf{s:} Treat as stun strength (see \ref{sec:stun-pistol}). 
        \textbf{t:} Apply hits in accordance with the Hit Table.
        However, after a number of minutes (four Action Rounds apiece)
        equal to the roll of one die, the effects of the hits
        disappear (unless the individual dies as a result of the
        hits).

        \medskip

        See \ref{sec:toxin-effects-matrix} for explanation of use.}

      \medskip
    \end{minipage}}
\end{table}


\subsection[Hit Table]{The Hit Table is used to determine hits suffered by
  individuals and damage suffered by objects.}
\label{sec:hit-table-1}

See Table \vref{tab:hit}.


\begin{table}[htbp]
  \centering
  \fbox{%
    \begin{minipage}{5.25in}
      \centering
      \caption{Hit Table}
      \label{tab:hit}
      
      \medskip
      
      \begin{tabular}{llccccc}
        & \multicolumn{1}{c}{PHYSICAL} &&&&& DAMAGE\\
        \multicolumn{1}{c}{ONE DIE PLUS} &
        \multicolumn{1}{c}{CHARACTERISTIC} & \multicolumn{4}{c}{ARMOR
          RESULT} & TO\\
        \multicolumn{1}{c}{HIT STRENGTH} & \multicolumn{1}{c}{RATING
          POINTS LOST} & \multicolumn{4}{c}{(NUMBER OF FIRES)} &
        EQUIPMENT\\
        &&&&& 5 &\\
        && 1 & 2* & 3,4 & or more &\\
        \rowcolor{grey}
        1 or less & No Effect & - & - & - & - & NE\\
        2 & 1 Strength & - & - & - & - & NE\\
        \rowcolor{grey}
        3 & 1 Dexterity & - & - & - & - & NE\\
        4 & 2 Endurance & 1 & - & - & - & NE\\
        \rowcolor{grey}
        5 & 2 Agility & 1 & - & - & - & NE\\
        6 & 3 Endurance & 1 & 1 & - & - & NE\\
        \rowcolor{grey}
        7 & 3 Strength & 2 & 1 & - & - & S\\
        8 & 4 Endurance & 2 & 1 & 1 & - & S\\
        \rowcolor{grey}
        9 & 4 Agility & 2 & 1 & 1 & - & S\\
        10 & 5 Endurance & 2 & 1 & 1 & - & S\\
        \rowcolor{grey}
        11 & 5 Strength & 2 & 1 & 1 & 1 & S\\
        12 & 6 Endurance & 3 & 2 & 1 & 1 & S\\
        \rowcolor{grey}
        13 & 6 Dexterity & 3 & 2 & 1 & 1 & S\\
        14 & 7 Endurance & 3 & 2 & 1 & 1 & S\\
        \rowcolor{grey}
        15 & 7 Agility & 3 & 2 & 1 & 1 & L\\
        16 & 8 Endurance & 3 & 2 & 2 & 1 & L\\
        \rowcolor{grey}
        17 & 8 Strength & 4 & 3 & 2 & 1 & L\\
        18 & 9 Endurance & 4 & 3 & 2 & 2 & L\\
        \rowcolor{grey}
        19 & 9 Agility & 4 & 3 & 2 & 2 & L\\
        20 & 10 Endurance & 4 & 3 & 3 & 2 & L\\
        \rowcolor{grey}
        21 & 10 Strength & 4 & 3 & 3 & 2 & L\\
        22 & 11 Endurance & 5 & 4 & 3 & 2 & L\\
        \rowcolor{grey}
        23 & 11 Dexterity & 5 & 4 & 3 & 2 & H\\
        24 & 12 Endurance & 5 & 4 & 3 & 3 & H\\
        \rowcolor{grey}
        25--26 & 13 Strength & 5 & 4 & 3 & 3 & H\\
        27--28 & 14 Endurance & 5 & 4 & 4 & 3 & H\\
        \rowcolor{grey}
        29--30 & 15 Agility & 5 & 4 & 4 & 3 & H\\
        31--33 & 16 Endurance & 6 & 5 & 4 & 4 & P\\
        \rowcolor{grey}
        34--37 & 17 Strength & 6 & 5 & 5 & 4 & P\\
        38 or more & 18 Endurance & 6 & 6 & 5 & 5 & T\\
      \end{tabular}

      \medskip

      \parbox{\textwidth}{* Use this column when involved in 
        any Close Combat or Accident.
        
        \medskip

        
        \textbf{NE:} No Effect.\\
        \textbf{S:} Superficial Damage.\\
        \textbf{L:} Light Damage.\\
        \textbf{H:} Heavy Damage.\\
        \textbf{P:} Partially Destroyed.\\
        \textbf{T:} Totally Destroyed.\\

        \medskip

        See \ref{sec:hit-table} for explanation of use.}

      \medskip
    \end{minipage}}
\end{table}

%%% Local Variables: 
%%% mode: latex
%%% TeX-master: "gm_guide"
%%% End: 

