%%
%% $Id: additional.tex,v 1.1.1.1 2004/10/04 19:39:59 goltz20707 Exp $
%%
\chapter{Additional Material}
\label{sec:additional-material}

This chapter contains additional material provided by \emph{Universe}
players and GMs.  None of the rules or suggestions is required for
play, but they can enhance the believability of the universe the GM is
trying to create.  GMs should consider all of this material optional,
and incorporate none, some, or all of the rules below as they see fit.




\section{\index{Additional Equipment}Additional Equipment}
\label{sec:additional-equipment}


\subsection{Miscellaneous Equipment}
\label{sec:misc-equipm}

\begin{description}
\item[Multitool.] A multitool is a small, hand-held mini-repair kit,
  basically a smaller version of the basic repair kit.  It can
  temporarily repair superficial damage to non-weapon equipment less
  that 15 kg in mass.  The base repair chance is the same as for any
  other repair (see \ref{sec:repair}), except that after the first
  successful attempt, additional success rolls must be made \emph{once
    per day} at random (no additional repair time is needed after the
  first roll). If the second or subsequent rolls fail, the repair must
  be done again. The same damage can be repaired by a multitool
  \emph{no more than three times}; after the third repair fails, the
  damage must be repaired by a basic repair kit or better.

  After any failed attempt to repair with a multitool, the difference
  between the percentile dice roll and the success chance is
  \emph{subtracted} from the success chance, as if a repair attempt
  had failed (see Chapter \ref{cha:skills}).

  Civ level: 5.  Weight:
  0.5 kg.  Price: 30 Mils.

  \textbf{Note:} A multitool is also known as a ``swizarm'' or
  ``joatknife'' (the latter name derived from ``joat'',
  jack-of-all-trades).

\item[Star sailor wrist chronometer.] A wrist computer devoted solely
  to telling time.  Because star sailors travel to many different
  worlds, keeping track of local time as well as ship's time can be
  complicated.  The wrist chronometer tracks ship time (Universal Star
  Sailor Time) as well as the local planet's standard time.  It can
  receive time signals transmitted from Federal Star Sailor faclities
  (all class 3 and 4 spaceports) and can be manually set to track time
  according to any arbitrary scheme.

  Civ level: 6.  Weight: 0.1 kg. Price: 1.5 Trans.

  For purposes of pricing on the Actual Price Table (table
  \ref{tab:actual-price} on page \pageref{tab:actual-price}), the
  wrist chronometer should be classed with computer components.

  \textbf{Note:} The GM may wish to make wrist chronometers a benefit
  to all in the Star Sailor profession, and possibly other professions
  of a multi-system nature (interstellar traders, for example). See
  \ref{sec:benefits}.

\end{description}


\section{Rule interpretations}
\label{sec:rule-interpretations}


\subsection{Hyperjump (Sections \ref{sec:psionicskills},
  \ref{sec:interstellar-travel})}
\label{sec:hyperjump}

A successful jump must terminate outside of a star's gravity well, at
a distance equal to or greater than that given in
\ref{sec:interstellar-travel}.  It need not terminate at the ``jump
point'', and can in fact terminate in interstellar space, but must
terminate at the jump point for the starport class increases listed in
\ref{sec:psionicskills} to take effect.  (These increases have to do
with ``familiarity'' with a star system, and this familiarity requires
a known configuration of planet-star-starship.)

\emph{Justification:} Section \ref{sec:jump-point} states:

\begin{quote}
  A jump point for departing or arriving at a given planet lies above
  or below the plane of the ecliptic for the star system, such that a
  line drawn from the point to the planet would be perpendicular to
  the ecliptic.
\end{quote}

Obviously, then, there can be more than one jump point in a system.

Further, Section \ref{sec:interstellar-travel} says:

\begin{quote}
  Failure of a navigator to wrest the monopoles to his mental command
  often results in a jump error and/or psychic damage to the psion.
  Gravitational fields increase this risk by disrupting the stability
  of the monopole screen that the navigator projects his mind onto.
  Because of this, spaceships must travel to the outskirts of a star
  system by conventional propulsion before a jump may be conducted.
  Conversely, the destination of a jump must be outside the gravity
  wells of the system so that the navigator's mind and the integrity
  of the ship will not be destroyed by monopoles reacting to forces
  other than his own thoughts.  The point outside a star system that
  is closest to a given planet and yet far enough away from all
  gravitational fields in the system to conduct a safe hyperjump from
  is called that system's jump point.  It is to this point that a
  navigator will bring a spaceship as the result of a perfect jump.
\end{quote}

Also, we see in the hyperjump table (table \ref{tab:hyperjump}) that
jumps can terminate beyond a jump point, but almost never within the
gravity well of a star (and there is spacecraft damage in that case).

The implication to me is that the main concern is that the jump must
be made outside the gravitational influence of a star, i.e. farther
from the star than the distances given in \ref{sec:jump-point}.  The
fact that the jumps begin and end directly above the plane of the
ecliptic from a given planet is merely a matter of convenience: that's
the closest point to the planet which is still the given minimum
distance away from a star.

In the case of a jump to an uncharted star, the jump can terminate
anywhere outside the minimum distance from the star.  If there's some
idea what planets circle that star, then the jump might terminate as
close to one of them as possible.  But for a star with no known
planets (or no planets at all), the only thing known is that a
successful jump will terminate outside the minimum distance.


\subsection{Planetology (Section \ref{sec:scientific-skills})}
\label{sec:planetology}

The base time required to perform a given planetology task is two
hours \emph{plus} two hours for each task above the listed task.  For
example, if a player attempts to determine the atmosphere of a world
(which, if successful, also determines the land/water distribution and
temperature of all environs), the base time is six hours (two for the
atmosphere determination, plus two each for land/water distribution
and temperature).


\emph{Justification:} It is assumed that these tasks are being done
from orbit.  Each task takes longer on a larger world (more
surface area to cover), but can be done more quickly (the ship orbits
more quickly), so effects due to  world size cancel each other out.
We can assume that tasks with a lower chance of success take longer
than easier tasks, and since the tasks are arrayed from largest to
smallest base chance, the lower tasks in the list should take longer
than the higher ones.

Since each task requires determining something about the entire world,
the shortest time for a task should be one orbit.  This depends on
planet size and altitude, but in general a larger planet has a shorter
orbital period (more mass), so once again these effects cancel each
other out.

Given a fairly reasonable orbital period of two hours, that means that
(for example) determining the general resources of a world requires
four orbits, or eight hours.  (The minimal orbital period for Earth is
90 minutes; a two-hour orbit has an altitude of about 1000 miles.)




\section{Optional Rules}
\label{sec:optional-rules}



\subsection[World Axial Tilt]{The GM may determine the axial tilt of a
  world being generated.}
\label{sec:world-axial-tilt}

When generating worlds (see Chapter \ref{cha:world-generation}), the
GM may wish to determine the \emph{axial tilt} of a world --- the
angle between its orbital plane and its equator.  The axial tilt
affects how much sunlight reaches a given spot on the planet at a
given point in that planet's orbit.  This can dramatically affect
climate; a world with little or no axial tilt will have little or no
seasonal changes, while a planet with an axial tilt near 90\textdegree\
would have drastically different seasons.  Each hemisphere of such a
planet would be without sunlight for half of its orbit and without
darkness for the other half.  Essentially, its ``Arctic Circle'' would
be at its equator.

It is recommended that, if axial tilt is generated for planets, day
length (section \ref{sec:day-length}) be determined as well.

To determine axial tilt, the GM refers to the \textbf{World Axial Tilt
  Table} (page \pageref{tab:axial-tilt}) and locates the column which
matches the world type.  He then rolls one die and records the result.
A result with an \textbf{R} next to it means that the planet's
rotation is \emph{retrograde}: it spins on its axis in a direction
opposite to its motion around its star.  This will not have much
effect except for planets whose day length is a substantial fraction
of its year length.  \textbf{Example:} In the Solar System, the planet
Mercury has a year of 88 Earth days and a ``day'' (rotation period) of
59 days.  Mercury's rotation is \emph{prograde} (the opposite of
retrograde), so its solar day (the period from one sunrise to the
next) is 186 days --- twice as long as its year!  If its rotation were
retrograde instead, its solar day would be only 37 days.  Its
retrograde rotation essentially helps the sun to ``catch up'' faster.

Planets with an axial tilt greater than about 15\textdegree\ will be
warmer in summer and colder in winter than the temperature
designations on the World Logs indicate.  (Remember that if one
hemisphere is in summer, the other is in winter, and vice versa.)
Planets with an axial tilt greater than about 40\textdegree\ will also
have drastic weather during summer and winter, as one hemisphere will
be constantly sunlit over most of its surface and the other in
darkness.

\begin{table}[htbp]
  \caption{World Axial Tilt Table}
  \centering
  \fbox{%
    \begin{minipage}{3.5in}
      \centering
      \label{tab:axial-tilt}

      \medskip

      \begin{tabular}{cccc}
        & \multicolumn{3}{c}{WORLD TYPE}\\
        ONE DIE & EARTH-LIKE & TOLERABLE & HOSTILE\\
        \rowcolor{gray}
        1 & 10 & 0 & 50\\
        2 & 14 & 0 & 55\\
        \rowcolor{gray}
        3 & 16 & 5 & 60\\
        4 & 18 & 35 & 75\\
        \rowcolor{gray}
        5 & 20 & 40 & 90\\
        6 & 22 & 45 & 90\\
        \rowcolor{gray}
        7 & 24 & 10R & 15R\\
        8 & 26 & 5R & 40R\\
        \rowcolor{gray}
        9 & 28 & 0R & 45R\\
        10 & 30 & 0R & 70R

      \end{tabular}

      \medskip

      \parbox{\textwidth}{Numbers represent axial tilt in degrees.\\
        \textbf{R:} Retrograde.

        See \ref{sec:world-axial-tilt} for explanation of use.}
    \end{minipage}}
\end{table}

%%% Local Variables:
%%% mode: latex
%%% TeX-master: "gm_guide"
%%% End:
