%%
%% $Id: introduction.tex,v 1.1.1.1 2004/10/04 19:40:00 goltz20707 Exp $
%%
\externaldocument[DV-]{delta_vee}

\chapter{Introduction}
\label{sec:introduction}


\section{The Universe and the Future}
\label{sec:universe-future}

Rakal, the Expedition Chief, surveyed the hot, virulent terrain.  The
movement had set him on his guard; he worried for the eight other
members of the Survey Team.  Looking at his Bio Officer he asked,
``What form do you think it takes?''

``I think the creature is in great pain,'' replied Mylina.  ``The
readings on my bio scanner indicate a skittishness; perhaps our
scanning has caused it harm.  May I send my Scoutbot over to inspect
it at closer range?''

``Yes,'' answered Rakal hesitantly.  At Mylina's verbal command the
Bot whirred through the thin atmosphere and began to traverse the 100
meters to the creature.  ``Do we know the chemistry of its energy
transferences?''

``It is an alkaloid life form, Chief.'' The robot's metallic twang was
clear through the respirator helmet speaker.  This model had been with
the crew a long time, Rakal thought.  He squinted as he looked into
the harsh blue-white sun, following the bot's flight.

Rakal was concerned.  He had seen too many specimens damaged through
ignorance.  Equipment couldn't think for itself --- man still must
rely all too much on his instinct.  If only the Psion were along on
the Survey.  Why \emph{had} he allowed Kk'rral, his navigator, to
remain shipboard? ``The Shuttlecraft needs no navigator to reach the
surface,'' the psion had scorned.  ``Leave me my astrogation chores
and our courses will be plotted by your return.  Good hunting!''

Kk'rral, however, might have been able to communicate with this
creature\ldots where was the Scoutbot, anyway? It should have returned
or at least given readings.

``Mylina, any readings from the Bot?'' Mylina was busy staring into
the distance and paid Rakal no heed.  ``Mylina, any readings?''

He quickly looked around the perimeter at the remainder of the Survey
Team.  They, too, were staring at some point in the distance.  Rakal
quickly drew his paint gun.  His instinct not to watch the creature
had probably just saved his mind.  His mind, however, now was filled
with moans of great pain; pain that was not his own\ldots he ran.
Looking for cover by the ridge, he was stopped by an outcropping of
cyan rock, so he turned north.  The moaning increased.  He dared not
look back but knew the creature followed.  Rakal needed to circle back
to the Shuttle and return to the ship; unless he somehow managed to
slow the creature down this would be impossible.

As he searched for higher ground, the moaning increased painfully.
Realizing it was now or never, he turned to face his pursuer.  He
aimed the weapon at the beast.  He touched the trigger lightly to
activate the aiming beam.  The creature approached rapidly.  The aimer
found the cortex, or what appeared to be the cortex; Rakal fired.  The
coherent light splashed as the creature's natural armor splayed the
beam everywhere.  Rakal retreated and fired again.  Again the beam
scattered wildly\ldots Rakal felt a burn on himself from a reflection.
Rakal thought of running, but the beast no longer seemed ugly; what a
lovely melody he found running through his brain.  Wasn't that the
creature's song? He stood up as it drew very close.  Looking up into
the enormous open maw, his last thought was how warm and soft it would
be once he got inside.

\bigskip

\begin{center}
  \rule{8cm}{0.5mm}
\end{center}

\bigskip

It is the 24th Century.  Mankind, searching for its destiny, has begun
to explore and colonize the Milky Way.  The stars themselves are now
man's playground; they are also the seat of his greed.

After having overcome the temptation to annihilate himself, man found
the time to let science proceed with its work.  The most important
discovery of the 21st Century was the isolation of the chromosome
controlling the intensity of psionic talent.

Once this was isolated and brought out of its dormant state, the
psionic abilities of some individuals could be awakened.  This led to
a long series of experiments in telekinesis, psychokinesis and,
ultimately, in forms of aportation.  It was thus that faster than
light travel via mass teleportation (called hyperjump) was discovered.
The Psions were found to act as catalysts for both travel and
communications between the stars.  They became very wealthy and aloof,
as their talents tended to isolate them from society.  Feared by the
ignorant and the untalented alike, they guarded many secrets and
controlled the throttle on man's outward expansion.

At last the dream of leaving behind the earth was within mankind's
grasp.  The final frontier was before him.  The earth, politically
united at last, funded the first colonizing missions.  Each major
national power sent its own ships to habitable worlds to explore and
colonize.  To govern these new worlds, the earth founded the
Federation of Planets.  Not to be caught in the old mistakes once
again, the federation was a loose conglomeration of these worlds
rather than a tight-fisted ruling body.  Some of the larger colonies
set themselves up as independent states, practically free from federal
authority.  The federal influence was exerted at the spaceports within
the star systems.  No direct control was insisted upon; rather light
but very strictly enforced taxation was the method of the times.  The
armed might of the federation was considerable, consisting of
far-reaching army and naval forces (both space-faring).  Local forces
resented their presence, but were usually no match in any combat
situation.

The federation still controlled, to a large extent, the monetary
systems.  This was to insure fair trade amongst the stars.  The unit
of currency was the \emph{Tran}, the abbreviation for one Transfer.
%% Original:
%% This was the equal of 500 1980 dollars.
%% New:
This was the equivalent of 1000 dollars in the year
2000.\label{sec:universe-future-change-tran}
%% End change
The Tran was subdivided into one thousand Milli-transfers or Mils.

With fair (or at least reasonably fair) trade ensured by federal
measures, business and technology flourished.  Robots became
economically feasible, as well as technologically, and they were as
commonplace as the home computer of the 1980's.  Business became the
major sponsor of scientific missions, mainly to discover new resource
sites and discover the means of bringing these resources back.  This
exploration spawned a new curiosity on the part of humankind: the
capturing and exhibition of alien creatures discovered on these new
worlds.  Trade involving live specimens was lucrative, and every week
brought the news of a new find.

\index{languages|bb}The initial colonies met with mixed success; some floundered while
others blossomed.  Many kept their own national flavor; Russian,
Chinese and Hindustani could just as easily be heard on a colony world
as \emph{Universal}, the business language of the era.  Universal had
its roots in the English of the 20th century, but was as closely
related to that as English was to the Middle English of the 11th
Century.  Amongst the languages that still existed were German,
Japanese, Chinese, French, Italian, Spanish, Russian, Swedish,
%% Original:
%% Afrikaans,
%% New:
Swahili,\label{sec:universe-future-change-swahili}
%% End change
Hindustani, Portuguese, Danish, Dutch, Arabic, Greek, and Arctican.

As mixed as the national flavor of the colonies was, so was their
level of economic and technological development.  Although exploration
was speeded exponentially due to hyperjumping, commerce between stars
was still slow due to the long transit time from the jump points into
a system and the world itself.  This jump point had to be outside the
system's gravity well; thus the distance was often very great.

The pressures facing mankind's first colonists were tremendous; the
mettle of mankind's best and brightest was sorely tested.  Some worlds
found new strengths under this pressure while others reverted to old
ways for the peace that they hoped simplicity would bring.  Still
others just could not match the space faring technology of the era
with the resources at hand in the sometimes hostile environments.
Thus the level of civilization in each colony found its own level.
Worlds developed into major powers within their sector next to others
who found it quite enough just to simply survive.  As always, business
found ways to make either situation pay off.  And since business was
now fronting almost all the scientific missions, the federals kept
their interference to a minimum.


And what of Earth and its future history? The planet, much battered
and misused, had survived pretty much intact into the era.  Much of
its life and livelihood revolved above the surface; the age of active
space station technology had moved industry into orbit.  Along with
the many spaceports launching one colony ship after another, these
working space factories cluttered the orbital lanes around the earth.
This development had allowed much of the land to be reclaimed and
enjoyed once again by the people; massive refertilization projects
were undertaken and a green and beautiful Earth once again flourished.

No contact with intelligent alien life had occurred.  Long having been
both the dream and the nightmare of mankind's new star-faring
mobility, discoveries of ruins and numerous artifacts and burnt out
spaceship hulls of unknown design saw hopes (and fears) rise.


Civilization had progressed, but still man needed to strive and take
risks to advance.  The Universe held wonder and surprise; danger
lurked around every outcropping of ore.

This vision of the future lies before you, as you own this game.
These rules are your guide and your key to the Universe; the journey
is yours to take.


\section{The Gamemaster and the Players}
\label{sec:gamemaster-players}

\emph{Universe} is a role-playing game, and in such a game there is no
winner or loser; rather, the \emph{Gamemaster} and the \emph{players}
interact in a non-combative way to resolve \emph{adventures}.  The
Gamemaster acts as a referee.  Within the framework of the rules, he
objectively determines the effects of the players' actions.  The
players act out the part of their character as if that character
actually existed in the universe the GM has created.  The actions that
the players take within the GM's universe may be grouped together to
form \emph{adventures}, which are similar in length and complexity to
a short story.

The GM is a master storyteller, a weaver of tales that deal in the
unknown: unknown worlds, uncharted star systems, unmapped reaches of
the galaxy.  These stories, strung together, form the campaign.  Players
yearn to lose themselves in this ``alternate existence,'' and the GM
is the one who creates it.

The rules of \emph{Universe} are intended as a framework in which the
GM creates adventures for the players.  These adventures can exist
individually or be strung together to form a campaign.  Either method
is viable; a coherent campaign takes much more of the GM's time to
create and maintain but there are certain joys that come from seeing
characters grow, story lines interweave, and history actually being
created.  The story is the GM's, and he has final say concerning
anything to do with that story.  This includes the rules, which he is
free to alter to fit his individual needs.  However, the GM should not
take free license with the rules; they were not published to be
disregarded.  Rather, careful inspection by a conscientious GM will
yield what modifications he can make without unbalancing play.  The
players are great sources of feedback on this; they will let the GM
know loud and clear if something about play bothers them.

To feel confident enough to alter any rules, and to GM \emph{Universe}
well, the GM must know the rules intimately.  Since he is the final
arbiter, this knowledge is a must.  Many concepts in the game will be
foreign, and it is up to the GM to know and explain them to his
players.  If this means playing ``unofficial'' sessions to solve
problems, do it! Without this familiarity with the rules, play will
not flow smoothly.

This flow of play is critical to a well run adventure.  Since the GM
is an entertainer, he is putting on a performance.  All the players
must be kept interested and involved throughout play.  If this is not
done, play will bog down and become dull.  An interesting flow of play
supersedes almost all else; let the technical details slide.  The
players won't mind a detail flubbed if the story was really
interesting.  GM's should discourage things that slow down play (i.e.,
players leaving the room, talking too much amongst themselves, the GM
having to look up too many rules).

Every player should be treated equally and fairly by the GM; all
characters should be given chances to perform tasks, as this is how
characters advance in \emph{Universe}.  Players respect a GM who is
impartial and as interested in seeing the characters advance as he is
in seeing his story work out.  If the players think the GM is not
treating them fairly, this will sow a seed of discontent, which will
end the campaign or instigate arguments.  The players need to trust
the GM.

In preparing the adventures for the players, give thought to the
balance of danger and challenge.  No player wishes to solve every
problem all the time; on the other hand, no player wants to be beaten
all the time.  The GM should constantly challenge his players'
abilities, both mentally and emotionally.  Encourage role-playing by
enacting the non-player characters to the hilt; the GM should use them
to make the players think on their feet by engaging them in direct
conversation.

The adventures must be varied, and the GM will need a lot of input to
remain creatively fresh.  Use many different sources: science fiction
literature, television, movies, and your players.  Take ideas and
inspirations from these sources to make your own.  The players will
tire of a series of adventures all dealing with a similar theme; the
ideas must be varied.  If a campaign is created, very often ideas for
new adventures will materialize right out of the play itself --- the
story takes on a life of its own.  Players have many responsibilities
of their own; the creation of a well-run science fiction campaign is
the sum of all who play.  The voices of the players must be heard loud
and clear.  After all, why play the game if not for enjoyment and
escape? The GM must be made aware of what the players desire out of
\emph{Universe}.

If the players wish the GM to listen to their ideas and desires, they
must be willing to take on their share of the burden for the game.
The players must know the rules that they will use all the time (use
of a skill, for instance).  Also the players must respect the GM as
the final arbiter of the rules and on events that occur within his
universe.  It is his creation and if he didn't know the secret reasons
for things, who would? This knowledge, which the players are not
entitled to, may cause things to occur the players do not understand.
They must accept this convention and abide by all final rulings the GM
may deliver.  The player must take responsibility for keeping the
record pertaining to his character.  This burden should not concern
the GM, who certainly has enough to worry about without keeping tabs
on how much is in any character's bank account, for instance.  And
finally, the players must have respect for the time and enjoyment of
the other players and the GM.



\section{Sequence of Events}
\label{sec:sequence-events}

When players and a GM get together to play \emph{Universe}, a certain
sequence of events will usually occur.  These are outlined below, and
most games will follow this sequence, more or less.  The following are
not rules; rather take them as guidelines, which may or may not be
adhered to.

\begin{enumerate}
\item The GM prepares the adventure.  This may involve many hours of
  pre-play preparation on his part, creating the scenario the players
  will be involved in.  The GM generates non-player characters (or
  NPC's), worlds, creatures, etc., in order to give the adventure the
  elements needed to flesh it out.  This preparation is the most
  important part of the sequence for the GM, as mistakes or
  assumptions made at this point will be almost impossible to correct
  once play begins.  The serious GM spends as least as much time
  preparing an adventure, as it will take to play it.
  
\item The GM and the players agree to meet and play.  All concerned
  parties should set aside a block of time 4 to 6 hours in length.
  This is the average amount of time it takes to play through a
  short-to-medium length adventure, and two of these sessions should
  finish off almost any adventure.  Remember that the time from
  beginning to end is not all spent playing (human beings being what
  they are), and the GM should be aware of this when he estimates how
  long an adventure would take to complete.
  
\item The players then choose or generate characters.  If the
  adventure is part of an extant campaign, most players will have
  characters already.  A new player or a player whose character just
  died will have to generate a new one.  It is advised that these
  individuals arrive early and take care of this, but it can be done
  just before the adventure starts.  Any monetary maintenance that
  characters have to take care of is done now.  This includes room and
  board, equipment purchase, spaceship upkeep, etc.
  
\item The GM presents the adventure.  This may be done any number of
  ways.  Some of the more common are through a pre-arranged meeting
  with a sponsor, picking up a story line left over from a previous
  adventure session, or letting the players go do what they wish and
  let them find adventure.  A non-player character for instance, could
  approach the players and offer employment (see the adventure in the
  \emph{Adventure Guide}, \emph{Lost on Laidley}), or coerce the players, or
  drop a clue as to some interesting occurrence the players might
  become curious about.  The goal for the GM is to entice (not force)
  the players into going on the adventure he had planned for them.
  Players do not often like being forced into doing something.
  Sometimes it is the only way (as in a hijacking scenario), but a
  constantly maneuvered player is an unhappy player.  If the adventure
  created by the GM is interesting enough, the players will want to go
  without much provocation.
  
\item The adventure begins and continues until resolved.  This may
  force the playing of an adventure beyond the time that a session
  must end.  In that case, the GM ``freezes'' time, and the players
  pick up where they left off the next time they meet.  Play will
  normally continue until either the characters succeed, they fail, or
  they aren't sure and return from whence they came.  During play, the
  GM must act as narrator, describing the events as an impartial
  observer and giving the players all the information they would
  ordinarily become entitled to.  The GM plays the parts of various
  NPC's, describes graphically locations the characters find
  themselves in, resolves combat (taking the side of the enemy or
  creature), and tells the players the results of their actions.
  During the adventure, the players have as much if not more control
  than the GM due to the decisions they must make.
  
\item The adventure is resolved.  The characters have succeeded,
  failed, or staged a strategic withdrawal.  Any adjustments to the
  characters' records are made and plans discussed to meet and play
  again.  If this was a single adventure, not tied in with an extant
  campaign, the GM may wish to give the players the answer to the
  adventure (such as it might be), if they did not find it themselves.
  In a campaign situation this would not occur, as many answers are
  yet to be discovered through continued play.
\end{enumerate}


\section{Requirements for Play}
\label{sec:requirements-play}


%% Original:
%% \subsection[Contents of Game]{A complete copy of \emph{Universe}
%%   should include:}
%% \label{sec:contents}

%% \begin{itemize}
%% \item 1 22'' x 33'' Interstellar Display
%% \item 1 Gamemaster's Guide Book
%% \item 1 Adventure Guide Book
%% \end{itemize}

%% The following parts are included in the boxed version only:

%% \begin{itemize}
%% \item One 17'' x 22'' Tactical Space Combat game map
%% \item One \emph{Delta Vee} rules booklet (tactical space combat)
%% \item One 200 die-cut \emph{Delta Vee} counter sheet
%% \item Two 20-Sided dice
%% \item One Counter tray
%% \item One Game Box
%% \end{itemize}

%% New:
\subsection[Contents of Game]{The game \emph{Universe} consists of the
  following:}
\label{sec:contents}

\begin{itemize}
\item The \emph{Gamemaster's Guide}, containing general information
  for the GM and players
\item \emph{Delta Vee}, the space combat rule system
\item The \emph{Adventure Guide}, containing information on encounters
  that is private to the GM
\item The \emph{Interstellar Display} (see \vref{sec:interst-display})
\end{itemize}

The GM and players will have to supply the following:

\begin{itemize}
\item At least 20-sided or 10-sided dice
\item Hex grid maps for use in \emph{Delta Vee}, if space combat
  becomes necessary (see \emph{Delta Vee}\ref{DV-sec:game-map})
\item \emph{Delta Vee} chits for use on the above maps
\end{itemize}
%% End change

\subsection[The Interstellar Display]{The Interstellar Display shows
  the positions, in three  dimensions, of every known star within 30
  Light Years of earth.}
\label{sec:interst-display}

Each star's location is shown using three coordinates (x, y, z).  Each
coordinate represents a distance in light years from our sun (Sol).
The x and y coordinates are also shown visually, by the star's actual
position on the display.  The z coordinate is a positive or negative
number representing the star's distance above or below the plane of
the display (the plane of the earth's equator).  Also included on the
display is a chart listing the distances between other stars.  The
Interstellar Display is not used as a playing surface.  It is intended
as an information source for the players and the GM.  The GM chooses
stars from the display (and uses the information provided with each
star) when he generates worlds.


\subsection[Photocopiable Masters]{The Gamemaster's Guide includes
  many masters of logs and records that must be photocopied before
  use.}
\label{sec:photocopy-masters}

These include the Character Record, Star System Log, Environ Hex Map,
and eight pages of different sized World Logs.  The logs in this book
should be carefully removed to facilitate photocopying.  Note that
World Logs 8 and 9 take up two pages each.  SPI grants permission to
photocopy all this material for personal use.


\subsection[Die Rolls]{All die rolls in  \emph{Universe} are conducted
  with one or two 
  20-sided dice only.}
\label{sec:die-rolls}

In a given situation, the GM or one of the players will be called upon
to roll dice in one of three different ways: roll \emph{one} die, roll
\emph{two} dice, or roll \emph{percentile} dice.

\begin{description}
\item[One Die.] Roll one 20-sided die and read the result.  A result of
  \textbf{0} is \emph{always} considered a \textbf{10}.  Thus, a range
  of numbers from \textbf{1} to \textbf{10} is possible when rolling
  one die.
\item[Two Dice.] Roll two 20-sided dice and \emph{add} the two results
  together.  On both dice, a result of \textbf{0} is considered a
  \textbf{10}.  A range of numbers from \textbf{2} to \textbf{20} is
  possible when rolling two dice.  Example: If one die result is a
  \textbf{4} and the other is a \textbf{7}, the two-dice result is
  \textbf{11}.
\item[Percentile Dice.] Roll one die and read the result as the
  \textbf{10}'s result (i.e., multiply the result by \textbf{10}).
  Roll the second die and add its result to the first.  When rolling
  percentile dice, a result of \textbf{0} is considered a \textbf{0}
  \emph{unless both dice} show a \textbf{0} result, in which case the
  result is read as \textbf{100}.  A range of numbers from \textbf{1}
  to \textbf{100} is possible when rolling percentile dice.
  \textbf{Example:} If the first die result is \textbf{5} and the
  second die result is \textbf{9}, the percentile dice result is
  \textbf{59}.
\end{description}

\textbf{Note:} By using two dice of different colors, a percentile
number may be rolled quickly.  One die is declared as the \textbf{10}'s
die, and then both dice are rolled together.


\subsection[Play Aids]{The GM and the players must provide some
  miscellaneous play aids.}
\label{sec:gm-players-must}

All concerned should have a pencil with a good eraser.  The GM should
have a set of colored pencils or markers to draw world logs, environ
hex maps, and other game displays.  Plenty of scrap paper is also
fervently recommended.

A large hex grid map (with 19 mm or 25 mm hexes) is recommended for
use as the Action Display (see \ref{sec:gm-track-location}).  The
tactical space combat maps may be used if nothing else is available.
Miniature figurines or cardboard counters are recommended for
conducting Action Rounds on the Action Display.

Additional 20-sided dice are helpful in speeding up play.  Some
players (and GM's) prefer their own personal pair of dice.  The GM
will find a pocket calculator most helpful.




%%% Local Variables: 
%%% mode: latex
%%% TeX-master: "gm_guide"
%%% End: 
