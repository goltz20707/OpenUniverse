%%
%% $Id: skills.tex,v 1.4 2004/11/04 15:32:02 goltz20707 Exp $
%%

% ZZYZZ hyperlinked to here Fri -10-01 1344

%
% The following two new counters are necessary to keep \LaTeX\ from
% complaining.
\newcounter{creature}
\newcounter{npc}
\externaldocument[DV-]{delta_vee}
\externaldocument[AG-]{adventure_guide}

\chapter{Skills}
\label{cha:skills}

\index{skills}Characters use skills in all aspects of play.  Each skill provides a
character with expertise in a specific area of endeavor.  The skill
descriptions in this Chapter explain when and how each skill can
affect play.  Each character receives skills during character
generation and, once he begins adventuring, a character may improve
those skills and acquire others.

\index{skills!maximum level}Each skill description lists the \emph{maximum level} that may be
achieved in that skill (for example, the maximum level that may be
achieved in the geology skill is \textbf{7}).
\index{skills!characteristic limit}Each skill also lists a
\emph{characteristic limit}, which is the highest level a character
may achieve in a skill when his Characteristic Rating appropriate to a
particular skill is lower than the maximum limit for that skill (thus,
a character with an Intelligence Rating of \textbf{5} can increase his
Geology skill to a maximum level of \textbf{5}).  These limits are
summarized on the Character Record.

When the GM is organizing an adventure or offering employment to the
characters, he should take their skills and professions into account
(for example, a character with high scientific Skill Levels should
have no problem finding a sponsor for a scientific expedition).  During
play, situations will surely arise that call for character abilities
not specifically explained in these skills.  After familiarizing
himself with how the skills work, the GM should easily be able to
determine the chances and effects of any task the characters wish to
attempt.

\index{skills!tasks}Many skills are presented in the form of one or more \emph{tasks},
each representing a particular use to which a character may put the
skill during play.  Each task is noted with a \tasklistmark\ symbol and
has a \emph{base chance} of success expressed as a percentage.  Each
task is carried out according to the following procedure and any
special rules listed with the skill or task description.

\renewcommand{\theenumi}{\arabic{enumi}}
\begin{enumerate}
\item The character declares which specific task listed in the
  appropriate skill description he is undertaking.  In most cases, he
  must have access to a particular piece of equipment.
\item One of the character's \emph{Characteristics Ratings} (specified
  in the skill description) is added to the \emph{square} of his
  \emph{Skill Level}, and the total is \emph{added} to the \emph{base
    chance} listed for the task.  In some instances, the Skill Level
  may be increased by a piece of equipment or decreased if the task is
  especially difficult.  The base chance may be further modified by the
  task description or at the GM's discretion (in some cases, he may
  apply a modifier secretly).
\item The character rolls percentile dice (the GM should roll if he
  wishes the outcome to remain secret).  If the dice result is
  \emph{equal to} or \emph{less than} the modified chance, the
  character succeeds at the task.  When a task requires a certain
  amount of time to perform (listed with the equipment description or
  with the task description), the time required to perform the task is
  \emph{reduced} by \textbf{1\%} for \emph{each} percentage point
  below the chance the dice result indicates.  If the dice result is
  \emph{greater than} the modified chance, the character either fails
  outright, must spend more time at the task to succeed, or suffers
  some other disappointment.  Each Skill description lists different
  effects for failing a task.  Many skills introduce a variety of
  results depending on how far above or below the modified chance a
  dice result falls.
\end{enumerate}

Unless specifically prohibited, a character that does not possess a
given skill \emph{may} attempt a task or other use of the skill.
However, the base chance of success is \emph{not} increased by a
Characteristic Rating or any other attribute of the character, nor
will equipment increase his Skill Level. \textbf{Exception}:
Treatment, \ref{sec:scientific-skills}.

\index{skills!tasks!retry after failure}A character that fails at a task in a given situation may \emph{not}
attempt that specific task again.  Another character \emph{may} attempt
the task.  However, his chance of success is \emph{reduced} by
\textbf{1\%} for each percentage point by which the previous attempt
failed (for example, if a character had a \textbf{60\%} chance of
success at a task and failed by rolling an \textbf{80}, the next
character to attempt the task has his chances reduced by \textbf{20}).
This reduction is cumulative; if a third character attempted the task,
his chance would be reduced for both the first and second characters'
failures.  The GM should not allow an additional task attempt until all
the time required for the previous attempt has passed.


\section{Acquiring and Improving Skills}
\label{sec:acqu-impr-skills}

\index{skills!improving}Once generated, a character possesses a variety of skills at Level 1
or higher.  During play, he will be able to improve these skills and
acquire others by collecting \emph{Experience Points} (EPs).  As a character
conducts actions related to a skill, he slowly amasses EPs in that
skill; when he has collected enough EPs, he expends them to increase
his Skill Level.  When a Skill Level is increased, further improvement
becomes more difficult as the number of EPs required to advance to
the next level increases with each advance.

\index{skills!unskilled}A character that has no Skill Level in a given skill is
\emph{unskilled} in that area.  He is also unskilled if he has an
\textbf{\textsf{X}} in the skill space on the Character Record, but he
is somewhat \index{skills!familiar (X)}familiar with the skill and may learn it more easily than
a character with nothing marked in that skill space.  The fact that a
character is unskilled does not prevent him from conducting actions
related to that skill or gaining EPs for the skill.
\textbf{Exceptions:} An unskilled character may not undertake a task
from a \emph{psionic} skill at all (see \ref{sec:psionicskills}).  An
unskilled character may not undertake a task from a \emph{scientific}
skill unless he is familiar with the skill (has an \textbf{\textsf{X}}
in the skill space; see \ref{sec:scientific-skills}).

Each character keeps track of all his current Skill Levels and the
number of EPs he has for each skill on his Character Record.  Pencil
should be used, since Skill Levels change and EPs are collected and
expended.


\subsection[Experience Points]{Each time a character uses a skill, he
  has a chance of 
  receiving an \index{experience points}Experience Point.}
\label{sec:experience-points}

Certain die results obtained by a character when using a skill reward
him with one Experience Point.  Each skill description or section lists
the die result needed and the type of rolls that are eligible to
provide the character with an EP in that skill.
\textbf{Example:} As explained in \ref{sec:sci-skills-ep}, a character
that is attempting a \emph{treatment} task receives an EP if either
die of his percentile dice result shows a \textbf{0}, \textbf{1}, or
\textbf{2}.  Thus, he would receive one EP if the dice result were
\textbf{01} through \textbf{32} or any higher result ending in
\textbf{0},\textbf{1}, or \textbf{2}.

 

When a character receives an EP, he immediately notes it in the EP
space for the skill just used on the Character Record.

No more than one EP may be gained from a single die or dice roll, even
if the appropriate number appears on both dice of a percentile dice
roll.  An EP may be gained by a character whether or not the skill was
used successfully, as long as the appropriate number is rolled.
(\textbf{Exception:} An \emph{unskilled} character who is attempting a
\emph{scientific} task receives an EP only if he \emph{succeeds} at
the task and rolls the appropriate number.) Certain skill descriptions
contain additional methods by which Experience Points may be gained.


\subsection[Increasing Skill Levels]{A character may increase a given
  \index{skills!increasing}Skill Level by amassing a 
  number of EPs equal to the next level in that skill.}
\label{sec:char-may-incr}

Each character keeps track of the EPs he has gained in each skill on
his Character Record.  When he has collected a number of EPs for one
skill \emph{equal to the next level} in that skill, he erases all
those EPs and increases the Skill Level by \textbf{1}.  Thus, a
character must acquire \textbf{2} EPs to increase a Skill Level from
\textbf{1} to \textbf{2}, \textbf{3} \emph{more} EPs to go from level
\textbf{2} to \textbf{3}, \textbf{4} more EPs to go from level
\textbf{3} to \textbf{4}, and so on.  Skill Level increase may not take
place during an \emph{Action Round} or procedure; the, character
should wait until a pause in the action to implement an increase.

\textbf{Exceptions:} A character that does \emph{not} possess a skill,
but has an \textbf{\textsf{X}} in the skill's space on his Character
Record (noted during character generation, see
\ref{sec:skill-points-are}) must collect \textbf{3} EPs to attain
Level 1 in that skill.  A character that does \emph{not} possess a
skill, and does \emph{not} have an \textbf{\textsf{X}} in the skill's
space, must collect \textbf{8} EPs to attain Level 1 in the skill.  If
a character reaches Level 1 in a skill by either of these methods,
subsequent Skill Level increases occur normally.

When an EP is gained for using a skill, it may only be assigned to
that skill.  Note that the four \emph{vehicle} skills are divided into
\emph{sub-skills} (see \ref{sec:vehsubskills}).  An EP gained for a vehicle
skill is assigned to the skill, not to a specific sub-skill.

When a skill reaches the maximum level that a character may attain, he
may no longer earn any EPs for that skill.  Skill Levels may never be
permanently reduced, even if the characteristic limit for the skill
drops below the current Skill Level attained by the character.  This
could happen if a Characteristic Rating is permanently reduced by
injury (see \ref{sec:rate-of-healing}).  Temporary reduction of a
characteristic has no effect on Skill Level increase; characteristic
limits for Skill Levels are based on the Characteristic Ratings of a
character at his best health.

Experience Points should not be confused with \emph{Skill} Points and
\emph{Initial Skill} Points, both of which are used during character
generation only.



\bigskip

\newcommand{\skill}[2]{%

\bigskip

\noindent\textbf{\sffamily\large\MakeUppercase{#1}}\\
\textbf{#2}\index{#1}}

%\renewcommand{\thefootnote}{\fnsymbol{footnote}}
\skill{perception}{9 Levels/Limit: None}\label{sec:skill-perception}

Perception is a measure of the character's intuition developed as a
result of his adventuring experience.  It is used to determine if a
character notices a detail about a situation, sees something in the
distance, hears a footfall, etc.

Every character automatically possesses the Perception Skill when
generated.  All human characters initially possess this Skill at level
\textbf{2}.  No initial Skill Points or Professional Skill Points need
be spent to acquire the skill, nor may the skill be improved by
expenditure of such points.  Perception may be improved during play by
amassing Perception Experience Points.  Perception is used in play in
either of two ways:

\begin{enumerate}
\item If the GM wishes to allow the chance to notice something not
  readily apparent that is related purely to the senses (sight, smell,
  hearing, etc.), he instructs the character to conduct a
  \emph{Perception Check} by rolling one die.  If the die result is
  equal to or less than the character's Perception level, the GM
  informs him of the occurrence or item.  If the die result is greater
  than the character's Perception level, the GM does not provide any
  information.
\item If the GM wishes to allow a character a chance to notice
  something related to a particular area of expertise, he instructs
  the character to conduct a Perception Check by rolling
  \emph{percentile dice}.  The chance of success equals \textbf{10\%} plus the
  character's Perception level plus the level he has with the skill
  associated with the item or event.  The \emph{higher} of these two
  levels is \emph{squared before adding}.
  
  Example: A skimmer is flying by the character at the edge of view.
  He has Perception \textbf{3} and Air Vehicles \textbf{5}, so he has
  a \textbf{38\%} chance ($10+5^2+3$) of noticing the vehicle and correctly
  identifying it as a skimmer.
\end{enumerate}

Any one of many different skills may be associated with a Perception
Check; the most common would be a scientific, technical, environ, or
vehicle skill.  However, any skill might apply, depending on the
situation.
%% Original:
%% A character who rolls a one on either die when conducting a
%% Perception Check receives one Perception Experience Point.
%% New:
A character who rolls a \textbf{1} on either die when conducting a
Perception Check receives one Perception Experience
Point.
%% \label{sec:char-may-incr-change-zero}
%% End change

\section{Military Skills}
\label{sec:military-skills}

\index{skills!military}Military skills include all those used in character combat (fighting
NPC's, creatures, or other characters on a planet or in a spaceship),
space combat (spaceship vs. spaceship), and other action-related
situations.  Specific personal weapon skills are explained in
\ref{sec:weapon-skills-allow}, and space combat skills are explained
in \ref{sec:foll-four-skills}.  Explanations of other military skills
follow.

\skill{ambush}{7 Levels/Limit: Agility}\label{sec:skill-ambush}

The character may move silently and swiftly and may approach and
attack a target undetected.  The location of a character that conducts
ambush successfully will be unknown to those he is hiding from.  The
character's \emph{Agility Rating}, \emph{Battlefield Skill Level},
\emph{Environ Skill Level}, and the \emph{square} of his \emph{Ambush
  Skill Level} are added to the base chance of any ambush task.  The
highest \emph{Environ Skill Level} and the \emph{square} of the
highest \emph{Battlefield Skill Level} in the \emph{enemy} force
are \emph{subtracted} from the base chance.  The GM may secretly apply
other modifiers as the situation warrants.  A character who is
performing ambush movement may move as far as normally allowed (see
\ref{sec:action-round-movement}), but may not fire a weapon while
moving.  An unskilled character may attempt only the first task listed,
and nothing is added to his base chance.

\begin{tasklist}
\item Move secretly during an Action Round in which the enemy is
  unaware of the character's current location: (\textbf{15$\times$}
  the Terrain Value)\%, with addition for darkness, if applicable.
\item Move secretly during an Action Round in which the enemy is aware
  of the character's location: (\textbf{5$\times$} the Terrain
  Value)\%.
\item Attack enemy being in close combat without being detected (see
  \ref{sec:close-combat}): \textbf{30\%}.  Character must be
  undetected by all except victim in order to attempt this task.
\item If currently undetected, fire at enemy being with a silent
  ranged weapon (such as a bow or thrown knife) without being
  detected: \textbf{50\%}.
\end{tasklist}

An ambush task may not be attempted in an area with a Terrain Value of
less than \textbf{2} (including any modifier for darkness).  When a
character successfully performs an ambush attack, the GM controls the
enemy force as if they had no idea of the character's where abouts.  A
dice result for any ambush task that is above the modified chance
indicates failure; the character's location is known.

A character who rolls a \textbf{0} or \textbf{1} on either die when
attempting an ambush task receives an Experience Point.


\skill{artillery}{6 Levels/Limit: Intelligence}\label{sec:skill-artillery}

The character is familiar with all aspects of mounted gun use.  He may
spot targets and aim stationary artillery, tank guns, and
self-propelled artillery.  The base chance to hit a target with
artillery fire is \textbf{50\%}.  From this chance, \textbf{1} is
\emph{subtracted} for every \emph{100 meters} away the target is
located.  The character's \emph{Intelligence Rating} and the
\emph{square} of his \emph{Skill Level} are added to the base chance.
If the percentile dice result is greater than the modified chance for
a given artillery fire, the shell strikes \textbf{2} \emph{hexes} (ten
meters) away from the target for each percentage point over the chance
the dice result indicates (GM determines direction).

An unskilled character adds nothing to his base chance when firing
artillery and, if he misses is target, the shell strikes \textbf{4}
hexes away from the target for each percentage point over the chance
the dice result indicates.

A character who rolls a \textbf{0} or \textbf{1} on either die when
attempting to strike a target with artillery fire receives an
Experience Point.

\skill{battlefield}{6 Levels/Limit: Leadership}\label{sec:skill-battlefield}

The character is experienced in ground combat and the execution of
successful strategies and tactics.  He recognizes the signs of battle
and the warning signs of battles to come.  Battlefield skill affects
various procedures during encounters with NPC's and creatures.

\begin{itemize}
\item The highest Battlefield Skill Level among the characters in a
  party is subtracted from the \emph{awareness chance} during a
  creature encounter.  The \emph{square} of the highest Battlefield
  Skill Level among the characters in a party is subtracted from the
  \emph{awareness chance} during an NPC encounter (see
  \ref{sec:awareness}).
\item The character chosen as the party's \emph{leader} during an
  Action Round (see \ref{sec:initiative}) adds his Battlefield Skill
  Level to his \emph{Initiative} die roll.  If NPC's are the enemy,
  \emph{twice} his \emph{Skill Level} is added.
\item A character may use his Battlefield Skill Level \emph{instead}
  of his Mental Power Rating when making a \emph{Willpower Check} (see
  \ref{sec:will-power-check}).  The battlefield skill is not used when
  making a \emph{Shock} Check.
\end{itemize}

Battlefield skill also aids a character who is planning strategy for a
battle ahead of time.  The GM should take the character's Skill Level
into account, as well as the quality of his stated plans, when
determining the reaction and performance of the enemy force.

A character who rolls a \textbf{0} on the die when making an
\emph{Initiative Check} receives one Experience Point for his
battlefield skill.  The GM may also give a player who successfully
plans an attack (as described in the previous paragraph) an Experience
Point.  No other die or dice rolls concerning a character's battlefield
skill may provide him with an Experience Point.


\skill{blades}{7 Levels/Limit: Dexterity}\label{sec:skill-blades}

The character is skilled in the use of daggers and swords.

When attacking or defending in close combat, a skilled character adds
his \emph{Dexterity} or \emph{Agility Rating} (his choice) and the
\emph{square} of his \emph{Blade Skill Level} to the Hit Strength of
his blade (see \ref{sec:close-combat}).  An unskilled character uses
the strength of his blade only.

A dagger (but not a sword) may be thrown as a ranged weapon as
described on the Weapon Chart and in \ref{sec:weapon-skills-allow}.

A character that rolls a \textbf{0} on the die when attacking with a
blade in close combat receives an Experience Point.  A character who
rolls a \textbf{0} on either die when attempting to hit a target with
a thrown knife receives an Experience Point.

\skill{body armor}{6 Levels/ Limit: Agility}\label{sec:skill-body-armor}

The character has experience maneuvering and fighting in 
body armor and other protective attire.  The body armor skill
\emph{contains all the attributes of the EVA skill},  plus the
following:

\begin{itemize}
\item The movement rate of a skilled character wearing augmented body
  armor may be increased if his Skill Level is high enough (see
  \ref{sec:foot-travel} and \ref{sec:action-round-movement}).
\item A character with body armor skill may add his \emph{Strength
    Rating} and the \emph{square} of his \emph{Skill Level} to the Hit
  Strength of his body armor or battle sleeve when participating in
  close combat (see \ref{sec:close-combat}).
\end{itemize}

A character who rolls a \textbf{0} or \textbf{1} on either die when
attempting to avoid a body armor accident receives an Experience
Point.  A character who rolls a \textbf{0} on the die when using body
armor to attack in close combat receives an Experience Point.

\skill{demolitions}{6 Levels/Limit: Dexterity}\label{sec:skill-demolitions}

The character is skilled in the use of plastic explosives and dynamite
to blow holes through walls and doors or simply destroy a structure.
If the character has sufficient demolitions equipment, he may attempt
to prepare explosives so that when detonated, they will destroy all
that the character wishes destroyed while leaving surrounding
structures unharmed.  When setting explosives, he should declare
whether detonation will be triggered by radio, wire, or timer.

The base chance to destroy a declared structure and nothing but that
structure is \textbf{45\%}.  The GM may reduce this chance if the
structure to be destroyed is smaller than a trap door.  The character's
\emph{Dexterity Rating} and the \emph{square} of his \emph{Skill
  Level} are added to the base chance.  A percentile dice result that
is greater than the modified chance by \textbf{30} or less indicates
that the explosion occurs when planned but is too powerful (if an even
dice result) or too weak (if an odd dice result).  The effect of such a
result is left up to the GM.  A percentile dice result that is greater
than the modified chance by more than \textbf{30} indicates that the
explosives failed to detonate and are now defective (if an even dice
result) or that the explosives went off while being set, thus injuring
the character (if an odd dice result).  The nature of such an injury is
left up to the GM.

An unskilled character adds nothing to his base chance when setting
explosives and, if his dice result is greater than the base chance at
all, the explosives go off while being set (whether the result is even
or odd).

A character who rolls a \textbf{0}, \textbf{1} or \textbf{2} on either
die when attempting to set explosives receives an Experience Point.

%% Because of the mixed-case nature of the EVA skill heading, we
%% have to do this one by hand.

% \skill{eva {(extra vehicular activity)}}{6 Levels/Limit:
%   Agility}\label{sec:skill-eva}

\bigskip

\noindent\textbf{\sffamily\large EVA (Extra-Vehicular Activity)}\\
\textbf{6 Levels/Limit: Agility}\index{skills!EVA}

The character is able to operate and maneuver in an expedition suit on
the surface of a planet and in a zero-G environment.

When a character is wearing body armor or an expedition suit with an
Encumbrance Rating (see the Protective Attire Chart,
\ref{tab:protective-attire}), his EVA Skill Level is \emph{subtracted}
from the Rating to determine the Movement Rate (see
\ref{sec:foot-travel} and \ref{sec:action-round-movement}).  A
character wearing augmented body armor may not use his EVA skill to
increase his movement (the body armor skill is used for augmentation).
A character without EVA skill suffers the full effects of an
Encumbrance Rating.

A skilled character's \emph{Agility Rating} and the \emph{square} of
his \emph{Skill Level} are added to his base chance to avoid a
suit/armor accident (see \ref{sec:accidents}).  The base chance to
avoid an accident and the procedure undertaken are similar to that of
a vehicle accident (see \ref{sec:vehaccidents}).  An unskilled
character has nothing added to his base chance of avoid ing such an
accident.

A character who rolls a \textbf{0} or \textbf{1} on either die when
attempting to avoid an accident in an expedition suit or respirator
helmet receives an Experience Point.

\skill{jet pack}{6 Levels/Limit: Agility}\label{sec:skill-jet-pack}

The character is able to operate a jetpack, a device strapped to the
back that allows flight.  A character's Jet Pack Skill Level affects
the speed and maneuverability he may attain with the pack (see
\ref{sec:other-actions}).  A skilled character's \emph{Agility Rating} and the
\emph{square} of his \emph{Skill Level} are added to his base chance
to avoid a jet pack accident (see \ref{sec:accidents}).  An unskilled
character has nothing added to his base chance of avoiding such an
accident.  A character who rolls a \textbf{0} or \textbf{1} on either
die when attempting to avoid a jetpack accident receives an Experience
Point.

\skill{unarmed combat}{8 Levels/Limit: Agility}\label{sec:skill-unarmed-combat}

The character is skilled in fighting with his hands and body.  When
attacking or defending in close combat, a skilled character uses his
\emph{Dexterity}, \emph{Strength}, or \emph{Agility Rating} (his
choice) plus the \emph{square} of his \emph{Unarmed Combat Skill
  Level} (see \ref{sec:close-combat}).  An unskilled character uses one
half (round up) of his Strength, Dexterity, or \emph{Agility Rating}
only.

A character in close combat with a creature has his Unarmed Combat
\emph{Skill Level} reduced by \textbf{1} (to a minimum of \textbf{1}).

A character that rolls a \textbf{9} or \textbf{0} on either die when
attacking unarmed in close combat receives an Experience Point.


\subsection[Weapons Skills]{Weapon skills allow a character to use a
  weapon effectively during combat.}
\label{sec:weapon-skills-allow}

\index{skills!weapons}All weapon fire is conducted in accordance with
\ref{sec:firing-weapons} and the Weapon Chart (\ref{tab:weapon}).  The
chance to hit a target with a weapon is equal to the \emph{base
  chance} listed for the weapon on the Fire Chart plus the character's
\emph{Dexterity Rating} plus the \emph{square} of his \emph{Skill
  Rating} with the weapon.  A character that is skilled with a weapon
may often fire it more than once in a single Action Round (depending
on the weapon).

A character that is not skilled with a weapon may use the weapon with
the following restrictions:

\begin{itemize}
\item His chance to hit is equal to the base chance listed on the
  Weapon Chart \emph{only}; nothing is added for his Characteristic
  Ratings.
\item He may fire the weapon only \emph{once} per Action Round.
\item He may \emph{not} fire while moving or controlling a vehicle (he
  may fire while riding in a vehicle, see
  \ref{sec:fire-modifier-chart}).
\end{itemize}

The following skill descriptions list the specific types of weapons
that each skill allows the character to use (if more than one).

\skill{arc gun}{8 Levels/Limit: Dexterity}\label{sec:skill-arc-gun}

The character may fire an arc gun.  

\skill{bows}{7 Levels/Limit: Dexterity}\label{sec:skill-bows}

The character may shoot a short bow, long bow, or crossbow.  When
shooting a long distance with a short bow or long bow, the character's
Strength Rating (as well as his Dexterity Rating) is added to his hit
chance (see Weapon Chart).

\skill{handguns}{5 Levels/Limit: Dexterity}\label{sec:skill-handguns}

The character may fire a pistol, needle pistol, laser pistol, or stun
pistol.  Note that the latter two weapons are also included in the
laser/stun pistol skill listing.  A character may use either Skill
Level when firing a stun or laser pistol; however, any Experience
Points gained when doing so may only be applied to the laser/stun
pistol skill.

\skill{grenades}{8 Levels/Limit: Dexterity}\label{sec:skill-grenades}

The character may throw fragmentation, smoke, illumination, and gas
grenades.  When throwing a grenade a long distance, the character's
Strength Rating (as well as his Dexterity Rating) is added to his hit
chance (see Weapon Chart).

\skill{laser/stun pistol}{5 Levels/Limit: Dexterity}\label{sec:skill-laser-stun-pistol}

The character may fire a laser or stun pistol.  

\skill{longarms}{6 Levels/Limit: Dexterity}\label{sec:skill-longarms}

The character may fire a musket, rifle, car bine, or needle rifle.

\skill{machine guns}{5 Levels/Limit: Dexterity}\label{sec:skill-machine-guns}

The character may fire a sub-machine gun or an emplaced machine gun
(see \ref{sec:other-actions}).

\skill{paint gun}{7 Levels/Limit: Dexterity}\label{sec:skill-paint-gun}

The character may fire a paint gun.

\bigskip

A character rolling a \textbf{0} on either die when attempting to hit
a target with a ranged weapon receives an Experience Point.  A
character may not receive an Experience Point when using the Hit Table
or rolling dice for any purpose other than actually attempting to make
his hit chance.  A character conducting more than on e fire in a single
Action Round considers only his \emph{first} hit chance dice roll of
the Round for possible Experience Point gain.  If a character is firing
in a non-combat, non-pressure situation (such as putting holes in an
immobile, helpless target), the GM should invalidate any dice rolls he
conducts for EP purposes.


\subsection[Spacecraft Skills]{The following four skills are used
  aboard a spaceship or Battlecraft during space combat.}
\label{sec:foll-four-skills}

\index{skills!spacecraft}These skills modify the procedures outlined in the \emph{Delta Vee}
rules and \ref{sec:space-combat}.  Any character aboard a spaceship or
Battlecraft that is participating in combat may be assigned to the
functions that any of these skills entail.  However, if the character
is not skilled at his function, his performance will threaten the
spaceship's chances of survival.

A character may use two space combat skills at the same time, if he is
skilled at both, in the following instances:

\begin{itemize}
\item \emph{Pilot} and \emph{gunnery} when in a battlecraft.    
\item \emph{Pilot} and \emph{space tactics} when in a spaceship's bridge.    
\item \emph{Missile guidance} and \emph{space tactics} when in a
  weapon or arsenal pod.
\item \emph{Gunnery} and \emph{space tactics} when in a spaceship's
  bridge, weapon, or arsenal pod.
\end{itemize}

When a character is using two skills at once, the level of each is
\emph{reduced by} \textbf{2} (to a minimum of \textbf{1}).

\skill{gunnery}{9 Levels/ Limit: Dexterity}\label{sec:skill-gunnery}

The character may effectively use shipboard laser and particle weapons
against enemy craft.  A character may be assigned to any single weapon
aboard the spaceship (for instance, to the ship's burster, or to one
of its pods that contains laser and particle weapons or to a
battlecraft's burster.  The character's Skill Level modifies any fire
conducted from his assigned location as follows:

\begin{description}
\item[Unskilled:] The Target Value is increased by \textbf{4}.  No
  Target Program modifier is applied.
\item[Level 1:] No Target Program modifier is applied.
\item[Level 2:] One half the normal Target Program modifier is
  applied.
\item[Level 3, 4:] Fire conducted normally.
\item[Level 5:] A \textbf{--4} modifier or the Target Program
  modifier (whichever is greater) is applied.
\item[Level 6:] A \textbf{--6} modifier is applied (instead of the TP
  modifier).
\item[Level 7:] A \textbf{--6} modifier is applied (instead of the TP
  modifier) and may be used to reduce the range (exception to
  \emph{Delta Vee} \ref{DV-sec:targeting-program-rel-vel}). 
\item[Level 8:] A \textbf{--8} modifier is applied (instead of the TP
  modifier) and may be used to reduce the range.
\item[Level 9:] As in Level 8.  In addition, every hit achieved by the
  gunner is considered a critical hit.
\end{description}

A gunner assigned to an arsenal pod may conduct two fires in a single
Fire Phase (see \emph{Delta Vee} \ref{DV-sec:number-types-fires}).
When doing so, his Skill Level is reduced by \textbf{1}.  A gunner may
also conduct two fires in a single Fire Phase if assigned an
additional fire from a battle communications pod.

A character receives an Experience Point for gunnery each time he
rolls a \textbf{1} when using the Fire Results Table (\emph{Delta Vee}
\ref{DV-sec:fire-results-table}).  He may not receive an Experience
Point when using the Hit Table.

\skill{missile guidance}{9 Levels/Limit: Intelligence}\label{sec:skill-missile-guidance}

The character may effectively launch missiles, control their course,
and spot their targets.  Any character may be assigned to oversee
missile fire in a hunter, weapon, or arsenal pod of a spaceship.  The
character's Skill Level affects the launch and performance of missiles
from that pod.

\begin{description}
\item[Unskilled:] MIMS and Intelligent Missiles may not be launched at
  all.  A guided missile may not receive Maneuver Commands once
  launched. \textbf{2} is subtracted from every missile interception
  chance.
\item[Level 1:] MIMS may not be launched at all.  A guided missile may
  receive only one Maneuver Command for each Control Guided Missile
  Command issued to its spaceship.  \textbf{1} is subtracted from
  every missile interception chance.
\item[Level 2:] MIMS may not be launched at all.  \textbf{1} is
  subtracted from every missile interception chance.
\item[Level 3:] \textbf{1} is subtracted from every missile
  interception chance.
\item[Levels 4,5:] No modifications.
\item[Level 6:] Prepare Missile Command not required to launch
  unguided or guided missile.
\item[Level 7:] As in Level 6 and \textbf{1} is added to every missile
  interception chance.
\item[Level 8:] Prepare Missile Command not required to launch
  unguided, guided, or intelligent missile.  Each missile possesses
  \textbf{1} extra Energy Unit.  \textbf{1} is added to every missile
  interception chance.
\item[Level 9:] Prepare Missile Command not required to launch any
  type of missile.  Each missile possesses \textbf{2} extra Energy
  Units. \textbf{1} is added to every missile interception chance.
\end{description}

A character receives an Experience Point for missile guidance each
time a missile he launched successfully intercepts a target (even if
the target has an active force field).  In addition, at the end of a
space combat, the character rolls percentile dice.  If the result is
equal to or less than the total number of missiles launched by the
character (counting a MIMS as one missile), he receives an Experience
Point.

\skill{pilot}{9 Levels/Limit: Intelligence}\label{sec:skill-pilot}

The character may effectively maneuver a spaceship or Battlecraft.  Any
one character may be assigned to pilot a spaceship or Battlecraft, and
when doing so is considered to be in the bridge.  The character's Skill
Level modifies the number of Maneuver Commands he may issue to the
ship each Command Phase and affects the performance limits of the
craft.

\begin{description}
\item[Unskilled:] No more than one Maneuver Command may be issued in a
  single Command Phase; no Weave Commands may be issued.  The spaceship
  is destroyed upon entering a planet hex and automatically collides
  with an asteroid upon entering an asteroid field (see
  \emph{Delta Vee} \ref{DV-sec:asteroid-collision}).
\item[Level 1:] No more than two Maneuver Commands may be issued in a
  single Command Phase; no Weave Commands may be issued.  The spaceship
  may not receive ``free'' Maneuver Commands upon entering a planet
  hex (see \emph{Delta Vee} \ref{DV-sec:planets-maneuvering}).
\item[Level 2:] No more than three Maneuver Commands may be issued in
  a single Command Phase; no Weave Commands may be issued.
\item[Level 3:] No Weave Commands may be issued.
\item[Level 4, 5:] No modifications.
\item[Level 6:] Ship's Maneuver Rating increased by \textbf{1}.
\item[Level 7:] Ship's Maneuver and Velocity Rating increased by
  \textbf{1} each.
\item[Level 8:] As in level 7, and the chances of missile interception
  and asteroid collision are reduced by \textbf{1} each.
\item[Level 9:] As in level 7, and the chances of missile interception
  and asteroid collision are reduced by \textbf{3} each.
\end{description}

Pilot skills may also be used when controlling a shuttle or any craft
in low planet orbit, as described in \ref{sec:vehicleskills}.

A character receives an Experience Point for pilot skill at the
conclusion of any space battle in which he piloted a craft that was
fired upon.  He may also receive an Experience Point when attempting to
avoid an accident while controlling a craft in low planet orbit, as
explained in \ref{sec:vehicleeps}.

\skill{space tactics}{9 Levels/Limit: Leadership}\label{sec:skill-space-tactics}

The character may effectively direct battle strategies undertaken by a
spaceship he is commanding during space combat.  Space tactics may not
be used aboard a Battlecraft.  Any one character may be assigned to
space tactics, and a character so assigned must occupy the ship's
battle communications pod, if the ship has one.  If not, he must be in
the bridge, a weapon pod or an arsenal pod.  The character's Skill
Level affects the number of Battle Commands that may be issued to the
ship and the number of fires that may be conducted from the ship.

\begin{description}
\item[Unskilled:] No more than \textbf{1} Battle Command may be issued
  to the spaceship in a single Command Phase and no more than
  \textbf{1} fire may be conducted from the spaceship in a single Fire
  Phase.  Active Search, Rendezvous and Tractor Beam Commands may not
  be issued at all.
\item[Level 1:] No more than \textbf{1} Battle Command may be issued
  in a single Command Phase and no more than \textbf{1} fire may be
  conducted in a single Fire Phase.
\item[Level 2:] No more than \textbf{2} Battle Commands may be issued
  and no more than \textbf{2} fires may be conducted.
\item[Level 3:] No more than \textbf{3} Battle Commands may be issued
  and no more than \textbf{3} fires may be conducted.
\item[Level 4, 5:] No modifications.
\item[Note:] If the number or Commands of fires a ship may normally
  receive or conduct is less than listed above, the ship's limitations
  take precedence.
\item[Level 6:] \textbf{1} additional Battle Command may be issued.
\item[Level 7:] \textbf{1} additional Battle Command may be issued and
  \textbf{1} additional fire may be conducted.
\item[Level 8:] As in level 7, and the Civ Level of the ship is
  increased by \textbf{2} when attempting to activate a force field
  during missile interception (see \emph{Delta Vee}
  \ref{DV-sec:missile-force-field}).
\item[Level 9:] \textbf{2} additional Battle Commands may be issued,
  and \textbf{2} additional fires may be conducted, and the ship's
  Civ Level is increased as in Level 8.
\end{description}

A character receives one Experience Point for space tactics at the
conclusion of any space battle in which he commanded a spaceship that
was fired upon.

\section{Psionic Skills}
\label{sec:psionicskills}

\index{skills!psionic}Psionic skills allow a character to use the powers of his mind in
tangible ways.  These skills are \emph{restricted}; that is, unless a
character possesses a psionic skill or is eligible to possess the
skill (has an \textbf{\textsf{X}} in its space), he may \emph{not}
attempt the skill \emph{at all}.  Furthermore, a character with a
Mental Power Rating of \emph{less than} \textbf{4} may never receive
Experience Points for any psionic skill he uses.

With the exception of navigation, none of the psionic skills require
any special equipment.  However, a \emph{psionic rig} may enhance a
character's use of certain psionic skills.

Using certain psionic skills puts a terrific strain on the character's
mind.  If the skill is used poorly, the character may suffer
\emph{psionic backlash}, the effects of which range from a temporary
blackout to insanity or death.

\skill{life sense}{8 Levels/Limit: Intelligence}\label{sec:skill-life-sense}

The character is sensitive to the living energy emanations of all
living beings.

During the Encounter Procedure, the \emph{square} of the single
highest \emph{Life Skill Sense Level} among the characters in the
party is automatically subtracted from a creature's or NPC's Surprise
Ambush Chance (see \ref{sec:awareness}).  If the GM rolls a \textbf{9}
on either die when checking for surprise/ambush, he should inform the
character that contributed his life sense (if any) that he has gained
an Experience Point.

\begin{tasklist}
\item When \emph{perception} of a creature is possible (see
  \ref{sec:encounter-procedure}), a character with life sense may
  attempt to perceive if the creature is \emph{intelligent}:
  \textbf{20\%}.  The creature's \emph{Intelligence Rating} (if any),
  the character's \emph{Mental Power Rating}, and the \emph{square} of
  his \emph{Skill Level} are added to the base chance.  If the attempt
  is successful, the GM immediately reveals the creature's
  Intelligence Rating (or lack thereof) to the character.  A dice
  result that is greater than the modified chance indicates failure;
  the character cannot perceive the creature's mind and may not try
  again.  A character who rolls a \textbf{0} on either die when
  attempting to perceive creature intelligence receives an Experience
  Point.
\end{tasklist}


\skill{mind control}{9 Levels/Limit: Intelligence}\label{sec:skill-mind-control}

The character may attempt to interfere with or actually take control
of another individual's mind.  During an Action Round, a skilled
character may attempt to control any character, NPC, or creature
(however, see \ref{sec:psicreatures}) that is within his natural
sight.  \index{psionic rig!mind control}A character wearing a psionic rig may attempt to control an
individual with in 100 km whose exact location is known to him.  A
character attempting mind control may not move, fire, or attack during
the Action Round.

The base chance of successful mind control is \textbf{10\%}.  To this
is added the character's \emph{Mental Power Rating} and the square of
his \emph{Mind Control Skill Level}.  The \emph{square} of his
target's \emph{Mental Power Rating} is subtracted from the chance.  The
attempting character rolls percentile dice.  If the dice result is
greater than the modified chance, the attempt fails; \emph{check for
  psionic backlash}.  If the dice result is equal to or less than the
modified chance, \textbf{1} is \emph{temporarily subtracted} from the
target's Mental Power Rating for every \textbf{10} (or fraction
thereof) below the chance the dice result indicates.  One of the
following two procedures is then carried out:

\begin{itemize}
\item \emph{If the target's modified MP Rating is now} \textbf{1}
  \emph{or higher}, the target must immediately check for shock (see
  \ref{sec:shock-check}), using the modified Rating.  If shocked, the
  target does not collapse; he remains motionless in place for one
  Action Round.
\item \emph{If the target's modified MP Rating is now} \textbf{0}
  \emph{or less}, the target is controlled by the attempting
  character.  A controlled individual may be moved or commanded to
  perform any other actions possible in an Action Round (see
  \ref{sec:other-actions}) in any way the controller wishes.  A
  controlled individual may not be directed to do anything suicidal
  (such as walking off a cliff or firing a weapon at himself).  He may
  be directed to do such things as attack his allies, toss aside his
  weapons, or run wildly away.  An individual remains controlled for
  one Action Round only.
\end{itemize}

A character who rolls a \textbf{0} or \textbf{1} on either die when
attempting mind control receives an Experience Point.

\skill{navigation}{9 Levels/Limit: Intelligence}\label{sec:skill-navigation}

The character has learned the secrets of hyperdrive thought.  He is
familiar with the concepts, design and use of spaceship hyper-drives.
If aboard a spaceship that has an explorer, hunter, jump, or augmented
jump pod --- and the spaceship is at a valid \emph{jump point} --- the
character may attempt a hyperjump, within the limitations of
\ref{sec:interstellar-travel}.  The character declares his destination
star system (and planet within the system, if known) and calculates
his chance of success as follows:

The \emph{square} of his \emph{Navigational Skill Level} (a jump or
augmented pod jump increases the Skill Level), \emph{plus} the
\emph{square} of his \emph{Mental Power Rating}, \emph{plus}
\textbf{10$\times$} the highest \emph{Starport Class} in the
destination system (if the character has been to the star system
before, increase the Starport Class by \textbf{1}; if the character
frequents the star system, increase the Starport Class by \textbf{2};
both of these increases may be applied to a \textbf{0} Class Starport,
but may not increase the Class of any Starport above \textbf{4}),
\emph{minus} the distance between the spaceship's current position and
the destination star system (in light years).

The character rolls percentile dice, and the GM refers to the
Hyperjump Table (see \ref{tab:hyperjump}), using the difference
between the dice result and the calculated chance to locate the
outcome of the hyperjump.  If the outcome includes Psionic Backlash
Modifier, the GM rolls percentile dice again, adds the modifier to the
dice result, and locates the result on the Psionic Backlash Table
(\ref{sec:psibacklashtable}).

A character who rolls \textbf{0}, \textbf{1}, \textbf{2} or \textbf{3}
on either die when attempting a hyperjump receives an Experience
Point.  An EP may not be gained from a dice roll for the Psionic
Backlash Table.

\skill{psionic boost}{8 Levels/Limit: Intelligence}\label{sec:skill-psionic-boost}

The character may attempt to call upon the powers of his subconscious
to improve his performance in a stress situation.  In any situation
where the character is called upon to use a military skill or a
vehicle skill, he may declare that he is attempting to temporarily
improve that skill with a psionic boost.  Only those skills in
\ref{sec:military-skills} and \ref{sec:vehicleskills} that the
character possesses at Level \textbf{1} or higher are eligible for
psionic boost.

\index{psionic rig!psionic boost}The base chance of successful psionic boost is \textbf{25\%}.  To this
is added the character's \emph{Mental Power Rating} and the
\emph{square} of his \emph{Psionic Boost Skill Level} (\emph{may be
  increased} if wearing a psionic rig).  The character rolls percentile
dice.  For every \textbf{18} points (or fraction thereof) \emph{below}
the modified chance the dice result indicates, the declared Skill
Level is \emph{increased} by one.  Thus a character with a
\textbf{50\%} chance of psionic boost who rolled a \textbf{36} would
increase the declared skill by \textbf{2} levels.  For every
\textbf{10} points (or fraction thereof) \emph{above} the modified
chance the dice result indicates the declared Skill Level is
\emph{decreased} by one.  If an unsuccessful boost attempt reduces the
declared Skill Level below \textbf{1}, the character is considered
unskilled at the task.  The effects of psionic boost (successful or
unsuccessful) last for one use (one die roll) of the declared skill
only.

A character who rolls a \textbf{8} on either die when attempting
psionic boost receives an Experience Point.  A character may not
receive an EP for a skill if he attempted to improve its current use
with psionic boost.


\skill{psionic communication}{8 Levels/Limit: Intelligence}\label{sec:skill-psionic-communication}

The character can send and receive thoughts with other individuals.  At
any point during play, the character may declare that he is attempting
to send a psionic (unspoken) message to another character, NPC, or
creature (however, see \ref{sec:psicreatures}) within his sight.
\index{psionic rig!psionic communication}A character
wearing a psionic rig may attempt to send a psionic message to an
individual anywhere on the same world or within 200,000 km.
The character declares the intended receiver and the GM calculates the
chance of success as follows: To the base chance of \textbf{10\%} is added the
character's \emph{Mental Power Rating}, plus the \emph{square} of his
\emph{Psionic Communication Skill Level}, plus the \emph{receiver's MP
  Rating}, plus the \emph{square} of the \emph{receiver's Psionic
  Communication Skill Level}.

The GM rolls percentile dice.  The GM subtracts the modified chance
from the dice result and locates the difference in one of the
following outcomes:

\begin{description}
\item[\textbf{--20 or less}]: Successful communication has been established.
  The two individuals converse silently for the equivalent of one
  minute (adjudicated by GM).
\item[\textbf{--19 to 0:}] The character may send one message to the receiver.
  The receiver may not return a message except to acknowledge receipt.
\item[\textbf{+1 to +10:}] The receiver is aware that the character is
  attempting to contact him, but cannot receive a message.
\item[\textbf{+11 to +20:}] The receiver is aware that someone, somewhere is
  attempting to contact him.  The sender is not aware of this fact.
\item[\textbf{+21 or more:}] No contact is achieved at all.
\end{description}

A character may not attempt psionic communication with a specific
individual more than once per day.  

A character that rolls a
\textbf{0} on either die when attempting psionic communication
receives an Experience Point.  If successful psionic communication is
established and the \emph{receiver} has (or is eligible to acquire) the
psionic communication skill, he also receives an Experience Point on a
roll of \textbf{0}.  The GM should not allow EP gain for psionic
communication if the skill is being used when normal or radio
conversation could be safely used.

\skill{psychokinesis}{9 Levels/Limit: Intelligence}\label{sec:skill-psychokinesis}

The character is able to move and manipulate objects with the power of
his mind.  The character may declare that he is attempting to lift
and/or move any unattached object within his sight.  A character
attempting psychokinesis during an Action Round may not move or fire.

The base chance of successful psychokinesis is \textbf{10\%} minus the
size (0--9) of the world the character is on (considered \textbf{0} if
in a zero-G environment).  The character's \emph{Mental Power Rating}
and the \emph{square} of his \emph{Skill Level} are added to the
chance.  The character rolls percentile dice.  If the dice result is
\emph{less} than the modified chance by \emph{at least} a number of
percentage points equal to the \emph{kilogram weight} of the object
(rounded up) to be moved, it is successfully lifted.  Any
\emph{additional} amount that the dice result falls below the chance
may be converted to movement of the object: Divide the difference by
the whole kilogram weight of the object (rounding the quotient to the
nearest whole number) to determine the number of hexes (five-meter
increments) the object may be moved in one Action Round (ten seconds).
\textbf{Example:} A character with a \textbf{45\%} chance of lifting a
three kg object rolls a \textbf{25}.  He may lift the object and has
\textbf{17} percentage points with which to move it.  Dividing
\textbf{17} by \textbf{3} provides the character with the ability to
move the object \textbf{6} hexes in an Action Round.  If the object had
weighed between \textbf{14} and \textbf{20 kg} , he would be able to
lift it a short distance but would not be able to move it.

A character that is able to move an object with psychokinesis may hurl
the object at a target.  The Hit Strength of a hurled object is
calculated as follows: ( 10 $+$ the \emph{Agility Rating of the
  target})\index{agility!and evading psychokinetic attack} is subtracted from the \emph{speed} of the hurled object
(the number of hexes it moves in an Action Round).  If this number is
\textbf{0} \emph{or less}, the object may not harm the target.  If the
number is \textbf{1} \emph{or greater}, multiply it by the whole
kilogram weight of the object and then divide this product by
\textbf{20}.  If this quotient is \textbf{1} \emph{or greater} (after
rounding fractions \emph{down}) it is used as the Hit Strength on the
Hit Table or the Equipment Damage Table (depending on the type of
target); see \ref{sec:hits-and-damage}.  If this quotient is 
\textbf{less than 1}, the target is not harmed.  In summary: 

$$\frac{\mathrm{Weight}\times\left[\mathrm{Speed} - \left(10 +
      \mathrm{Target\ Agility}\right)\right]}{20}$$

    
A dice result for a psychokinesis attempt that is above the modified
chance at all in dicates failure; check for psionic backlash.

\index{psionic rig!psychokinesis!}If a character is wearing a psionic rig while attempting psychokinesis
rolls less than his modified chance, the number of percentage points
below the chance the die result indicates is \emph{multiplied by}
\textbf{10}.  Thus, if the character in the preceding example were
wearing a psionic rig, his dice result would be considered
\textbf{200} \emph{percentage points} below his chance.  He could then
move the \textbf{3 kg} object \textbf{66} \emph{hexes} in one Action
Round and strike an NPC (Agility Rating of \textbf{6}) with a Hit
Strength of \textbf{7}.
A psionic rig does not increase the actual psychokinesis chance.

A successful psychokinesis attempt lasts one Action Round \emph{only}.
If a character wishes to continue controlling an object, he must
conduct another attempt.

A character who rolls a \textbf{0} on either die when attempting
psychokinesis receives one Experience Point.  However, a character may
receive no more than one EP when controlling an object through more
than one Action Round.

\subsection[Psionic Rig]{A character may enhance certain of his
  psionic skills by wearing a \hypertarget{tag:psionic-rig}{psionic rig}.}
\label{sec:psionicrig}

\index{psionic rig|bb}If wearing a psionic rig while attempting \emph{psionic boost}, a
character's skill level is \emph{increased by} \textbf{2}.  A Psionic
Rig greatly increases the range (to \textbf{100 km}) over which a
character may attempt \emph{mind control} and greatly increases the
range (anywhere on the world) over which the character may attempt
\emph{psionic communication}.  A psionic rig increases the effect of
successful \emph{psychokinesis} by a factor of \textbf{10}.  These
effects are detailed in the specific skill descriptions.  A psionic rig
does not aid navigation, psion tech, or life sense.  If a psionic rig
is being worn by a character that must use the Psionic Backlash Table,
10 is \emph{subtracted} from the result on the table.

A psionic rig weighs 3 kg and may be purchased in a psionic institute
for \textbf{25} Trans.

\subsection[Psionic Skills and Creatures]{If a character is aware that
  a creature has an Intelligence Rating, he may attempt mind control
  or  psionic communication with the creature.}
\label{sec:psicreatures}

\index{skills!psionic!and creatures}Such a creature is considered to possess a Mental Power Rating of
\textbf{1} unless otherwise specified.  A character's Mind Control and
Psionic Communication Skill Levels are reduced by \textbf{2} when
dealing with a creature.  If a creature has no Intelligence Rating or
the characters are not aware that it has an Intelligence Rating, mind
control and psionic communication cannot be attempted with it.

\subsection[Psionic Backlash]{A character must sometimes check for
  psionic backlash after attempting navigation, mind control, or
  psychokinesis.}
\label{sec:psibacklash}

\index{psionic backlash}If a dice result for mind control or psychokinesis is above the
modified chance, the difference is located on the Psionic Backlash
Table and the listed effect is applied to the character (no additional
dice roll is made).  If the outcome of a hyperjump includes a psionic
backlash modifier, the character must roll percentile dice again,
apply the modifier to a roll, and locate the modified result on the
Psionic Backlash Table to determine the effect on the character.

\index{psionic rig!psionic backlash}\index{psionic backlash!psionic
  rig, effect on}If a character is wearing a psionic rig, \textbf{10} is subtracted
from the result on the Psionic Backlash Table.  However, the rig may
become damaged, as listed in certain outcomes.

A shock result from the
table is carried out in accordance with \ref{sec:shock-check}.  If a stunned
character is not in an action situation, it can be assumed that he
blacked out for a few moments and then came to with no long term
effects.

\index{mental power!loss of}A character that suffers Mental Power loss as a result of psionic
backlash may regain the lost points only if he is healed at a psionic
institute (see \ref{sec:psiinstitute}).  Lost Mental Power Points may be
regained at the rate of one per month (once healing has begun).  If a
character's Mental Power Rating is reduced by \textbf{2} or more, he
may use none of his skills (psionic or otherwise) until healed; he is
temporarily insane.  The manifestation of his insanity is determined by
the GM.

\subsection[Psionic Backlash table]{The Psionic Backlash Table is
  used to determine the effects of psionic backlash.}  
\label{sec:psibacklashtable}

See table \ref{tab:psibacklash}.

\begin{table}[htbp]
  \centering
  \fbox{%
    \begin{minipage}{4in}
      \centering
      \caption{Psionic Backlash Table}
      \label{tab:psibacklash}

      \medskip

      \begin{tabular}{lp{2.5in}}
        \multicolumn{1}{c}{Percentile Dice}\\
        \multicolumn{1}{c}{Result Minus}\\
        \multicolumn{1}{c}{Chance} &
        Effect On Psionic Character\\
        \hline
        +10 or less & No effect.\\
        +11 to +20 &   Shock check (see \ref{sec:shock-check}).\\
        +21 to +30 &  Character is automatically shocked.\\
        +31 to +40 & Character is shocked; loses one die roll of points
        from his Endurance Rating. \\
        +41 to +55 & Character passes out; Endurance Rating reduced to
        \textbf{0}.  Psionic rig suffers superficial damage. \\
        +56 to +70 & Character passes out; Endurance Rating reduced to
        \textbf{0}.  Mental Power reduced by \textbf{1} (see
        \ref{sec:psibacklash}).  Psionic rig   
        suffers light damage. \\
        +71 to +85 & Character passes out; Endurance Rating reduced to
        \textbf{0}.  Mental Power reduced by \textbf{2} (see
        \ref{sec:psibacklash}).  Psionic 
        rig suffers heavy damage. \\
        +86 to +100 & Character passes out; Endurance Rating reduced
        to \textbf{0}.  Mental Power reduced to \textbf{1} (See
        \ref{sec:psibacklash}).  The character  
        may use no Psionic skills until cured (see
        \ref{sec:psiinstitute}).  Psionic 
        rig partially destroyed.\\
        +101 or more &  Character is dead.  Psionic rig destroyed. \\
      \end{tabular}

    \medskip

    \parbox{0.9\textwidth}{When checking for psionic backlash  after
      a hyperjump attempt, roll  percentile dice and add the modifier
      from  the Hyperjump Table
%% Original:
%% (nothing)
%% New:
      (table \ref{tab:hyperjump})\label{tab:psibacklash-change-ref}
%% End change
      to the
      dice result.  See 
      \ref{sec:psiinstitute} for 
      detailed explanation of use.}
  \end{minipage}}
\end{table}

\subsection[Psionic Institutes]{A psionic institute is a secluded
  center of research and meditation controlled by psions.}
\label{sec:psiinstitute}

\index{psionic institute}Any world with a class \textbf{3} or \textbf{4} spaceport has a
psionic institute (see \ref{sec:spaceport-class}).  The location of an
institute on a given world is determined by the GM.  Any character that
is a psionic navigator or a thinker may always enter a psionic
institute.  Any character with a Mental Power Rating of \textbf{3} or
greater may enter a psionic institute if accompanied by a psionic
navigator or thinker.  \index{psionic backlash!healing
  from}\index{mental power!recovery of}A character in an institute may be healed of any
ill effects of psionic backlash and any physical ailments as well.
Psionic rigs may be purchased and repaired at an institute.


\section{Vehicle Skills}
\label{sec:vehicleskills}

\index{skills!vehicle}Vehicle skills allow a character to safely drive or pilot all types of
planet-based vehicles on the ground, in the atmosphere, or on or
below liquid.

Any character may attempt to operate a vehicle.  However, in situations
that require skillful maneuvering or quick decisions, a character with
the proper skill will be much more likely to see himself and his
passengers through safely.

Vehicle skills are organized in a different way than other skills.
There are four vehicle skills: ground vehicles, air vehicles, marine
vehicles and military vehicles.  Each of these skills is divided into
five or six sub-skills, each representing proficiency with a
particular type of vehicle in the skill category.  As a character
increases a vehicle skill, he receives Experience Points to assign to
its sub-skills.

The following vehicle skills and sub-skills are available to the
characters:

\skill{air vehicles}{9 Levels/Limit: None}\label{sec:skill-air-vehicles}

The character is familiar with the theories of atmospheric flight and
the operation of all types of air vehicles, broken into the following
sub-skills:

\begin{description}
\item[Direct Lift.]  Any jet-powered craft designed for point take-off
  and landing.
\item[Glider.] Any air vehicle powered only by air currents or human
  strength.  Also includes ``mechanical birds,'' such as an
  ornithopter.
\item[Helicopter.]
\item[Jet Plane.]
\item[Propeller Plane.]
\item[Shuttle.]  A rocket-powered vehicle designed to fly from a
  planet surface to low orbit and back.  A character with pilot skill
  is considered to have this skill at the same level.
\end{description}

\skill{ground vehicles}{9 Levels/Limit: None}\label{sec:skill-ground-vehicles}

The character is experienced with all unarmed ground vehicles, broken
into the following sub-skills:

\begin{description}
\item[All-Terrain Vehicle.]
\item[Animal Drawn.] Includes all vehicles drawn by horses, oxen, and
  alien beasts of burden, and the riding of any such animals.
\item[Automobiles.] 
\item[Sled.] Any powered or non-powered vehicle designed for
   travelling over snow and ice.   If an animal-drawn sled is being
   used, the driver uses the  lower of his Animal Drawn and Sled
   Skill Levels.
 \item[Tractor.] 
 \item[Truck.] Any vehicle, designed for road use, with more than two
   axles.
 \end{description}
 
\skill{marine vehicles}{9 Levels/Limit: None}\label{sec:skill-marine-vehicles}

The character is familiar with all aspects of maritime transport and
the operation of a wide range of marine vessels, broken into the
following sub-skills:

\begin{description}
\item[Motorboat.] Any small engine-powered craft.
\item[Oar boat.] Any craft powered by human strength.
\item[Sailing ship.] A craft of any size powered by wind.
\item[Submarine.] Any submersible vessel.
\item[Supervessel.] Any large engine-powered ship, such as an ocean
  liner, supertanker, or aircraft carrier.
\end{description}

\skill{military vehicles}{9 Levels/Limit: None}\label{sec:skill-military-vehicles}

The character is familiar with the operation of a wide range of
military vehicles, broken into the following sub-skills:

\begin{description}
\item[Armed All-Terrain Vehicle.]  A character with this sub-skill may
  operate an unarmed ATV at the same skill level.  When doing so
  however, any Experience Points gained must be applied to the Ground
  Vehicles Skill, not to Military Vehicles.
\item[Armored Personnel Carrier.] 
\item[Half Track.] 
\item[Self-propelled Artillery.] 
\item[Tank.]
\end{description}

\subsection[Sub-Skill Levels]{Each level a character achieves in a
  vehicle skill allows him to increase his sub-skills by a number of
  points equal to the new Skill Level.}
\label{sec:vehsubskills}

\index{skills!vehicle!sub-skills}For example, a character at Skill Level 1 in Ground Vehicles could
assign \textbf{1} point to the automobile sub-skill.  When he achieves
Level 2, he could assign \textbf{1} additional point to the automobile
sub-skill and \textbf{1} point to the truck sub-skill.  Upon reaching
Level 3, he assigns \textbf{3} more points to any of the ground
vehicle sub-skills, and so on until he reaches Level 9, when he
receives \textbf{9} points to assign to any of the sub-skills (he
would then have a total of \textbf{45} points assigned to all the
ground vehicle sub-skills).  Points received when reaching a new Skill
Level may be assigned to sub-skills in any manner as long as a single
sub-skill does not exceed Level 9, and as long as points are assigned
only to sub-skills for vehicles that have been used by the character
in some capacity since the last Skill Level increase.

\textbf{Note:} When choosing a vehicle skill during character
generation, the player receives and assigns points to sub skills as
follows: At Skill Level 1 he receives \textbf{1} sub-skill point; at
Level 2 he receives a total of \textbf{3} points; at Level 3, a total
of \textbf{6} points; and at Level 4, a total of \textbf{10} points.

\subsection[Vehicle Accidents]{When a potential vehicle accident is
  imminent, the character driving the vehicle may attempt to overcome
  the hazard, using his vehicle skill.}
\label{sec:vehaccidents}

\index{accidents!vehicle!avoiding}The base chance to avoid an accident is \textbf{25\%}, \textbf{50\%},
or \textbf{75\%} (see \emph{Adventure Guide} section
\ref{AG-sec:accidents}).  The Performance Modifier of the vehicle is
added to the base chance.  The character's \emph{Dexterity Rating} and
the \emph{square} of his \emph{Sub-Skill Rating} are added to the base
chance.  If an unskilled character is driving the vehicle, nothing is
added to the base chance.

\begin{enumerate}
\item The character driving the vehicle rolls percentile dice.  If the
  dice result is less than or equal to the modified chance, no
  accident occurs at all, and this procedure is concluded (the GM
  might wish to describe some sort of ``close call'' to the players).
  If the result is greater than the chance, an accident occurs and the
  following steps are conducted.
\item The GM subtracts the modified chance from the dice result and
  locates the difference on the Hit Table (\ref{tab:hit}) to secretly
  determine the type of damage incurred by the vehicle.  He does not
  add a die roll to the difference (as is stated on the table).  The
  GM should then describe the nature of the accident to the players in
  colorful terms.
\item If the vehicle incurs \emph{heavy} damage or worse as a result
  of the accident, any characters aboard may be hurt.  Each character
  must roll one die, applying the following modifiers: subtract the
  character's \index{agility!and accidents}Agility Rating; add \textbf{10} if the vehicle is
  partially destroyed; add \textbf{20} if the vehicle is totally
  destroyed.  \index{accidents!vehicle!hits from}Locate the modified die result on the Hit Table and
  apply any hits incurred by the character as explained in
  \ref{sec:hits}.  Characters are not hurt in a vehicle accident
  resulting in \emph{superficial} or \emph{light damage}.
\end{enumerate}

\subsection[Vehicle Experience Points]{A character who rolls a
  \textbf{0} or \textbf{1} on either die when attempting to avoid a
  vehicle accident receives one Experience Point.}
\label{sec:vehicleeps}

The character may also check for Experience Point gain after every
\textbf{30} hours (or so) of driving time in which no accident check
occurs.  The GM and/or the driving character should keep track of
``safe'' driving time since the last accident check for this purpose.
After \textbf{30} hours have passed, the player rolls percentile dice
and gains an Experience Point if a \textbf{0} or \textbf{1} appears on
either die.  The dice roll has no purpose aside from checking for
Experience Point gain.

\subsection[Vehicle Types and Sub-Skills]{The type of vehicle
  sub-skill used for a  specific vehicle is listed on the Vehicle
  Chart.}
\label{sec:vehtypes}

Two sub-skills are listed for certain vehicles on the chart.  In this
case, the driver must use the one sub-skill that applies to the
current use of the vehicle.  If the GM introduces a vehicle into play
that is not covered by the Vehicle Chart (\ref{sec:land-vehicles},
\ref{sec:marine-vehicles}, or \ref{sec:air-vehicles}), he must assign
one (or two) of the applicable sub-skills to it, and announce this to
the players.


\section{Scientific Skills}
\label{sec:scientific-skills}

\index{skills!scientific}Scientific Skills allow a character to attempt a wide variety of
analyses, syntheses, studies and treatments that will often be of
vital importance to the party.

A character may undertake a scientific task only if he possesses the
appropriate skill, or if he is eligible to acquire the skill (that is,
if there is an \textbf{\textsf{X}} in the Skill Level space on his
Character Record, see \ref{sec:skill-points-are}).
\textbf{Exception}: Any character may attempt to \emph{diagnose} an
ailing person.

Unless otherwise stated, all scientific tasks require a particular
lab, scanner, or other piece of equipment.  Some of these devices
provide the character with a temporary increase in his Skill Level (as
explained on the Personal Equipment Table).  This increase is not
applied if the character does not possess the skill required for the
task.  \textbf{Exception}: An unskilled character attempting
\emph{treatment} may receive an increase when using a mediscanner (his
Skill Level is considered to be \textbf{0} for this purpose).

Each task requires a certain amount of time to perform.  In most cases
the time required is listed with the description of the equipment
that must be used.  If the task requires no equipment other than for
its primary function, the time required is listed in the task
description.  The time required to perform a task may be reduced or
increased, as explained in the chapter introduction.

The GM may have an NPC or service that the party has encountered
attempt a task that the party previously failed.  The GM should
discourage repetitive use of a task (such as scanning for geological
resources every 50 meters) by informing a character do ing so that he
is not eligible to receive Experience Points.

\skill{astronomy}{6 Levels/Limit: Intelligence}\label{sec:skill-astronomy}

The character is learned in the study of celestial bodies and the
geography of known space.  His services are required when attempting
to locate an unexplored planet or when attempting to locate one's own
position after a hyperjump error.  All spaceships contain equipment
necessary to survey the stars.  A Civ Level 8 spaceship or explorer pod
increases a character's Astronomy Skill Level by \textbf{1}.  A survey
pod increases the Level by \textbf{2}.

\begin{tasklist}
\item Locate unexplored planet (when in system space): \textbf{90\%}.
\item Locate uncharted planet (when in system space): \textbf{70\%}.
\item Locate own position after minor jump error: \textbf{60\%}.
\item Locate own position after major jump error: \textbf{40\%}.
\item Locate own position after randomized  jump: \textbf{20\%}.
\end{tasklist}

The time required for any of the above tasks is \textbf{6} hours.  A
dice result for any of the above tasks that is no more than
\textbf{10} above the modified chance indicates success with a
\textbf{20\%} increase in the time required for each extra percentage
point.  A dice result that is more than \textbf{10} above the modified
chance indicates failure.

\skill{biology}{9 Levels/ Limit: Intelligence}\label{sec:skill-biology}

The character is familiar with the science of living matter in all its
forms, and is learned in botany, zoology, biochemistry, and
xenobiology (the study of alien life).  By observing a creature, he
may discover its unique at tributes, the danger it presents (if any),
how it eats, and where it fits into its ecological ``niche.'' By
examining a creature with a bioscanner, the character may learn
details of its inner structure (it could be edible or of commercial
value).  A character's biology skill is reduced by \textbf{2} (to a
minimum of \textbf{1}) when dealing with non-carbon creatures (in some
cases this fact is know only by the GM).

\begin{tasklist}
\item During a creature encounter, when a character gets his first
  sight of the creature, he may attempt to perceive information about
  it (see \ref{sec:encounter-procedure}): \textbf{10\%}.  A bioscanner
  is not required for perception, nor is time expended.  For every
  \textbf{10} percentage points or fraction thereof that a dice result
  for perception is greater than the modified chance, the information
  is revealed to the character one Action Round (\textbf{10} seconds)
  later.  Thus, if the dice result were \textbf{22} higher than the
  chance, the GM would reveal the information three Rounds after
  perception was attempted, or when the Action Rounds are concluded
  (whichever comes first).
\item Perceive a creature with a bioscanner: \textbf{20\%}.  Same as
  preceding; however, the character must be within five meters of the
  creature (in the same hex) and must spend one Action Round at the
  task (this time may not be reduced).  This task may be performed
  after the above task against a given creature, if the information
  has not yet been revealed.
\item Examine creature with a bioscanner: \textbf{30\%}.  Requires one
  hour.  This task may be performed only if the creature is dead,
  unconscious, or safely restrained.  If examination is successful,
  the GM reveals the appropriate information about the creature to the
  character (see \emph{Adventure Guide} \ref{AG-sec:creatures}).  A
  dice result for examination that is no more than \textbf{10} above
  the modified chance indicates success with a \textbf{20\%} increase
  in the time required for each extra percentage point.  A dice result
  that is more than \textbf{10} above the modified chance for
  examination indicates failure.
\end{tasklist}

Biology skill is also required to diagnose and/or treat ailments
suffered by an alien life form (see diagnosis and treatment).

\skill{chemistry}{9 Levels/Limit: Intelligence}\label{sec:skill-chemistry}

The character is knowledgeable in all aspects in the study of chemical
substances and elements.  If he has a chemlab, he may analyze
atmosphere, soil samples, or liquid samples for all chemical elements
and compounds.  If the character has a chemsynthesizer and the proper
raw materials, he may attempt to synthesize any chemical compound.  A
dice result for chemical \emph{analysis} that is no more than
\textbf{10} above the modified chance indicates successful analysis
with a \textbf{10\%} increase in the time required for each extra
percentage point.  A dice result that is more than \textbf{10} above
the modified chance for analysis indicates failure.  A dice result for
chemical \emph{synthesis} that is above the modified chance at all
indicates failure.

\begin{tasklist}
\item Analyze sample to find all abundant chemicals: \textbf{70\%}.
\item Analyze sample to find all abundant and trace chemicals and
  complex compounds: \textbf{25\%}.
\item Synthesise simple compound (water, oxygen, explosives, acids):
  \textbf{20\%}.
\item Synthesise complex compounds (such as drugs or edible proteins):
  \textbf{-5\%}.
\end{tasklist}

\skill{diagnosis}{9 Levels/Limit: Intelligence}\label{sec:skill-diagnosis}

The character is familiar with the theories of medicine and the nature
of all ailments suffered by humans.  His services are essential before
a patient may be healed.  If the character has a first aid kit or a
mediscanner, he may attempt to diagnose injury or disease suffered by
another person (not himself) and thus allow and aid the medical
treatment of that person.

\begin{tasklist}
\item Diagnose ailing being for treatment: \textbf{90\%}.  From this
  chance the GM \emph{subtracts} ($3 x$ the total number of hits
  received by the patient); \textbf{7} hits would be a subtraction of
  \textbf{21}.  If the patient is suffering from something other than
  hits to his physical characteristics (such as poison or disease),
  the GM secretly determines how much is subtracted from the chance.
  The modified chance may exceed \textbf{100\%} after adding to it for
  the character's skill.  No time is required to perform this task.
  The time listed on the Equipment Chart to use a first aid kit or
  mediscanner is for treatment only.
\end{tasklist}

A dice result for diagnosis that is above the modified chance
indicates that diagnosis has failed; treatment may \emph{not} be
conducted by the characters at all.  If the dice result is \emph{less}
than the modified chance by \emph{more than} \textbf{20}, the Skill
Level of the character that will treat the patient is \emph{increased
  by} \textbf{1} (this may be the diagnosing character or any other
with treatment skill).  If the dice result is less than the modified
chance by \emph{more than} \textbf{40}, the Treatment Skill Level is
\emph{increased by} \textbf{2}.  If the dice result is less than the
modified chance by more than \textbf{70}, the Treatment Skill Level is
\emph{increased by} \textbf{3}.

A character without diagnosis skill may attempt to diagnose.  When
doing so, the \textbf{90\%} chance (with the subtraction for the
ailment) may not be increased at all.  Successful diagnosis by an
unskilled character may not increase a character's Treatment Skill
Level.

When attempting to diagnose an ailing alien life form, a character
uses the \emph{lower} of his \emph{Diagnosis} and \emph{Biology Skill
  Levels}.  If he does not possess both of these skills, he is
considered unskilled at the task.  The GM may also apply this rule when
a character is diagnosing a human with an alien disease.

\skill{geology}{7 Levels/Limit: Intelligence}\label{sec:skill-geology}

The character can identify all known types of rocks and minerals and
has studied the forces that compose and control planetary crusts and
mantles.  If he has a geolab or geoscanner, he may analyze a sample for
mineral and resource content.  If he has a geoscanner, on a planet
surface, he may also survey the area for minerals, other resources,
fissures, or volcanic activity.  A dice result for any geology task
that is no more than \textbf{10} above the modified chance for the
task indicates success with a \textbf{10\%} increase in the time
required for each extra percentage point.  A dice result that is more
than \textbf{10} above the modified chance indicates failure.

\begin{tasklist}
\item Analyze sample to find all abundant minerals and resources:
  \textbf{70\%}.
\item Analyze sample to find all abundant and trace minerals and
  resources: \textbf{30\%}.
\item Scan area to locate three minerals or resources that the
  geoscanner is set for: \textbf{60\%}.
\item Scan area to locate all tunnels and fissures at least three
  meters wide: \textbf{50\%}.
\item Scan area to locate all tunnels and fissures: \textbf{20\%}.
\item Scan area to locate volcanic activity: \textbf{60\%}.
\end{tasklist}

All resources listed on the World Resource Table
(\ref{tab:world-resource}) may be found with geology skill and
geological equipment, except for light-fiber plants, wood, arable
land, edible plants, and edible game.  If a character has declared
that he is scanning an area for a resource that is located only at a
site, the resource is found only if the dice result is more than
\textbf{30} \emph{less} than the modified chance to locate resources.

\skill{physics}{6 Levels/Limit: Intelligence}\label{sec:skill-physics}

The character is educated in the study of matter, energy, motion, and
force.  If he has an energy scanner, he may analyze an object, an area
or an occurrence for the type of forces and energy that caused or
might affect it.  Unlike most other scientific tasks, the time required
to conduct a physics task depends on the task itself, not on the
attributes of the energy scanner, A dice result for any physics task
that is above the modified chance at all indicates failure.

\begin{tasklist}
\item Determine the type, intensity, and possible danger of energy
  picked up by the energy scanner in an area: \textbf{80\%}.  Time
  required: \textbf{1} Action Round (no reduction possible).
\item Determine the type of energy powering an unknown device or a
  non-protein based creature: \textbf{70\%}.  Time required:
  \textbf{1} Action Round (no reduction possible).
\item Determine, the type of force or energy that caused a phenomenon
  (such as a blast crater or some other unobserved act of
  destruction): \textbf{50\%}.  Time required: \textbf{1} hour.
\item Determine in advance what the application of a given force or
  energy might do to an object: \textbf{40\%}.  Time required:
  \textbf{1} hour.
\item Tap an energy source for use by the party: \textbf{20\%}.  Time
  required: \textbf{2} hours.  When attempting this task, a character
  uses the lower of his Physics and Energy Tech Skill Levels.  If he
  does not possess both of these skills, he is considered unskilled at
  the task.  The kit that would normally be used to repair the object
  that the character is attempting to provide power for is also
  required.  The character must identify the energy source (with one
  of the above tasks) before he may attempt to tap it.  Whether or not
  an energy source may be tapped is left up to the GM.  This task is
  not required in order to use a common energy source, such as an
  electrical outlet or battery pack.
\end{tasklist}

The physics skill is also required in order to attempt repair of a
force field (see Energy Tech, \ref{sec:technical-skills}).

\skill{planetology}{7 Levels/Limit: Intelligence}\label{sec:skill-planetology}

The character is well versed in the geography, meteorology, and other
general physical features that make up a world.  If he is orbiting a
world in a spaceship capable of carrying \emph{at least} 4
\emph{pods}, or that has an \emph{explorer} or \emph{survey} pod, he
may analyze the world to gain information about its climate,
atmosphere, geographical layout, and natural resource distribution.
When the character wishes to analyze a world, he chooses one of the
following tasks.  That task, and all listed above it (if not already
known), may be determined in a single analysis attempt.  If the
character's analysis dice result is greater than the modified chance
listed for the chosen task, analysis is still successful, but the time
required is increased by \textbf{10\%} for each percentage point over
the chance the dice result indicates.

\begin{tasklist}
\item Determine hydrograph percentage and distribution of land and
  liquid masses on entire world: \textbf{85\%}
\item Determine temperatures of all environs of world: \textbf{75\%}.
\item Determine atmosphere of world: \textbf{65\%}
\item Determine general resources of world (all those resources that
  exist throughout two or more environs, not number of environs per
  resource or environ location): \textbf{50\%}.
\item Determine the contour of all environs individually (peaks,
  mountains, hills, or flat): \textbf{40\%}.
\item Determine the dominant terrain feature of each environ
  individually (such as barren, forest, craters, ice,): \textbf{20\%}.
\item Determine the presence of general resources in all environs
  individually (all resources except those that exist only at a site
  in an environ, see \ref{sec:world-resource-table}): \textbf{10\%}.
\item Determine the detailed geography/geology of all environs
  individually, including the presence of resources found only at a
  \emph{site} (the GM may provide the character with a detailed
  environ map or maps, if available): \textbf{-5\%}.
\end{tasklist}

An \emph{explorer} or \emph{survey pod} is required for any world
analysis with less than a \textbf{40\%} base chance.  A survey pod is
required for any analysis with less than a \textbf{10\%} base chance.
The GM may wish to reduce some of these percentages if the world has
thick cloud cover, or has a side that never receives light from its
star.  The GM should provide the players with a world log that varies
in detail, depending on how much of the world they have analyzed.
Survey of a world from orbit will not reveal the exact location of any
resource, and will not reveal the presence of the following resources
at all: spices, light fiber plants, wood, edible plants, and edible
game.

\skill{programming}{8 Levels/Limit: Intelligence}\label{sec:skill-programming}

The character is familiar with the dynamics and operation of computers
and robots.  A character with programming skill may always use a
computer or robot that he owns to its full potential (no chances are
assigned and no dice rolls are made).  He may use a robot that he does
not own in the same way, as long as he has the robot's
\emph{controller} (see \ref{sec:robot-controllers}).  If the character
has access to a computer or robot owned by another person, company, or
government agency, he may attempt the following tasks.  When dealing
with a robot, a character's Programming Skill Level is \emph{reduced
  by} \textbf{2} (to a minimum of \textbf{1}).

\begin{tasklist}
\item Gain control of robot not controlled by character (without
  controller): \textbf{30\%}.  Time required: \textbf{1} hour.  An
  electrokit or robot kit is necessary, although neither provides a
  Skill Level increase.
\item Call up unrestricted information in computer: \textbf{90\%}.   
\item Call up restricted information in computer: \textbf{30\%}.   
\item Call up top-level secrets in computer:  \textbf{10\%}.  
\item Alter protected information in computer: \textbf{10\%}.   
\end{tasklist}

A dice result for any of the preceding tasks that is above the
modified chance in dicates failure.  A dice result that is more than
%% Original:
ten
%% New:
\textbf{10}\label{sec:scientific-skills-change-ten}
%% End change
above the modified chance may cause the computer or robot to alert its
owners (openly or secretly) that the character is using the device in
a way that may not be to their liking.  The repercussions of such an
occurrence are left up to the GM.  A character must have the
\emph{compu/robot tech} skill in order to attempt any task not listed
above that involves computer or robot \emph{hardware}, or he must work
with a character who has the tech skill (as ruled by the GM).

\skill{treatment}{9 Levels/Limit: Intelligence}\label{sec:skill-treatment}

The character is familiar with all forms of paramedical and surgical
procedures.  If he has a first aid kit or a mediscanner, \emph{and
  successful diagnosis has been performed,} the character may treat an
ailing person, thus speeding his recovery or even saving his life.

\begin{tasklist}
\item Treat an ailing being: \textbf{1\%}.
\end{tasklist}

Effects of treatment (whether successful or not) are explained in
\ref{sec:rate-of-healing}.  A character's Treatment Skill Level may be
increased if diagnosis was successfully performed (in addition to any
increase for using a mediscanner; see diagnosis, above).  A character
who is eligible to acquire treatment skill but has not \emph{may}
attempt to treat a patient.  When doing so, his Skill Level
(\textbf{0}) may be increased by successful diagnosis and use of a
mediscanner (unlike most tasks performed by an un skilled character).
However, treatment by an unskilled character is not as effective (see
\ref{sec:rate-of-healing}).

When attempting to treat an alien life form, a character uses the
lower of his Treatment and Biology Skill Levels.  If he does not
possess both of these skills, he is considered unskilled at the task.

\subsection[Additions To Base Chance]{A character who is skilled in
  any scientific task he is attempting adds his Intelligence Rating
  and the square of his Skill Level to the base chance.}
\label{sec:sci-skills-add-base-chance}

An eligible unskilled character who is attempting a scientific task
adds nothing (not even his Intelligence Rating) to the base chance.

\subsection[Experience Points]{A character receives an Experience
  Point when attempting a scientific task as follows:}
\label{sec:sci-skills-ep}

\begin{itemize}
\item If he rolls a \textbf{0} or \textbf{1} on either die for a
  \emph{chemistry}, \emph{geology}, or \emph{biology} task.
\item If he rolls a \textbf{0}, \textbf{1}, or \textbf{2} on either
  die for a \emph{diagnosis}, \emph{treatment}, \emph{programming} or
  \emph{physics} task.
\item If he rolls a \textbf{0}, \textbf{1}, \textbf{2} or \textbf{3}
  on either die for a \emph{planetology} or \emph{astronomy} task.
\end{itemize}

\textbf{Note}: A character that is unskilled at a scientific task
receives an Experience Point for attempting the task only if it is
successful \emph{and} fulfills the preceding requirements.

\subsection[Rate Of Healing]{The rate at which a character heals
  from wounds incurred depends on his Endurance Rating and the quality
  of treatment he receives.}
\label{sec:rate-of-healing}

\index{hits!healing
  from}\index{injury|see{hits}}\index{wounds|see{hits}}\index{healing|see{hits,
  healing from}}\begin{itemize}
\item A wounded character who receives \emph{no} treatment or who
  receives \emph{unsuccessful treatment} from an \emph{unskilled}
  character regains lost Physical Characteristic points at the rates
  indicated on the \textbf{Treatment Results Table} (Table
  \vref{tab:treatment}).
\item A wounded character who receives \emph{successful treatment}
  from an \emph{unskilled character} has the time required to regain
  each characteristic point divided by \emph{one half} his \emph{full}
  Endurance Rating (rounded down).  Thus, a character with a full
  Endurance Rating of \textbf{6} (regardless of his current rating)
  who had lost five points from his Strength Rating would regain one
  point every two days.
\item A wounded character who receives \emph{unsuccessful} treatment
  from a \emph{skilled} character has the time required to regain each
  characteristic point divided by his full Endurance Rating.
\item A wounded character who receives \emph{successful} treatment
  from a \emph{skilled} character has the time required to regain each
  characteristic point reduced as follows: the number of percentage
  points below the modified treatment chance the dice result shows is
  added to the full Endurance Rating of the character.  The time
  required for the character to regain each characteristic point is
  then divided by this sum.  \textbf{Example:} A character with a full
  Endurance Rating of \textbf{6} has lost \textbf{5} points from his
  Strength Rating.  A skilled character that is treating him with a
  \textbf{60\%} chance of success rolls a \textbf{40} (a difference of
  \textbf{20}).  The base time required to regain each lost point
  (\textbf{6} days) is divided by \textbf{26}, so that one point is
  regained approximately every \textbf{5} hours.
\end{itemize}

A character that is regaining points lost from more than one
characteristic must regain points for each characteristic as evenly as
possible.  Thus, a character that has lost points from his Endurance
and Agility must regain a point in each before regaining a second
point in either (until one or the other has returned to its full
rating).

\index{characteristics!reduced to zero}If one or more of a wounded character's Physical Characteristics are
at \textbf{0}, a check must be made every \emph{game hour} to
determine if permanent, untreatable damage occurs.  Every hour
(beginning one hour after the character incurred the wounds) he rolls
percentile dice for each characteristic at \textbf{0}.  If the dice
result is \textbf{10} \emph{or less}, the full rating for the
characteristic is \emph{permanently reduced by} \textbf{1}.  If the
\emph{Endurance} Rating is currently at \textbf{0}, and the dice
result is a \textbf{01} or \textbf{02}, the character \emph{dies}.
These checks are made every hour until all Physical Characteristics
are increased above \textbf{0} by healing.  As long as a character's
Endurance Rating remains at \textbf{0}, he is considered unconscious
(see \ref{sec:hits}).

A character whose healing time is reduced by treatment begins healing
at the new rate when the time required for treatment has passed.

\begin{table}[htbp]
  \centering
  \fbox{%
    \begin{minipage}{2.5in}
      \centering
      \caption{Treatment Results Table}
      \label{tab:treatment}

      \medskip

      \begin{tabular}{cc}
        \rowcolor{grey}
        Total & Days required to regain\\
        \rowcolor{grey}
        Points Lost & one Characteristic Point\\
        \rowcolor{white}
        1, 2 & 1 Day\\
        \rowcolor{grey}
        3, 4 & 3 Days\\
        5--7 & 6 Days\\
        \rowcolor{grey}
        8--10 & 10 Days\\
        11--14 & 16 Days\\
        \rowcolor{grey}
        15 or more & 24 Days
      \end{tabular}
    \end{minipage}}
\end{table}

\section{Technical Skills}
\label{sec:technical-skills}

\index{skills!technical}\index{skills!repair|see{skills, technical}}Technical skills allow a character, to repair weapons, robots,
vehicles, and other equipment damaged during play.

When a device is damaged (as a result of weapon fire, other combat
actions, or accident) the GM secretly determines the extent of damage
(superficial, light, heavy or partially destroyed; see
\ref{sec:equipment-damage}) and informs the players how the damage
appears to them, without actually letting them know the category of
damage.  Any character may volunteer to repair a damaged device, but
unless he has the appropriate Tech Skill, he will rarely be able to
repair anything more than superficial damage to small items.

A character with a Tech Skill is familiar with the technology,
materials, and operation of all devices related to the area of his
skill.  Aside from repair work, the GM may allow a character's Tech
Skill to come into play in such other situations, as when the party is
inspecting unknown equipment or, if the proper materials are
available, when the character is attempting to build a device related
to his Tech Skill.

\index{equipment!repair kits}A party may attempt to repair an item only if it has the requisite
\emph{kit}.  A \emph{basic repair kit} usually allows repair of
superficial damage to any item smaller than a large ground vehicle,
and superficial or light damage to any item that may be held by a
character.  Certain exceptions to this rule, and the type of kits
required for repair of more extensive damage are listed with the
appropriate Tech Skill description.  Certain kits may increase a
character's Tech Skill Level for purposes of a given repair attempt.
In such instances, the increase is applied to the Skill Level before
any reductions a re made for especially difficult repair jobs (as
noted in certain Tech Skill descriptions).  Detailed explanations of
the attributes of all kits can be found in \ref{sec:tech-kits}.


\skill{compu/robot tech}{9 Levels/Limit: Intelligence}\label{sec:skill-compu-robot-tech}

The character may repair all types of computers, portable and
installed.  He may also repair robots with a reduction of two to his
Skill Level (to a minimum of \textbf{1}).  An electrokit is required
for repair of a computer that has incurred more than superficial
damage.  A robot kit is required for repair of a robot that has
incurred more than superficial damage.

\skill{construction}{6 Levels/Limit: None}\label{sec:skill-construction}

The character is familiar with the construction of houses, shelters
and buildings sealed from harsh environments.  He may repair any such
structure.  Building materials (made available at the GM's discretion)
are required for repair of any small structure (survival hut, storage
shed) that has incurred more than superficial damage, and any large
structure (office building, barracks) that has incurred any damage.
Certain kits may be used in specialized repair of structures (such as
an armor kit for a damaged pill box).

\skill{electro tech}{8 Levels/Limit: Dexterity}\label{sec:skill-electro-tech}

The character may repair all types of handheld, non-weapon devices,
including scan ners, portable labs, cameras, holographers, radios, and
all other types of small electronic equipment.  The electro tech skill
may not be used to repair interstellar commlinks, psionic equipment,
and computer systems.  A character's Electro Tech Skill Level is
reduced by \textbf{2} (to a minimum of \textbf{1}) when repairing or
in specting any Civ Level 8 device.  An electro-kit is required to
repair any of these items that has incurred more than light damage.
Electro Tech also allows a character to operate a two-way radio
skilfully.

\skill{energy tech}{6 Levels/Limit: Dexterity}\label{sec:skill-energy-tech}

The character is familiar with all types of power systems.  He may
repair heating and cooling systems, electrical systems, air systems,
and all non-combustion drive systems (including spaceship engines).
The kit required for repair depends on the type of system undergoing
repair.  An electrokit would be used for most portable systems, a
vehicle kit for damage to a vehicle climate-control system or engine,
and a spaceship kit would be used for a spaceship engine or other
spaceship system.  A basic repair kit may not be used to repair power
systems at all.  

The character may also attempt to repair a damaged
force field.  When doing so he uses the lower of his \emph{Energy Tech}
and \emph{Physics Skill Levels}.  If he does not possess both of these
skills, he may not attempt repair.  An electrokit is required to repair
a personal force field.  A vehicle or spaceship kit (as appropriate) is
required to repair a larger force field.

\skill{psion tech}{8 Levels/Limit: Dexterity}\label{sec:skill-psion-tech}

The character may repair interstellar commlinks, psionic rigs, and
other psionic equipment.  He may also repair psionic navigation
equipment in the jump pod of a spaceship with a reduction of
\textbf{2} to his Skill Level (to a minimum of \textbf{1}).  An
electrokit is required to repair psionic navigation equipment that has
incurred \emph{any} damage.

\skill{spaceship tech}{9 Levels/Limit: Intelligence}\label{sec:skill-spaceship-tech}

The character may repair damage incurred by spaceship hulls and pods
(including Battlecraft).  Each damaged part of a spaceship must be
repaired separately.  When repairing damage to sp aceship engine or to
spaceship armor, the character's Skill Level is reduced by \textbf{2}
(to a minimum of \textbf{1}).  Psionic equipment may not be repaired
with the spaceship tech skill.  Repair of superficial damage requires a
Civ Level 6 spaceship kit, light damage a Civ Level 7 spaceship kit,
and heavy damage a Civ Level 8 spaceship kit.  A partially destroyed
part of a spaceship may only be repaired at a Class 4 spaceport.

\skill{suit tech}{8 Levels/Limit: Dexterity}\label{sec:skill-suit-tech}

The character may repair all types of expedition suits, respirators,
respirator helmets, and body armor.  He may also repair armor on small
vehicles.  However, when repairing armor his Skill Level is reduced by
\textbf{2} (to a minimum of \textbf{1}).  A character that is repairing
body armor may declare that he is repairing punctures only.  If he does
so, his Skill Level is not reduced, but any reductions to the
projectile and beam defense strength of the armor may not be repaired.
A suit kit is required to repair an expedition suit or body armor that
has incurred more than superficial damage.  A suit kit may not be used
to repair the projectile and beam defense strength of armor.  An armor
kit is required to do full repair work on body armor and vehicle
armor.

\skill{vehicle tech}{8 Levels/Limit: Dexterity}\label{sec:skill-vehicle-tech}

The character may repair all types of vehicles listed in
\ref{sec:vehicles}.  When he is repairing a military vehicle, an air
vehicle, or armor on any vehicle, his Skill Level is reduced by
\textbf{2} (to a minimum of \textbf{1}).  A vehicle kit is required in
order to repair a small vehicle that has incurred any damage, or a
large vehicle that has incurred any damage.

\skill{weapon tech}{8 Levels/Limit: Dexterity}\label{sec:skill-weapon-tech}

The character is familiar with the workings of projectiles and beam
weapons, both hand held and mounted.  He may repair any type of weapon
listed in \ref{sec:weapons}.  He may also repair artillery and spaceship
missile, laser and particle fire systems with a reduction of
\textbf{2} to his Skill Level (to a minimum of \textbf{1}).  A weapon
kit is required to repair any handheld weapon that has incurred more
than light damage and any larger weapon that has incurred more than
superficial damage.  A spaceship kit is required to repair any damage
incurred by spaceship weapon systems.

\subsection[Repair]{One character may attempt to repair an item that
  is damaged, whether or not he possesses the  appropriate Tech
  Skill.}
\label{sec:repair}

\index{equipment!repair}\index{repair}When a character declares that he wishes to do so, and announces the
type of kit he is using, the following steps are undertaken:

\begin{enumerate}
\item The GM secretly determines the \emph{base repair chance}, depending on
  the type of damage incurred by the item, as indicated on the \textbf{Damage
  Repair Table} (Table \vref{tab:damage-repair}).
\item The GM secretly determines the \emph{maximum repair time} by
  multiplying the \emph{Base Repair Time} (listed in the description
  of the item under repair) by the appropriate \emph{Repair Time
    Multiplier} listed above.\label{item:1}
\item The  GM determines the  \emph{actual repair chance}  using the 
following formula:

$$\mathrm{Base\ Repair\ Chance} + \mathrm{Tech\ Skill\ Level}^2 +
   \mathrm{Intelligence\ Rating}^2$$
   
   A character's Tech Skill Level may be increased (before squaring)
   for this purpose if the appropriate kit is being used.  The actual
   repair chance may exceed \textbf{100\%}.
 \item The character attempting repair rolls percentile dice.
   \renewcommand{\theenumii}{\Alph{enumii}}  If the
   dice result is greater than the actual repair chance, the attempt
   fails; conduct \ref{item:3} below.  If the dice result is equal to or
   less than the actual repair chance, the attempt succeeds; conduct
   \ref{item:4} 
   below.\label{item:2}
   \begin{enumerate}
   \item The GM determines how much time is spent in the futile
     attempt to repair the item.  He divides the maximum repair time
     (as calculated in Step \ref{item:1}) by the character's Tech Skill Level
     (plus any increase the kit allows) \emph{squared}, \emph{or} by his
     \emph{Intelligence Rating} (not squared), whichever is higher.  The
     amount of time derived from this calculation passes as the party
     waits for the repairer to realize that he cannot do the job.  If
     this period of time is long, an encounter may even occur.\label{item:3}
   \item The GM determines how much time is spent successfully
     repairing the item.  The dice result obtained in Step
     \ref{item:2} is subtracted from the \emph{actual} repair chance.
     The difference is applied as a percentage reduction to the
     maximum repair time to determine the \emph{actual} repair time.
     Example: The actual repair chance to repair an item with a
     maximum repair time of \textbf{24} hours is \textbf{70\%}.  The
     player rolls a \textbf{30}, which is less than the actual repair
     time by \textbf{40} (this can also be expressed as 60\% of the
     maximum repair time).  The actual repair time is then \textbf{14.5}
     hours.\label{item:4}
   \end{enumerate}
\end{enumerate}
 
If the dice result in a successful repair attempt is less than the
actual repair chance by more than \textbf{90\%}, the maximum repair time is
reduced by \textbf{90\%} only.

A character without the appropriate Tech Skill may only attempt to
repair superficial or light damage to an item.  When doing so, his
actual repair chance is equal to the base repair chance.  He receives
no adjustments for his Intelligence Rating or the kit he is using.

The GM may implement the passage of time when a character is
attempting to repair an item in one of two ways: he may announce the
amount of time at the outset and skip directly to the point in time
that repair is accomplished or failure is realized; \emph{or} he may allow
time to pass normally and not reveal the result of the repair attempt
until the moment of realization is reached.  If he chooses the latter,
and repair is successful but lengthy, he should inform the players
that the attempt will be successful and how long it will take well
before the repair is accomplished.  The amount of time that passes
before revealing successful repair may be calculated as in Step \ref{item:3}
above, except of course, that the GM announces success instead of
failure.

\newcolumntype{G}{%^^A
   >{\color{black}\columncolor{grey}%^^A
      \raggedright}c}
\begin{table}[htbp]
  \centering
  \fbox{%
    \begin{minipage}{3.25in}
      \centering
      \caption{Damage Repair Table}
      \label{tab:damage-repair}

      \medskip
      
      \begin{tabular}{lcc}
        & Base Repair & Repair Time\\
        Type Of Damage & Chance & Multiplier\\
        \rowcolor{grey}
        Superficial         & 50\%    & $\times$ 1\\
        Light               & 20\%    & $\times$ 2\\
        \rowcolor{grey}
        Heavy               & --10\% & $\times$ 4\\
        Partially Destroyed & --40\% & $\times$ 8\\
        \rowcolor{grey}
        Totally Destroyed   & \multicolumn{2}{G}{Repair Impossible}
      \end{tabular}
    \end{minipage}}
\end{table}

\subsection[Experience Points]{A character who rolls a \textbf{0},
  \textbf{1}, or \textbf{2} on either die when attempting repair
  receives an Experience Point.}
\label{sec:tech-skills-ep}

The GM may also give a character an Experience Point in this manner
when the character is rolling percentile dice for some other use of
his tech skill.

\subsection[Repair Service]{A damaged item that the characters are
  unable to repair may be taken to a repair service.}
\label{sec:repair-service}

A town or suburb area on a planet is considered to have services for
repairing any item of a Civ Level equal to or less than that of the
planet.  An urban area is considered to have services for repairing any
item of a Civ Level up to \textbf{1} greater than that of the planet.

\textbf{Exception}: Spaceships and spaceship parts may be repaired
only at a spaceport.  A Class 2 spaceport has facilities for repair of
superficial and light damage, a Class 3 spaceport for heavy damage,
and a Class 4 spaceport for partially destroyed spaceships.  Psionic
equipment may be repaired only at a Psionic institute (see
\ref{sec:psiinstitute}).

An item taken to a repair service is automatically repaired.  The
repair time is always equal to the \emph{Base Repair Time} multiplied
by the \emph{Repair Time Multiplier}.  This product, when expressed in
hours, also represents the \emph{cost} to repair the item in 100's of Mils.
Thus, an item with a Base Repair Time of \textbf{6} hours that has
suffered heavy damage would take \textbf{24} hours of work time to
repair and the job would cost \textbf{2400 Mils} (2.4 Trans).
The price of repair service may not be haggled over.


\section{Interpersonal Skills}
\label{sec:interpersonal-skills}

\index{skills!interpersonal}Interpersonal skills are used by the character when dealing directly
with society, in business, leisure, legal, and communication matters.
More so than with other skills, the GM should consider the \emph{player's}
actual interplay with NPC's or authorities when using one of these
skills.  For example, a character may have a high Diplomacy Skill
Level, but if the player blatantly insults the individual that he is
conversing with, the skill should not do him much good.

\skill{diplomacy}{6 Levels/Limit: Empathy}\label{sec:skill-diplomacy}

The character is experienced in all manner of official conversation
and negotiation, and is generally well-spoken and tactful.  He will be
most effective when dealing with those from the \emph{local establishment} or
a higher social standing (see table \ref{sec:social-standing-table}).

\index{skills!diplomacy!and communication with NPCs}\index{skills!streetwise!and communication with NPCs}When a character is acting as party \emph{spokesman}, he may use his
streetwise and/or diplomacy skill to aid the establishment of friendly
communications with an NPC or group of NPCs (see
\ref{sec:npc-encounters}).  During Step \ref{item:npc-interact-7}
of the NPC Encounter
Procedure, the chance of successful communication is calculated by the
GM: To the base chance of \textbf{40\%} is added \emph{twice} the spokesman's
\emph{Empathy Rating}.  The spokesman declares whether he is using
his \emph{streetwise} or \emph{diplomacy} skill.  If he declares use
of the skill that the NPC's \emph{social standing} responds to, the
\emph{square} of the Skill Level is added to the chance.  If he declares the
skill that the NPC's social standing does \emph{not} respond to, the
Skill Level (not squared) is added to the chance.  If the spokesman is
unskilled in both streetwise and diplomacy, his Empathy Rating (not
doubled) is added to the base chance \emph{only}.  The GM rolls
percentile dice and applies the outcome to the NPC Reaction Table
(\ref{tab:npc-reaction}) in terms of \emph{shifts}, as explained in
the NPC Encounter Procedure.

In addition to the communications task, the GM should take a
character's streetwise and/or diplomacy skill into account when the
character is participating in any sort of extended dialogue with
NPC's.  These skills do not aid a character in financial negotiations
(the \emph{trading} skill is used for money matters).

A character who rolls a \textbf{0}, \textbf{1}, or \textbf{2} on
either die when using his streetwise or diplomacy skill receives an
Experience Point.

\skill{disguise}{8 Levels/Limit: Dexterity}\label{sec:skill-disguise}

The character can control his voice, mimic a wide variety of postures,
and alter his facial appearance through the use of make-up, latex,
skin injections and dyes.  It is assumed a character with this skill
possesses the requisite materials to alter his appearance; however,
accessories such as clothing and insignia must be acquired by the
character when necessary.  The base time required to prepare a disguise
is four hours.  The character's \emph{Dexterity Rating} and the
\emph{square} of his \emph{Skill Level} are added to the base chance
of the following tasks.

\begin{tasklist}
\item Disguise self to gain anonymity: \textbf{60\%}.
\item Disguise self to resemble person that has been extensively
  observed: \textbf{40\%}.
\item Disguise self to resemble person that has been seen briefly or
  in pictures only: \textbf{25\%}.
\end{tasklist}

The GM rolls percentile dice secretly and compares the dice result to
the modified chance:

\begin{itemize}
\item \emph{Result under chance by} \textbf{more than 20}: The
  disguise fools all except those intimately familiar with the
  subject.  If the character is unskilled, the next outcome is used
  instead.
\item \emph{Result equal to or under chance by} \textbf{20 or less}:
  The disguise fools those who do not have everyday contact with the
  subject.
\item \emph{Result over chance by} \textbf{30 or less}: Anyone who
  gets a good look at the disguise, or hears the character say more
  than a few words will not be fooled.  If the character is skilled, he
  is informed of this fact.  If he is not skilled, he is told that the
  disguise looks fine.
\item \emph{Result over chance by more than} \textbf{30}: The character fails
  his attempt and is told that it will not work.
\end{itemize}

A character that rolls a \textbf{0}, \textbf{1}, or \textbf{2} on
either die when attempting disguise receives an Experience Point.

\skill{economics}{8 Levels/Limit: Intelligence}\label{sec:skill-economics}

The character understands the complex economic systems of the future;
how the resources, shipping schedules and laws of supply and demand on
the worlds of the federation affect the value of any item from place
to place.  If he has a business computer, the character may attempt to
predict the price of any declared item in the future at a declared
location.  If the location is another star system, he must have access
to an interstellar commlink.  The base time required for price
prediction is \textbf{12} hours.

\begin{tasklist}
\item Determine price of an item at a declared time and location:
  \textbf{35\%} minus the number of weeks (or fraction thereof) in the
  future the price is requested for.  The character's
  \emph{Intelligence Rating} and the \emph{square} of his \emph{Skill
    Level} are added to the base chance.
\end{tasklist}

The GM rolls percentile dice.  If the dice result is equal to or less
than the modified chance, the attempt succeeds; the GM immediately
uses the Actual Price Table (\ref{tab:actual-price}) to determine the price
and announces it to the character.  A dice result above the modified
chance indicates failure (after \textbf{12} hours have passed).  If a
failing dice result is even, the character is told that the prediction
is unsuccessful.  If a failing dice result is odd, the GM reads the
player a random result from the Actual Price Table as if it were a
correctly predicted price.

A character who rolls a \textbf{0}, \textbf{1}, or \textbf{2} on
either die when attempting to predict a price receives an Experience
Point.

\skill{forgery/counterfeiting}{8 Levels/Limit: Dexterity}\label{sec:skill-forgery-counterfeiting}

The character is experienced in the art of forging documents used for
identification, shipping, and other business and government
transactions.  If he has the requisite materials (as determined by the
GM) and a model to work from, the character may attempt to copy a
document or piece of currency.  The character's \emph{Dexterity Rating}
and the \emph{square} of his \emph{Skill Level} are added to the base
chance of the following tasks.  The time required for each task is
determined by the GM, depending on the complexity of the document.

\begin{tasklist}
\item Forge commercial document: \textbf{50\%}.
\item Forge world or local government document: \textbf{30\%}.
\item Forge federal document: \textbf{10\%}
\item Counterfeit 100 Mil note: \textbf{25\%}
\item Counterfeit 1 Tran note: \textbf{0\%}
\end{tasklist}

The GM rolls percentile dice secretly and compares the result to the
modified chance:

\begin{itemize}
\item \emph{Result under chance by} \textbf{more than 20}: The
  document passes all inspections (visual and electronic).
  \textbf{Exception}: If the character is unskilled and/or the world
  has a Law Level of \textbf{4}, the next outcome is used instead.
\item \emph{Result equal to or under chance by} \textbf{20 or less}:
  Document passes all visual inspection.  \textbf{50\%} chance that
  the document will be found false each time it undergoes electronic
  inspection.
\item \emph{Result over chance by} \textbf{30 or less}: Document found
  false by any electronic inspection.  When undergoing visual
  inspection, the GM rolls percentile dice; if the result is less than
  or equal to the inspector's \emph{Intelligence Rating} plus the
  \emph{square} of his Forgery/Counterfeit \emph{Skill Level}, the
  document is found false.  \textbf{Note}: A character with
  forgery/counterfeit skill may attempt to detect false documents made
  by others in the same way.
\item \emph{Result over chance by} \textbf{more than 30}: Document
  will not fool anybody.  If the forger is skilled, he is told of this
  fact.
\end{itemize}

Any repercussions of character's forged documents being found out are
left up to the GM.

A character who rolls a \textbf{0}, \textbf{1}, \textbf{2} or
\textbf{3} on either die when attempting forgery receives an
Experience Point.

\skill{gambling}{6 Levels/Limit: Intelligence}\label{sec:skill-gambling}

The character is familiar with all common games of chance.  If he is at
a casino, some other gaming establishment, a bar or inn with an
informal game, or with another character or NPC who wishes to play
against him, the character may gamble.  He must state his \emph{bet
  size} (if playing against another character or NPC, both agree on a
bet size).  The bet size is not the total amount to be risked, but
rather an amount that will be continuously risked over the four or
five hour period that each gambling attempt represents.  The
character's \emph{Intelligence Rating} and \emph{Gambling Skill Level
  (not squared)} are added to the base chance.

\begin{tasklist}
\item Gamble at a casino or established gaming house: \textbf{35\%}.
  Minimum bet size: \textbf{10 Mils}; Maximum bet size: \textbf{1
  Tran}.
\item Gamble at an informal gathering in public place: \textbf{40\%}.
  Minimum bet size: \textbf{1 Mil}; Maximum bet size: \textbf{100
    Mils}.
\end{tasklist}

If the dice result is less than the modified chance, character wins an
amount equal to the difference multiplied by the bet size.  If the dice
result is greater than the modified chance, the character loses the
amount.

When gambling against another character or NPC, both individuals roll
percentile dice separately, adding their Intelligence Rating and
Gambling \emph{Skill Level} to the dice result.  The difference between
the dice results is multiplied by the bet size, and the character with
the lower dice result must pay the product to the character with the
higher result.

\index{skills!gambling!cheating}\index{cheating at
  gambling|see{skills, gambling, cheating}}A character may declare that he is \emph{cheating} in any gambling
attempt.  If he does so, his Skill Level is \emph{doubled}, but if his
dice result is \emph{even}, the GM checks for detection by his
opponent.  He rolls two dice.  If the result is \emph{less than} the
opponent's Intelligence Rating plus his \emph{Skill Level}, minus the
cheating character's \emph{Skill Level}, the character's cheating is
revealed.  A casino is considered to have a combined Intelligence and
gambling skill of \textbf{18}, and an informal gathering of
\textbf{15} for this purpose.  The consequences of revealed cheating
and/or a character's inability to pay a gambling debt are up to the
GM.

A character who rolls a \textbf{0} or \textbf{1} on either die when
gambling receives an Experience Point.

\skill{law}{8 Levels/Limit: Intelligence}\label{sec:skill-law}

The character is learned in the structure of most federal and local
laws and judicial systems.  He is a skilful speaker and is eligible to
practice law in court.  The character may attempt any of the following
tasks, adding his \emph{Intelligence Rating} and the \emph{square} of
his \emph{Skill Level} to the base chance.  The chances of all these
tasks assume that a basically honorable legal system is being dealt
with; the GM should apply modifiers if the system is corrupt.

\begin{tasklist}
\item Bribe authority to ignore trespassing, illegal possessions or
  other criminal act: \textbf{20\%} \textbf{plus 1} for \emph{every
    100 Mils} offered, \emph{minus} \textbf{10$\times$} the 
  \emph{Law Level} of the world.  In addition, the GM should secretly
  reduce the chance by anywhere from \textbf{0} to \textbf{50}, based
  on the loyalty of the individual being bribed toward his employer or
  government.  \textbf{Example}: \textbf{50} would be subtracted for
  an elite federal soldier while a local security guard might cause no
  subtraction at all.
\item Convince authority that a criminal act is not illegal or that
  the party is exempt: \textbf{10\%} \emph{minus} the authority's
  \emph{Intelligence Rating} and minus \textbf{10$\times$} the Law
  Level of the world.
\item Successfully defend innocent party in court: \textbf{35\%}
  \emph{plus} \textbf{10$\times$} the Law Level of the world.
\item Successfully defend guilty party in a court: \textbf{35\%}
  \emph{minus} \textbf{10$\times$} the Law Level of the world.
\end{tasklist}

Any dice result under the modified chance indicates success.  Any dice
result over the modified chance indicates failure.  If a bribery
attempt fails by more than \textbf{10} percentage points, the
authority will either accept the bribe and report the party anyway, or
will report the party's bribe attempt as well as their original
transgression, at the discretion of the GM.

A character who rolls a \textbf{0}, \textbf{1}, or \textbf{2} on
either die when attempting a law task receives an Experience Point.  An
unskilled character may not defend a party in court.

\skill{linguistics}{8 Levels/Limit: Intelligence}\label{sec:skill-linguistics}

\index{languages}The character is fluent in other languages other than Universal.  He
can speak, read, and write in a number of additional languages equal
to his Skill Level.  A character that receives the linguistics skill
during character generation should choose any languages from among
those listed in \ref{sec:universe-future} and note them on the back of his
Character Record.  \textbf{Exception}: A language written in a
non-roman alphabet may not be chosen until Skill Level 3 is reached.
All characters are considered fluent in Universal.

When the party must speak with an NPC in an unknown language, a
skilled character may attempt to comprehend the NPC and make himself
understood.  A \emph{translator} (see \ref{sec:comm-gear}) is not required
to use the linguistics skill, but the device does increase the user's
Skill Level (even if he is \emph{unskilled}).  A translator does not
increase the number of languages a character may speak.  The chance of
successful communication equals the sum of the NPC's
\emph{Intelligence Rating}, the character's \emph{Intelligence
  Rating}, (if skilled) and the \emph{square} of the character's
\emph{Skill Level}.  The GM secretly rolls percentile dice and
compares the result to the chance:

\begin{itemize}
\item \emph{Result under chance by} \textbf{more than 20}: All
  information that the NPC and characters wish to exchange is
  understood.
\item \emph{Result equal to or under chance by}  \textbf{20 or less}:
  Simple  direction, numerical and identification information is
  exchanged.
\item \emph{Result over chance by} \textbf{20 or less}: Names and
  small numbers may be exchanged only.
\item \emph{Result over chance by} \textbf{more than 20}: Absolutely
  nothing is understood by either party or, if the GM wishes, one
  side totally misunderstands the information or intent of the other.
\end{itemize}

The linguistics skill may also be used to attempt communication with
certain creatures, once the desire to communicate has been
established.  See \ref{tab:npc-reaction}, Procedure CC.

A character who rolls a \textbf{0} or \textbf{1} on either die when
attempting to communicate in an unknown language receives an
Experience Point.  Each time a character attains a new level in
linguistics he may choose a new language.  The new language should be
one he attempted to use since his last Skill Level increase.  A
character \emph{never} receives an Experience Point for a language he
already understands.

\skill{recruiting}{6 Levels/Limit: Empathy}\label{sec:skill-recruiting}

The character is skilled in the administrative and personnel side of
business.  When the party seeks NPC's to aid in any venture they are
undertaking, the character may attempt to hire individuals with good
qualifications.  The character declares how many NPC's he wishes to
hire and the pay he is offering.  Two hours must be spent recruiting
for each NPC declared, with a minimum time of \textbf{6} hours.  The
character's Empathy Rating and the square of his Skill Level are added
to the base chance.

\begin{tasklist}
\item Hire NPC's in a Starport or urban area: \textbf{50\%}
\item Hire NPC's in a suburban area or town: \textbf{35\%}
\end{tasklist}

The GM may modify the base chance, depending on the pay offered,
skills requested, the danger and legality of the job, and the means of
seeking new employees (working through an employment agency would be
helpful).  A decent weekly salary for an individual is \textbf{100
  Mils} $\times$ the Civ Level of the equipment he will be dealing with.  If
the job is dangerous, the pay should be increased by \textbf{500 Mils}
to \textbf{1 Tran} per week.  An average individual for hire possesses
a Mental Power of \textbf{1} and an \index{aggression!individuals for hire}Aggression of \textbf{6}.  His
other characteristics average \textbf{4}.  He has a skill level of
\textbf{3} in his home environ, of \textbf{1} in his home gravity, and
has eight Skill Points distributed among all other skills by the GM.

If the dice result is equal to or less than the modified chance, the
declared number of NPC's are found.  The attributes of the NPC's may be
improved by the GM as follows: For every percentage point below the
modified chance the result indicates, each NPC is improved by one
Skill Point of one Characteristic Point.  The GM should apply these
increases to characteristics and skills that would be helpful in the
upcoming job.  If the dice result is greater than the modified chance,
fewer NPC's than requested are found and/or their attributes are
reduced, at the discretion of the GM.

A character who rolls a \textbf{0}, \textbf{1}, \textbf{2} or
\textbf{3} on either die when attempting to recruit receives an
Experience Point.  Only one recruitment attempt is allowed in a single
area for a given purpose.



\skill{streetwise}{4 Levels/Limit: Empathy}\label{sec:skill-streetwise}

\index{skills!streetwise!and communication with NPCs|see{skills,
    diplomacy}}The character is up on the slang and friendly expressions used among
the common folk throughout the federation.  He will be most effective
when dealing with those from the skilled tech class or a lower social
standing.  The streetwise skill is used for the \emph{communications} task
(see diplomacy skill description).

\skill{teaching}{6 Levels/Limit: Empathy}\label{sec:skill-teaching}

\index{skills!increasing!with an instructor}The character has experience teaching and/or tutoring and is able to
pass knowledge he has acquired onto others.  He may attempt to aid
another character (hereafter called the student) increase his
expertise in any skill that the teacher possesses at a \emph{higher level}
than the student.  One week of both characters' time is required to
teach a skill.  This time may not be reduced but may be increased, as
explained below.

When the teacher and student have declared the particular skill they
wish to study together, the chance of successful teaching is
calculated: To the base chance of \textbf{30\%} is added the student's
\emph{Intelligence Rating}, the teacher's \emph{Empathy Rating} and
the \emph{square} of the teacher's \emph{Teaching Skill Level}.  The
teacher rolls percentile dice.  The student receives \textbf{1}
Experience Point in the studied skill for every \textbf{10} percentage
points (or fraction thereof) below the modified chance the dice result
indicates (for example, if the modified chance were \textbf{55\%} and
the teacher rolled a \textbf{31}, the student would immediately gain
\textbf{3} EP's for the declared skill).  However, a student may never
increase a Skill Level by more than \textbf{1} in a single ``study
session'' (any excess EP's are lost).  If the dice result is no more
than \textbf{10} above the modified chance, the student gains
\textbf{1} EP and the teaching time is increased by \textbf{10\%} for
each percentage point over the chance the result indicates.  A result
that is more than \textbf{10} above the modified chance indicates
failure; one week is expended and the student gains no EP's.

No more than one student may be taught at a time, and any equipment
necessary to use the skill being taught must be available (such as a
weapon, vehicle, or tech kit).  A teacher may attempt to teach a
specific student a specific skill once only.  An unskilled character
may attempt to teach (using the base chance of \textbf{30\%} only);
however, his student may receive no more than \textbf{1} EP from the
study session.  A psionic skill may be taught only if both the teacher
and the student possess a Mental Power Rating of \textbf{4} or higher
and the student has or is eligible to acquire the skill.  A character
with teaching skill may charge any fee he can get for his teaching
services.  Conversely, the GM may have an NPC offer to teach a
character a skill as a favor or for pay.

A character who rolls \textbf{0}, \textbf{1} or \textbf{2} when at
tempting to teach receives an Experience Point.

\skill{trading}{6 Levels/Limit: Empathy}\label{sec:skill-trading}

The character is a skilled bargainer.  He can get the most out of a
transaction through his understanding of commerce and his ability to
negotiate.  When the GM is using the Actual Price Table
(\ref{tab:actual-price}) to determine the price of an item or service
that a character is attempting to purchase or sell, the character may
attempt to alter the price in his favor by bargaining.  Any purchases
from a federal establishment may not be bargained; such prices are
set.

The base chance of successful bargaining is \textbf{20\%}.  To this is
added the character's \emph{Empathy Rating} and the \emph{square} of
his \emph{Skill Level}.  The character rolls percentile dice.  If the
dice result is greater than the modified chance, the attempt fails;
\textbf{10} is \emph{added} to the GM's \emph{Actual Price} roll.  If the
dice result is less than the modified chance, the difference is
\emph{subtracted} from the GM's Actual Price roll.

One character should check for bargaining each time such a situation
arises.  If the character is unskilled, nothing is added to his base
chance.

A character who rolls a \textbf{0} or \textbf{1} on either die when
bargaining receives an EP.

\section{Environmental Skills}
\label{sec:environmental-skills}

\skill{agriculture}{8 Levels/Limit: None}\label{sec:skill-agriculture}

The character is skilled at farming tillable soil and in the science
of hydroponics (growing without soil).  Breakthroughs in fertilizers
and genetic research also enable the growth of crops to be greatly
accelerated.  When working with a hydroponic farm, \textbf{2} is
\emph{subtracted} from the character's skill level (to a minimum of
\textbf{1}).  A skilled character may always grow any plant (as long
as he has the proper shoots or seeds) in an environ with arable land
or in an established hydroponic garden.  If the character wishes to
grow a plant in a more exotic location or wishes to accelerate the
growth of a crop, the GM should assign a base chance to the declared
task.  The character's \emph{Intelligence Rating} and the
\emph{square} of his \emph{Skill Level} are added to the base chance.
As a guideline, the simplest of agriculture tasks would have a base
chance of \textbf{95\%} while an attempt to grow a fruit tree in an
arctic environment with a poisonous atmosphere would have a base
chance of \textbf{-5\%}.  A robot with an agriculture system increases
a character's skill level by \textbf{2}.

A character who rolls a \textbf{0}, \textbf{1}, or \textbf{2} on
either die when attempting an agriculture task receives an Experience
Point.

\skill{asteroid mining}{6 Levels/Limit: None}\label{sec:skill-asteroid-mining}

The character is familiar with the business and techniques of mining
and processing resources from asteroids and small planetoids.  The
tasks and procedures of this skill are identical to those of the
\emph{mining} skill.

\skill{environs}{6 Levels/Limit: None}\label{sec:skill-environs}

As explained in character generation, each character receives positive
or negative Skill Levels in all 33 environs shown on the Environ Skill
Display.  A character's Skill Level in the environ he is in is used
during an encounter for the following:

\begin{itemize}
\item \emph{Twice} the highest Environ Skill Level among the
  characters in the party is \emph{subtracted} from the awareness
  chance during a creature or NPC encounter (see \ref{sec:awareness}).
\item The character chosen as the party's \emph{leader} during an Action
  Round (see \ref{sec:initiative}) adds his Environ Skill Level to his
  \emph{initiative die roll}.
\item A character's Environ Skill Level is added to his chance to
  perform an \emph{ambush} task (see the ambush skill,
  \ref{sec:military-skills}). 
\end{itemize}

The Experience Point system explained in \ref{sec:acqu-impr-skills} is
not used to improve Environ Skill Levels.  Instead, the GM ``hands
out'' Environ Skill Level increases.  If a character spends one week
(give or take a day, at the GM's discretion) adventuring in a
particular environ away from urbanized areas, the GM should reward him
with a Skill Level increase of one in that environ.  No single environ
skill may be increased beyond Level \textbf{6}.

\skill{gravity}{5 Levels/Limit: None}\label{sec:skill-gravity}

As explained in character generation, each character receives positive
and negative Skill Levels in all four gravity types shown on the
Gravity Skill Display.  A character's Skill Level in the gravity type
he is in is added to his \emph{Action Round Movement Rate} (see
\ref{sec:action-round-movement}).  

A character's Gravity Skill Level
also affects his chance of avoiding a gravity-related accident (see
\emph{Adventure Guide} \ref{AG-sec:accidents}).

Experience Points are not used to improve Gravity Skills Levels.  If a
character spends eight weeks (give or take a week, at the GM's
discretion) adventuring in a particular gravity type, the GM should
reward him with a Skill Level increase of \textbf{1} in that gravity
type.  No single gravity skill may be increased beyond Level
\textbf{5}.

\skill{mining}{6 Levels/Limit: None}\label{sec:skill-mining}

The character is familiar with the business and technique of planetary
mineral and metal mining and processing.  After a minable resource has
been found in an environ (see the geology skill,
\ref{sec:scientific-skills}), the character may attempt to separate a
quantity of it from the ground in raw form.  Simple digging tools, a
\emph{rock blaster} or a robot with a \emph{miner} system are
required.  The latter two items provide the character with a Skill
Level increase.  After raw ore has been mined, the character may refine
it if he has a robot with a miner system (the Skill Level increase
applies).  The character's \emph{Intelligence Rating} and the
\emph{square} of his \emph{Skill Level} are added to the base chance.

\begin{tasklist}
\item Mine raw ore from an identified source: \textbf{25\%}.
\item Refine previously mined raw ore: \textbf{0\%}.
\end{tasklist}  

Each of these task requires \textbf{12} hours to attempt.  This time
may be reduced if the attempt is successful (see the task procedure in
the chapter introduction).  For every percentage point \emph{over} the
modified chance the dice result indicates, the amount of ore mined or
processed is reduced by \textbf{5\%} (the time required is not
increased).  If the dice result is greater than the chance by
\textbf{20} or more, the attempt fails completely.

The GM determines the quantity of ore that a character may mine in a
single attempt as follows: locate the ore on the World Resource Table
(\ref{tab:world-resource}) and note the number of environs the Table states
that the ore may appear in (if the ore is abundant on the world,
double this number).  The \emph{square} of the number represents the
number of \emph{kilograms of refined ore} that may be mined in a
single attempt (its actual weight in its raw form will be considerably
more). \textbf{Exception}: If the World Resource Table states that the
ore exists at a \emph{site} only, \emph{one gram} of the ore may be
mined in a single attempt.

These tasks are not used when the character is dealing with larger
mining facilities.  In such a case, his Skill Level would effect the
administration of the factory and the efficiency with which it
operates.

A character who rolls a \textbf{0}, \textbf{1} or \textbf{2} on either
die when attempting to mine or process ores receives an Experience
Point.

\skill{survival}{8 Levels/Limit: Intelligence}\label{sec:skill-survival}

The character is experienced in ``living off the land'' and staying
alive with a minimum of supplies in the wild.  If the character is in a
party that has exhausted its supply of food and water, he may attempt
to forage for the basic necessities to sustain himself and his
comrades.  The survival skill will \emph{not} aid a party that has exhausted
its oxygen supply.  A survival task takes \textbf{6} hours to conduct
and if successful, lasts for one full day (including the time spent
foraging).  The character chooses the most favourable task listed below
that applies to the party's situation.  He adds his \emph{Survival}
Skill Level to the \emph{highest Environ Skill Level in the party} and
\emph{squares} the \emph{sum} (a negative sum is considered \textbf{0} for
this purpose).  The result of this calculation \emph{and his
Intelligence Rating} are added to the base chance.

\begin{tasklist}
\item Survive in environ with edible game and/or plants: \textbf{90\%}
\item Survive in environ with arable land: \textbf{60\%}
\item Survive in environ with any type of vegetation: \textbf{30\%}
\item Survive on world that contains water: \textbf{--50\%}
\item Survive on world that contains no water: \textbf{--150\%}
\end{tasklist}

For every \textbf{10} (or fraction thereof) below the modified chance
the dice result indicates, one character may be kept alive and well.
If the dice result is over the modified chance at all, no characters
receive the basic necessities of life.  The GM determines the effect of
lack of food and water, depending on the party's current situation.

A character who rolls a \textbf{0}, \textbf{1} or \textbf{2} on either
die when using his survival skill receives an Experience Point.

\skill{urban}{6 Levels/Limit: None}\label{sec:skill-urban}

A character may receive an Urban Skill Level during character
generation.  A character's urban skill is used and improved when the
character is in a built-up area (an area where the natural features
have been entirely replaced with artificial structures and technology)
exactly as an environ skill is used and improved.

%%% Local Variables: 
%%% mode: latex
%%% TeX-master: "gm_guide"
%%% End: 
