%%
%% $Id: ag_lost_on_laidley.tex,v 1.1 2004/10/07 18:26:39 goltz20707 Exp $
%%
\chapter{Adventure: Lost on Laidley}
\label{cha:advent-lost-laidl}
% XI

This chapter presents a complete star system and adventure for use by
the GM and a suggested five players. It is recommended that this
adventure be played before the GM starts designing his own worlds and
adventures, so that all concerned get a good grasp of the many game
systems and opportunities that are available. Before beginning the
adventure session, the GM should read all the Sections of the Chapter.
He should allow the players to see Sections \ref{sec:orionis-system}
and \ref{sec:players-characters}, the Orionis Star System Log, the
three World Logs, and the Environ hex map only.  The GM should have a
good grasp of the rules in the GM Guide and the players should at
least be aware of the importance of their characteristics, skills and
equipment.

The adventure takes place in the Orionis star system (see
\ref{sec:orionis-system}).  However, use of the star system need not
stop with this adventure.  The GM may wish to begin his own adventures
in this system or borrow ideas from it when creating his own systems.
The Orionis system was generated in exact accordance with the rules of
GM guide chapter \ref{GM-cha:world-generation} and serves as an
example when learning the world generation system.


\section{The Orionis System}
\label{sec:orionis-system}
% [51.0]

The GM and the players should look over the Orionis Star System Log
and the three World Logs included in this booklet. All the basic
information about the system is explained on the logs. Historical
information and additional details can be found in this section. With
the exception of some undiscovered resources on Laidley, all the
information would be known by the characters.

Orionis is an F6 type star (yellow-white) 26 light years from Sol. It
is the most distant system to be extensively colonized by the
federation. Eighty years ago (in 2251), exploration and colonization
of the system opened with the arrival of an expedition financed by a
South American conglomerate. Its efforts to colonize Titus were soon
overshadowed by the arrival of a Scandinavian coalition on Kryo.  The
Nordic settlers quickly turned their small, cool planet into the
centre of activity in Orionis. The system is peaceful, and federal
control is thus light. A volunteer Astroguard squadron operates out of
the Kryo spaceport, and a federal ranger brigade bivouacs in New Oslo
(with a battalion detached to Titus). If necessary, federal naval and
trooper forces will hyper-jump in from sector headquarters.

Aside from the information on the System Log, no details are provided
for Acarpous, a huge, hostile world. If the GM wishes to use the
planet or its moons for an adventure, he should complete the
procedures outlined in \ref{GM-sec:geographical-features} and
\ref{GM-sec:population-technology} of World Generation in the GM
Guide.


\subsection[Titus]{Titus is a large, humid planet with 20,000 settlers
  living 
  on its scattered islands.}
\label{sec:titus-large-humid}

When colonization of the Orionis system began, Titus was considered
the prime candidate for system supremacy. However, the lack of high
value resources and its restrictive escape velocity retarded
development; Kryo became the centre of Orionis commerce. Titus has a
spectacular set of rings and, orbiting further out, a small moon
(\textbf{1HP}, not shown). A small spaceport is located in Orion, the
planet's capital and largest town, with limited freight service to
Kryo, but no interstellar facilities. The colonists generally live in
a manner similar to life in 1900 --- some industrialization combined
with heavy agriculture. The major resource of Titus is its delicious
high-protein fruits, found in great abundance throughout the temperate
climates.  Demand for this fruit keeps merchants coming to Orion to
take the delicacies to other worlds (where they are sold at exorbitant
prices).  Surface shipping is the main form of transportation on the
calm seas of Titus, and most of the settlements hug the coastlines.

The following environs of Titus are of special interest. Titusian
fruits exist in each environ with edible plants.

\begin{description}
\item[n02] Edible plants
\item[n03] 2,000 people, iron, edible plants
\item[n06] Edible plants
\item[n09] 9,000 people (including 6,000 in Orion and a federal ranger
  battalion), edible plants, caesium, iron
\item[n12] 3,000 people, edible plants, copper
\item[n18] Iron
\item[n20] Helium
\item[n23] 1,000 people, copper
\item[s02] Edible plants
\item[s04] Chlorine
\item[s07] Edible plants
\item[s08] Edible plants
\item[s10] Edible plants
\item[s11] Edible plants, 5,000 people
\end{description}


\begin{figure}[htbp]
  \centering
  \fbox{%
    \begin{minipage}{0.9\textwidth}
      \centering
      \caption{Titus world log}
      \label{fig:titus}
      
      \medskip
      
      \includegraphics[width=0.85\textwidth]{titus}
  \end{minipage}}
\end{figure}

\subsection[Kyro]{Kryo, the capital world of the Orionis system, is a
  small cool planet 
with 10 million inhabitants.}
\label{sec:kryo-capital-world}

A coalition of Scandinavian countries began colonization of Kryo soon
after Titus was settled. Taking advantage of the planet's slight
gravity and its abundance of natural resources, the colonists quickly
set up a viable industrial base. Kryo's major exports include
precision machine tools, tech kits, light weaponry, and high-quality
optical equipment. Kryo is prevented from full fi nancial independence
by its lack of resources necessary to manufacture armor, spaceships,
medicines, and related goods. These things and such basics as wood and
fertilizers must be imported. Kryo's orbiting spaceport provides
freight service and limited passenger service to and from Sol and a
few nearby systems. The installation is small, supporting an
Astroguard squadron, light repair facilities, and a combined
Kryo/Federal staff of 60. A shuttle service flies between the
spaceport and New Oslo twice daily.

The environs of Kryo contain the following: 

\begin{description}
\item[n01] Iron, argon
\item[n02] 1 million people, edible plants, light fibre plants, carbon
  chemicals
\item[n03] 7 million people (including 1 million in New Oslo), federal
  ranger brigade, carbon chemicals, edible plants, light fibre plants,
  iron, phosphorus
\item[n04] 200,000 people, aluminium, sulphur
\item[n05] Iron, aluminium, carbon chemicals
\item[s01] Iron
\item[s02] 500,000 people, nitrogen chemicals, carbon chemicals,
  edible plants, iron, silver
\item[s03] Iron, aluminium 
\item[s04] 1.3 million people, nitrogen chemicals, edible plants,
  carbon chemicals, iron
\item[s05] 1,000 people, nitrogen chemicals, carbon chemicals, edible plants, iron 
\end{description}


\subsection[Laidley]{Laidley is an unsettled planet on which most of the
  adventure occurs.}
\label{sec:laidley-an-unsettled}

As described in \ref{sec:gm-should-read}, Laidley recently hosted an
unsuccessful exploration mission of about 100 explorers and pioneers
from Kryo. One side of the planet always faces Orionis; thus,
temperatures vary on the planet much more than on most worlds (see
below). Although Laidley has no appreciable surface water, the warmer
environs show traces of moisture. A network of small streams has been
observed in a lightly vegetated area (Environ 7).

The environs of Laidley have the following average temperatures and
resources:

\begin{description}
\item[n01] 0\textdegree 
\item[n02] 25\textdegree, nitrogen chemicals 
\item[n03] --25\textdegree, radioactives 
\item[n04] --25\textdegree, silver 
\item[n05] 25\textdegree, nitrogen chemicals, silver 
\item[n06] 75\textdegree, nitrogen chemicals, silver, phosphorus 
\item[n07] 50\textdegree, nitrogen chemicals, edible plants, adamantine 
\item[n08] 0\textdegree, caesium 
\item[n09] --50\textdegree, radioactives 
\item[n10] --75\textdegree, ammonia 
\item[n11] --50\textdegree, mercury 
\item[n12] 0\textdegree, silver 
\item[n13] 50\textdegree, nitrogen chemicals, edible plants 
\item[s01] 0\textdegree, germanium 
\item[s02] 25\textdegree, silver, germanium 
\item[s03] --25\textdegree, germanium 
\item[s04] --25\textdegree, phosphorus, germanium 
\item[s05] 50\textdegree, silver, nitrogen, phosphorus 
\end{description}

\textbf{Note:} Many of these resources may be undiscovered. If the GM
wishes, he may conceal their presence from the players. These
resources have little effect on the enclosed adventure, but the GM may
come up with a geological expedition to Laidley to hunt for
undiscovered resources (radioactives, silver, and adamantine would
certainly be finds).


\section{The Players and the Characters}
\label{sec:players-characters}
% [52.0]

Once the GM has assembled the players who are going to participate in
this adventure, they should decide whether or not to generate their
own characters. Lost on Laidley is designed as a learning experience
for all concerned, and it might be better if each player is free to
experiment with the game systems without the fear that a mistake would
cause his character's death. To facilitate this, five pre-generated
characters are detailed in \ref{sec:char-1:-expl} through
\ref{sec:character-5:-ex}.  These characters are varied in their
characteristics and skills to allow many game systems to be used
during the adventure. They are also designed to mesh well with the
mission as it stands. These characters may only be used until the end
of the adventure on Laidley. After that the GM should have players
generate their own.

If there are five players, the GM assigns each player one character or
allows them to be chosen randomly. With fewer than five players,
certain players may play more than one character or the GM may play
any which are left over. It is not normally recommended that a player
control more than one character, but for this adventure the GM may
allow it. If the GM decides not to use all five characters, they
should be chosen by priority. Each character is listed in order of his
importance to the successful completion of the mission. Character \#1
is the most important, character \#5 the least. The players should be
informed of this information only if they question the selection of
characters; otherwise they may receive clues as to the proper way to
resolve the adventure they might not have thought of themselves.

If the players wish to generate their own characters, the GM should
allow it, reminding the players of the rationale discussed above. To
aid in the successful completion of the mission, the GM may assign one
or two of the pre-generated characters to aid the fledgling party.
Characters \#1 or \#2 (or both) would be ideal choices. As before, the
players could run these additional characters, or the GM may run them
as NPC's.

Once the players have their characters, the GM should examine their
Character Records for completeness and accuracy.  If the players are
using the pre-generated characters, the players should give their
character a name.

\textbf{Note:} The Environ Skill Level is listed for each character's
Home Environ only; all other environ Skill Levels should be
extrapolated as explained in Case 5.4.

\subsection[Player Background]{The GM should read or paraphrase the
  following background 
  of the adventure for the characters.}
\label{sec:gm-should-read}

This is the story of the ill-fated expedition that Darmath Svenson,
the adventure's sponsor, will present to the characters as he heard it
from the expedition's commander. The characters have been hired by
Svenson to go on this mission. If the players are using the
pre-generated characters, it was through their explorer character that
the initial contact was made with Darmath's secretary. If the players
are using their own characters, the GM should invent the connection.
The characters are escorted into Darmath's personal office, located
somewhere in downtown New Oslo. It is the office of a man who is
president of his own trading firm, a firm which is doing very well
financially. Darmath is a dashing middle-aged man in good shape. For a
complete description, see \ref{sec:darm-svens-hires}. He will ask the
characters to make themselves comfortable and will then relate this
tale:

One year ago, a scientific expedition set out from New Oslo for
Laidley (see \ref{sec:orionis-system} for information concerning the
system, which should be shown to the players). The expedition was
concerned with exploration and mapping of that world in preparation
for possible colonization. They landed without incident, set up
Laidley Base 1, and then proceeded to investigate the area (Environ 13
of the World Log).  Laidley proved very inhospitable, having a poison
atmosphere and almost no water. A great deal of the expedition's time
was spent synthesizing oxygen and water. Some readings seemed to
indicate there was some water underground. One favourable attribute
was the plant life; not only was it edible, but very flavourful as
well. Another interesting item was a strange creature which the
expedition called an Auroch. It was a docile herd animal, vaguely
swine-like, which roamed the world. The Aurochs never threatened the
expedition, being herbivores and appearing very stupid. Various
sighting of other creature types excited zookeepers on Kryo, and a
biological survey mission had been planned for early next year. These
detailed creature reports had been sent back by the expedition's
biologist, Mordecala Svenson, Darmath Svenson's sister.  

After eight months of painstaking exploration and examination, the
expedition was ready to give up on Laidley. There was not enough water
to justify active colonization.  Then one day scanning revealed traces
of surface water 8,000 km west of Laidley Base 1.  Mordecala was
chosen to lead a small scouting party to the site to investigate. They
chose a Floater to avoid the volcanic range to the west.

This scouting party consisted of 10 expedition members, all of whom
disappeared.

Their journey took 200 consecutive hours of flying. The trip was
uneventful until the end. (At this point, Svenson will play a tape of
the scouting party's last radio transmission. They had been having
trouble with the radio's transmitter and the message is garbled as a
result.)

\begin{quote}
  \emph{``\ldots having a\ldots with monopole electrical
    sy\ldots.possibly\ldots land the Floater
    here\ldots.water!\ldots.oh, no\ldots.Cord, help steady
    the\ldots.going to land\ldots the lake\ldots---''}
\end{quote}

The voice was a female, and Darmath will tell the characters the voice
was Mordecala's.

Three days after the last transmission, the volcanic range became
active, and the expedition's safety was in jeopardy. Because the world
showed so little promise, the expedition commander (Svenson here
indicates extreme disgust with the choice) chose to leave the world.
Knowing the fate of the scouting party was in doubt (but fearing the
worst) the commander ordered a fly-by of the area from where the
party's last transmission was received. Upon seeing no trace of them,
not even the wreckage of the Floater, the commander ordered a state
funeral for the 10 and returned to Kryo. Questioned upon his return,
the commander admitted that, because no trace at all had been spotted,
there was doubt as to the fate of the scouting party. The funerals
were cancelled. All investigators admitted the curious fact was the
absence of any metal debris, but most concluded the party's Floater
had sunk into the lake, which was mentioned.

Svenson believes there were no remains because the party had landed
the Floater, repaired it, and proceeded to return to Laidley Base 1.
Their radio was inoperative, so they could not have radioed in, and
the expedition left without them. At least, this is the hope he is
clinging to. The accident occurred four weeks ago.

Svenson will supply the party with all the necessary equipment and
transportation to Laidley. He will drop them off on the world at
Laidley Base 1 on his way out-system to complete a business deal. His
ship, the \emph{Star Vision} (Corco \emph{Iota} class), will carry the
characters, a crawler, and plenty of equipment to Laidley, and the
ship's Lander will provide transportation to the surface. If the
missing expedition has not returned to the base, the characters will
have 18 days to complete their mission, for then Svenson will be
returning to Kryo, and will stop at Laidley to pick them up. The
characters will be able to contact the ship from anywhere on Laidley
via their vehicle's radio and they will be picked up wherever they are
at the moment.

Their mission is to first find out what happened to Mordecala. Svenson
wishes concrete proof of one form or another; he wants to be sure of
what happened to her. Second, any unusual creatures the characters can
capture and bring back with them they will be paid for (Svenson is not
interested in the Aurochs).

The characters are offered 5 Trans apiece for completing the mission,
and double that if any interesting creatures are captured.  Svenson
will pay triple that if concrete evidence of Mordecala's fate is
brought back.

Svenson will gladly answer any questions the characters present to
him, then he will urge them to prepare themselves for the mission, as
he wishes to leave tomorrow.


\subsection{Character \#1: Explorer}
\label{sec:char-1:-expl}

\textbf{ST:} 5 \textbf{EN:} 4 \textbf{DX:} 4 \textbf{AY:} 5
\textbf{IN:} 7 \textbf{MP:} 3 \textbf{LD:} 7 \textbf{EM:} 4
\textbf{AG:} 8 \textbf{SS:} Family fallen on hard times \\
\textbf{Age:} 28 \textbf{Money:} 600 Mils \\
\textbf{Skills:} Urban \textbf{1}; Grav \textbf{NW (-1)}, \textbf{LT
  (2)}, \textbf{HY (-1)}, \textbf{EX (-3)}; Temp \textbf{NL}; Environ
\textbf{MN/LV (4)}; EVA \textbf{2}; Handguns \textbf{2}; Pilot
\textbf{2}; Survival \textbf{2};
Ground Vehicles \textbf{ATV 3}; Biology \textbf{1}; Geology
\textbf{1}; Streetwise \textbf{1}. \\
\textbf{Possessions:} None except clothing.


\subsection{Character \#2: Scientist}
\label{sec:char-2:-scient}

\textbf{ST:} 4 \textbf{EN:} 5 \textbf{DX:} 5 \textbf{AY:} 3 \textbf{IN:} 9 \textbf{MP:} 2 \textbf{LD:} 3 \textbf{EM:} 4 \textbf{AG:} 3 \textbf{SS:} Skilled Tech Family. \\
\textbf{Age:} 32 \textbf{Money:} 10 Trans, 500 Mils \\
\textbf{Skills:} Urban \textbf{0}; Grav \textbf{NW (1)}, Lt\textbf{ (-1)}, \textbf{HY (-3)}, \textbf{EX (-5)}; Temp \textbf{CD}; Environ \textbf{PK/WD (3)}; Chemistry \textbf{3}; Planetology \textbf{3}; Biology \textbf{3}; 
Geology \textbf{1}; Laser/Stun Pistol \textbf{1}; Ground Vehicles
\textbf{Truck 1}; Vehicle Tech \textbf{1}; Streetwise \textbf{1}. \\
\textbf{Possessions:} Civ Level 6 Chemsynthesizer, clothing. 


\subsection{Character \#3: Space Trooper}
\label{sec:character-3:-space}

\textbf{ST:} 6 \textbf{EN:} 6 \textbf{DX:} 7 \textbf{AY:} 6 \textbf{IN:} 4 \textbf{MP:} 2 \textbf{LD:} 6 \textbf{EM:} 3 \textbf{AG:} 8 \textbf{SS:} Lower Middle Class. \\
\textbf{Age:} 32 \textbf{Money:} 3 Trans, 600 Mils. \\
\textbf{Skills:} Urban \textbf{1}; Grav \textbf{NW (-3)}, \textbf{LT
  (0)}, \textbf{HY (2)}, \textbf{EX (0)}; Temp \textbf{HO}; Environ \textbf{HL/VO (4)}; Body Armor \textbf{4}; Machine Gun \textbf{2}; Paint Gun \textbf{3}; 
Suit Tech \textbf{1}; Ground Vehicles \textbf{ATV 1}; Laser/Stun Pistol \textbf{3}. \\
\textbf{Possessions:} Reflect/impact armor (Civ 
Level 7), clothing. 


\subsection{Character \#4: Doctor}
\label{sec:character-4:-doctor}

\textbf{ST:} 4 \textbf{EN:} 4 \textbf{DX:} 8 \textbf{AY:} 4 \textbf{IN:} 9 \textbf{MP:} 2 \textbf{LD:} 2 \textbf{EM:} 6 \textbf{AG:} 4 \textbf{SS:} Local Establishment. \\
\textbf{Age:} 36 \textbf{Money:} 4 Trans, 600 Mils. \\
\textbf{Skills:} Urban \textbf{2}; Grav \textbf{NW (-1)}, \textbf{LT (1)}, \textbf{HY (-1)}, \textbf{EX (-3)}; Temp \textbf{NL}; Environ \textbf{FL/JU (2)}; Teaching \textbf{3}; Diagnostics \textbf{4}; Treatment \textbf{3}; 
Ground Vehicles Automobiles \textbf{1}; Laser/ Stun Pistol \textbf{1}; Electro Tech \textbf{2}; Physics \textbf{1}. \\
\textbf{Possessions:} Mediscanner (Civ Level 6), clothing. 


\subsection{Character \#5: Ex-Scout}
\label{sec:character-5:-ex}

\textbf{ST:} 4 \textbf{EN:} 3 \textbf{DX:} 4 \textbf{AY:} 4 \textbf{IN:} 6 \textbf{MP:} 2 \textbf{LD:} 3 \textbf{EM:} 7 \textbf{AG:} 5 \textbf{SS:} Skilled Tech Family. \\
\textbf{Age:} 24 \textbf{Money:} 1 Tran, 300 Mils. \\
\textbf{Skills:} Urban \textbf{0}; Grav \textbf{NW (1)}, \textbf{LT
  (-1)}, \textbf{HY (-3)}, \textbf{EX (-5)}; Temp \textbf{HO}; Environ \textbf{PK/VO (2)}; EVA \textbf{2}; Handguns \textbf{2}; Pilot \textbf{2}; Survival \textbf{2}; 
Air Vehicles \textbf{Glider 1}, \textbf{Direct Lift 1}; Treatment \textbf{1}; Biology \textbf{1}; Suit Tech \textbf{2}. \\
\textbf{Possessions:} None except clothing. 


\subsection[Equipment]{Darmath Svenson will provide the characters
  with the 
  following equipment for their use on Laidley.}
\label{sec:darmath-svenson-will}

{\large Crawler} (ATV). See GM guide \ref{GM-sec:land-vehicles} and
the Land Vehicle Chart for all specifications on this vehicle. The
crawler has an oxygen synthesis system programmed to work in the
poison atmosphere of Laidley. Four weeks of food and water for five
are aboard. The rear of the vehicle is sealed off from the passenger
compartment to provide 10 cubic meters of space to store creatures in
their own atmosphere.  An air lock large enough for one man separates
the two compartments. Built into the crawler are a planetary Shortwave
and a Neuroscanner.

The following items are loaded into the crawler: six Civ Level 7
expedition suits (character \#3 will probably want to use his armor
instead); one Civ Level 8 Bioscanner; one Civ Level 8 chemsynthesizer;
one first aid kit; one energy scanner; one basic repair kit; one suit
kit; one Civ Level 8 Electrokit; one Civ Level 6 vehicle kit; four
force cages; one super cage (eight cubic meters); two metal cages (one
cubic meter each); three Civ Level 8 laser pistols; two Civ Level 5
pistols; two Civ Level 8 stun pistols; one Civ Level 6 paint gun; one
Civ Level 8 needle pistol; and binoculars. The characters may also
bring along any equipment they already own. If they wish to leave any
behind, Svenson's associates in New Oslo will watch it.

\section{The Gamemaster's Background}
\label{sec:gm-background}
% [53.0]

The situation as Darmath Svenson described it to the characters is
accurate as far as it goes. His suppositions concerning the fate of
the missing expedition are erroneous, however. Nothing of the fate of
the missing explorers is known to anyone but its five surviving
members, and the players should not be able to find out any of it
unless they find the explorers.

The description of Svenson is a guide as to how his part should be
played by the GM. The final interpretation of his personality is up to
the GM, as he is the one who will be "acting" the part of Svenson for
the player's benefit.


\subsection*{The Fate of the Missing Explorers}
\label{sec:fate-miss-expl}

The floater carrying Mordecala Svenson, Cord Tenon, and their eight
companions suffered electrical problems as it flew over the jagged
peaks of the Kishtu Heights. After their radio failed, they spotted
the lake (hex 1622 of the environ map) and chose to land the craft on
its surface. Immediately after the floater settled on the lake, acidic
agents in the water began corroding the metallic underside of the
craft. The explorers threw a large life raft in the water, correctly
assuming the acid would not eat through its synthetic construction.
Wearing expedition suits and carrying what equipment they could, the
10 made for the shore 500 meters away. Looking back, they saw a
transparent slug-like thing (\emph{the} slug in mutant form, see
\ref{sec:slug-dang-creat}) drag their dissolving floater beneath the
surface of the rolling lake.

Mordecala lead the party away from the deadly lake northeast to a cave
in hex 1522. There Cord set up their chemsynthesizer to purify water
from a nearby stream and to synthesize oxygen from the atmosphere.
However, it soon became clear that the initial supply of oxygen in
their suits and the amount the synthesizer could produce would not be
enough to support more than five people after five days. They were
saved from a grisly decision when the four days were nearly up by the
horrifying arrival of the slug in its new incarnation. Three explorers
died right away during the slug's attack and two others died due to
exposure to the poison atmosphere when their suits were punctured in
combat. Mordecala discovered that beam weapons hurt the creature
terribly and thus was able to drive it away. Rather than chase the
slug, Cord, Mordecala and the other three survivors tended to their
fallen comrades and then moved their camp to a large clearing with a
good field of vision.

The five remained there for days, exerting themselves as little as
possible so as to conserve oxygen. There was plenty of purified water
and they knew that some of the plants around them were edible. The
slug returned a few times but was easily driven off with beam and stun
weapons. Ten days after the crash, they were finally able to repair
their radio and attempted to contact Laidley Base.  Unbeknownst to the
survivors, the Laidley expedition had already departed (they had been
in the cave during the flyover). Totally unaware of the departure
plans, Mordecala and Cord could only assume that their radio had an
undetectable transmission problem.

Unless the characters are delayed, they will touch down at Laidley
Base 30 days after the crash of the floater. The survivors are totally
bewildered by the lack of a rescue attempt, but refuse to give up
hope. Occasionally they will shoot off a flare or transmit with their
seemingly defective radio. The slug has not bothered them for the past
15 days.



\subsection[Darmath Svenson]{Darmath Svenson hires the characters to
  undertake the 
  adventure while in New Oslo on the planet Kryo.}
\label{sec:darm-svens-hires}

A wealthy interstellar trader (dealing in high quality optics),
Darmath Svenson had very little to worry him. He had cut quite a swath
through the society of New Oslo. Having inherited the business from
his entrepreneur father, his own acute trading instincts had set him
up for life financially, and his taste for sporty ground cars and even
sportier women was legendary. His exotic whims ranged from various
spices to strange alien creatures. This devil-may-care existence was
shattered with the news his sister Mordecala had been lost on an
expedition to the planet Laidley.

News of her demise was sketchy at best, but Darmath was determined to
discover what had happened. Along with his sincere concern for his
sister was his curiosity regarding rumours of very strange creatures
inhabiting the planet. He instructed his secretary to examine his
company's files for an explorer he could hire for this mission. She
discovered a man who had grown up in an environ similar to the one in
which his sister was lost, and also knew how to operate an ATV. A man
of such qualifications could retrieve any creatures encountered and
also search for any trace of Mordecala. Darmath was not quite sure his
sister was dead (although he feared the worst), and would not truly
accept her fate without concrete proof. His characteristic ratings
are:

\textbf{\large Darmath Svenson}

\textbf{ST:} 5 \textbf{EN:} 6 \textbf{DX:} 3 \textbf{AY:} 3 \textbf{IN:} 8 \textbf{MP:} 2 \textbf{LD:} 6 \textbf{EM:} 4 \textbf{AG:} 2 \textbf{SS:} Independent Trading Family.\\
\textbf{Skills:} Urban \textbf{2}; Gravity \textbf{LT 1}; Temp \textbf{NL}; Environ \textbf{NW (2)}; Pilot \textbf{2}; Economics \textbf{4}; Trading \textbf{5}; Automobile \textbf{3}; Shuttle \textbf{2}; Motorboat \textbf{2}; 
Gambling \textbf{2}; Biology \textbf{1}. 

Darmath is a handsome male of Scandinavian descent; he is middle-aged,
but in very good shape. He does not wish to actually risk his own skin
and instructs the party as such. His little sister Mordecala was/is
very close to him and resembles him slightly (enough so that the
characters would notice).


\subsection[Start Of Adventure]{The adventure begins as the characters
  board a transporter 
  shuttle with Darmath Svenson at the New Oslo terminal to fly to the
  Kryo orbiting spaceport.} 
\label{sec:adventure-begins-as}

The trip to the spaceport takes about an hour. Six hours after the
characters arrive at the spaceport, the \emph{Star Vision} is ready to
depart.  While waiting, the characters will undergo a short courteous
federal inspection (Svenson is known and trusted at the Kryo
spaceport).  The rest of the time is the character's to do with what
they will. The journey from Kryo to Laidley in the \emph{Star Vision}
takes about four days (calculated in accordance with GM guide
\ref{GM-sec:interplanetary-travel}).

The \emph{Star Vision} is a Corco \emph{Iota} Class luxury freighter
with the following pods: light weapon, augmented jump, energy, full
service luxury, crew, Lander, standard cargo, buffered cargo, living
cargo.  All pods are armour Class 1. The ship is fully crewed; the
characters may relax during the voyage.

After going into orbit around Laidley, the characters will be directed
to the ship's Lander pod where they will board the \emph{Spectral
  Dancer}.  \emph{Dancer's} pilot informs the characters that the
craft cannot land in steep, mountainous terrain (anything greater than
Terrain Value 2), so when the characters call for pick-up in 18 days,
they should be in an accessible area. The Lander will depart the
\emph{Star Vision} and deposit the party, and all their equipment
about 100 meters from the remains of Laidley Base 1. No sign of the
missing expedition will be found at all. The characters are on their
own.

During the transit from New Oslo to Laidley Base 1, the GM should
check for encounters in accordance with GM guide
\ref{GM-sec:encounters}, using the appropriate column of the Encounter
Table. It is possible that an encounter on the way to Laidley will
delay or alter the conduct of the mission.


\section{Laidley}
\label{sec:laidley}
% [54.0]

Once the characters have landed at Laidley Base and the Lander has
departed, they have entered into the heart of the adventure. The GM
should read this Section carefully, to familiarize himself with the
overall situation on Laidley so that he may properly interact with the
characters as they seek their goals.

Environ 13 (containing Laidley Base) and the environs lying between it
and the survivors' environ do not have environ hex maps. The
characters will probably plot a route of travel between the base and
the lake area in Environ 7. A likely route will take them through
2,000 km of Environ 13, 2,000 km of Environ 5, and 2,000km of Environ
2, from which they would enter Environ 7 in hex 0130 to 0135.  Of
course, the GM should not suggest this route, but should keep in mind
that if the party is not travelling the entire length or width of an
environ, he should estimate the distance they traverse.

Laidley Base is a group of simple huts, which may be sealed, from the
poison atmosphere. Some heavy equipment has been left behind,
including an operable oxygen synthesizer that will provide for at
least 30 people. The party will find nothing of use that may be
carried in the crawler (the synthesizer is the size of a small hut).


\subsection[Environ 7 Hex Map]{A hex map of Environ 7 on Laidley is
  provided for use by 
  the GM and the players.}
\label{sec:hex-map-environ}

The GM should show the players this map before starting the adventure.
It is assumed all the information on the map was gathered during an
orbital mapping flight well before the character's mission.

Environ 7 is a light vegetation/mountainous environ, and most of the
map shows features of this type. Some variation in the form of peaks,
hills, flat areas, barren areas, and woods are also present. Some
volcanic areas exist on the eas t and south edges (adjacent to
environs 6 and 18). The mass of peaks in the middle of the environ
form the \emph{Kishtu Heights}, a series of canyons and steep
mountains.  Winding through the Kishtu is a network of small streams
and dry stream beds that lead to the lake in which the missing
explorers lost their floater (hex 1622). The lake has a diameter of
about 40 km. The GM should remember that the terrain and contour of
any hex the party currently occupies is considered their environ for
purposes of encounters and movement, regardless of the overall nature
of Environ 7.


\subsection[Survivors]{The five surviving members of the missing
  expedition are encamped in a clearing in hex 1522 of the  
  Environ 7 hex map.}
\label{sec:five-surv-memb}

Two of the survivors are Mordecala Svenson (Darmath Svenson's sister),
and Cord Tenon, the original mission's chemist. Mordecala was the
biologist, and her skill and knowledge saved the remaining members
from the slug during the attack in which it killed the five others.
Both Mordecala and Cord are residents of New Oslo, and knew each other
before this ill-fated mission. They have become very close due to the
fear and tension, and will try and protect each other in any pressure
situation. Cord's skill as a chemist has kept the group alive by
synthesizing water and oxygen with his lab. Mordecala knows all the
information concerning the slug as listed for the GM except the
tunnelling ability. As far as they are aware, the thing is still in
the lake.

\textbf{Mordecala's} characteristic ratings are: \\
\textbf{ST:} 3 \textbf{EN:} 4 \textbf{DX:} 4 \textbf{AY:} 4
\textbf{IN:} 7 \textbf{MP:} 2 \textbf{LD:} 7 \textbf{EM:} 6
\textbf{AG:} 7 \textbf{SS:} Independent Trading Family. \\ 
\textbf{Skills:} Urban \textbf{1}; Grav \textbf{LT 1}; Temp
\textbf{NL}; Environ \textbf{NW (3)}; Laser/Stun Pistol \textbf{2};
Direct Lift \textbf{1}; Programming 
\textbf{1}; Biology \textbf{6}; Geology \textbf{1}.

\textbf{Cord's} characteristic ratings are: \\
\textbf{ST:} 4 \textbf{EN:} 5 \textbf{DX:} 4 \textbf{AY:} 4
\textbf{IN:} 6 \textbf{MP:} 3 \textbf{LD:} 5 \textbf{EM:} 5
\textbf{AG:} 5 \textbf{SS:} Skilled Tech Family. \\ 
\textbf{Skills:} Urban \textbf{1}; Grav \textbf{LT 1}; Temp
\textbf{NL}; Environ \textbf{NW (2)}; Laser/Stun Pistol \textbf{1};
Handguns \textbf{2}; Pilot \textbf{1}; ATV \textbf{1}; Chemistry
\textbf{6}; Programming \textbf{1};  
Electro Tech \textbf{1}. 

The possessions for the group as a whole: planetary Shortwave radio;
basic repair kit; expedition suits; laser pistols (Civ Level 8); stun
pistols (Civ Level 7); pistols (Civ Level 5); Chemsynthesizer (Civ
Level 8); flare gun and 1 dozen rocket flares, life raft, sundry minor
miscellaneous items (knives, personal effects).

The three other remaining members should be named and personalized by
the GM, remembering that if any die (as described in
\ref{sec:laidley-encounters}), these will be the first to go. Their
characteristic ratings are:

\textbf{ST:} 3 \textbf{EN:} 3 \textbf{DX:} 3 \textbf{AY:} 5 \textbf{IN:} 5 \textbf{MP:} 1 \textbf{LD:} 4 \textbf{EM:} 2 \textbf{AG:} 2 \textbf{SS:} Skilled Tech Family. \\
\textbf{Skills:} Urban \textbf{0}; Grav \textbf{LT 1}; Temp
\textbf{NL}; Environ \textbf{NW (2)}; Laser/Stun Pistol \textbf{1};
Handguns \textbf{1}; Direct Lift \textbf{1}; Biology \textbf{1};
Geology \textbf{1}; Any skills  
the GM sees fit \textbf{1}. 


\subsection[The Slug]{The slug is a dangerous creature that wanders
  the Kishtu
  Heights and surrounding area in Environ 7.}
\label{sec:slug-dang-creat}

The slug is encountered in accordance with
\ref{sec:laidley-encounters} and \ref{sec:foll-events-occur}.

\textbf{Size:} Two Hex (large in first hex, one hex in second hex). \\
\textbf{Combat:} 14 \textbf{AY:} 4 (12 when in water) \textbf{AG:} 11
\textbf{IN:} No \textbf{Initiative:} 10\% \\ 
\textbf{Composition:} Carbon-based \\
\textbf{Powers:} Heightened smelling, regeneration and kinetic absorption. 

\textbf{Dart shooting:} May shoot one metal spike from above its mouth
as an attack each Action Round. Chance to hit man is 80\% minus 10  
percentage points for each hex distant. Hit Strength and range as in
power description. \\
\textbf{Acid:} Teeth and mouth secrete acid for dissolving metal. Acid
will not harm expedition suit. Reduce effectiveness of character's
armour  
by one when attacked by slug. 
\textbf{Tunnelling:} Will always have a tunnel
entrance within 10 hexes each time encountered. Agility Rating  
doubled when in tunnel. See notes below for further information. 

\begin{itemize}
\item \textbf{Warning:} A glistening trail of rusty powder leads
  towards a large quivering shape in the distance; or slurping sounds
  are heard within the entrance to a tunnel encrusted with red powder.
\item \textbf{Sight:} Eight-meter long slug-like creature covered with
  a sticky, rust-coloured powder constantly dropping off its body and
  forming a trail behind it. Blue-green splotches are visible beneath
  the coating. Large cog-like teeth at its front end seem to make up
  its mouth.
\item \textbf{Perception:} The slug's teeth drip an organic acid
  capable of dissolving metals. The acid neutralizes after a few
  minutes in the open.  The creature's coating and trail are its
  perspiration and excrement, the metallic residue of its digestive
  process. A small orifice directly above its mouth looks as if it
  emits matter rather than taking it in.
\item \textbf{Examination:} The slug lives on the metallic elements it
  ingests. Some of its intake is converted into small adamantine
  spikes that the creature shoots to break rock as it tunnels,
  searching for new concentrations of food. The acidic water of the
  lake and the streams running into the lake are the source of the
  creature's power. If the creature is hurt and returns to its water
  before dying, it regenerates immediately (it also regenerates when
  out of water, at a slower rate). The creature dislikes plastics and
  other non-metallic elements and may be easily restrained in two
  force cages or a sturdy, non-metallic container (if large enough).
\end{itemize}


\subsection*{History and Additional Information:}
\label{sec:hist-addit-inform}

Prior to the arrival of the stranded explorers, the slug existed in
another form. It floated through a lake (hex 1622 of the environ hex
map) like a jellyfish, ingesting metallic elements brought into the
lake by many small streams. The slug was very similar to the lake it
festered in; they both dissolved and ingested metals using powerful
acids. The creature however, had a distinct form and a mind. When the
explorer's floater settled on the l ake and started to corrode, the
slug went wild; never before had it received so much concentrated food
at one time. Along with the floater's hull, the slug took in the
craft's source of power, a plate of magnetic monopoles. This caused
violent reactions and mutations in the creature, which almost killed
it. The creature did not die, however; after three days it slithered
ashore in a new amphibian form with abilities that enabled it to
travel and tunnel in its craggy environment.

Though aggressive, the creature likes to live and will flee if it is
being hurt. The creature will always head for water or, if no water is
nearby, to a tunnel entrance.

If the characters encounter the slug when they have the crawler, the
slug will be much more interested in attacking the vehicle than the
characters. If the characters leave the crawler behind when an
encounter with the slug occurs, the GM may have the creature destroy
the crawler, unbeknownst to the characters.


\subsection[Laidley Encounters]{While the characters are on Laidley,
  the GM uses the ``100 Thousand or less'' column of the Encounter  
  Table for all encounter checks.}
\label{sec:laidley-encounters}
% [54.4]

The procedure in GM guide \ref{GM-sec:encounters} is used with the
following modifications: 

\begin{description}
\item[Common NPC:] No NPC encounter. The survivors are transmitting
  with their radio. If the characters are in the crawler and its radio
  is on, they receive the message and may respond. The survivors can
  give the characters exact co-ordinates of their location.
\item[Rare NPC:] Treat as no encounter. One of the five survivors
  dies.
\item[Unique NPC:] Choose a unique NPC from \ref{sec:unique-npcs}.
\item[Common Creature:] Always treat as an encounter with a herd of
  \emph{Aurochs} (creature \#\ref{sec:creatures-1} in
  \ref{sec:creatures}). Roll 
  percentile dice to determine how many are present.
\item[Rare Creature:] Choose an appropriate creature from
  \ref{sec:creatures}.  \textbf{Exception:} If the party is within
  sight of a stream on the Environ 7 hex map, they encounter the slug.
\item[Unique Creature:] Choose an appropriate creature from
  \ref{sec:creatures}. \textbf{Exception:} If the party is within 200
  km of a stream on the Environ 7 hex map, they encounter the slug.
\item[Common, rare or unique accident:] Choose an appropriate accident
  from \ref{sec:accidents}.
\item[No event:] Locate the dice result among the following list of
  events:
  \begin{description}
  \item[82--88.] If the party is within 300 km of a stream on the
    environ 7 hex map, they come across a trail of rust-like powder
    about three meters wide (the slug's path). Following the path in
    one direction will lead the party to the nearest stream; the other
    direction will lead them to a tunnel entrance (roll percentile
    dice for the number of kilometres distant).
  \item[89--96.] The survivors shoot off a flare. If the characters
    are in hex 1522, they may follow the flare's trail to the
    survivor's camp. If the characters are within two hexes of 1522,
    they see a reddish glow on the horizon in the general direction of
    1522. The GM should keep track of flares fired by the survivors.
    When 12 have been used, no more are available.
  \item[97--98.] The survivors have turned their radio on for
    \textbf{1} hour to listen for transmissions. If the characters
    transmit with the crawler's radio during this time, contact is
    established. The survivors can give the characters exact
    coordinates of their location.
  \item[99.] The survivors have turned their radio on for \textbf{2}
    hours to listen for transmissions. See above.
  \item[100.] The survivors have turned their radio on for \textbf{4}
    hours to listen for transmissions. See above.
  \end{description}
\end{description}

\subsection[Automatic Events]{The following events occur automatically
  if the appropriate conditions are met.}
\label{sec:foll-events-occur}

\begin{enumerate}
\item Each time the party comes to a stream hex side, the GM rolls two
  dice and subtracts the hex side's distance (in hexes) from the lake
  (1622) from the dice result. Locate the modified result in the
  following list:

  \begin{description}
  \item[0 or less:] The stream is dry. 
  \item[1--4:] The stream is a trickle of water or puddles no more
    than 1/2-meter wide and 10 cm deep. The water is drinkable if
    boiled.
  \item[5--7:] The stream is a flow of water about two meters wide and
    1/2-meter deep. Easily traversable. The water is drinkable if
    purified (a simple synthesis task; refer to the chemistry skill).
  \item[8--10:] The stream is a rushing brook with steep banks.
    Dangerous to cross on foot, no problem for crawler. The water is
    slightly acidic but may be made drinkable with heavy purification
    (advanced synthesis task; see chemistry skill).
  \item[11 or more:] The stream has cut a deep wide gorge and is
    totally impassable. The water is very acidic and may not be made
    drinkable.  The GM may wish to check for damage to any metallic
    equipment that comes in contact with the water.
  \end{description}
  
  \textbf{Exception:} A stream hex side may not have less water than a
  part of the same stream fork further away from the lake. For
  example, if a stream hex side of 0716 has a trickle of water, a
  stream hex side of 0918 may not be dry if encountered by the
  characters later on.
  
\item If the party comes to the shore of the lake, and the slug has
  not yet been killed, they automatically encounter the creature here.
  The water in the lake has the same properties as that in an
  impassable stream, (see preceding).
  
\item If the party travels along the stream separating hex 1522 from
  1623, they will encounter the survivors 200 meters from the north
  side of the stream (in 1522).
  
\item Exactly 18 days after the party is dropped off at Laidley Base,
  Darmath Svenson's spaceship will be in orbit around Laidley
  attempting to contact them. Radio contact may be established with
  the crawler's radio or with the survivor's radio. Once contact is
  achieved, the GM rolls one die to determine how many hours will pass
  before the shuttle touches down to pick them up. \textbf{Note:} The
  shuttle may not land in an area with a Terrain Value above 2.
\end{enumerate}

\section{Adventure Afterword}
\label{sec:adventure-afterword}
% [55.0]

\emph{Lost on Laidley} was designed to be a teaching adventure and an
introduction to \emph{Universe} for both the players and the GM. The
GM must keep this in mind while running the adventure, and freely
discuss any problems, which he or the players might have.

The information presented in the adventure cannot cover all possible
contingencies, and the GM must be prepared to flesh these areas out.
If any of the players own a copy of \emph{Universe}, it would be wise
to insist they do not read Lost on Laidley, as it would greatly
detract from their enjoyment of the adventure. In this regard, the GM
may want to alter a few pertinent details to prevent any player from
becoming overconfident.

The Orionis system holds many interesting adventures in addition to
the one presented. Perhaps the biological survey mission to Kryo could
be played, for instance. The GM should use this star system to its
fullest extent, thus giving himself time to set up adventures of his
own and become confident in his abilities as a GM.

Once the campaign has begun, the GM should feel free to alter any
details mentioned concerning the settlement and colonization of the
Orionis system to be consistent with his own ideas.


%%% Local Variables: 
%%% mode: latex
%%% TeX-master: "adventure_guide"
%%% End: 
