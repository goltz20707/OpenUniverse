%%
%% $Id: worldgen.tex,v 1.2 2004/11/27 13:42:12 goltz20707 Exp $
%%
\externaldocument[AG-]{adventure_guide}

\chapter{World Generation}
\label{cha:world-generation}

%% ZZYZZ checked (except tables) to here Sun -09-26 1710

Once the GM has introduced the players to the game with the enclosed
adventure, he must create other star systems. The world generation
system is designed to provide the GM with only as much information
about a world as he wants, and the process is accordingly punctuated
with convenient stopping places.

If, while creating a world, the GM decides to finish the details
himself without rolling dice, he is encouraged to do so. This chapter
is intended as a guide for the creation of a world; the GM need not
let the dice determine his universe. Logic and creativity, in
proportions, should be injected into this system.

Generating each star system may yield planets, with each planet's
size, type, and position determined. Moons may be similarly generated.
At this point, the GM will have a graphic layout of the whole system
and enough information to give the players if they are examining the
system from above or below the ecliptic. It is up to the GM to name
the stars and worlds. The names of the stars listed on the Stellar
Display are sometimes very dry and technical; the GM should feel free
to invent interesting names.

Generating the geographic features of each world yields the
atmosphere, mean temperature, hydrographic percentage, distribution of
land and water, and the specific gravity of each environ. At this
point, the GM will have an actual map of each world to show the
players if they are in orbit around the world.

Generating the population and technology of each world yields the
total population, type of settlement. Law Level, Spaceport Class, Civ
Level, distribution and development of resources, and the distribution
of the population. At this point the GM will have enough information
to referee any adventure involving the world.


\paragraph{GLOSSARY}

The following terms are used throughout this chapter: 

\begin{description}
\item[Binary or Trinary Star.] Two or three stars revolving around a
  common center of gravity; may limit the number of possible planets. 
\item[Biosphere.] The most habitable zone (for humans) of a star system;
affects the type of planet or moon within it. 
\item[Civilization (Civ) Level.] A number from \textbf{1} to
  \textbf{8} measuring a world's economic contribution to the
  Federation.
\item[Environ.] One of a variable number of areas on a world's
  surface. Each environ is a square \textbf{4,000 km} to a side. An
  environ may be drawn at a scale of \textbf{100 km} per hex on an
  Environ Hex Map.
\item[Hostile Zone.] The most inhospitable zone (for humans) of a star
  system; affects the type of planet or moon within it. 
\item[Kuiper Belt.] A disk shaped region of heavier bodies orbiting
  the outer limits of the Star system. For Sol this region is
  considered to start outside the orbit of Neptune and is where Pluto
  has thought to have originated. The region contains planetesimals
  and icy asteroids.
\item[Law Level.] A number from \textbf{1} to \textbf{5} measuring the
  quality and degree of law enforcement on a world.
\item[Moon.] A body that orbits a planet; also referred to as a
  world.
\item[Neutral Zone.] The area of a star system not in the biosphere or
  hostile zone; affects the type of planets or moons within it.
\item[Oort Cloud.] A spherical region of space at the limits of a
  star's gravitational influence. For Sol this extends out to about 3
  light years and contains the flotsam and jetsam of the solar system
  particularly comets.
\item[Planet.] A body orbiting a star; also referred to as a world.
\item[Resource.] A natural feature of value found on a world.
\item[Settlement Status.] A term summarizing the type and extent of
  human settlement on a world.
\item[Site.] A small location within an environ of special interest;
  natural sites are not usually noticeable without extensive
  exploration.
\item[Spaceport Class.] A number from \textbf{0} to \textbf{5}
  measuring the quality of facilities available at a spaceport.
\item[Spectral Class.] A standard astronomical term quantifying a
  star's luminosity. Used in Universe to determine a star's likelihood
  of having planets.
\item[Star System.] A star with its orbital bodies, including planets,
  moons, asteroids, and comets. Graphically represented in Universe on
  a Star System Log.
\item[World.] Any planet, moon, asteroid belt, or other heavenly body
  on which an adventure may take place; graphically represented on a
  World Log.
\end{description}

\section{Star Systems}
\label{sec:star-systems}
 
The GM is responsible for choosing a star to generate. The
Interstellar Display shows every known star within 30 Light Years of
Sol, each star's Spectral Class, and whether it is able to support
planets. Some stars form binary or trinary systems; their
peculiarities are detailed in \ref{sec:binary-trinary}. The Star
System Log is used to record the information concerning the star and
its planets. To generate a complete star system, the GM conducts the
following steps:

\begin{enumerate}
\item Choose a star from the Interstellar Display. Note the star's
  coordinates, its distance from Sol, and its Spectral Class on the
  Star System Log.
\item Determine the number and positions of the planets in the system,
  and record them on the Log.
\item Determine the size and type of each planet in the system.
\item Determine the number of moons for each planet and record the
  results on the Log. The size and type of each moon are determined,
  and that information is also recorded.
\item Note the gravity of each world.
\end{enumerate}

\subsection[Interstellar Display]{The GM chooses a star from the
    Interstellar Display.}
\label{sec:interstellar-display}
  
  Each star listing includes its Cartesian coordinates (\textbf{\textsf{X}},
  \textbf{Y}, \textbf{Z}), its name. Spectral Class, and whether or
  not it can sustain planets.  When choosing 'a star system to
  generate, the GM should keep in mind that the farther a star is from
  Sol, the less its chance of having a highly developed civilization
  and quality interstellar trade routes.  Also, the farther away from
  Sol the greater the chance of undiscovered resources and life forms.
  The Spectral Class of the star affects the number of planets and the
  habitability of each planet.
  
  The GM would do well to spend time examining all the charts and
  tables detailed in this Chapter before picking a star.

\subsection[Binary/Trinary Systems]{The binary and trinary star
  systems contain abnormalities.}
\label{sec:binary-trinary}

These systems have exceptions to the normal distribution of planets.
The anomalies are due to a number of factors including the stars'
Spectral Classes and the distance between the stars. The types of
restrictions limit the planet positions that can be rolled (for
instance, \textbf{1--7} indicates that only the first seven positions
can be rolled for). Star systems restricted in this manner are listed
in the \textbf{Binary/Trinary Star System Summary} (see below).

\begin{table}[htbp]
  \centering
  \fbox{%
    \begin{minipage}{4in}
      \centering
      \caption{Binary/Trinary Star System Summary}
      \label{tab:binary-trinary-stars}

      \medskip

      \begin{tabular}{cccc}
        \multicolumn{1}{l}{STAR} & Pos & \multicolumn{1}{l}{STAR} & Pos\\
        \rowcolor{grey}
        Alpha Centauri A: & 1--2 &  CD --8\textdegree A: &  2--12 \\
        \rowcolor{grey}
        --2, --1, --4 & & --6, --20, --3 &\\
        61 Cygni A: & 1--6 & Rho Eridani A: & 1--2 \\
        \rowcolor{grey}
        61 Cygni B: &  1--5 & Rho Eridani B: & 1--2 \\
        \rowcolor{grey}
        +6, --6, +7 & & +11, +5, --18 &\\
        Jim: & 1--10 & 41 Arae A: &  1--2 \\
        +6, --6, +7 & & --3, --18, --19\\
        \rowcolor{grey}
        Eta Cassiopeiae A: & 1--7 & Gamma Leporis A: & 1--11 \\
        \rowcolor{grey}
        +10, +2, +15 &&&\\
        WX Ursae Majoris A: & 1--8 & Gamma Leporis B:  & 1--8 \\
        --13, +4, +13 & & +2, +25, --10 \\
        \rowcolor{grey}
        BD+53\textdegree 1320: & 1--7 & CD --36\textdegree 13A: & 1--2 \\
        \rowcolor{grey}
        & & +8, --13, --12 &\\
        BD+ 53\textdegree 1321 &  1--7 \\
        --9, +8, +16
      \end{tabular}

      \medskip

      \textbf{Pos:} Roll only for Planets in the indicated positions
      for these stars.  
    \end{minipage}}
\end{table}


\subsection[Star System Log]{The GM should familiarize himself with
  the use of the Star System Log.}
\label{sec:star-system-log}

Space is given on this log to record the star's Spectral Class, name,
and distance from Sol. Also listed are the biospheres, neutral zones,
and hostile zones for all five Spectral Classes. Each planetary
position (\textbf{12} in all) gives the Planet Size Modifier and the distance
from the star in Astronomical Units (AUs).

The GM records the information concerning the star on the log. As each
planet is generated, he notes the name, size, type, and number of
moons for the planet. Below these listings is additional space to note
detailed information about the worlds (both planets and moons).

The log is also used to record general information concerning the
system itself, such as interworld and interstellar trade routes,
governmental types, amount of federal intervention. This information
is derived from \ref{sec:population-technology},
\ref{sec:interstellar-travel}, and \ref{sec:interplanetary-travel}.

\subsection[Planet Creation Table]{The GM determines the number of
  planets in the system.}
\label{sec:planet-creation-table}

Refer to the \textbf{Planet Creation Table} (page
\pageref{tab:planet-creation}) and roll two dice. for each possible
planet position shown on the Star System Log. If the result matches
the range listed, a planet exists at the position. If a planet does
not exist, put an X through the circle at the position. The circles of
all the existing planets are left blank for the time being. A total of
12 dice rolls are made (unless using a restricted binary or trinary
system).

If a position has no planet, this does not mean there is no world of
any kind there; but that there is no world of any interest there. The
GM is free to place dead worlds (gas giants, volcanic worlds, etc.)
wherever he wishes in order to fill out a star system.

\begin{table}[htbp]
  \centering
  \fbox{%
    \begin{minipage}{2.25in}
      \centering
      \caption{Planet Creation Table}
      \label{tab:planet-creation}
      
      \medskip

      \begin{tabular}{lc}
        \multicolumn{1}{c}{SPECTRAL}  & CLASS \\
        \multicolumn{1}{c}{PLANET} & RESULT \\
        \rowcolor{grey}
        A (0--4) & 2--5 \\
        A (5--9) & 2--7 \\
        \rowcolor{grey}
        F (0--4) & 2--8 \\
        F (5--9) & 2--9, 17 \\
        \rowcolor{grey}
        G (0--4) & 2--11 \\
        G (5--9) & 2--10 \\
        \rowcolor{grey}
        K (0--4) & 2--9 \\
        K (5--9) & 2--7, 17 \\
        \rowcolor{grey}
        M (0--4) & 2--6 \\
        M (5--9) & 2--3         
      \end{tabular}

      \medskip

      \parbox{\textwidth}{\textbf{PLANET RESULT:}\\ 
      Planet exists at 
      position if the sum of 
      2-dice is between the 
      range listed next to 
      the spectral class of 
      the star.
      See \ref{sec:planet-creation-table} for 
      explanation of use.}
    \end{minipage}}
\end{table}

\subsection[Size And Type]{The GM determines the size and type of
  each planet.}
\label{sec:size-and-type}
 
For each existing planet, refer to the \textbf{Planet Size and Type
  Table} (page \pageref{tab:planet-size-type}).

Roll one die twice to determine the planet's size, habitability, and
resource category. One or both die results may be modified, as listed
on the table and on the Star System Log. For each Spectral Class
listed in the upper left-hand corner of the log, there is a line
reading across the planet listings. This indicates which positions for
that Spectral Class are within the biosphere, neutral zone, or hostile
zone.  The biosphere gives a \textbf{--2} modifier, and the neutral
zone a \textbf{+2} both to the second die roll only. The hostile
zone indicates the second die is not rolled; the roll is treated as a
\textbf{10} and that row is used to locate the result. Record all
attributes derived from the table on the log.  The GM should invent a
name for each planet and record it on the log.

The abbreviated attributes are defined as follows:

\begin{description}
\item[E:] Earth-like; similar to Earth in most natural features.
\item[T:] Tolerable; can be inhabited with a certain amount of
  technological aid.
\item[H:] Hostile; habitation is very difficult and very expensive.
\item[A:] Asteroid belt; possibly a broken up planet such as exists in
  the Sol system. Treated as a size 4 planet for purposes of resource
  determination.
\item[r:] Resource rich; abundant resources in easily accessible
  locations.
\item[p:] Resource poor; resources are either scarce or very
  inaccessible.
\item[1 through 9:] Planet size; affects gravity (see 23.7), number of
  environs, and determines which World Log is used.
\end{description}

\begin{table}[htbp]
  \centering
  \fbox{%
    \begin{minipage}{5.5in}
      \centering
      \caption{Planet Size and Type Table}
      \label{tab:planet-size-type}
      
      \medskip
      
      \begin{tabular}{ccccccccccc}
        FIRST DIE & 1 & 2 & 3 & 4 & 5 & 6 & 7 & 8 & 9 & 10 \\
        SECOND DIE \\
        \rowcolor{grey}
        1 & 3:Er & 4:Ep & 4:Er & 4:Er & 4:Er & 4:Er & 5:Er & 5:Ep & 7:Ep & 7:Er\\
        2 & 3:Ep & 3:Ep & 4:Er & 4:Er & 4:Er & 4:Er & 5:Er & 8:Er & 8:Ep & 8:Ep\\
        \rowcolor{grey}
        3 & 4:Er & 4:Ep & 5:Ep & 5:Er & 5:Er & 5:Ep & 5:Er & 5:Ep & 7:Er & 7:Ep\\
        4 & 2:Tr & 3:Tp & 3:Tp & 3:Tr & 6:Er & 6:Er & 6:Er & 6:Er & 6:Ep & 6:Ep\\
        \rowcolor{grey}
        5 & 2:Tp & 3:Tr & 3:Tr & 4:Tr & 4:Tr & 4:Tp & 5:Tr & 5:Tr & 8:Tr & 8:Tp\\
        6 & 1:Hr & 3:Tp & 3:Tp & A:Hr & 4:Tp & 4:Tp & 5:Tr & 5:Tp & A:Hp & 8:Tp\\
        \rowcolor{grey}
        7 & 1:Hr & 2:Hr & 4:Hr & 4:Hr & A:Hr & 5:Tp & 5:Tp & A:Hp & A:Hr & 6:Tr\\
        8 & 1:Hr & 2:Hr & 3:Hr & A:Hr & 4:Hp & A:Hr & 5:Tp & 4:Hp & A:Hr & 6:Tp\\
        \rowcolor{grey}
        9 & 1:Hr & 2:Hr & 2:Hr & 2:Hr & 3:Hr & 3:Hr & 6:Tr & 6:Tp & 8:Tr & 6:Tp\\
        10 & 1:Hp & 2:Hp & 3:Hp & A:Hp & 5:Hp & 6:Hr & 7:Hp & 8:Hp & 7:Tr & 9:Hr
      \end{tabular}

      \medskip

      \parbox{0.90\textwidth}{%
        \#: Planet size. 
        E: Earth-Like; 
        T: Tolerable; 
        H: Hostile; 
        A: Asteroid Belt; 
        r: Resource Rich, 
        p: Resource Poor. 

        \textbf{Modifiers:}\\
        \textbf{\emph{Add or Subtract}} Planet Size Modifier (See Star System
        Log) from first die result. \textbf{\emph{Subtract} 2} from second  
        die result if planet is in biosphere. \\
        \textbf{\emph{Add} 2} to second die result if planet is not in
        biosphere and not in hostile zone.\\
        \textbf{\emph{Do not roll}} second die if planet is in hostile zone
        (See Star System Log); use \textbf{10} row or the table to obtain 
        planet size and type.\\
        Treat all modified rolls of less than \textbf{1} as \textbf{1}, and
        all modified die rolls of more than \textbf{10} as \textbf{10}.\\
        See \ref{sec:size-and-type} for detailed explanation of use.}
    \end{minipage}}
\end{table}

\subsection[Moons]{The GM determines the number of moons for each
  planet.}
\label{sec:moons}

Use the \textbf{Moon Generation Table} (page \pageref{tab:moon-gen}) for
each planet created; size 1 and 2 planets cannot possess moons. Cross
reference one die roll with the planet size to find the number of the
planet's moons; record the number on the System Log.

Determine each moon's size and type using the \textbf{Moon Type Table}
(page \pageref{tab:moon-type}) and \textbf{Moon Size Table} (page
\pageref{tab:moon-size}). The concepts used in this procedure are similar
to those in \ref{sec:size-and-type}. The GM should realize here the
only distinction for game purposes between a planet and a moon is the
body that they orbit. Also, it is possible to generate a size 0
moon, but a size 0 planet cannot be generated.

\renewcommand{\thetable}{\thesection A}
\begin{table}[htbp]
  \centering
  \fbox{%
    \begin{minipage}{2.25in}
      \centering
      \caption{Moon Generation Table}
      \label{tab:moon-gen}
      
      \medskip
      
      \begin{tabular}{cccccccc}
        & \multicolumn{7}{l}{PLANET SIZE}\\
        DIE & 3 & 4 & 5 & 6 & 7 & 8 & 9 \\
        1, 2 & 0 & 0 & 0 & 0 & 0 & 0 & 0 \\
        \rowcolor{grey}
        3, 4 & 0 & 0 & 0 & 0 & 1 & 1 & 1 \\
        5, 6 & 0 & 0 & 1 & 1 & 2 & 2 & 3 \\
        \rowcolor{grey}
        7, 8 & 0 & 1 & 1 & 2 & 3 & 4 & 5 \\
        9 & 1 & 1 & 2 & 3 & 4 & 5 & 6 \\
        \rowcolor{grey}
        10 & 1 & 2 & 3 & 4 & 5 & 6 & 7 
      \end{tabular}

      \medskip
      
      \parbox{0.9\textwidth}{Results are number of moons orbiting
        planet. A Size 1 or 2 Planet or moon may not possess moons.
        See \ref{sec:moons} for explanation.}
    \end{minipage}}
\end{table}

\renewcommand{\thetable}{\thesection B}
\begin{table}[htbp]
  \centering
  \fbox{%
    \begin{minipage}{2in}
      \centering
      \caption{Moon Type Table}
      \label{tab:moon-type}

      \medskip

      \begin{tabular}{ccccc}
        & \multicolumn{4}{c}{MOON SIZE}\\
        DIE & 0, 1 & 2 & 3 & 4, 5 \\
        1 & Hr & Tr & Er & Ep\\
        \rowcolor{grey}
        2 & Hp & Tp & Ep & Ep\\
        3 & Hr & Op & Tr & Er\\
        \rowcolor{grey}
        4 & Hp & Hr & Tp & Tr\\
        5 & Hr & Hp & Op & Tp\\
        \rowcolor{grey}
        6 & Hp & Hr & Hp & Tr\\
        7 & Hr & Hp & Hr & Or\\
        \rowcolor{grey}
        8 & Hp & Hr & Hp & Hp\\
        9 & Hr & Hp & Hr & Hp\\
        \rowcolor{grey}
        10 & Hp & Hr & Hp & Hr 
      \end{tabular}
      
      \medskip
      
      \parbox{0.9\textwidth}{\textbf{E:} Earth-Like. \textbf{T:}
        Tolerable. \textbf{H:} Hostile.  \textbf{O:} Ring. \textbf{r:}
        Resource Rich. \textbf{p:} Resource Poor.
        
        \emph{Subtract} \textbf{2} or \emph{add} \textbf{2} to die
        result depending on whether moon's planet is in or out of the
        Biosphere.
        
        If moon is in hostile zone of star system, do not roll;
        cross-reference moon size with \textbf{10} row to yield moon
        type.  Treat all modified rolls of less than \textbf{1} as
        \textbf{1} and all rolls of greater than \textbf{10} as
        \textbf{10}.}
    \end{minipage}}
\end{table}

\renewcommand{\thetable}{\thesection C}
\begin{table}[htbp]
  \centering
  \fbox{%
    \begin{minipage}{2.5in}
      \centering
      \caption{Moon Size Table}
      \label{tab:moon-size}

      \medskip

      \begin{tabular}{cc}
        1 DIE + PLANET SIZE & MOON SIZE \\
        4--8 & 0 \\
        \rowcolor{grey}
        9--12 & 1 \\
        13--15 & 2 \\
        \rowcolor{grey}
        16--17 & 3 \\
        18 & 4 \\
        \rowcolor{grey}
        19 & 5
      \end{tabular}
    \end{minipage}}
\end{table}
\renewcommand{\thetable}{\thesection}

\subsection[Gravity]{The gravity of each world is determined by the
  size of the world.}
\label{sec:gravity}

On each World Log, the gravity is listed for that size world. Gravity
is expressed in Gs, with Earth's gravity equalling \textbf{1.0 G}. The
gravity types range from \textbf{none} to \textbf{2.5 G}; see
\textbf{World Gravity Table} (below) for summary.

On worlds with a ``trace'' gravity, objects will slowly settle to the
surface. However, any object propelled with a velocity equal to or
greater than that of a pitched baseball will escape the gravitational
field.

\begin{table}[htbp]
  \centering
  \fbox{%
    \begin{minipage}{2.5in}
      \caption{World Gravity Table}
      \label{tab:gravity}

      \medskip

      \centering
      \begin{tabular}{ccc}
        WORLD SIZE & GRAVITY & CLASS \\
        0 & None & NW \\
        \rowcolor{grey}
        1 & Trace & NW \\
        2 & 0.2 & NW \\
        \rowcolor{grey}
        3 & 0.4 & NW \\
        4 & 0.7 & LT \\
        \rowcolor{grey}
        5 & 1.0 & LT \\
        6 & 1.3 & HY \\
        \rowcolor{grey}
        7 & 1.7 & HY \\
        8 & 2.0 & EX \\
        \rowcolor{grey}
        9 & 2.5 & EX
      \end{tabular}
    \end{minipage}}
\end{table}

\section{Geographical Features}
\label{sec:geographical-features}

The generation of geographical features for a world will yield not
only factual information but an actual pictorial representation of the
world as it would be seen from orbit. This is accomplished by use of
the World Log, which is a graphic layout of the world as viewed from
the poles. The information generated in Section \ref{sec:star-systems}
and the features determined in this Section are recorded on this log.
It is recommended that the GM use colored markers or pencils when
drawing the world map. To generate the geographical features, the GM
conducts' the following steps and records the information generated on
the World Log corresponding to the appropriate world size:

\begin{enumerate}
\item Determine the world's atmosphere.
\item Determine the world's mean temperature.
\item Determine the hydrograph percentage.
\item Note the number and type of land/water environs for the world
  and distribute them amongst the environs of the World Log.
\item Draw in the land masses and water bodies on the world map,
  creating continents, oceans, rivers, islands, etc.
\item Determine the environ type for each environ on the log. Record
  this information or draw out the environ type using markers or
  pencils.
\item Determine the day length (optional).
\end{enumerate}

\subsection[World Log]{The GM should familiarize himself with the
  use of the World Log.}
\label{sec:world-log}

There are, in this booklet, \textbf{10} World Logs, which represent
variously sized worlds.

Each World Log is a graphic representation of a world as it is viewed
from the poles. The size 0 log has only one view due to its small
size. These views are divided into environs, which represent areas on
the surface 4,000 km across. Each environ is numbered, and each ring
of environs represents a different temperature ring on the world. Each
ring contains a temperature modifier, which is applied to the mean
temperature generated for the world to determine the mean temperature
of that ring.

Each log has space to list the world's name, type, atmosphere,
temperature, hydrograph percentage, day length, population, Civ Level,
Law Level, Spaceport Class, and resources. The size and gravity of
each world is already listed at the top.  The map of the world is
divided into environs. The GM should examine the log carefully to note
which environs are adjacent to one another. Special attention should
be paid to how the rings, which form the outer circumference of the
north and south views, connect with each other.

The GM should create two logs, one containing all the fine detail for
himself, and one which is mainly pictorial in nature for the players.
Record all the raw data on a piece of scrap paper or an extra copy of
the log and make the final copy only after all information has been
determined. For each environ, the land/water' distribution, environ
type, resources, and population must be recorded.

\subsection[Atmosphere]{The GM determines the atmosphere for the
  world.}
\label{sec:atmosphere}

Referring to the \textbf{World Atmosphere Table} (page
\pageref{tab:world-atmosphere}), the GM rolls one die and adds the
world size to the result (a world of size 0 or 1 automatically
possesses no atmosphere). The modified die result is cross-referenced
with the world type and the atmosphere is recorded on the World Log.
The atmosphere types and their effects follow:

\begin{description}
\item[None.] Full expedition suit required at all times. All
  structures must be airtight with air-lock door systems.
\item[Thin.] Respirator required. Survival without respirator possible
  for up to 2 hours (or up to GM's discretion). Higher quality
  structures are airtight; most structures have air compressors.
\item[Thin, Contaminated.] Respirator required. Small possibility of
  disease. Higher quality structures are airtight. All structures must
  have air compressors and filter doors.
\item[Normal.] No special equipment or structures are required.
\item[Normal, Contaminated.] Filter mask required. All structures have
  filter doors and small compressors.
\item[Poison.] Respirator required; expedition suit recommended. All
  structures are airtight with oxygen supply. Disease, deterioration,
  and frequent storms are possible; low visibility is common.
\item[Corrosive.] Full expedition suit required; body armor
  recommended. All structures must be airtight with air-lock door
  systems. Higher quality structures are armored. Constant storms are
  possible; no visibility. Deterioration and disease are likely.
\end{description}

\begin{table}[htbp]
  \centering
  \fbox{%
    \begin{minipage}{4in}
      \centering
      \caption{World Atmosphere Table}
      \label{tab:world-atmosphere}
      
      \medskip
      
      \begin{tabular}{clll}
        DIE + WORLD & \multicolumn{3}{c}{WORLD TYPE}\\
        SIZE & \multicolumn{1}{c}{EARTH-LIKE} &
        \multicolumn{1}{c}{TOLERABLE} & \multicolumn{1}{c}{HOSTILE}\\ 
        \rowcolor{grey}
        3--5 & Thin & None & None\\
        6 & Thin (C) & Thin (C) & None\\
        \rowcolor{grey}
        7 & Normal & Thin & None\\
        8 & Thin (C) & Poison & None\\
        \rowcolor{grey}
        9 & Thin & Thin (C) & Poison\\
        10 & Thin & Poison & Corrosive\\
        \rowcolor{grey}
        11 & Normal & Thin & Poison\\
        12 & Normal (C) & Corrosive & Corrosive\\
        \rowcolor{grey}
        13 & Poison & Thin (C) & Poison\\
        14 & Normal & Poison & Corrosive\\
        \rowcolor{grey}
        15 & Normal (C) & Normal (C) & Poison\\
        16 & Normal & Normal (C) & Corrosive\\
        \rowcolor{grey}
        17 & Normal & Poison & Poison\\
        18 & Normal (C) & Poison & Corrosive\\
        \rowcolor{grey}
        19 & Normal & Normal (C) & Poison
      \end{tabular}
      
      \medskip

      \parbox{\textwidth}{A world of size \textbf{0} or \textbf{1}
        automatically possesses no atmosphere.

        \textbf{(C):} Contaminated.

        \medskip

        See \ref{sec:atmosphere} for explanation.}
    \end{minipage}}
\end{table}

\subsection[Temperature]{The GM determines the mean temperature of
  the world.}
\label{sec:world-temp}

Using the \textbf{World Temperature Table} (page
\pageref{tab:world-temp}), the GM rolls one die and applies modifiers
listed on the table. The result is cross referenced with the world
type and the temperature is recorded on the World Log. This
temperature is the mean for the entire world; the actual temperature
varies from one environ ring to another. All temperatures are in
Fahrenheit.

Each World Log lists the temperature modifier for each ring of
environs on the world. This modifier is added to the mean temperature
of the world to determine the average temperature of all environs in
that ring. For example, a size 4 world has a mean temperature of
\textbf{50\textdegree}. Reference to the World Log shows environs
\textbf{1} and \textbf{14} will have a temperature of
\textbf{25\textdegree}; \textbf{2} through \textbf{5} and \textbf{15}
through \textbf{18} will have a temperature of \textbf{50\textdegree};
and \textbf{6} through \textbf{13} will have a temperature of
\textbf{75\textdegree}. The specific temperature of an environ is used
with the \textbf{Environ Type Table} when determining the nature of an
environ.

When creating a world with an odd axial tilt (for instance, pointing
toward the star) or with no rotation, the GM should change the
distribution of the temperature modifiers on the World Log and/or
widen the range of modifiers. The modifiers are designed for a world
with an axis nearly perpendicular to the plane of the star system
ecliptic and with axial rotation. The GM should feel free to alter
them to fit the individual pattern.


\begin{table}[htbp]
  \centering
  \fbox{%
    \begin{minipage}{4in}
      \caption{World Temperature Table}
      \label{tab:world-temp}
      
      \medskip\centering
      
      \begin{tabular}{crrr}
        & \multicolumn{3}{c}{WORLD TYPE}\\
        ONE DIE & EARTH-LIKE & TOLERABLE & HOSTILE \\
        \rowcolor{grey}
        --5, --4 & 125 & 175 & 200+\\
        --3, --2 & 100 & 150 & 200\\
        \rowcolor{grey}
        --1, 0 & 100 & 125 & 175\\
        1, 2 & 75 & 125 & 175\\
        \rowcolor{grey}
        3, 4 & 75 & 100 & 150\\
        5, 6 & 50 & 25 & --25\\
        \rowcolor{grey}
        7, 8 & 50 & 0 & --25\\
        9, 10 & 25 & 0 & --50\\
        \rowcolor{grey}
        11, 12 & 25 & --25 & --50\\
        13, 14 & 0 & --25 & --75\\
        \rowcolor{grey}
        15, 16 & 0 & --50 & --75 or lower
      \end{tabular}

      \medskip

      \parbox{\textwidth}{Results represent average temperatures of 
        the world; all temperatures are expressed 
        in \textdegree F.

        See \ref{sec:world-temp} for explanation. 

        \medskip

        \textbf{Die Roll Modifiers:}

        If the planet is closer to the star than the 
        star's biosphere, \emph{subtract} \textbf{6} from the die 
        result.

        If the planet is further from the star than the 
        star's biosphere, \emph{add} \textbf{6} to the die roll.}
    \end{minipage}}
\end{table}

\subsection[Hydrograph Percentage]{The GM determines the hydrograph
  percentage of the world.}
\label{sec:world-hydro-pct}

The GM refers to the \textbf{World Hydrograph Table}
(\ref{tab:world-hydro-pct}). Using
the world's size and mean temperature to ascertain which column to
look in, the GM rolls one die and records the hydrograph percentage on
the World Log. Note that worlds that possess no atmosphere or whose
temperature is below \textbf{0\textdegree} or above
\textbf{125\textdegree} automatically have no freestanding water.

The GM then refers to the \textbf{Land and Water Distribution Chart} 
(\ref{tab:land-water-dist}). Seven
types of land-water distribution that might exist  
in an environ are listed across the top of the chart (see explanation 
following). Using the line that  corresponds to the world size and 
hydrograph percentage, the GM reads across to find how many 
environs of each type exist on the world. Note the total of all the 
numbers on a single line equals the number of environs on the 
world. The following seven types are included: 

\begin{description}
\item[Water.] All liquid with no land.
\item[Water with Minor Islands.] 90 percent water with scattered small
  islands.
\item[Water with Major Islands.] 75 percent water with islands or land
  masses possibly connecting to a larger land mass out of environ.
\item[Water with Land.] 50 percent water and landmasses.
\item[Land with Major Water.] 75 percent land with water; such as a
  coast line or major lake within the land mass.
\item[Land with Minor Water.] 90 percent land with small water; a
  small part of a coastline, a river network, small lakes, etc.
\item[Land.] Land with no bodies of water.
\end{description}

These descriptions are intended as guides only; the GM should 
decide what elements make up the distribution type. See the 
Adventure Guide for visual examples of these distributions. 

The GM should actually draw these environs on the log, assigning 
them to whichever environ seems consistent, logical, or 
aesthetically pleasing. Many water-only environments could be 
combined adjacently to form a large ocean. Coastlines from en- 
viron to environ should be contiguous, and the GM should be 
constantly aware of which environs are actually adjacent although 
graphically separated on the log. This drawing process should be 
done roughly in pencil at first, with a final colored version 
executed when all decisions are finalized. 


\renewcommand{\thetable}{\thesection A}
\begin{table}[htbp]
  \centering
  \fbox{%
    \begin{minipage}{4in}
      \centering
      \caption{World Hydrograph Table}
      \label{tab:world-hydro-pct}
      
      \medskip
      
      \begin{tabular}{crrrr@{\hspace{3em}}rrrrr}
        & \multicolumn{9}{c}{WORLD TEMPERATURE}\\
        & \multicolumn{4}{c}{50, 75} & \multicolumn{5}{c}{0, 25, 100, 125}\\
        & \multicolumn{4}{c}{WORLD SIZE} & \multicolumn{5}{c}{WORLD SIZE}\\
        ONE\\
        DIE & 3 & 4 & 5 & 6--9 & 2 & 3 & 4 & 5 & 6--9\\
        \rowcolor{grey}
        1 & 0 & 0 & 20 & 20 & 0 & 0 & 0 & 0 & 0\\
        2 & 0 & 20 & 20 & 40 & 0 & 0 & 0 & 0 & 20\\
        \rowcolor{grey}
        3 & 0 & 20 & 40 & 40 & 0 & 0 & 0 & 20 & 20\\
        4 & 20 & 20 & 40 & 40 & 0 & 0 & 0 & 20 & 20\\
        \rowcolor{grey}
        5 & 20 & 40 & 60 & 60 & 0 & 0 & 20 & 20 & 40\\
        6 & 20 & 40 & 60 & 60 & 0 & 20 & 20 & 40 & 40\\
        \rowcolor{grey}
        7 & 20 & 40 & 60 & 80 & 0 & 20 & 20 & 40 & 60\\
        8 & 40 & 60 & 80 & 80 & 20 & 20 & 40 & 40 & 60\\
        \rowcolor{grey}
        9 & 40 & 60 & 80 & 100 & 20 & 40 & 40 & 60 & 80\\
        10 & 40 & 80 & 100 & 100 & 20 & 40 & 60 & 60 & 80\\
      \end{tabular}

      \medskip

      \parbox{\textwidth}{Results represent percentage of the world's
        surface covered with  
        water.

        This table is not use for worlds that  possess no atmosphere, or 
        possess average temperatures below \textbf{0} or above \textbf{125}. 
        Such worlds automatically have no water. 

        See \ref{sec:world-hydro-pct} for detailed explanation of use.}
    \end{minipage}}
\end{table}


\renewcommand{\thetable}{\thesection B}
\begin{table}[htbp]
  \centering
  \fbox{%
    \begin{minipage}{3.5in}
      \centering
      \caption{Land and Water Distribution Chart}
      \label{tab:land-water-dist}
      
      \medskip
      
      \begin{tabular}{crrrrrrrr}
        WORLD  & 
        \rotate{HYDROGRAPH PERCENTAGE} &
        \rotate{WATER} &
        \rotate{WATER/MINOR ISLANDS} &
        \rotate{WATER/MAJOR ISLANDS} &
        \rotate{WATER/LAND} &
        \rotate{LAND/MAJOR WATER} &
        \rotate{LAND/MINOR WATER} &
        \rotate{LAND}\\
        SIZE \\
        \rowcolor{grey}
        2 & 20 & 0 & 0 & 0 & 0 & 2 & 2 & 2\\
        3 & 20 & 0 & 0 & 0 & 1 & 3 & 2 & 4\\
        \rowcolor{grey}
        3 & 40 & 0 & 0 & 3 & 2 & 2 & 2 & 1\\
        4 & 20 & 0 & 0 & 0 & 1 & 6 & 4 & 7\\
        \rowcolor{grey}
        4 & 40 & 0 & 0 & 5 & 4 & 4 & 4 & 1\\
        4 & 60 & 1 & 4 & 6 & 4 & 2 & 1 & 0\\
        \rowcolor{grey}
        4 & 80 & 7 & 6 & 4 & 1 & 0 & 0 & 0\\
        5 & 20 & 0 & 0 & 0 & 3 & 8 & 5 & 10\\
        \rowcolor{grey}
        5 & 40 & 0 & 1 & 6 & 6 & 5 & 5 & 3\\
        5 & 60 & 3 & 5 & 7 & 5 & 3 & 3 & 0\\
        \rowcolor{grey}
        5 & 80 & 10 & 8 & 5 & 3 & 0 & 0 & 0\\
        5 & 100 & 23 & 3 & 0 & 0 & 0 & 0 & 0\\
        \rowcolor{grey}
        6 & 20 & 0 & 0 & 0 & 4 & 11 & 8 & 15\\
        6 & 40 & 0 & 1 & 8 & 9 & 8 & 8 & 4\\
        \rowcolor{grey}
        6 & 60 & 4 & 8 & 10 & 8 & 4 & 4 & 0\\
        6 & 80 & 15 & 11 & 8 & 4 & 0 & 0 & 0\\
        \rowcolor{grey}
        6 & 100 & 34 & 4 & 0 & 0 & 0 & 0 & 0\\
        7 & 20 & 0 & 0 & 1 & 4 & 14 & 11 & 20\\
        \rowcolor{grey}
        7 & 40 & 0 & 1 & 13 & 11 & 10 & 10 & 5\\
        7 & 60 & 5 & 11 & 14 & 10 & 5 & 4 & 1\\
        \rowcolor{grey}
        7 & 80 & 20 & 16 & 9 & 4 & 1 & 0 & 0\\
        7 & 100 & 45 & 5 & 0 & 0 & 0 & 0 & 0\\
        \rowcolor{grey}
        8 & 20 & 0 & 0 & 1 & 6 & 19 & 14 & 26\\
        8 & 40 & 0 & 1 & 18 & 14 & 13 & 13 & 7\\
        \rowcolor{grey}
        8 & 60 & 7 & 14 & 18 & 13 & 7 & 6 & 1\\
        8 & 80 & 26 & 21 & 12 & 6 & 1 & 0 & 0\\
        \rowcolor{grey}
        8 & 100 & 60 & 5 & 1 & 0 & 0 & 0 & 0\\
        9 & 20 & 0 & 0 & 1 & 7 & 24 & 17 & 33\\
        \rowcolor{grey}
        9 & 40 & 0 & 1 & 23 & 17 & 17 & 16 & 8\\
        9 & 60 & 8 & 17 & 24 & 17 & 8 & 7 & 1\\
        \rowcolor{grey}
        9 & 80 & 33 & 26 & 15 & 7 & 1 & 0 & 0\\
        9 & 100 & 75 & 6 & 1 & 0 & 0 & 0 & 0
      \end{tabular}
    \end{minipage}}
\end{table}
\renewcommand{\thetable}{\thesection}

\subsection[Terrain and Contour]{The GM determines the terrain
  feature and contour of each environ.}
\label{sec:terrain-contour}

Using the \textbf{Environ Type Chart} (Table \vref{tab:environ-type}),
the GM 
determines the topography of each environ with some land. He locates
the column containing the correct land/water distribution and the
correct temperature of the environ (remembering that temperatures vary
from environ to environ, see \ref{sec:world-temp}). The GM rolls
percentile dice and reads down the column until locating the numerical
result that most nearly equals the dice result without being less
than the roll. The environ type corresponding to the numerical result
is the predominant terrain feature and contour of that environ (for
example, a roll of \textbf{34} in the first column would yield a
result of barren/peaks). Repeat this procedure for every environ on
the world until all have been determined. The GM should graphically
depict in each environ the predominant terrain feature and contour.

After generating each environ's type, the GM should look at the world
and feel free to juggle environs around to form a cohesive whole.  For
example, if ice is generated in a non-polar environ and there is no
ice at the pole, the GM should swap the environs to make up for the
anomalies created by the random dice system.

For each environ type listing, the first feature is the terrain
feature, and the second is the contour.


\paragraph{CONTOURS:}

\begin{description}
\item[Flat.] Land is perfectly flat and provides no cover whatsoever.
\item[Hills.] Gently rolling hills, very little hindrance to travel;
  provides some cover at long range.
\item[Mountain.] Heights which erosion has smoothed over, some
  hindrance to travel. Cover fairly easy to find except at extremely
  close range.
\item[Peaks.] Jagged mountains with precipices, travel almost
  impossible; cover is available virtually everywhere.
\end{description}


\paragraph{TERRAIN FEATURES:}

\begin{description}
\item[Volcanic.] Active volcanos exist throughout the environ; travel
  is possible with care; cover varies. Note that it is impossible to
  have volcanos in a flat environ.
\item[Crater.] Land is churned up, travel is difficult; cover is easy
  to find.
\item[Barren.] Totally featureless cracked dry earth; no hindrance to
  travel; no cover.
\item[Light Vegetation.] Sparse grass, lichen, brush; travel
  unaffected; no cover.
\item[Woods.] Widely dispersed trees with undergrowth; travel possible
  with care; no cover except at long range.
\item[Forest.] Dense trees with undergrowth; travel slowed; cover easy
  to find.
\item[Jungle.] Land is, choked with foliage and undergrowth; travel
  except on foot is nearly impossible; cover very easy to find.
\item[Marsh.] Bits of vegetation; some trees and undergrowth; travel
  slowed greatly; some cover available on occasion.
\end{description}


\paragraph{ICE ENVIRONS:}

\begin{description}
\item[Ice/Flat.] Treat as barren/flat; travel with care is possible;
  no cover.
\item[Ice/Hills.] Very jagged ice formations; travel almost
  impossible; cover available at all except close range.
\end{description}

The following environ types are included in the World Logs in the
Adventure Guide. Refer to them for possible methods of drawing and
coloring them.


\paragraph{On the Planet \emph{Titus:}}

\begin{description}
\item[n02] Water with minor islands; woods/hills.
\item[n03] Land/water; light veg/mountains;
\item[n05] Water with minor islands; volcanic/hills.
\item[n09] Water with major islands; forest/mountains.
\item[n12] Water with major islands; jungle/mountains.
\item[n20] Land/water; barren/mountain.
\item[s01] Water; ice/flat.
\item[s04] Water with major islands; marsh/hill.
\item[s13] Water with major islands; barren/peaks.
\end{description}

\paragraph{On the Planet \emph{Kryo:}}


\begin{description}
\item[n01] Land/water; ice/hill,
\item[n03] Land with major water; woods/mountains.
\item[s02] Land with major water; light veg/flat.
\item[s03] Land with minor water; barren/peaks.
\end{description}

\paragraph{On the Planet \emph{Laidley:}}


\begin{description}
\item[n02] Land only; barren/mountain,
\item[n05] Land only; crater/flat.
\item[s05] Land only; volcanic/ mountains.
\end{description}

When the GM has finished drawing out the world on the log, he will
have a picture he may show to his players that will represent the
world as they might see it through their view screens.


\begin{table}[htbp]
  \scriptsize
  \centering
  \fbox{%
    \begin{minipage}{\textwidth}
      \centering
      \caption{Environ Type Chart}
      \label{tab:environ-type}
      
      \medskip
      
      \begin{tabular}{l|rrrr|rrrr|rrrr|rrrr|rrr}
        & &&& & \multicolumn{4}{l|}{WATER/LAND OR} & &&& & \multicolumn{4}{|l|}{LAND ONLY} & \multicolumn{3}{l|}{LAND ONLY}\\
        & \multicolumn{4}{|l|}{WATER WITH} & \multicolumn{4}{l|}{LAND
        WITH} & &&& & \multicolumn{4}{|l|}{(WATER} & \multicolumn{3}{l|}{(NO WATER}\\
        LAND/LIQUID & \multicolumn{4}{|l|}{MINOR OR MAJOR} & \multicolumn{4}{l|}{MAJOR WATER} & \multicolumn{4}{l|}{LAND WITH MINOR} &
        \multicolumn{4}{|l|}{ELSEWHERE ON} & \multicolumn{3}{l|}{ANYWHERE}\\ 
        DISTRIBUTION & \multicolumn{4}{|l|}{ISLANDS} & \multicolumn{4}{l|}{BODIES} & \multicolumn{4}{l|}{WATER BODIES} & \multicolumn{4}{|l|}{WORLD)} & \multicolumn{3}{l|}{ON WORLD)}\\
        \hline
        \multicolumn{1}{r|}{TEMPERATURE} & 0 &&& 100 & 0 &&& 100 & 0 &&& 100 & 0 &&& 100 &
        0 & 25 & 100 \\
        & to &&& to & to &&& to & to &&& to & to &&& to & to & 50
        &to\\ 
        ENVIRON TYPE & 25 & 50 & 75 & 125 & 25 & 50 & 75 & 125 & 25 &
        50 & 75 & 125 & 25 & 50 & 75 & 125 & 25 & 75 & 125\\
        \rowcolor{grey}\hline
\multicolumn{1}{l|}{See Note} & 1 & 1 & 2 & 3 & 1 & 1 & 1 & 2 & 1 & 2 & 2 & 3 & 2 & 3 & 3 & 4 & 3 & 5 & 7\\
Volcanic/Hills & 3 & 4 & 6 & 9 & 2 & 3 & 3 & 4 & 3 & 5 & 6 & 7 & 4 & 7 & 8 & 10 & 8 & 12 & 17\\
\rowcolor{grey}
Volcanic/Mountains & 4 & 6 & 9 & 14 & 3 & 4 & 5 & 6 & 5 & 7 & 9 & 10 & 6 & 9 & 11 & 14 & 13 & 19 & 26\\
Volcanic/Peaks & 5 & 8 & 11 & 17 & 4 & 5 & - & 7 & 6 & 8 & 10 & 12 & 7 & 10 & 13 & 16 & 14 & 21 & 28\\
        \hline
\rowcolor{grey}
Crater/Flat & - & - & - & - & - & - & - & - & 7 & 9 & 11 & 13 & 9 & 12 & 15 & 18 & 24 & 31 & 37\\
Crater/Hills & - & - & - & - & - & - & - & - & 8 & 10 & 12 & 14 & 11 & 14 & 17 & 20 & 39 & 46 & 52\\
\rowcolor{grey}
Crater/Mountains & - & - & - & - & - & - & - & - & 9 & 11 & 13 & 15 & 13 & 16 & 19 & 22 & 51 & 58 & 64\\
Crater/Peaks & - & - & - & - & - & - & - & - & 10 & 12 & 14 & 16 & 14 & 17 & 20 & 23 & 53 & 60 & 66\\
        \hline
\rowcolor{grey}
Barren/Flat & 11 & 10 & 12 & 26 & 9 & 6 & 6 & 16 & 20 & 19 & 22 & 34 & 29 & 27 & 34 & 42 & 62 & 69 & 74\\
Barren/Hills & 23 & 15 & 15 & 42 & 18 & 8 & 7 & 32 & 32 & 27 & 31 & 54 & 49 & 40 & 52 & 66 & 75 & 82 & 86\\
\rowcolor{grey}
Barren/Mountains & 33 & 20 & 17 & 55 & 25 & 10 & 8 & 45 & 40 & 33 & 37 & 66 & 59 & 48 & 61 & 79 & 84 & 91 & 94\\
Barren/Peaks & 36 & 22 & 18 & 59 & 27 & 11 & - & 47 & 44 & 36 & 40 & 71 & 63 & 52 & 65 & 84 & 86 & 93 & 96\\
        \hline
\rowcolor{grey}
Light Veg/Flat & 41 & 28 & 22 & 67 & 36 & 18 & 14 & 57 & 48 & 43 & 47 & 77 & 67 & 61 & 72 & 87 & 87 & 94 & 97\\
Light Veg/Hills & 46 & 33 & 27 & 75 & 46 & 25 & 20 & 68 & 54 & 52 & 55 & 84 & 72 & 71 & 80 & 91 & 88 & 95 & 98\\
\rowcolor{grey}
Light Veg/Mountains & 56 & 40 & 32 & 83 & 53 & 31 & 25 & 78 & 58 & 58 & 61 & 88 & 76 & 78 & 86 & 93 & 89 & 96 & 99\\
Light Veg/Peaks & 59 & 42 & 34 & 85 & 55 & 33 & 26 & 80 & 60 & 61 & 63 & 90 & 77 & 81 & 88 & 94 & 90 & 97 & 00\\
        \hline
\rowcolor{grey}
Woods/Flat & 62 & 46 & 38 & 90 & 60 & 39 & 32 & 87 & 64 & 66 & 68 & 93 & 80 & 85 & 92 & 95 & - & - & -\\
Woods/Hills & 65 & 51 & 42 & 95 & 66 & 46 & 38 & 94 & 68 & 73 & 74 & 97 & 83 & 91 & 96 & 97 & - & - & -\\
\rowcolor{grey}
Woods/Mountains & 71 & 58 & 49 & 99 & 70 & 52 & 44 & 99 & 71 & 77 & 78 & 99 & 85 & 97 & 99 & 99 & - & - & -\\
Woods/Peaks & 73 & 60 & 52 & 00 & 71 & 54 & 46 & 00 & 72 & 79 & 79 & 00 & 86 & 99 & 00 & 00 & - & - & -\\
        \hline
\rowcolor{grey}
Forest/Flat & 75 & 64 & 55 & - & 73 & 60 & 52 & - & 74 & 82 & 83 & - & - & - & - & - & - & - & -\\
Forest/Hills & 77 & 70 & 59 & - & 76 & 69 & 58 & - & 76 & 86 & 87 & - & - & - & - & - & - & - & -\\
\rowcolor{grey}
Forest/Mountains & 80 & 72 & 67 & - & 78 & 75 & 64 & - & 78 & 88 & 89 & - & - & - & - & - & - & - & -\\
Forest/Peaks & 81 & 79 & 69 & - & 79 & 77 & 66 & - & 79 & 89 & 90 & - & - & - & - & - & - & - & -\\
        \hline
\rowcolor{grey}
Jungle/Flat & - & 82 & 76 & - & - & 81 & 74 & - & - & 90 & 92 & - & - & - & - & - & - & - & -\\
Jungle/Hills & - & 87 & 88 & - & - & 86 & 83 & - & - & 91 & 94 & - & - & - & - & - & - & - & -\\
\rowcolor{grey}
Jungle/Mountains & - & 89 & 95 & - & - & 88 & 90 & - & - & 92 & 95 & - & - & - & - & - & - & - & -\\
        \hline
Marsh/Flat & - & 95 & 98 & - & - & 95 & 96 & - & - & 96 & 98 & - & - & - & - & - & - & - & -\\
\rowcolor{grey}
Marsh/Hills & - & 99 & 00 & - & - & 99 & 00 & - & - & 99 & 00 & - & - & - & - & - & - & - & -\\
        \hline
Ice/Flat & 90 & - & - & - & 89 & - & - & - & 88 & - & - & - & 92 & - & - & - & 94 & 98 & -\\
\rowcolor{grey}
Ice/Hills & 00 & 00 & - & - & 00 & 00 & - & - & 00 & 00 & - & - & 00 &
        00 & - & - & 00 & 00 & -\\
        \hline
      \end{tabular}

      \medskip

      \parbox{\textwidth}{\textbf{How to Read the Result:} Roll
        percentile dice and locate the column corresponding to the
        Land/Liquid/Temperature combination of the environ. Read down
        the column until you located the result that most nearly
        equals the percentile roll without being less than the roll.
        Roll example, rolling a 34 in the first column would yield an
        environ type of Barren/Peaks.
        
        Note: All environs of same water and temperature type which
        have not yet been assigned an environ type will be the type
        determined by the next percentile roll. (\textbf{-}): Environ
        type impossible; proceed down the column. \textbf{Peaks:} If
        the world size is \textbf{7} or greater, treat as Hill.
        \textbf{Mountains:} If the World size is \textbf{9}, treat as
        Flat. If the Land/Liquid distribution is water only, this table
        is not used.  However, if the GM wishes to check for ice in
        water-only environ, roll using the water with minor islands
        column and ignore all non-ice results.
        
        See \ref{sec:terrain-contour} for detailed explanation of use.}
    \end{minipage}}
\end{table}

\subsection[Day Length]{The GM may determine the world's day
  length.}
\label{sec:day-length}

This procedure is optional as there will be complications caused by
having worlds with different day lengths. The GM refers to the \textbf{World
Day Length Table} (page \pageref{tab:world-day-length}) and locates
the column that matches the 
world type. He then rolls one die and records the result on the World
Log.

If the GM wishes a simpler-solution, he may ignore the table and
assume all worlds have a 24-hour day. The effects of having days of
differing lengths must be judged before such a decision can be
reached. Many game systems are designed to measure their time
expenditures in terms of hours. Thus, if the world's hour does not
match the character's hours, two separate tracks must be kept to
measure time. Also, the GM must be ready to apply the physical and
psychological effects of different day lengths to the characters in
order to simulate the problems encountered. One possible answer is to
measure all time expenditures according to the spaceship's clock, and
let the world's day vary as it might.


\begin{table}[htbp]
  \centering
  \fbox{%
    \begin{minipage}{3.5in}
      \centering
      \caption{World Day Length Table}
      \label{tab:world-day-length}
      
      \medskip
      
      \begin{tabular}{cccc}
        & \multicolumn{3}{c}{WORLD TYPE}\\
        ONE DIE & EARTH-LIKE & TOLERABLE & HOSTILE\\
        \rowcolor{grey}
        1 & 12 & 6 & 4\\
        2 & 15 & 9 & 6\\
        \rowcolor{grey}
        3 & 18 & 12 & 8\\
        4 & 21 & 15 & N\\
        \rowcolor{grey}
        5 & 24 & 18 & N\\
        6 & 24 & 24 & 72\\
        \rowcolor{grey}
        7 & 27 & 36 & 4d\\
        8 & 30 & 48 & 6d\\
        \rowcolor{grey}
        9 & 33 & 60 & 8d\\
        10 & 36 & 72 & 10d
      \end{tabular}

      \medskip

      \parbox{\textwidth}{Numbers represent length of 
        day in hours (including daylight 
        and darkness). 
        
        \textbf{d:} Day measured in Earth days.\\
        \textbf{N:} No rotation. 
        
        See  \ref{sec:day-length} for explanation of use.}
    \end{minipage}}
\end{table}

\subsection[Environ Hex map]{The GM may decide to create an Environ
  Hex Map.}
\label{sec:environ-hex-map}

When it becomes desirable to enlarge an environ and reproduce it on a
hex map, the GM should use the hex map provided with the game. This
map is drawn at a scale of 100 km per hex, and on it the GM creates
the major land masses and water bodies of the environ as he has
indicated on the World Log. The terrain features and contours are
added, but in much greater detail than before. The GM may place sites
on this map in particular hexes. A site indicates a small location not
apparent on the World Log of special interest to the area. Sites the
GM might wish to consider include the following:

Rain Forests, Roads, Swamps, Trails, Beaches, Cities, Rivers, Mines,
Caves, Raw mineral deposits, Alien Ruins, Lakes, Old Settlements,
Ponds, Glaciers, Cliffs, Meteor Craters, Volcanos, Psionic
Institutes, Lava Fields, Abysses, Towns, etc.

When drawing in the details of the environ type on the hex map, not
only that type should appear. The GM should refer to the Environ Skill
Display on the Character Record and locate the box corresponding to
the predominant environ type; the eight adjacent boxes indicate
environ types which may also occur in the environ. The GM should cover
about two-thirds of the Environ Hex Map with details of the
predominant environ type, and the rest may be divided up as he sees
fit. He may also place terrain features along the edges of the hex map
that match those of neighboring environs on the World Log.

During the character generation process, a home environ skill is
determined for every character in a particular environ type. It is to
be assumed the character came from an environ in which that type was
predominant. The GM may place the character's home on a world that
contains the appropriate environ type and has some form of settlement.
Due to the detail with which the character would know this environ,
the GM may desire to create an Environ Hex Map of that environ and
possibly show it to the player, depending how much information the
player desires.

Refer to the Adventure Guide for an example of an Environ Hex Map.

\section{Population and Technology}
\label{sec:population-technology}

The final details of a world include the elements of population and
technology. These items will breathe life into an otherwise colorless
world. The GM is encouraged to use the information available in this
Section as a springboard from which he may fully realize a world's
potential for enjoyable play. Each world will be given general
indications of total population, Law Level, Spaceport Class, Civ
Level, resources available, locations of those resources, and the
effects of those resources on prices and economics. These indications
should be utilized to guide the GM in reaching his own final
conclusions about the world, its inner workings, and how each
settlement and citizen fits into the whole.

The GM conducts the following steps to determine a world's population
and technology:

\begin{enumerate}
\item Determine the settlement status, population total. Law Level,
  Spaceport Class, and Civ Level.
\item Determine the amount and type of resources and their
  availability on the world.
\item Assign the resources to various environs.
\item Assign the population to sites in the various environs. 
\end{enumerate}

\subsection[World Development]{The GM uses the World
  Development Table to determine the settlement status, human
  population, Law Level, Spaceport Class, and Civ Level.}

As explained on the table, the GM determines the Development Value
based on the world's atmosphere, temperature, and hydrograph
percentage. The GM rolls two dice and adds the Development Value to
the result. He then locates this sum on the table and records the
listed information on the World Log. For example, a resource rich
world with a thin atmosphere, 75\textdegree temperature, hydrograph of
40\% that is 20 light years from Sol would have a Development Value of
\textbf{11}. Rolling a \textbf{15} on two dice and adding the
Development Value of \textbf{11} would yield a result of \textbf{26},
which indicates a \textbf{minor state}, a population of \textbf{1
  billion}, a Law Level of \textbf{4}, a Spaceport Class of
\textbf{3}, and a Civ Level range of \textbf{6--8}. The GM should
choose an appropriate Civ Level from the range given.


\subsection{World Development Table}
\label{sec:world-devel-table}

See Table \vref{tab:world-development}.

\begin{table}[htbp]
  \small
  \centering
  \fbox{%
    \begin{minipage}{0.95\textwidth}
      \centering
      \caption{World Development Table}
      \label{tab:world-development}
      
      \medskip
      
      \begin{tabular}{rlrccccc}
        \multicolumn{1}{c}{TWO} & & \multicolumn{1}{c}{HUMAN} & LAW &
        SPACE & PATROL & SECURITY & CIV LEVEL\\
        \multicolumn{1}{c}{DICE} & \multicolumn{1}{c}{SETTLEMENT
          STATUS} & POP. & LEVEL & PORT & RATING & RATING & RANGE\\ 
        0 & Uncharted & 0 & 0 & 0 & 0 & 2 & None\\
        \rowcolor{grey}
        1 & Unexplored & 0 & 0 & 0 & 0 & 2 & None\\
        2 & Unexplored & 0 & 0 & 0 & 0 & 2 & None\\
        \rowcolor{grey}
        3 & Unexplored & 0 & 0 & 0 & 0 & 2 & None\\
        4 & Explored and Abandoned & 10 & 0 & 0 & 0 & 2 & 1--2\\
        \rowcolor{grey}
        5 & Explored and Abandoned & 100 & 0 & 0 & 0 & 2 & 1--2\\
        6 & Abandoned Pioneer Colony & 200 & 0 & 0 & 0 & 2 & 1--3\\
        \rowcolor{grey}
        7 & Active Exploration & 100 & 0 & 0 & 0 & 2 & 1--4\\
        8 & Active Exploration & 1000 & 1 & 0 & 1 & 3 & 2--4\\
        \rowcolor{grey}
        9 & Active Exploration & 2000 & 1 & 0.5 & 2 & 4 & 2--4\\
        10 & Pioneer Colony & 1000 & 1 & 0 & 1 & 3 & 1--4\\
        \rowcolor{grey}
        11 & Pioneer Colony & 10000 & 1 & 0.5 & 2 & 4 & 2--5\\
        12 & Pioneer Colony & 20000 & 2 & 0.5 & 3 & 5 & 2--5\\
        \rowcolor{grey}
        13 & Subsidized Scientific Colony & 10000 & 1 & 0.5 & 2 & 4 & 4--6\\
        14 & Subsidized Scientific Colony & 100000 & 2 & 1 & 3 & 5 & 4--6\\
        \rowcolor{grey}
        15 & Subsidized Scientific Colony & 200000 & 2 & 1 & 3 & 5 & 5--7\\
        16 & Subsidized Working Colony & 100000 & 2 & 1 & 3 & 5 & 3--5\\
        \rowcolor{grey}
        17 & Subsidized Working Colony & 1 Million & 2 & 2 & 4 & 6 & 4--6\\
        18 & Subsidized Working Colony & 2 Million & 3 & 2 & 5 & 7 & 4--6\\
        \rowcolor{grey}
        19 & Self-Sufficient Colony & 1 Million & 2 & 2 & 4 & 6 & 4--7\\
        20 & Self-Sufficient Colony & 10 Million & 3 & 2 & 5 & 7 & 5--7\\
        \rowcolor{grey}
        21 & Self-Sufficient Colony & 20 Million & 3 & 3 & 6 & 8 & 5--7\\
        22 & Full-Tech Colony & 10 Million & 3 & 2 & 5 & 7 & 5--8\\
        \rowcolor{grey}
        23 & Full-Tech Colony & 100 Million & 3 & 3 & 6 & 8 & 6--8\\
        24 & Full-Tech Colony & 200 Million & 4 & 3 & 7 & 9 & 6--8\\
        \rowcolor{grey}
        25 & Minor State & 100 Million & 4 & 3 & 7 & 9 & 6--8\\
        26 & Minor State & 1 Billion & 4 & 3 & 7 & 9 & 6--8\\
        \rowcolor{grey}
        27 & Minor State & 2 Billion & 4 & 4 & 8 & 10 & 7--8\\
        28 & Major State & 1 Billion & 4 & 4 & 8 & 10 & 7--8\\
        \rowcolor{grey}
        29 & Major State & 3 Billion & 4 & 4 & 8 & 10 & 8\\
      \end{tabular}
      \begin{tabular}{llcrcr}
        \multicolumn{2}{l}{MODIFIERS:}\\
        \multicolumn{2}{c}{WORLD ATMOSPHERE} & \multicolumn{2}{c}{WORLD
          TEMPERATURE} & \multicolumn{2}{c}{WORLD HYDROGRAPH}\\
        \rowcolor{grey}
        0 & None & --75, --50 & 0 & 0\% & --1\\
        2 & Thin Corrosive & --25 & 1 & 20\% & 2\\
        \rowcolor{grey}
        3 & Normal Corrosive & 0 & 3 & 40\% & 4\\
        4 & Thin Corrosive & 25 & 4 & 60\%--80\% & 5\\
        \rowcolor{grey}
        5 & Normal & 50, 75 & 5 & 100\% & 2\\
        1 & Poison & 100 & 4\\
        \rowcolor{grey}
        --1 & Corrosive & 125 & 3\\
        \multicolumn{2}{c}{} & 150 & 1\\
        \rowcolor{grey}
        \multicolumn{2}{c}{} & 175, 200 & --1
      \end{tabular}

      \medskip

      \parbox{\textwidth}{\textbf{Procedure:} Take the appropriate
        number from each of the columns and add them together. If the
        world is resource rich, double this total (\textbf{Exception:}
        If the total is negative, and the world is resource rich, divide
        the total by two, rounding toward zero.) Subtract one half
        (rounded down) of the world's star's distance from Sol from this
        new total.  The result is the Development Value. Roll two dice
        and add the Development Value to the roll, and apply this result
        to the Table. \textbf{Patrol Rating:} The frequency of
        Astroguard patrols. \textbf{Security Rating:} The probability of
        enforcement agencies discovering illegal commodities once a
        spacecraft has landed.}
    \end{minipage}}
\end{table}

\subsection[Settlement Status]{The settlement status and population
  indicate the general extent of human presence in the world.}
\label{sec:settlement-status}

The human settlement of a world may be in any of the following states: 

\begin{description}
\item[Uncharted.] Never mapped; unnamed.
\item[Unexplored.] Charted, but not traversed.
\item[Explored and Abandoned.] Surface has been traversed, but
  colonization never occurred; GM's discretion as to why not.
\item[Abandoned Pioneer Colony.] Colonization was started, but
  environment proved too harsh or resources dried up.
\item[Active Exploration.] Extant investigation by one or more groups;
  no governmental structure as yet.
\item[Pioneer Colony.] Just beginning to establish permanent
  population and develop resources; federation begins to take notice.
\item[Subsidized Scientific Colony.] Taking in much more raw material
  than it produces, but interest in the world's secrets and potentials
  make the output of knowledge a sufficient payoff. Federal presence
  exists.
\item[Subsidized Working Colony.] Takes in more than it produces, but
  shows promise of becoming a profitable commercial venture.  Federal
  presence.
\item[Self-sufficient Colony.] Stands on its own economically and
  accordingly draws the attention of the federation consistently.
\item[Full Tech Colony.] Commercial hub of a few systems, but has not
  yet been recognized as a minor state; the federation has not
  released the control it exercises.
\item[Minor State.] Commercial hub of a few systems, but federation
  has relinquished control to the point where the system operates more
  or less on its own.
\item[Major State.] Major commercial center of many systems.
  Federation does little except keep a watchful eye open.
\end{description}

The GM should use the human population figure as a guide. The total
number of humans indicated is an approximate figure that the GM
should adjust as he sees fit. This population may be divided into any
social and/or political factions that are consistent with the world's
settlement status.

The languages that any population may speak, are chosen by the GM.
As is mentioned in \ref{sec:universe-future}, there are a variety of
languages, and this choice should be integrated into the overall
flavor of the settlement.

It is to be assumed that any official federation representatives will
speak Universal, as will most local law enforcement and governmental
officials.

\subsection[Law Level]{The degree to which the federal laws are
  enforced is indicated by the Law Level of the world.}
\label{sec:law-level}

Throughout the universe, the laws have remained the same. The actions
that are criminal in one sector are criminal to the same degree in
another. What varies is the intensity and quality of enforcement and
the way punishment is meted out. The Law Levels and their effects
follow:

\begin{description}
\item[1.] A cavalier attitude toward justice; the maximum punishment
  for capital offenses is incarceration for a short time. Very often a
  fine is the only penalty prescribed. Illegal weapons or items are
  simply confiscated. Very few enforcement agents are present.
\item[2.] The quality of justice varies with the individual agent. The
  maximum punishment for capital offenses is heavy fining and in-
  carceration. Illegal weapons or items may result in imprisonment.
  The number and quality of law enforcement agents rises.
\item[3.] The judicial system is fair and reasonably accurate. The
  maximum punishment for capital offenses varies in proportion to the
  illegal act. Illegal weapons or items will result in imprisonment.
  Enforcement agents are seen regularly.
\item[4.] The search for truth and justice supersedes all else.
  Maximum punishment for capital offenses is death. Illegal weapons or
  items will result in long term imprisonment. Highly intelligent
  enforcement agents abound.
\item[5.] Strict adherence to judicial codes and practices results in
  accurate justice meted out swiftly. Maximum punishment for capital
  offenses is death. Possession of illegal weapons or items may be
  classified as a capital offense. Enforcement is of the highest
  quality and training. This level occurs only in Class 4 Spaceports
  and on Earth.
\end{description}

The Law Level of a world also influences the distribution of
encounters concerning law enforcement agents.

The \textbf{Enforcer Encounter Table} (page
\pageref{tab:enforcer-encounter}) indicates how often an encounter
should be with a local law enforcer of some type. These agents
represent local authorities who will be inspecting parcels, luggage,
equipment; looking out for trespassers; protecting the rights of
citizens; apprehending criminals; and so forth. See
\ref{sec:creating-encounters} and \emph{Adventure Guide} section
\ref{AG-sec:non-play-char}.


\begin{table}[htbp]
  \centering
  \fbox{%
    \begin{minipage}{5in}
      \centering
      \caption{Enforcer Encounter Table}
      \label{tab:enforcer-encounter}
      
      \medskip
      
      \parbox{2in}{\begin{tabular}{cl}
          LAW \\
          LEVEL & FREQUENCY \\
          \rowcolor{grey}
          0 & No Authorities \\
          1 & 1 out of 8 encounters \\
          \rowcolor{grey}
          2 & 1 out of 5 encounters \\
          3 & 1 out of 4 encounters \\
          \rowcolor{grey}
          4 & 1 out of 3 encounters \\
          5 & 1 out of 2 encounters 
        \end{tabular}}
      \hspace{0.25in}
      \parbox{2in}{The Enforcer Encounter Table indicates how 
        often an encounter should be with a local law 
        enforcer of some type. These agents represent 
        local authorities who will be inspecting parcels, 
        luggage, equipment; looking out for trespassers; 
        protecting the rights of citizens; apprehending 
        criminals; and so forth. 

        See \ref{sec:creating-encounters} and \emph{Adventure Guide}
        \ref{AG-sec:non-play-char}.}
    \end{minipage}}
\end{table}

\subsection*{Patrol and Security Rating. (Optional)}

Derived from SPI's \emph{Star Trader}: \textbf{Patrol Rating:} The
frequency of Astroguard patrols, the sum of the Spaceport Class and
Law Level. \textbf{Security Rating:} The probability of enforcement
agencies discovering illegal commodities once a spacecraft has landed,
the sum of the Spaceport Class and Law Level found above, plus
\textbf{2}. Both represent the probability out of 10 and the GM can
modify the chance of evasion as he sees fit.

\subsection[Spaceport Class]{The Spaceport Class represents the
  sophistication of facilities available for spaceship maintenance.}
\label{sec:spaceport-class}

Spaceports orbit around the world they serve, acting much in the same
fashion as the 20th Century airport. Worlds with a Spaceport Class of
$\frac12$, however, have no orbital station; rather, they have a
landing strip on the surface. Thus, ships that are not streamlined,
cannot land there. Orbital stations have a shuttle service to the
world's surface; the number of flights per day equals the square of
the Spaceport Class.

The Spaceport Class also affects the degree of trade and commerce a
world can engage in. The Spaceport Classes are:

\begin{description}
\item[0.] No facility whatsoever.
\item[$\frac12$.] Landing strip on the world surface. Energy for
  emergency use only. No security force stationed. Administered by one
  or two people. No repair service at all.
\item[1.] Energy is sometimes (50\% chance) available. No repair
  service, A small federal detachment administers (10--15 people).
\item[2.] Energy is usually (85\%) available. May repair superficial
  or light damage to pods and spaceships; a few used ships and pods
  may be available. No hyperjump maintenance. A full federal customs
  and security detachment (50--150 people). If it is the major port in
  the system, it includes an Astroguard patrol squadron.
\item[3.] Energy is always available. May repair superficial, light,
  or heavy damage to spaceships or pods; many standard ship types and
  pods are available. Hyperjump maintenance available at a Psionic
  Institute (see \ref{sec:psiinstitute}). Reinforced federal customs
  security, and 
  administrative force (200--500 people). Astroguard patrol squadron
  present. If the major port in a system, it also includes a federal
  navy force.
\item[4.] Energy always available. May repair any type of damage; full
  ship construction available. Hyperjump maintenance is available at a
  Psionic Institute (see \ref{sec:psiinstitute}). It is the center of
  federal activity; endless customs, security, and administrative
  forces (500--2,000 people).  Astroguard command post. Federal naval
  command (fleet headquarters).
\item[5.] \textbf{Earth.} Same as Class 4, but also includes the
  federal headquarters from which all military forces are
  administered.
\end{description}

If a spaceport is the highest class in a system, it is also considered
the center of the federal administration of the system itself. The
population of a world includes the population inhabiting the
spaceport.

The Spaceport Class affects the type of trading route existing between
the world and other worlds within the system, and between the system
and other star systems (see Chapter \ref{cha:space-travel}).

\subsection[Civ Level]{The Civ Level of a world indicates the degree
  of that settlement's contribution to the federation.}
\label{sec:civ-level}

The Civ Level of a world corresponds roughly to centuries in Earth's
past and indicates the level of industrial output of the world. It
does not necessarily indicate the sophistication of the population,
nor does it reflect the intelligence of the individuals living on the
world. A scientific colony, for instance, would have all the latest
equipment, but would not be able to survive if the equipment broke
down; they need their technology imported.

\pagebreak[2]

The Civ Levels and their corresponding Earth Centuries are: 

\begin{quote}
  Level 1 (1600) \\
  Level 2 (1700) \\
  Level 3 (1800) \\
  Level 4 (1900) \\
  Level 5 (2000) \\
  Level 6 (2100) \\
  Level 7 (2200) \\
  Level 8 (2300)
\end{quote} 

Any experimental equipment or scientific breakthroughs developing
during play would be considered Civ Level 9.

Most individuals found on any world will be aware that high-tech items
exist, and such items may be found on those worlds. However, in order
to maintain or produce those items, the world must be of an equivalent
or higher Civ Level.

\subsection[World Resource Table]{The GM uses the World Resource
  Table to determine the world's resources and then assigns them to
  various environs.}
\label{sec:world-resource-table}

The GM rolls percentile dice the number of times indicated and applies
any modifiers indicated on the Table. Every resource generated should
be recorded on the World Log, along with the number of environs in
which the resource appears. Note that rolling a resource twice
indicates that resource is \emph{abundant}; rolling it once indicates
the resource is \emph{limited}. A resource cannot be rolled more than
twice; if one is generated a third time, the dice are re-rolled.
Rolling a site (S) listing twice equals one environ listing.

After the correct number of rolls have been conducted for the world,
the resources are placed in the various environs of the world or at
sites as indicated. All placements are the province of the GM. Once
the resources are placed, the lettered results of each resource
generated are examined. Every lettered result for the world's Civ
Level and all lower Civ Levels apply for that resource on this world.
These lettered results are explained on the World Resource Table. If
no letters exist for a resource at a given Civ Level, that resource
has not been discovered.

\textbf{Example:} One of the 13 rolls on an Earth-like size
\textbf{5}, resource rich world results in chromium existing in one
environ. The Civ Level of the world is \textbf{5}, so the lettered
results \textbf{A}, \textbf{S}, \textbf{R}, and \textbf{D} apply to
chromium there.

The GM should use the explanation of the lettered results as both
factual information and as guidelines concerning the industrial output
of the world, in terms of what the world does and does not produce.
The explanations will guide the GM in general, and do imply specific
incidents in the history of the world. Logic should rule all ambiguous
situations, and the world's consistency should be maintained.

For example, using the world generated above, chromium lettered
results indicate the resource has been discovered in every environ in
which it occurs (as per result \textbf{A}); the \textbf{D} result is
superseded by the A result; chromium has been refined in every environ
in which it occurs (result \textbf{R}). Also, if iron is available,
chromium has been used to manufacture Level 5 impact armor (result
\textbf{S}). If the world settlement status was that of a subsidized
scientific colony, for example, the GM would have to decide how much
(if any) impact armor is being manufactured and whether the armor can
be repaired there. Given the nature of a subsidized scientific colony,
the answer might be that enough is manufactured for repair only.

The resources and Civ Level on a world affect what products are
available on the world and what prices are asked for those items (see
\ref{sec:economic-guidelines}). The World Resource Table lists prices
for all resources in a refined state. For trading purposes, it is
wiser to sell products or resources to worlds where their availability
is limited or nil. Prices for resources in a raw state are one-half to
one-quarter their price in a refined state.


\subsection{World Resource Table}
\label{sec:world-resource-table-1}

See Table \vref{tab:world-resource}.

\begin{table}[htbp]
  \scriptsize
  \centering
  \fbox{%
    \begin{minipage}{0.93\textwidth}
      \centering
      \caption{World Resource Table}
      \label{tab:world-resource}
      
      \medskip
      
      \begin{tabular}{lrlccccccccccc}
&&& \multicolumn{8}{c}{CIV LEVEL OF WORLD} & \multicolumn{3}{c}{\# OF
  ENVIRONS}\\
\rotate{PERCENTILE DICE} & PRICE & RESOURCE & 1&2&3&4&5&6&7&8 &
\rotate{EARTHLIKE} & \rotate{TOLERABLE} & \rotate{HOSTILE}\\
\rowcolor{grey}
1--8 & 0.5/T & Iron & A & R & M, JJ & - & - & - & - & - & - & - & 7\\
9--14 & 2.0/T & Aluminum & - & - & A , R & L, G & - & - & - & - & - & - & 6\\
\rowcolor{grey}
15--19 & 40.0/T & Radioactives & - & - & - & D & A, F & - & - & - & - & - & 5\\
20--23 & 3.5/T & Copper & D, R & - & A & - & - & X & - & KK & - & - & 4\\
\rowcolor{grey}
24--26 & 10.0/T & Chromium & - & - & D & S,R & A & NN & - & - & - & - & 3\\
27--29 & 0.5/K & Silver & D & - & A & P & - & - & - & - & - & 3 & 3\\
\rowcolor{grey}
30 & 1.0/G & Gold & D & - & - & - & - & AA & - & - & - & 1 & 1\\
31 & 1.0/G & Platinum & - & D & A & R & C & - & - & - & - & 1 & 1\\
\rowcolor{grey}
32--33 & 10.0/T & Titanium & - & - & - & D & Z,J & A,Y & - & - & - & 2 & 2\\
34--35 & 50.0/T & Cesium & - & - & - & D & F,H & A & - & - & - & 1 & 1\\
\rowcolor{grey}
36--37 & To 21/T & Other Metals\footnotemark[1] & - & - & D & - & - & A,N & - & KK & - & 2 & 2\\
38--40 & 15.0/T & Ammonia & A & - & - & K & LL & - & - & - & - & 1 & 1\\
\rowcolor{grey}
41 & 2.0/G & Magnetic Monopoles & - & - & - & - & - & E & D,V & A & - & S & S\\
42 & 1.0/G & Crystals & - & D & - & - & - & A,MM & - & - & - & S & S\\
\rowcolor{grey}
43--45 & 1.0/T & Phosphorus & - & - & - & D,R & A,W & - & - & - & - & 3 & 3\\
46--47 & 2.0/T & Germanium & - & - & - & D,R & - & A,EE & - & - & - & 2 & 2\\
\rowcolor{grey}
48 & 2.0/T & Silicon & - & - & D & R,CC & A & - & - & - & - & 2 & 2\\
49--50 & To 4/T & Other Non-Metals\footnotemark[2] & - & - & - & D,R & BB & - & A,V & - & - & 3 & 3\\
\rowcolor{grey}
51--57 & 0.5/T & Iron & A & R & M,JJ & - & - & - & - & - & 4 & 4 & 7\\
58--63 & 2.0/T & Aluminum & - & - & A,R & L,G & - & - & - & - & 3 & 3 & 6\\
\rowcolor{grey}
64--68 & 40.0/T & Radioactives & - & - & - & D & A,F & - & - & - & 2 & 2 & 5\\
69--72 & 3.5/T & Copper & D,R & - & A & - & - & X & - & KK & 2 & 2 & 4\\
\rowcolor{grey}
73--75 & 10.0/T & Chromium & - & - & D & S,R & A & NN & - & - & 1 & 2 & 3\\
76--78 & 0.5/K & Silver & D & - & A & P & - & H,MM & - & - & 1 & 3 & 3\\
\rowcolor{grey}
79 & 1.0/G & Gold & D & - & - & - & - & AA & - & - & S & 1 & 1\\
80 & 1.0/G & Platinum & - & D & A & R & C & - & - & - & S & 1 & 1\\
\rowcolor{grey}
81--82 & 10.0/T & Titanium & - & - & - & D & Z,J & A,Y & - & - & 1 & 2 & 2\\
83 & 50.0/T & Cesium & - & - & - & D & F,H & A & - & - & S & 1 & 1\\
\rowcolor{grey}
84--85 & To 21/T & Other Metals\footnotemark[1] & - & - & D & - & - & A,N & - & KK & 1 & 2 & 2\\
86--88 & 1.0/T & Phosphorus & - & - & - & D,R & A,W & - & - & - & 1 & 3 & 3\\
\rowcolor{grey}
89--90 & 2.0/T & Germanium & - & - & - & D,R & - & A,EE & - & - & 1 & 2 & 2\\
91--93 & 2.0/T & Silicon & - & - & D & R,CC & A & - & - & - & 1 & 2 & 2\\
\rowcolor{grey}
94--98 & To 4/T & Other Non-Metals\footnotemark[2] & - & - & - & D,R & BB & H,MM & A,V & - & 2 & 3 & 3\\
99 & 1.0/T & Exotic Spices & - & A & LL & - & - & - & - & S & S & S\\
\rowcolor{grey}
100--104 & 5.0/T & Organic Chemicals &&&&&&&&&&&\\
\rowcolor{grey}
& & (carbon) & - & - & D,FF & A,U & - & - & - & - & 3 & 3 & S\\
105--109 & 1.0/T & Organic Chemicals &&&&&&&&&&&\\
& & (nitrogen) & - & D & A,HH & - & - & - & - & - & 3 & 3 & -\\
\rowcolor{grey}
110--113 & 3.0/T & Light Fiber Plants & D,T & A & - & - & - & - & - & - & 2 & 2 & -\\
114--119 & 0.8/T & Wood-like Plants & D,Q & A & FF & - & - & - & - & - & 3 & 3 & -\\
\rowcolor{grey}
120--122 & Variable & Arable Land & D & A,GG & - & B & - & - & - & - & 7 & 2 & -\\
123--124 & To 5.0/T & Edible Plants & - & D,GG & A,B & - & - & - & - & - & 5 & 1 & -\\
\rowcolor{grey}
125 & To 1.0/T & Edible Game & D,DD & - & A & - & - & - & - & - & 3 & S & -\\
126--137 & Variable & Arable Land & D & A,GG & - & B & - & - & - & - & 7 & - & -\\
\rowcolor{grey}
138--146 & To 5/T & Edible Plants & - & D,DD & A,B & - & - & - & - & - & 5 & - & -\\
147--150 & To 1/T & Edible Game & D,DD & - & A & - & - & - & - & - & 3 & - & - 
      \end{tabular}

      \medskip

      \parbox{\textwidth}{\textbf{Modifiers:} Add \textbf{25} if world
        is Tolerable; add \textbf{50} if world is Earth-Like.\\ 
        Roll a number of time equal to the World Size if resource
        poor; roll a number of times equal to the World Size
        \textbf{+8} if resource rich.\\
        \textbf{Price:} Given in thousands of Credits per gram, kilo,
        or ton. \textbf{To \#/T:} Price fluctuates from that figure,
        down 50\%.\\
        \textbf{Variable:} Price varies extremely, depending on
        availability on the world.
        
        \textbf{Notes:} \textbf{1.} These include such metals as
        adamantine, beryllium, erbium, gadolinium, lead, manganese,
        mercury, nickel, potassium, rubidium, strontium, tin, and
        zinc. \textbf{2.} These include such non-metals as argon,
        barium, chlorine, cobalt, fluorine, helium, iodine, krypton,
        sulphur, and xenon. \textbf{S:} Site.

        See \ref{sec:world-resource-table} and page
        \pageref{tab:world-resource-explanation} for detailed
        explanation of use.}
    \end{minipage}}
\end{table}

\begin{table}[htbp]
  \small
  \centering
  \fbox{%
    \begin{minipage}{0.95\textwidth}
      \caption{World Resource Table Explanation Of Results}
      \label{tab:world-resource-explanation}
      
      \medskip\centering
      
      \begin{minipage}{0.95\textwidth}
        \begin{center}
          \parbox{0.95\textwidth}{After determining the resources for a world,
            the Civ Level of the world is used to determined the
            development of those resources.  Any lettered listing under
            that world's Civ Level or a lesser Civ Level applies to the
            resource. Results that indicate a product is available do not
            imply all products of that generic type are readily available;
            the Civ Level of the item in question must still be less than
            or equal to the Civ Level of the world. For example, lettered
            result E indicates psionic equipment is available; however ,
            if the Civ Level of the world was less than 8, an interstellar
            Commlink would not be available. For some lettered results,
            the item is readily available only if certain other resources
            are available. These resources are listed with the item. Also
            listed with the item is the resource from which it came.}
        \end{center}
        
        \medskip
        
        \begin{multicols}{3}
          \textbf{A:} The resource is automatically 
          discovered wherever it occurs on the 
          world.

          \textbf{B:} Vegetables and Fruit. 
          \textbf{Resource:} Edible Plants, Arable Land.

          \textbf{C:} All chemistry equipment, if Iron is also 
          available. 
          \textbf{Resource:} Platinum. 

          \textbf{D:} The resource has been discovered in 
          some (approximately half) of the 
          environs in which it occurred; the GM 
          should decide which environs.

          \textbf{E:} Psionic equipment (including Jump 
          Pods, Augmented Jump Pods, Hunter 
          Pods, Explorer Pods, and Psionic Rigs), 
          if Iron is also available. 
          \textbf{Resource:} Magnetic Monopoles. 

          \textbf{F:} Fuel for sub-light drives and fission 
          power plants, if Iron is also available. 
          \textbf{Resource:} Radioactives. 

          \textbf{G:} All non-jet aircraft (including gliders, 
          propeller planes, and helicopters), if Iron 
          is also available. 
          \textbf{Resource:} Aluminum. 

          \textbf{H:} Holographic equipment, if Iron is also 
          available. 
          \textbf{Resource:} Other Non-Metals, Cesium. 

          \textbf{J:} Jet air vehicles, if Iron is also 
          available. 
          \textbf{Resource:} Titanium. 

          \textbf{K:} Fertilizers. 
          \textbf{Resource:} Ammonia. 

          \textbf{L:} Ground and Marine Vehicles, if Iron is 
          also available. 
          \textbf{Resource:} Aluminum. 

          \textbf{M:}Machine Tools and Tech Kits. 
          \textbf{Resource:} Iron. 

          \textbf{N:} Unarmored Spacecraft Hulls and Pods, if 
          Iron is also available. 
          \textbf{Resource:} Other Metals. 

          \textbf{P:} Recording Equipment with the exception of 
          holographic equipment, See H), if Iron is also 
          available. 
          \textbf{Resource:} Silver. 

          \textbf{Q:} Wood Products and structures. 
          \textbf{Resource:} Wood-Like Plants. 

          \textbf{S:} All Impact Armor, if Iron is also available. 
          \textbf{Resource:} Chromium. 

          \textbf{T:} Textiles. 
          \textbf{Resource:} Light Fiber Plants. 

          \textbf{U:} Vision equipment and other plastics, if Iron is 
          also available. 
          \textbf{Resource:} Organic Chemicals (carbon). 

          \textbf{V:} Robot Hardware and Software if Iron is also 
          available. 
          \textbf{Resource:} Magnetic Monopoles, Other Non-Metals. 

          \textbf{W:}Artillery weapons and Explosives, if Iron is 
          also available. 
          \textbf{Resource:} Phosphorus. 

          \textbf{X:} Robot Chassis, if Iron is also available. 
          \textbf{Resource:} Copper. 

          \textbf{Y:} Armored Spacecraft, if Iron is also available. 
          \textbf{Resource:} Titanium. 

          \textbf{Z:} Armored ground vehicles, if Iron is also 
          available. 
          \textbf{Resource:} Titanium. 

          \textbf{AA:} All Body Armor, if Iron is also available. 
          \textbf{Resource:} Gold. 

          \textbf{BB:} Computer Components and 
          Software (including Robot Software), 
          if Iron is also available. 
          \textbf{Resource:} Other Non-Metals. 

          \textbf{CC:} Communications equipment, if 
          Iron is also available. 
          \textbf{Resource:} Silicon. 

          \textbf{DD:} Meat. 
          \textbf{Resource:} Edible Game. 

          \textbf{EE:} All Scientific equipment (except 
          Chemistry equipment, see C), if Iron 
          is also available. 
          \textbf{Resource:} Germanium. 

          \textbf{FF:} Fossil Fuels. 
          \textbf{Resource:} Wood-Like Plants, 
          Organic Chemicals (nitrogen). 

          \textbf{GG:} Grain. 
          \textbf{Resource:} Edible Plants, Arable 
          Land. 

          \textbf{HH:} Explosives, Cartridge 
          Ammunition, Grenades, if Iron is also 
          available. 
          \textbf{Resource:} Organic Chemicals 
          (nitrogen). 

          \textbf{JJ:} Projectile Weapons. 
          \textbf{Resource:} Iron. 

          \textbf{KK:} Force fields of all types, if Iron is 
          also available. 
          \textbf{Resource:} Copper, Other Non-Metals. 

          \textbf{LL:} Drugs and Poisons. 
          \textbf{Resource:} Crystals, Other Non-Metals. 

          \textbf{MM:}Beam Weapons, if Iron is also 
          available. 
          \textbf{Resource:} Crystals, Other Non-Metals. 

          \textbf{NN:} Expedition Suits, if Iron is also 
          available. 
          \textbf{Resource:} Chromium. 
        \end{multicols}
      \end{minipage}

      \vspace{0.05\textwidth}

    \end{minipage}}
\end{table}


\subsection[Population Distribution]{The GM assigns portions of the
        population to various environs on the world.}
\label{sec:population-dist}


The GM assigns population to the environs of the world in any way he
sees fit, so long as the total population assigned to all the environs
equals the population as indicated on the World Development Table. The
population of an environ is recorded by placing a number from 0 to 9
in the environ. This number represents a power of \textbf{10}. Thus,
if a 5 were recorded in a space, its population would be 100,000. A
population level between one exponent and the next may be recorded by
writing a multiple before the exponent. Thus, 3/5 would represent
300,000. While assigning population to a world, the following
restrictions must be adhered to.

\begin{itemize}
\item No more than 100 people may be placed in an environ with no
  vegetation and no water.
\item No more than 1,000 people may be placed in a 100\% water
  environ.
\item No more than 100,000 people may be placed in a water with minor
  land masses environ.
\item No more than 10 million people may be placed in a water with
  major land masses environ.
\item No more than 100 million people may be placed in a water/land
  environ.
\item No more than 1 billion people may be placed in a land with major
  water environ.
\item No more than 10 million people may be placed in a land with
  minor water environ.
\item No more than 10,000 people may be placed in a land with
  vegetation but no water environ.
\end{itemize}

The GM should relate the population centers on the world to various
resource concentrations of sites of interest in a logical, consistent
fashion.

%%% Local Variables:
%%% mode: latex
%%% TeX-master: "gm_guide"
%%% End:
