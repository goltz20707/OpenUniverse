%%
%% $Id: space_travel.tex,v 1.4 2004/10/15 15:43:39 goltz20707 Exp $
%%
\chapter{\hspace{5mm}Space travel}
\label{cha:space-travel}

There are two distinct methods of travelling between worlds.
Hyperdrive uses the Mental Power of a psionic navigator to travel
instantaneously from star system to star system. Unfortunately,
hyperdrive will not work when a spaceship is within the gravity well
of a star system. Thus, slower-than-light reaction drive (fission) is
used to travel from planet to planet within a star system. A character
conducting a hyperjump uses his navigator skill (this skill is not
used to navigate a spaceship through interplanetary space). Pilot
skill is used to control a spaceship through interplanetary space.

All the spaceships in \emph{Universe} are composed of a \emph{hull}
and a variable number of \emph{pods}. The hull determines the ship's
size and overall performance. The pods are attached to the hull and
give the ship a specific character: military, scientific, merchant,
passenger, etc. This concept is more fully explained in \emph{Delta
  Vee} Sections \ref{DV-sec:spacecraft} through \ref{DV-sec:pods-dv}.

The Gamesmaster creates spaceships from the various hulls and pods and
introduces them into play so that the characters may travel in them,
encounter them and, if wealthy enough, purchase them.

Two astronomical terms are used in this Chapter: 

\begin{description}
\item[Astronomical Unit (AU).] A standard measure of distance in
  interplanetary space. One AU equals 149 million km, the distance
  from Sol to Earth.
\item[Ecliptic.] The plane formed by a star and the orbital paths of
  all the planets around it. The mass of the worlds scattered around
  the ecliptic create a flattened sphere of gravity wells that may
  not be entered when hyper-jumping.
\end{description}

\section{Hulls and Pods}
\label{sec:hulls-pods}



Spaceship hulls and pods are manufactured on worlds with \textbf{Class
3} and \textbf{4} spaceports, and in orbiting factories
attached to such spaceports. Like all other manufactured goods,
spaceship parts come in a variety of Civ Levels (6,7,8) and will only
be readily available on a world of an equal or higher Civ Level.
Spaceships are far beyond the financial means of most people and,
thus, the characters will be little concerned with the economics of
spaceship construction and purchase. However, the characters will be
able to book passage in commercial spaceships and during an adventure
will sometimes receive free transport by an employer or may even be
loaned a ship for the completion of a specific mission.

The GM uses the list of hulls and pods in this Section and the
additional information in the \emph{Delta Vee} rules (especially
concerning military spaceships) to create spaceships suited to his
adventures. The hull/pod system allows the GM to design a wide variety
of ships with a minimum of trouble. When constructing a ship, the GM
should keep in mind the specifications of each hull and pod,
especially their availability. Civ Level, and crew requirements, so
that, when assembled, the ship may be logically (and legally) used by
the characters.

\subsection[Hull Classes]{Every class of spaceship hull contains the
  following:} 
\label{sec:hull-classes}

Each hull possesses a sub-light engine
(using radioactives as its energy source, see \ref{sec:energy-blocks}), a
bridge with basic navigation and communications equipment, living
quarters and first aid station for a crew necessary to keep the craft
running, an airlock leading out of the ship, and a docking port for
rendezvous with other craft, and all the attributes listed in
\emph{Delta Vee} \ref{DV-sec:spacecraft} (especially Table
\ref{DV-tab:spacecraft-attribute}, 
the Spaceship Attribute Chart). 


Four industrial concerns produce the spaceships: 

\textbf{Terwillicker Spaceworks, Inc.} manufactures the
\emph{Terwillicker-5000}, a high-quality two-person craft; and the
\emph{Terwillicker-X} fighter, an innovative adaptation of the 5000
designed for military use.

\textbf{Blades Research Institute} produces military craft under
long-term contract. The \emph{Dagger}, \emph{Sword}, and \emph{Spear}
Class ships are their most successful models.

\textbf{Harmonics, Inc.} specializes in finely crafted ships for
government and high-level corporate use. The \emph{Piccolo},
\emph{Flute}, and \emph{Clarinet} represent the top of their line.

The \textbf{Corco Group} manufactures a large line of commercial
vessels, often sacrificing performance for economy. The \emph{Gamma},
\emph{Zeta}, and \emph{Mu} Classes are well suited for transport in
safe regions. The \emph{Iota} is designed to appeal to merchants
working in dangerous areas.


Additional information is listed with each spaceship hull description
in \ref{sec:spacecraft-hulls}. This information includes the following:

\textbf{Availability} (Open, Restricted, or Closed). An \emph{open}
spaceship is available for purchase by anyone who has the funds. A
\emph{restricted} spaceship is available only with permission from the
federation or an independent world. Such ships, equipped with quality
defensive gear and light weapons, are preferred by merchant concerns
and government agencies operating in dangerous areas. A
\emph{military} spaceship is designed specifically for combat and is
available only to the federal navy, the Astroguard services, and the
transport branches of military ground forces.

\textbf{Crew Required.} The minimum number of crew members required to
keep the spaceship running and trouble-free, excluding gunner for the
ship's burster or any crew necessary to service any pods the ship
possesses.

\textbf{Passenger Capacity.} The maximum number of people the
spaceship may accommodate when no pods are attached. Accommodations
are basic: shared sleeping quarters, galley, and first aid station
(fulfils equipment requirement for diagnosis and treatment tasks,
provides no Skill Level increase). Any crew required to run the ship
takes up passenger space.

\textbf{Cargo.} The maximum metric tonnage of cargo that the spaceship can
transport. Cargo space may be increased by \textbf{0.1} ton per
passenger below passenger capacity carried. The cargo hold will not
sustain life.

\textbf{Cost.} Price in Trans for the spaceship hull when new if
purchased on a world of the same or higher Civ Level as the hull.
Price may fluctuate as described in \ref{sec:economic-guidelines}.

\textbf{Performance Modifier (PM).} A quantification of the ship's
responsiveness and structural integrity. Applied to the chance to
avoid an accident, in accordance with \ref{sec:vehaccidents}.

\textbf{Base Repair Time.} Spacecraft Engine, Bridge and Hulls have a
Base Repair Time of \textbf{24} hours while Spacecraft Pods and
Forcefields have a Base Repair Time of \textbf{12} hours.

\subsection[Spacecraft Hulls]{The following spaceship hulls are the
  classes most common 
  throughout the federation.}
\label{sec:spacecraft-hulls}


\paragraph{Terwillicker Spaceworks, Inc.}
\label{sec:hulls-terwillicker}


\begin{itemize}
\item\textbf{\hypertarget{tag:terwillicker-5000}{Terwillicker 5000}.}\\
  Favorite personal craft of wealthy individuals. Used as a light,
  short-range patroller by the Navy and the Astroguard, and as a
  courier by government and private concerns.  \textbf{Note:} This
  craft does not include a galley or first aid station.

\textbf{NUMBER OF PODS:} 0. \textbf{VELOCITY RATING:} 2.
\textbf{MANEUVER RATING:} 7. \textbf{ENERGY CAPACITY:} 15.
\textbf{ENERGY BURN RATE:} 1. \textbf{STREAMLINED:} \textbf{Yes.BURSTER
  CLASS:} 1. \textbf{ARMOR CLASS:} 1.  \textbf{FORCEFIELD CLASS:} 0.
\textbf{CIV LEVEL:} 7. \textbf{TARGET PROGRAM:} -2.
\textbf{AVAILABILITY:} Open. \textbf{CREW REQUIRED:} 1.
\textbf{PASSENGER CAPACITY:} 2. \textbf{CARGO CAPACITY:} 2.
\textbf{PERFORMANCE MODIFIER:} 0. \textbf{COST:} 3100.

\item\textbf{Lander.}  A modified version of the 5000 is used as a
  surface-landing vessel launched from the Lander pod of a larger
  spaceship. It has a Velocity Rating of 1, a Maneuver Rating of 4, no
  armor, burster, or targeting program.  \textbf{Note:} This craft
  does not include a galley or first aid station.
  
  \textbf{NUMBER OF PODS:} 0. \textbf{VELOCITY RATING:} 1.
  \textbf{MANEUVER RATING:} 4. \textbf{ENERGY CAPACITY:} 15.
  \textbf{ENERGY BURN RATE:} 1. \textbf{STREAMLINED:}
  Yes. \textbf{BURSTER CLASS:} 0. \textbf{ARMOR CLASS:} 0.
  \textbf{FORCEFIELD CLASS:} 0. \textbf{CIV LEVEL:} 8. \textbf{TARGET
    PROGRAM:} 0. \textbf{AVAILABILITY:} Open. \textbf{CREW REQUIRED:} 1.
  \textbf{PASSENGER CAPACITY:} 4. \textbf{CARGO CAPACITY:} 0.5.
  \textbf{PERFORMANCE MODIFIER:} -5. \textbf{COST:} 1500.

\item\textbf{Terwillicker-X.}\\
  Short-range fighter craft, usually launched from the Battlecraft pod
  of a large military spaceship. Orbiting space stations and
  commercial complexes often use the \emph{X} for security and
  scouting.  \textbf{Note:} This craft does not include a galley or
  first aid station.
  
  \textbf{NUMBER OF PODS:} 0. \textbf{VELOCITY RATING:} 3.
  \textbf{MANEUVER RATING:} 9. \textbf{ENERGY CAPACITY:} 15.
  \textbf{ENERGY BURN RATE:} 1. \textbf{STREAMLINED:}
  Yes. \textbf{BURSTER CLASS:} 2. \textbf{ARMOR CLASS:} 2.
  \textbf{FORCEFIELD CLASS:} 0. \textbf{CIV LEVEL:} 8. \textbf{TARGET
    PROGRAM:} -4. \textbf{AVAILABILITY:} Restricted. \textbf{CREW
    REQUIRED:} 1.  \textbf{PASSENGER CAPACITY:} 2. \textbf{CARGO
    CAPACITY:} 0.1. \textbf{PERFORMANCE MODIFIER:} 25. \textbf{COST:}
  6900.
\end{itemize}


\paragraph{Blades Research Institute}
\label{sec:hulls-blades}

\begin{itemize}
\item\textbf{Dagger.}\\
Standard military patrol and pursuit craft. 

\textbf{NUMBER OF PODS:} 2. \textbf{VELOCITY RATING:} 2.
\textbf{MANEUVER RATING:} 6. \textbf{ENERGY CAPACITY:} 48.
\textbf{ENERGY BURN RATE:} 4. \textbf{STREAMLINED:} Yes. \textbf{BURSTER
  CLASS:} 2. \textbf{ARMOR CLASS:} 2.  \textbf{FORCEFIELD CLASS:} 1.
\textbf{CIV LEVEL:} 8. \textbf{TARGET PROGRAM:} -4.
\textbf{AVAILABILITY:} Military. \textbf{CREW REQUIRED:} 2. PASSENGER
\textbf{CAPACITY:} 4. \textbf{CARGO CAPACITY:} 0.5.
\textbf{PERFORMANCE MODIFIER:} +15. \textbf{COST:} 12200.


\item\textbf{Sword.}\\
  Elite heavy cruiser, mainstay of the federal navy.
  
  \textbf{NUMBER OF PODS:} 5. \textbf{VELOCITY RATING:} 3.
  \textbf{MANEUVER RATING:} 8. \textbf{ENERGY CAPACITY:} 78.
  \textbf{ENERGY BURN RATE:} 6. \textbf{STREAMLINED:}
  No. \textbf{BURSTER CLASS:} 2. \textbf{ARMOR CLASS:} 2.
  \textbf{FORCEFIELD CLASS:} 2. \textbf{CIV LEVEL:} 8. \textbf{TARGET
    PROGRAM:} -4. \textbf{AVAILABILITY:} Military. \textbf{CREW
    REQUIRED:} 5. PASSENGER \textbf{CAPACITY:} 10. \textbf{CARGO
    CAPACITY:} 3. \textbf{PERFORMANCE MODIFIER:} +25. \textbf{COST:}
  22100.
  

\item\textbf{Spear.}\\
  Military command post. Often used as the core spaceship of a federal
  navy task force or as an Astroguard headquarters.
  
  \textbf{NUMBER OF PODS:} 8. \textbf{VELOCITY RATING:} 1.
  \textbf{MANEUVER RATING:} 4. \textbf{ENERGY CAPACITY:} 144.
  \textbf{ENERGY BURN RATE:} 12. \textbf{STREAMLINED:}
  No. \textbf{BURSTER CLASS:} 2. \textbf{ARMOR CLASS:} 2.
  \textbf{FORCEFIELD CLASS:} 2. \textbf{CIV LEVEL:} 8. \textbf{TARGET
    PROGRAM:} -4. \textbf{AVAILABILITY:} Military. \textbf{CREW
    REQUIRED:} 10. PASSENGER \textbf{CAPACITY:} 20. \textbf{CARGO
    CAPACITY:} 7. \textbf{PERFORMANCE MODIFIER:} +10. \textbf{COST:}
  27900.
\end{itemize}


\paragraph{Harmonics Inc.}
\label{sec:hulls-harmonics}

\begin{itemize}
\item\textbf{\hypertarget{tag:harmonics-piccolo}{Piccolo}.}\\
  Common high-performance craft. Popular with small businesses,
  independent explorers.
  
  \textbf{NUMBER OF PODS:} 1. \textbf{VELOCITY RATING:} 3.
  \textbf{MANEUVER RATING:} 8. \textbf{ENERGY CAPACITY:} 30.
  \textbf{ENERGY BURN RATE:} 3. \textbf{STREAMLINED:}
  Yes. \textbf{BURSTER CLASS:} 1. \textbf{ARMOR CLASS:} 1.
  \textbf{FORCEFIELD CLASS:} 0. \textbf{CIV LEVEL:} 7. \textbf{TARGET
    PROGRAM:} -2. \textbf{AVAILABILITY:} Open. \textbf{CREW REQUIRED:}
  1. PASSENGER \textbf{CAPACITY:} 6. \textbf{CARGO CAPACITY:} 5.
  \textbf{PERFORMANCE MODIFIER:} +5. \textbf{COST:} 5400.
  

\item\textbf{Flute.}\\
  A heavily-defended craft used by many branches of the military
  (especially when a low profile is desired) and by corporations
  operating in dangerous areas.
  
  \textbf{NUMBER OF PODS:} 4. \textbf{VELOCITY RATING:} 3.
  \textbf{MANEUVER RATING:} 6. \textbf{ENERGY CAPACITY:} 66.
  \textbf{ENERGY BURN RATE:} 6.  \textbf{STREAMLINED:}
  Yes. \textbf{BURSTER CLASS:} 1. \textbf{ARMOR CLASS:} 2.
  \textbf{FORCEFIELD CLASS:} 1.  \textbf{CIV LEVEL:} 8. \textbf{TARGET
    PROGRAM:} -4. \textbf{AVAILABILITY:} Restricted. \textbf{CREW
    REQUIRED:} 3.  \textbf{PASSENGER CAPACITY:} 12. \textbf{CARGO
    CAPACITY:} 3. \textbf{PERFORMANCE MODIFIER:} +20. \textbf{COST:}
  20700.
  

\item\textbf{Clarinet.}\\
  The premier deep-space trading vessel. Its size, economy, and combat
  adaptability make it popular with interstellar traders and pirates.
  
  \textbf{NUMBER OF PODS:} 7. \textbf{VELOCITY RATING:} 2.
  \textbf{MANEUVER RATING:} 7. \textbf{ENERGY CAPACITY:} 104.
  \textbf{ENERGY BURN RATE:} 8. \textbf{STREAMLINED:}
  No. \textbf{BURSTER CLASS:} 1. \textbf{ARMOR CLASS:} 1.
  \textbf{FORCEFIELD CLASS:} 0. \textbf{CIV LEVEL:} 8. \textbf{TARGET
    PROGRAM:} -4. \textbf{AVAILABILITY:} Open. \textbf{CREW REQUIRED:}
  4. PASSENGER \textbf{CAPACITY:} 20. \textbf{CARGO CAPACITY:} 6.
  \textbf{PERFORMANCE MODIFIER:} +10. \textbf{COST:} 14100.
\end{itemize}


\paragraph{Corco Group}
\label{sec:hulls-corco}


  
\begin{itemize}
\item\textbf{Corco Gamma.}\\
  Common small freight and passenger vessel.
  
  \textbf{NUMBER OF PODS:} 3. \textbf{VELOCITY RATING:} 1.
  \textbf{MANEUVER RATING:} 4. \textbf{ENERGY CAPACITY:} 54.
  \textbf{ENERGY BURN RATE:} 6. \textbf{STREAMLINED:}
  Yes. \textbf{BURSTER CLASS:} 1. \textbf{ARMOR CLASS:} 0.
  \textbf{FORCEFIELD CLASS:} 0. \textbf{CIV LEVEL:} 7. \textbf{TARGET
    PROGRAM:} -2. \textbf{AVAILABILITY:} Open. \textbf{CREW REQUIRED:}
  2.  \textbf{PASSENGER CAPACITY:} 8. \textbf{CARGO CAPACITY:} 2.
  \textbf{PERFORMANCE MODIFIER:} -10. \textbf{COST:} 6700.
  

\item\textbf{Corco Zeta.}\\
  One of the oldest commercial vessels, still widely used in safe
  areas.
  
  \textbf{NUMBER OF PODS:} 6. \textbf{VELOCITY RATING:} 1.
  \textbf{MANEUVER RATING:} 3. \textbf{ENERGY CAPACITY:} 80.
  \textbf{ENERGY BURN RATE:} 8. \textbf{STREAMLINED:}
  No. \textbf{BURSTER CLASS:} 1. \textbf{ARMOR CLASS:} 0.
  \textbf{FORCEFIELD CLASS:} 0. \textbf{CIV LEVEL:} 6. \textbf{TARGET
    PROGRAM:} 0. \textbf{AVAILABILITY:} Open. \textbf{CREW REQUIRED:} 4.
  \textbf{PASSENGER CAPACITY:} 20. \textbf{CARGO CAPACITY:} 5.
  \textbf{PERFORMANCE MODIFIER:} -20. \textbf{COST:} 6400.
  

\item\textbf{Corco Iota.}\\
  Economical, well-defended trading vessel designed for
  government-sponsored commerce.
  
  \textbf{NUMBER OF PODS:} 9. \textbf{VELOCITY RATING:} 2.
  \textbf{MANEUVER RATING:} 5. \textbf{ENERGY CAPACITY:} 120.
  \textbf{ENERGY BURN RATE:} 12. \textbf{STREAMLINED:}
  No. \textbf{BURSTER CLASS:} 1. \textbf{ARMOR CLASS:} 1.
  \textbf{FORCEFIELD CLASS:} 1. \textbf{CIV LEVEL:} 7. \textbf{TARGET
    PROGRAM:} -4. \textbf{AVAILABILITY:} Restricted. \textbf{CREW
    REQUIRED:} 4.  \textbf{PASSENGER CAPACITY:} 25. \textbf{CARGO
    CAPACITY:} 10. \textbf{PERFORMANCE MODIFIER:} 0. \textbf{COST:}
  17500.

\item\textbf{Corco Mu.}\\
Common large freighter and passenger vessel. 

\textbf{NUMBER OF PODS:} 12. \textbf{VELOCITY RATING:} 1.
\textbf{MANEUVER RATING:} 4. \textbf{ENERGY CAPACITY:} 176.
\textbf{ENERGY BURN RATE:} 16. \textbf{STREAMLINED:} No. \textbf{BURSTER
  CLASS:} 1. \textbf{ARMOR CLASS:} 0.  \textbf{FORCEFIELD CLASS:} 0.
\textbf{CIV LEVEL:} 7. \textbf{TARGET PROGRAM:} -2.
\textbf{AVAILABILITY:} Open. \textbf{CREW REQUIRED:} 5.
\textbf{PASSENGER CAPACITY:} 30. \textbf{CARGO CAPACITY:}
15. \textbf{PERFORMANCE MODIFIER:} -10. \textbf{COST:} 14500.


\item\textbf{Corco Omega.}\\
  Emergency craft launched from escape/EVA pod of a large spaceship.
  The \emph{Omega} contains a burnout hyperjump engine and must be
  replaced after one jump at a cost of \textbf{200} Trans.
  
  \textbf{NUMBER OF PODS:} 0. \textbf{VELOCITY RATING:} 1.
  \textbf{MANEUVER RATING:} 3. \textbf{ENERGY CAPACITY:} 10.
  \textbf{ENERGY BURN RATE:} 1. \textbf{STREAMLINED:} No.
  \textbf{BURSTER CLASS:} 0. \textbf{ARMOR CLASS:} 0.
  \textbf{FORCEFIELD CLASS:} 0.  \textbf{CIV LEVEL:} 7. \textbf{TARGET
    PROGRAM:} 0. \textbf{AVAILABILITY:} Open. \textbf{CREW REQUIRED:}
  1.  \textbf{PASSENGER CAPACITY:} 4.  \textbf{CARGO CAPACITY:} 0.2.
  \textbf{PERFORMANCE MODIFIER:} -15. \textbf{COST:} 1100.
\end{itemize}

\subsection[Additional Hulls]{The GM may design additional spaceship
  hulls to use during 
  play.}
\label{sec:additional-hulls}



Any hull designed should be given ratings and attributes comparable to
those found on the Spaceship Attribute Chart (\emph{Delta Vee} 
Table \ref{DV-tab:spacecraft-attribute}) and in
\ref{sec:spacecraft-hulls}. The 
spaceship may 
be made capable of carrying any number of pods. The Energy Burn Rate
of a spaceship should be 20\%to 50\%greater than the number of pods it
may carry. The cost of a spaceship hull is calculated by adding
together the costs of all the following attributes:

\begin{itemize}
\item 200 Trans x Velocity Rating x Number of pods. 
\item 100 Trans x Maneuver Rating x Number of pods. 
\item 100 Trans x Passenger capacity of hull. 
\item 10 Trans x Energy capacity of hull. 
\item 100 Trans for a Class 1 burster; 1,000 Trans for a Class 2 burster. 
\item Armor Class 1: 100 Trans x Number of pods. 
\item Armor Class 2: 1,000 Trans x Number of pods. 
\item Force Field Class 1: 200 Trans x Number of pods. 
\item Force Field Class 2: 1,000 Trans x Number of pods. 
\item Target Program: 500 Trans for every subtraction of 1. 
\item Cargo: 100 Trans per ton of capacity. 
\end{itemize}

If the spaceship is streamlined, increase all the preceding by 50\%.
These costs are based on the number of pods the ship is capable of
carrying, not the number it is actually carrying at any particular
time.

\subsection[Spacecraft Pods]{Each spaceship hull may carry a variable
  number of pods, 
  each containing a system that specializes or improves the ship's
  operation.}
\label{sec:pods}



The concept of how pods work and details on combat abilities of
certain pods can be found in \emph{Delta Vee} \ref{DV-sec:spacecraft}
and \ref{DV-sec:pods-dv}, especially in \ref{DV-tab:pod-attribute}
(the Pod Attribute Chart). The following pods are the types most
commonly used on spaceships.

Each description includes the Pod's Availability, the Crew Required to
operate the Pod, the Pod's Passenger Capacity and Cargo Capacity in
Tonnes, the Civ Level at which the Pod initially becomes available and
Cost in Trans.  These attributes are similar to those in
\ref{sec:spacecraft-hulls}.

Unless a pod is assigned a specific Armor Class by the Pod Attribute
Chart, it is considered to be Armor Class 0. At an additional cost of
50 Trans, a pod may be purchased at Armor Class 1. At an additional
cost of 400 Trans, a pod may be purchased at Armor Class 2. Any crew
required to run a pod without passenger capacity must be housed
elsewhere on the spaceship.

\begin{itemize}
\item\textbf{Arsenal.}\\
  Weapons Pod. Each Arsenal pod contributes one Battle Command to the
  ship's total. This pod may fire laser and particle bursts and
  barrages. This pod carries 8 Unguided, 7 Guided Missiles, 5
  Intelligent and 2 MIMS missiles with only the MIMS and Intelligent
  missiles requiring a Prepare Missile command before launching.
  
  \textbf{LASER/PARTICLE WEAPONS:} Yes. \textbf{UNGUIDED MISSILE:} 8.
  \textbf{GUIDED MISSILE:} 7. \textbf{INTELLIGENT MISSILE:} 5.
  \textbf{MIMS:} 2.  \textbf{HYPERJUMP:} No. \textbf{ARMOR:} 2. BATTLE
  \textbf{COMMANDS:} 1. \textbf{NUMBER OF FIRES:} 2. \textbf{TARGET
    PROGRAM:} -4.
  
  \textbf{Availability:} Military. \textbf{Crew Required:} 6.
  \textbf{Passenger Capacity:} 0.  \textbf{Cargo Capacity:} 0.
  \textbf{Civ Level:} 8. \textbf{Cost:} 5000.
  
\item\textbf{\hypertarget{tag:pod-augmented-jump}{Augmented Jump}.}\\
  Allows spaceship to hyperjump. Navigator required. Jump engine never
  requires overhaul. Skill Level of navigator increased by 2 during
  hyperjump.
  
  \textbf{Availability:} Restricted. \textbf{Crew Required:} 3
  (including navigator).  \textbf{Passenger Capacity:} 1 (private
  cabin for navigator, includes Interstellar Commlink). \textbf{Cargo
  Capacity:} 0.  \textbf{Civ level:} 8.
  \textbf{Cost:} 2,500.
  
  

\item\textbf{Battlecraft.}\\
  Docking, refuelling, and maintenance facilities for a Terwillicker
  5000, X or other Battlecraft. Does not come with the Battlecraft.
  See \emph{Delta Vee} .

  
  \textbf{Availability:} Open. \textbf{Crew Required:} 1.
  \textbf{Passenger Capacity:} 0. Cargo: 0. \textbf{Cost:} 500.
  \textbf{Civ Level:} 7.
  
  

\item\textbf{Battle Communications.}\\
  Allows one extra fire from any one pod or burster on the spaceship
  during the friendly Fire Phase. The player may conduct Active
  Searchmore effectively from the pod. The pod's Targeting Program
  allows a modifier of -6 for any laser or particle fire conducted
  from anywhere on the ship.
  
  \textbf{LASER/PARTICLE WEAPONS:} No. \textbf{UNGUIDED MISSILE:} 0.
  \textbf{GUIDED MISSILE:} 0.  \textbf{INTELLIGENT MISSILE:} 0.
  \textbf{MIMS:} 0.  \textbf{HYPERJUMP:} No. \textbf{ARMOR:} 2. BATTLE
  \textbf{COMMANDS:} 2. \textbf{NUMBER OF FIRES:} 1. \textbf{TARGET
    PROGRAM:} -6.

  
  \textbf{Availability:} Military. \textbf{Crew Required:} 8.
  \textbf{Passenger Capacity:} 1 (cabin suitable for the needs of the
  ship's commander).  Cargo Capacity: 0. \textbf{Civ Level:} 8.
  \textbf{Cost:} 3000.
  
  

\item\textbf{Buffered Cargo.}\\
  Pressurized, temperature controlled cargo hold. Facilities for
  fragile and organic (but not living) items included.

  
  \textbf{Availability:} Open. \textbf{Crew Required:} 1.
  \textbf{Passenger Capacity:} 0. Cargo: 20. \textbf{Cost:} 350.
  \textbf{Civ Level:} 7.
  
  

\item\textbf{Bio-Research.}\\
  Complete research laboratory for the study of alien life forms.
  Contains chambers for keeping living specimens in their natural
  environment conditions and computer library of all known life forms.
  Fulfils equipment requirements for biology, diagnosis, geology,
  physics, chemistry, and treatment tasks. Provides Skill Level
  increases of 1 when performing any of the above tasks (Skill Level
  increase of 3 for biology tasks).

  
  \textbf{Availability:} Open. \textbf{Crew Required:} 1.
  \textbf{Passenger Capacity:} 1 (cabin/study for scientist). Cargo: 2
  (contents of environment chambers). \textbf{Cost:} 1,800.
  \textbf{Civ Level:} 8.
  
  

\item\textbf{Crew.}\\
  Basic accommodations for 40 additional crew of a spaceship with
  recreation, shared sleeping quarters, first aid station and galley.

  
  \textbf{Availability:} Open. \textbf{Crew Required:} 1.
  \textbf{Passenger Capacity:} 40. Cargo: 0.5. \textbf{Cost:} 900.
  \textbf{Civ Level:} 6.
  
  

\item\textbf{Energy.}\\
  Capable of holding and processing 144 Energy Units of radioactives.
  A ship with an energy pod expends all the Energy Units in the pod
  before expending Energy Units in its hull.

  \textbf{Note:} A \emph{non-streamlined} spaceship hull may carry
  \emph{two} Energy Pods using only one point of pod
  capacity.\label{sec:add-two-energy}
  
  \textbf{Availability:} Open. \textbf{Crew Required:} 0.
  \textbf{Passenger Capacity:} 0. Cargo: 0. \textbf{Cost:} 140
  (excluding energy). \textbf{Civ Level:} 7.
  
  

\item\textbf{Escape/EVA.}\\
  Contains assortment of expedition suits, propulsion devices, and
  tethers for zero-G maneuver outside the ship. Has three airlocks and
  four launch tubes, each capable of holding a Corco Omega lifeboat
  (not included).

  
  \textbf{Availability:} Open. \textbf{Crew Required:} 0.
  \textbf{Passenger Capacity:} 0. Cargo: 0. \textbf{Cost:} 400.
  \textbf{Civ Level:} 7.  114 of 177
  
  

\item\textbf{\hypertarget{tag:pod-explorer}{Explorer}.}\\
  Combines certain elements of a survey pod and a bio-research pod,
  designed for small exploration missions. Contains burn-out hyperjump
  engine. The engine in a hunter or explorer pod may not be used when
  attached to a ship capable of carrying more than 3 pods. An explorer
  pod engine must be replaced after two jumps at a cost of 375
  Credits. Fulfils equipment requirements for biology, geology,
  physics, planetology, and chemistry tasks. Provides Skill Level
  increase of 1 when performing an astronomy or planetology task.
  Provides Skill Level increase of 2 when performing a biology task.
  Contains chambers for keeping living specimens in their natural
  environment conditions.

  
  \textbf{Availability:} Open. \textbf{Crew Required:} 2 (including
  navigator).  \textbf{Passenger Capacity:} 0. Cargo: 0.5 (contents of
  environment chambers).  \textbf{Cost:} 1,800. \textbf{Civ Level:} 8.
  
  

\item\textbf{Heavy Weapon.}\\
  Weapons Pod. Each Heavy Weapon pod contributes one Battle Command to
  the ship's total. This pod may fire laser and particle bursts and
  barrages. This pod carries 5 Unguided and 3 Guided Missiles all
  except the Unguided missiles require a Prepare Missile command
  before launching.
  
  \textbf{LASER/PARTICLE WEAPONS:} Yes. \textbf{UNGUIDED MISSILE:} 6.
  \textbf{GUIDED MISSILE:} 5.  \textbf{INTELLIGENT MISSILE:} 3.
  \textbf{MIMS:} 1.  \textbf{HYPERJUMP:} No. \textbf{ARMOR:} 2. BATTLE
  \textbf{COMMANDS:} 1. \textbf{NUMBER OF FIRES:} 1.  \textbf{TARGET
    PROGRAM:} -4.

  
  \textbf{Availability:} Military. \textbf{Crew Required:} 4.
  \textbf{Passenger Capacity:} 0.  \textbf{Cargo Capacity:} 0.
  \textbf{Civ Level:} 7. \textbf{Cost:} 3000.
  
  
  

\item\textbf{Hunter.}\\
  Weapons Pod. Contains burn-out hyperjump engine (see
  \ref{sec:hyperjump-engine}).  The jump engine in the Hunter pod may
  not be used when attached to a ship capable of carrying more than 3
  pods.  A Hunter pod engine must be replaced after three jumps at a
  cost of 500 Credits. This pod may fire laser and particle bursts and
  barrages. This pod carries 2 Unguided and 1 intelligent Missile
  neither of which require a Prepare Missile command before launching.
  
  \textbf{LASER/PARTICLE WEAPONS:} Yes. \textbf{UNGUIDED MISSILE:} 2.
  \textbf{GUIDED MISSILE:} 0.  \textbf{INTELLIGENT MISSILE:} 1.
  \textbf{MIMS:} 0.  \textbf{HYPERJUMP:} Yes. \textbf{ARMOR:} 2.
  BATTLE \textbf{COMMANDS:} 0. \textbf{NUMBER OF FIRES:} 1.
  \textbf{TARGET PROGRAM:} -4.

  
  \textbf{Availability:} Military. \textbf{Crew Required:} 2
  (including navigator).  \textbf{Passenger Capacity:} 0.
  \textbf{Cargo Capacity:} 0. \textbf{Civ Level:} 8. \textbf{Cost:}
  2000.
  
  

\item\textbf{Lander.}\\
  Contains airlocks, docking, refuelling, and maintenance facilities
  for surface landing vessel (a modified Terwillicker 5000 or
  Piccolo).  Lander not included in cost of pod.

  
  \textbf{Availability:} Open. \textbf{Crew Required:} 0.
  \textbf{Passenger Capacity:} 0. Cargo: 0.  \textbf{Cost:} 300.
  \textbf{Civ Level:} 7.
  
  

\item\textbf{\hypertarget{tag:pod-light-weapon}{Light Weapon}.}\\
  Weapons Pod. Each Light Weapon pod contributes one Battle Command to
  the ship's total. This pod may fire laser and particle bursts and
  barrages. This pod carries 5 Unguided and 3 Guided Missiles all of
  which require a Prepare Missile command before launching.
  
  \textbf{LASER/PARTICLE WEAPONS:} Yes. \textbf{UNGUIDED MISSILE:} 5.
  \textbf{GUIDED MISSILE:} 3.  \textbf{INTELLIGENT MISSILE:} 0.
  \textbf{MIMS:} 0.  \textbf{HYPERJUMP:} No. \textbf{ARMOR:} 1. BATTLE
  \textbf{COMMANDS:} 1. \textbf{NUMBER OF FIRES:} 1.  \textbf{TARGET
    PROGRAM:} -2.

  
  \textbf{Availability:} Military. \textbf{Crew Required:} 2.
  \textbf{Passenger Capacity:} 0.  \textbf{Cargo Capacity:} 0.
  \textbf{Civ Level:} 6. \textbf{Cost:} 17000.
  
  

\item\textbf{Living Cargo.}\\
  Contains 50 independently pressurized and heated compartments for
  holding all types of living cargo from plants to pets to alien life
  forms. Requisite life support systems included. Compartments range
  in size from 0.5 cubic meter to 30 cubic meters.

  
  \textbf{Availability:} Open. \textbf{Crew Required:} 1.
  \textbf{Passenger Capacity:} 0. Cargo: 10. \textbf{Cost:} 650.
  \textbf{Civ Level:} 7.
  
  

\item\textbf{Luxury Cabin.}\\
  Spacious single and double occupancy staterooms for first class
  passengers. Includes recreational, bar, and dining area; galley and
  first aid station.

  
  \textbf{Availability:} Open. \textbf{Crew Required:} 5.
  \textbf{Passenger Capacity:} 15. Cargo: 0.5. \textbf{Cost:} 1,500.
  \textbf{Civ Level:} 7.
  
  

\item\textbf{Medical.}\\
  Fully equipped medical laboratory/computer. Fulfils equipment
  requirements for all biology, diagnosis, and treatment tasks.
  Provides Skill Level increase of 1 when performing a biology task.
  Provides Skill Level increase of 3 when performing a diagnosis or
  treatment task. Alternatively, the computer may diagnose and treat a
  patient by itself, with diagnosis and treatment Skill Levels of 6.

  
  \textbf{Availability:} Restricted (medicines and drugs require an
  accredited doctor). \textbf{Crew Required:} 1. \textbf{Passenger
    Capacity:} 5 (four sickbeds and one cabin/study for doctor).
  Cargo: 0. \textbf{Cost:} 2,000. \textbf{Civ Level:} 8.
  
  

\item\textbf{Robot and Equipment.}\\
  Storage and maintenance facilities for robots and equipment used by
  the crew and passengers. Fulfils equipment requirements for all
  Compu/robot, Electro, Energy, Suit, Vehicle, and Weapon tech repair
  tasks. Provides Skill Level increase of 3 when attempting to repair
  any such item brought into the pod. Alternatively, the maintenance
  system may repair an item by itself, with a tech Skill Level of 6.
  (Exception: The system may not repair a robot by itself.) This pod
  does not aid the construction, Psion tech, and spaceship tech skills.

  
  \textbf{Availability:} Open. \textbf{Crew Required:} 1.
  \textbf{Passenger Capacity:} 0. Cargo: 2 (items under repair).
  \textbf{Cost:} 1,200. \textbf{Civ Level:} 8.
  
  

\item\textbf{Standard Cabin.}\\
  Double, triple, and quad occupancy rooms for standard passengers.
  Includes dining area, galley, and first aid station.

  
  \textbf{Availability:} Open. \textbf{Crew Required:} 3.
  \textbf{Passenger Capacity:} 30. Cargo: 0.5. \textbf{Cost:} 1,200.
  \textbf{Civ Level:} 7.
  
  

\item\textbf{Standard Cargo.}\\
  For all non-pressurised bulk materials haulage. Cargo in the pod
  must be all of the same type unless an Automatic Cargo Control (ACC)
  is fitted. Non-climate controlled cargo hold. Will not sustain life.

  
  \textbf{Availability:} Open. \textbf{Crew Required:} 0.
  \textbf{Passenger Capacity:} 0. Cargo: 35. \textbf{Cost:} 35.
  \textbf{Civ Level:} 6.
  
  

\item\textbf{Standard Jump.}\\
  Allows spaceship to hyperjump. Psi-Navigator required. Jump engine
  requires monopole replacement every 200 light years (cost of
  replacement: 15 Trans). Skill Level of navigator increased by 1
  during hyperjump.

  
  \textbf{Availability:} Open. \textbf{Crew Required:} 2 (including
  navigator).  \textbf{Passenger Capacity:} 0. \textbf{Cargo
  Capacity:} 0.  \textbf{Civ level:} 7.
  \textbf{Cost:} 1,500.
  
  

\item\textbf{Survey.}\\
  Complete research center for studying a world being orbited or
  approached. Contains computer library on all known planetary
  phenomena. Provides Skill Level increase of 2 when performing an
  astronomy task and a Skill Level increase of 3 when performing a
  planetology task.

  
  \textbf{Availability:} Open. \textbf{Crew Required:} 1.
  \textbf{Passenger Capacity:} 1 (cabin/study for Astronomer or
  Planetologist). Cargo: 0. \textbf{Cost:} 1,600.  \textbf{Civ Level:}
  8.
  
  

\item\textbf{Tractor Beam.}\\
  Allows the player to issue Maneuver Commands to another friendly or
  enemy spaceship or Battlecraft during his Command Phase, as if he
  controlled the unit. The player must issue a Battle Command to use
  the tractor beam. If he does so, a Civ Level 7 tractor beam may be
  used to issue one Maneuver Command to any one unit within four hexes
  of the ship with the tractor beam. A Civ Level 8 tractor beam may be
  used to issue two Maneuver Commands to any one ship within six hexes
  of the ship with the tractor beam. A tractor beam may not be used to
  issue Maneuver Commands to an enemy or friendly missile. Each
  Maneuver Command issued by using a tractor beam requires the
  expenditure of a number of Energy Units equal to twice the Energy
  Burn Rate of the target unit.

  
  \textbf{Availability:} Restricted. \textbf{Crew Required:} 2.
  \textbf{Passenger Capacity:} 0 .  Cargo: 0. \textbf{Cost:} 2,500.
  \textbf{Civ Level:} 7.
\end{itemize}

\section{Interstellar Travel}
\label{sec:interstellar-travel}



A psionic navigator conducts and controls instantaneous interstellar
travel, or hyperjumping, by manipulating magnetic monopoles with his
mind. A hyperjump occurs when the thought patterns of a psionic mind
concentrating on a remote destination are encoded into a plate of
monopoles. The mental image and the power of the navigator's mind
force the perpetually unidirectional particles to reverse their
polarity, causing a shift or jump to the point matching the monopole
pattern. This inexplicable phenomenon is the key to both humankind's
galactic expansion and the psionic community's continued economic well
being.

Failure of a navigator to wrest the monopoles to his mental command
often results in a jump error and/or psychic damage to the psion.
Gravitational fields increase this risk by disrupting the stability of
the monopole screen that the navigator projects his mind onto. Because
of this, spaceships must travel to the outskirts of a star system by
conventional propulsion before a jump may be conducted. Conversely,
the destination of a jump must be outside the gravity wells of the
system so that the navigator's mind and the integrity of the ship will
not be destroyed by monopoles reacting to forces other than his own
thoughts. The point outside a star system that is closest to a given
planet and yet far enough away from all gravitational fields in the
system to conduct a safe hyperjump from is called that system's
\emph{jump point}. It is to this point that a navigator will bring a
spaceship as the result of a perfect jump.


\subsection[Hyperjump Engine]{A spaceship must have a hyperjump engine
  in order to 
  travel between stars.}
\label{sec:hyperjump-engine}



A standard jump pod, an augmented jump pod, a hunter pod, and an
explorer pod each contains a jump engine. Any spaceship with one of
these pods may hyperjump. \textbf{Exception:} The engine in a hunter
or explorer pod may not be used when attached to a ship capable of
carrying more than \textbf{3} pods. The Corco \emph{Omega} lifeboat
also contains a jump engine.

A jump engine does not consume energy; however, with the exception of
the augmented jump pod, all hyperjump engines lose their stability
after a time. A standard jump pod must be serviced as described in
\ref{sec:pods}. The hunter pod, the explorer pod, and the \emph{Omega}
each contains a burn-out hyperjump engine. After a few jumps, such an
engine must be entirely replaced. A hunter pod engine must be replaced
after three jumps at a cost of \textbf{500} Trans. An explorer pod
engine must be replaced after two jumps at a cost of \textbf{375}
Trans. A Corco \emph{Omega} engine must be replaced after one jump at
a cost of \textbf{200} Trans.



\subsection[Jump Point]{A spaceship that is at a safe jump point and
  that has a
  psionic navigator aboard may hyperjump.}
\label{sec:jump-point}



A jump point for departing or arriving at a given planet lies above or
below the plane of the ecliptic for the star system, such that a line
drawn from the point to the planet would be perpendicular to the
ecliptic.  See diagram below.


\begin{figure}[htbp]
  \centering
  \includegraphics{jumppoint}
  \caption{Jump Point}
  \label{fig:jump-point}
\end{figure}

The length of this line depends on the Spectral Class of the planet's
star and the distance between the planet and the star. Subtract the
distance (in AU's) between the planet and the star from one of the
following numbers:

\textbf{Spectral Class A: 180 F: 130 G: 90 K: 60 M: 40}

The result of this subtraction is the minimum distance (in AU) from
the planet a spaceship must be to go into or come out of a jump.  For
example, the minimum jump point for a planet that is \textbf{7 AU}
from a Class \textbf{K} star is \textbf{53 AU}. This distance would
have to be traversed using normal propulsion and could take as long as
four weeks (for an \textbf{A} Class star); see \ref{sec:interplanetary-travel}.

\textbf{Note:} A spaceship must be stationary (at \textbf{0} velocity) at the
moment of hyperjump, and will thus be stationary when it comes out of
hyperjump. A hyperjump is conducted using the Hyperjump Table
(\ref{tab:hyperjump}) in accordance with the navigation skill description
(see \ref{sec:psionicskills}).

\subsection[Hyperjump Table]{The Hyperjump Table is used to resolve
  the outcome of a 
  hyperjump.}
\label{sec:hyperjump-table}

See page \pageref{tab:hyperjump}.


\begin{table}[htbp]
  \centering
  \fbox{%
    \begin{minipage}{5.7in}
      \centering
      \caption{Hyperjump Table}
      \label{tab:hyperjump}
      
      \begin{tabular}{crp{3.5in}}
        PERCENTILE &&\\
        DICE MINUS & PSIONIC  & JUMP OUTCOME \\
        HYPERJUMP  & BACKLASH & AND \\
        CHANCE     & CHECK    & SPACECRAFT LOCATION \\
        \hline
        -40 or less & No &
        \textbf{Perfect Jump:} Perpendicular to the plan of the 
        system ecliptic, directly above the destination planet 
        at the closest safe Jump Point \\
        -39 to -20 & -50 &
        \textbf{Perfect Jump:} Perpendicular to the plan of the 
        system ecliptic, directly above the destination planet 
        at the closest safe Jump Point \\
        -19 to 0 & -40 &
        \textbf{Perfect Jump:} Perpendicular to the plan of the 
        system ecliptic, directly above the destination planet 
        at the closest safe Jump Point \\
        +1 to +10 & -30 &
        \textbf{Good Jump:} Perpendicular to the plan of the 
        system ecliptic, directly above the destination planet 
        and 1 Die roll in AU's beyond closest jump point. \\
        +11 to +20 & -20 &
        \textbf{Good Jump:} Perpendicular to the plan of the 
        system ecliptic, directly above the destination planet 
        and 2 Die rolls in AU's beyond closest jump point. \\
        +21 to +30 & -10 &
        \textbf{Slight Jump Error:} Perpendicular to the plan of the 
        system ecliptic, directly above the destination planet 
        and a Percentile Die roll in AU's beyond closest 
        jump point. \\
        +31 to +40 & 0 &
        \textbf{Minor Jump Error:} Perpendicular to the plan of the 
        system ecliptic, directly above the destination planet 
        and 10x a Percentile Die roll in AU's beyond closest 
        jump point. \\
        +41 to +50 & +10 &
        \textbf{Major Jump Error:} A Perfect Jump to the star
        nearest the destination star with the same Spectral 
        Class letter. \\
        +51 to +60 & +20 &
        \textbf{Major Jump Error:} The GM uses one die to 
        determine the random destination. The destination 
        star is used as the zero point. \\
        +61 to +70 & +30 &
        \textbf{Jump Randomised:} The GM uses two dice to 
        determine the random destination. Sol is used as 
        the zero point. \\
        +71 to +80 & +40 & 
        \textbf{Jump Failure:} The Hyperjump does not occur and 
        the spacecraft's jump engine suffers Heavy Damage \\
        +81 to +90 & +50 &
        \textbf{Jump Randomised:} The GM uses percentile dice 
        to determine the random destination. Sol is used as 
        the zero point. \\
        +91 or more & +60 &
        \textbf{Jump Disaster:} Within the gravity wells of the 
        destination star system. Use the Equipment 
        Damage column of the Hit Table (30.9) to check for 
        Spacecraft damage; roll two dice and add 15 to the 
        dice result. 
      \end{tabular}
      \parbox{\textwidth}{NOTES: 

        \textbf{Random Destination:} The GM secretly rolls 
        the indicated die or dice three times to 
        determine three co-ordinates. The first roll 
        determines the X co-ordinate; the second roll 
        determines the Y co-ordinate; and the third 
        roll determines the Z co-ordinate. 

        If the result of a roll is an even number, the 
        co-ordinate is positive; if the result is an odd 
        number the co-ordinate is negative. 

        The GM secretly locates the three co-ordinates results on the
        Interstellar Display  
        and informs the character that they are lost in 
        space. 

        A character may attempt an \textbf{Astronomy} skill 
        task to determine their location (See
        \ref{sec:scientific-skills}). 

        It is possible that a random destination will lie 
        off the Interstellar Display. 

        See \ref{sec:psionicskills} and \ref{sec:jump-point} for
        explanation of use.}
    \end{minipage}}
\end{table}

\subsection[Commercial Interstellar Travel]{A character without a
  spaceship may travel between stars 
  by booking passage on a commercial vessel.}
\label{sec:commercial_interstellar_travel}


Commercial hyperjump voyages occur with a varying degree of regularity
between many stars. The frequency and reliability of a commercial
voyage between two stars depends on the distance between them and the
highest Spaceport Class of a world orbiting each star. Commercial
interstellar travel always begins and ends at the world in a star
system with the highest Spaceport Class (by federal law). If two or
more worlds in a system share the highest class, the GM chooses one as
the interstellar terminal.

The Interstellar Route Chart is used to determine the type of route
(if any) that exists between any two stars. The highest Spaceport
Class in the destination star system is cross-referenced with the
highest Spaceport Class in the system of departure to yield three
numbers, each defining the maximum distance (in light years) at which
a given type of route exists.

A \textbf{Green} jump route is a well-travelled commercial lane with
passenger and freight service occurring daily (or more often). If a
character or party wishes to travel this route, roll \emph{percentile}
dice to determine how many \emph{hours} he must wait for a ship with
available space.

An \textbf{Amber} jump route is an infrequently travelled commercial lane
traversed by freighters and an occasional passenger vessel. Roll \emph{two
dice} to determine how many \emph{days} a character must wait for available
space on this route.

A \textbf{Red} jump route is a rarely travelled course traversed by a
few exploratory and resupply ships. Roll \emph{percentile} dice to
determine how many \emph{days} a character must wait for available
space on this route.

When using the Hyperjump Table for a spaceship that a character has
booked passage on, assume the ship's navigator has a Skill Level of
\textbf{7}, a Mental Power Rating of \textbf{6}, is in a standard jump
pod, and either frequents or has previously visited both star systems
at some time.

The price of \emph{standard} passage on a commercial interstellar
spaceship is calculated with the following formula:

\begin{large}
$$(\mathrm{Distance~in~LY's}\times\mathrm{100~Mils}) +
(\mathrm{Distance~in~AU's~to~and~from~each~world's~jump~point}\times\mathrm{20~Mils})$$ 
\end{large}

Standard passage includes a small room shared with two or three other
passengers, a common toilet, a common dining area, and a baggage
allowance of \textbf{100 kg}. The price of a standard passage on a
\textbf{red} jump is increased by \textbf{50\%}.

A character travelling a green jump (only) may book \emph{luxury}
passage at \textbf{2.5} times the standard fare. A luxury passenger
receives a private room (single or double occupancy, as requested)
with private bath, a common recreation-bar-dining area featuring
entertainment and gourmet dining, and a baggage allowance of
\textbf{500 kg}.

The total time of an interstellar voyage equals the time to travel
from the world of departure to its jump point, and from the
destination world to its jump point, in accordance with
\ref{sec:interplanetary-travel}. The actual hyperjump takes no time.

\subsection{Interstellar Route Chart}
\label{sec:interstellar-route-chart}

See page \pageref{tab:interstellar-route}.


\subsubsection*{Calculating the distance between two stars in Light Years:}

Where $X$, $Y$ and $Z$ are the co-ordinates of the Origin Star System
and $x$, $y$ and $z$ are the coordinates of the Destination Star
System.

\begin{large}
$$\mathrm{Distance~in~Light~Years} = \sqrt{(X - x)^2 + (Y - y)^2 +
  (Z-z)^2}$$
\end{large}

\begin{table}[htbp]
  \centering
  \fbox{%
    \begin{minipage}{4.75in}
      \centering
      \caption{Interstellar Route Chart}
      \label{tab:interstellar-route}
      
      \medskip
      
      \begin{tabular}{ccccccc}
        & \multicolumn{6}{c}{DESTINATION STARPORT}\\
        ORIGIN &&&&&&5\\
        STARPORT & $\frac12$ & 1& 2& 3& 4 &EARTH \\
        \rowcolor{grey}
        $\frac12$ & None & None & 0/0/2 & 0/0/6 & 0/0/10 & 0/0/15\\
        1 & None & 0/0/5 & 0/0/10 & 0/2/15 & 0/6/20 & 2/10/15\\
        \rowcolor{grey}
        2 & 0/0/2 & 0/0/10 & 0/10/20 & 2/15/25 & 6/20/30 & 10/25/35\\
        3 & 0/0/6 & 0/2/15 & 2/15/25 & 10/20/30 & 15/25/35 & 20/30/45\\
        \rowcolor{grey}
        4 & 0/0/10 & 0/6/20 & 6/20/30 & 15/25/35 & 20/35/45 & 30/40/50\\
        \parbox{5em}{\centering 5\\EARTH} & 0/0/15 & 2/10/25 & 10/25/35 &
        20/30/45 & 30/40/50 & -\\ 
      \end{tabular}

      \medskip

      \parbox{\textwidth}{\textbf{NOTES:}
        
        If the distance (in light years) is equal to or less than the
        first number, a \textbf{green} jump route exists. If the
        distance is greater than the first number but less than or
        equal to the second number, an \textbf{amber} jump route
        exists. If the distance is greater than the first and second
        numbers, but less than or equal to the third number, a
        \textbf{red} jump route exists.

        See \ref{sec:commercial_interstellar_travel} for details.}

      \medskip
    \end{minipage}}
\end{table}


\section{Interplanetary Travel}
\label{sec:interplanetary-travel}



Travel between the worlds of a single star system, or between a world
and its jump point is conducted at slower than light speeds (STL). The
most common sub-light spaceship engine is a reaction drive, using
radioactive elements as fuel.



\subsubsection*{Constant 1G Acceleration}
\label{sec:const-1g-accel}

In an interplanetary journey, a spaceship accelerates at a constant
rate to the journey's mid-point and then decelerates at the same rate
until it reaches its destination. Thus, the longer the journey, the
higher the velocity the ship will attain at its ``turn over'' point.
All interplanetary distances are measured in AU's. The time required
for the journey depends on the spaceship's rate of acceleration. A
constant acceleration/deceleration of \textbf{1 G} is the usual travel
speed for a spaceship. The travel time using this speed is calculated
with the following equation:


\begin{quote}
  \textbf{Constant 1 G Acceleration:}\\
  $\mathrm{Time~(in~hours)} = 68\sqrt{d}$\\
  ($d$ = distance in AU's)\\
  Energy: 1 Block per 24 hours
\end{quote} 

\subsubsection*{Coasting 1G Acceleration}
\label{sec:coast-1g-accel}



An energy efficient variation of the above is termed ``Coasting''
where the spacecraft will accelerate to build up velocity during the
first part of the voyage, and only decelerate when the destination is
near with the intervening time spent coasting. This method is vital
for spacecraft that have energy limitations. Energy is consumed at a
rate of \textbf{2} Energy Blocks every \textbf{5} days. The unpleasant
consequences of this energy saving maneuver is that the middle
three-fifths of the voyage are spent in weightlessness. There is a
$(50 - [10\times\mathrm{NW~Gravity~Skill~Level}])$\% chance that a
character will experience bouts of extreme discomfort during these
parts of the trip (roll once each week of weightlessness). The travel
time using this method is calculated with the following equation:

\begin{quote}
  \textbf{Coasting 1 G Acceleration:}\\
  $\mathrm{Time~(in~hours)} = 85\sqrt{d}$\\
  ($d$ = distance in AU's)\\
  Energy: 1 Block per 60 hours
\end{quote} 

\emph{(Additional --- From ``Mongoose and Cobra'' adventure,
  \emph{Ares Magazine})}


\subsubsection*{Constant 2.5G Acceleration}
\label{sec:const-2.5g-accel}


A spaceship without special equipment manned by a healthy crew may
travel at a constant acceleration/deceleration as high as \textbf{2.5
  G}.  This speed equals an acceleration or deceleration of \textbf{1} per turn
in \emph{Delta Vee}.  The travel time using this method is calculated
with the following equation:

\begin{quote}
  \textbf{Constant 2.5 G Acceleration:}\\
  $\mathrm{Time~(in~hours)} = 43\sqrt{d}$\\
  ($d$ = distance in AU's)\\
  Energy: 1 Block per 6 hours
\end{quote} 


\subsubsection*{Constant 5G Acceleration}
\label{sec:const-5g-accel}



A high performance spaceship (Velocity Rating of 2 or 3) manned by a
crew who all have \emph{internal gravity webs} may travel at a
constant acceleration/deceleration as high as \textbf{5 G}. The speed
is often used in long range military pursuit and equals an
acceleration or deceleration of 2 per turn in \emph{Delta Vee}. The
travel time using this method is calculated with the following
equation:

\begin{quote}
  \textbf{Constant 5 G Acceleration:}\\
  $\mathrm{Time~(in~hours)} = 30\sqrt{d}$\\
  ($d$ = distance in AU's)\\
  Energy: 1 Block per hour
\end{quote} 


\subsection[Distance Between Planets]{The distance between two
  planets in a star system 
  varies widely due to their orbital paths.}
\label{sec:distance-between-planets}



This distance may be as little as the \emph{difference} between their
distances from the star, or as great as the \emph{sum} of their distances
from the star (if the two planets are on exact opposite sides of the
star). For example, the 11th and 12th planets on the Star System Log
could be as close to each other as 15 AU's and as far apart as 65
AU's. Unless the GM wishes to determine the length of each planet's
orbital year and set up an ever changing orbital model, he should use
the following simplification to determine planetary distances: The
distance between two worlds equals the distance from their star to the
world of the two that is furthest from the star. Thus, the distance
between the planets mentioned above would be 40 AU's.

For purposes of calculating interplanetary distances, a moon is
considered to occupy the same position as its planet. The distance
between a planet and any of its moons is left up to the GM. As a
guideline, our moon is \textbf{.0026 AU's} (360,000 km) from the
Earth. At a constant 1G-acceleration/deceleration, a journey from the
Earth to the moon would take about \textbf{2.5} hours.



\subsection[Energy Units and Energy Blocks]{As a spaceship travels
  through interplanetary space, it 
  consumes fuel in the form of Energy Units and/or  
  Energy Blocks.}
\label{sec:energy-blocks}



An Energy Unit for a reaction drive engine consists of one kilogram of
radioactives and costs \textbf{300 Mils}. Thus, it would cost
\textbf{43 Trans} to ``fill up'' an energy pod capable of carrying 144
Energy Units. Energy is always available at a Class \textbf{3},
\textbf{4}, or \textbf{5} spaceport. Energy may be available at a
Class \textbf{1} or \textbf{2} spaceport; see
\ref{sec:spaceport-class}. Any spaceport orbiting a world that
contains radioactives as a resource will always have spaceship energy.

As explained in \emph{Delta Vee} , the number of Energy Units a spaceship
expends to maneuver depends on its Energy Burn Rate, which corresponds
to the number of Energy Units in one \emph{Energy Block} for that
ship. The same concept applies to long-range interplanetary travel. A
spaceship that is travelling at a constant acceleration/deceleration
of \textbf{1 G} must expend 1 Energy Block \emph{every} \textbf{24}
\emph{hours} (or fraction thereof). A ship that is travelling at
\textbf{2.5 G} must expend 1 Energy Block every \textbf{6}
\emph{hours}. A ship that is travelling at \textbf{5 G} must expend 1
Energy Block \emph{every hour}. A spaceship that is lifting off from
the surface of a world must expend an additional number of Energy
Blocks equal to the size of the world (\textbf{1} to \textbf{9}).
Lift-off does not increase the voyage time.

The GM must make sure that any spaceship he enters into play possesses
the requisite energy to complete any voyage planned for it.  Sometimes
two or three energy pods will be required for a large ship travelling
to and from jump points.

\subsection[Internal Gravity Web]{Any person in a spaceship undergoing
  high G forces must 
  have an \hypertarget{tag:gravity-web}{internal gravity web}.}
\label{sec:gravity-web}



A gravity web contains and protects a person's organs and arteries
during the strain of high G acceleration and deceleration. The web is
inserted throughout the person's body in a complex series of
operations.  All star sailors are provided with a gravity web, as are
many members of the Astroguard. A character may undergo an operation
for gravity web implantation on any world with a Civ Level of 7 or
higher. The operation costs \textbf{15} Trans and the character will
be in the hospital for \textbf{7} days.

A character without a gravity web cannot survive an extended period of
acceleration or deceleration greater than \textbf{2.5 G}.  When
playing \emph{Delta Vee} , a character without a gravity web cannot survive a
velocity change of more than \textbf{2} (5 G) in a single Command
Phase.

\subsection[Commercial Interplanetary Travel]{A character without a
  spaceship may travel between worlds in a star system by booking
  passage on a  
commercial vessel.}
\label{sec:commercial-interplanetary-travel}


The frequency of commercial voyages between two worlds depends on the
class of their spaceports.  The Interplanetary Route Chart is used to
determine the type of route (if any) that exists between any two
worlds in the same system.  The Spaceport Class of the destination
world is cross-referenced with the Spaceport Class of the world of
departure to yield the type of route: \emph{course green},
\emph{course amber}, or \emph{course red}.  The frequency and quality
of service on these routes correspond to those of the jump routes of
the same color designations in
\ref{sec:commercial_interstellar_travel}.

The price of standard passage on a commercial interplanetary spaceship
equals the distance of the voyage (in AU's) multiplied by \textbf{40
  Mils}.  If the spaceship must lift off from the surface of a world
(Spaceport Class $\frac12$) or if the voyage is course red, the fare
is increased by \textbf{50\%}.  In any case, the minimum cost for any
interplanetary voyage is \textbf{250 Mils} (including a voyage from a
planet to one of its moons and vice versa).

A character travelling on a course green voyage of at least one AU may
book luxury passage at \textbf{2.5} times the standard fare.  He
receives the comforts of luxury passage described in
\ref{sec:commercial_interstellar_travel}.  A commercial vessel always
travels at a constant acceleration/deceleration of 1 G.


\subsection{Interplanetary Route Chart}
\label{sec:interplanetary-route-chart}

See Table \vref{tab:interplanetary-route}.

\begin{table}[htbp]
  \centering
  \fbox{%
    \begin{minipage}{5in}
      \centering
      \caption{Interplanetary Route Chart}
      \label{tab:interplanetary-route}
      
      \medskip
      
      \parbox{3in}{\begin{tabular}{ccccccc}
          & \multicolumn{6}{c}{DESTINATION STARPORT}\\
          ORIGIN &&&&&& 5\\
          STARPORT & $\frac12$ & 1 & 2 & 3 & 4 & EARTH\\
          \rowcolor{grey}
          $\frac12$ & - & R & R & R & A & G\\
          1 & R & R & A & A & A & G\\
          \rowcolor{grey}
          2 & R & A & A & G & G & G\\
          3 & R & A & G & G & G & G\\
          \rowcolor{grey}
          4 & A & A & G & G & G & G\\
          \parbox{5em}{\centering 5\\EARTH} & G & G & G & G & G & -
        \end{tabular}}
      \hspace{0.25in}
      \parbox{1.25in}{\textbf{NOTES:}
        
        \textbf{R:} Course Red. \\
        \textbf{A:} Course Amber. \\
        \textbf{G:} Course Green. \\
        \textbf{(-):} No Route. 
        
        See \ref{sec:interplanetary-route-chart} for details.}
    \end{minipage}}
\end{table}


\section{Space Combat}
\label{sec:space-combat}



When the characters are in a spaceship that encounters another
spaceship, combat may occur. If possible, the characters should avoid
spaceship combat; it is deadly. However, situations will certainly
arise where combat is inevitable. When this happens, the GM and the
players use the \emph{Delta Vee} tactical space combat system to
resolve the battle. \emph{Delta Vee} is complete unto itself and may
be used as is, if the characters are in a ship that they are not
controlling. The following rules modify the system, allowing
characters to participate directly in the conduct of combat. Combat
burns up energy at a much faster rate than steady, uninterrupted
travel. Conserving enough energy during battle so that the ship may
get somewhere if it survives should be as much on the characters'
minds as victory.

\subsection[Tactical Maps]{The GM sets up the tactical maps and the spaceship
  counters.}
\label{sec:tactical-maps}



Keeping in/mind that each hex on the map is \textbf{20,000 km} across,
the GM may arrange the maps in any configuration and may place
planets, asteroids, the characters' spaceship, and any spaceships
encountered on the maps to match the situation he is describing.
Unless the spaceships are near a world or jump point that they are
travelling to or from, they will be travelling much faster than the
velocities in \emph{Delta Vee}. In this case, the Velocity markers
assigned show their velocities relative to each other. The slowest
ship should be assigned a \textbf{0} Velocity marker, and the others
should have markers indicating their velocities in comparison with the
slowest ship. If the GM wishes, he may use markers of his own devising
to show velocities above \textbf{9}.

\subsection[Character Positions]{Before beginning battle, each
  character declares which 
  part of the ship he is in.}
\label{sec:character-positions}



When in the bridge or certain pods of a spaceship, a character may use
his space combat skills, as described in \emph{Delta Vee}
\ref{DV-sec:commands}.  At the
beginning of any friendly Command Phase during space combat, a
character may declare that he is moving to another part of the ship.
If he does so, his skills may not be-used at all for that Command
Phase and the next friendly Fire Phase. At the beginning of the
following friendly Command Phase, the character may again use his
skills (as allowed by his new location).

Unless the GM determines Skill Levels for the crew manning a spaceship
that the characters encounter, he should use the \emph{Delta Vee}
system unmodified for their maneuvers, commands, and fires.


\subsection[Spacecraft Hits]{A character in a compartment of a
  spaceship that is hit by 
  spaceship fire may suffer hits.}
\label{sec:spacecraft-hits-player}



A character in a compartment that becomes \emph{vulnerable} is not
harmed. A character in a compartment that becomes \emph{damaged} must
use the Hit Table, as if struck with a Hit Strength of \textbf{6} (see
\ref{sec:hit-table}). A character in a compartment that is
\emph{destroyed} must use the Hit Table, as if struck with a Hit
Strength of \textbf{20}. If he survives this injury, and he does not
pass out, he may immediately move to any other compartment. If he
passes out as a result of the injury he dies.

Spaceship armor in \emph{Delta Vee} uses a different classification
system than personal and vehicle armor in Universe. If a situation
should arise where a spaceship's armor is hit by forces other than
spaceship weaponry, the Armor Classes of each pod and compartment can
be translated into projectile and beam Defence Ratings as follows:
Armor Class \textbf{0}: 2/2. Armor Class \textbf{1}: 4/4.  Armor Class
\textbf{2}: 6/6. A spaceship's force field may never be pierced by any
projectile weapon.


  
%%% Local Variables: 
%%% mode: latex
%%% TeX-master: "gm_guide"
%%% End: 
