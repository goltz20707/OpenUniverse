%%
%% $Id: dv_text.tex,v 1.8 2004/11/29 19:48:04 goltz20707 Exp $
%%

%% \chapter{Delta Vee Tactical Space Combat}
%% \label{cha:delta-vee}



\section*{Read This First}
\label{sec:dv-read-this-first}


The rules to \emph{DeltaVee} are organized by major topics arranged in the
order in which they occur in the play of the game. Each such major
topic is given a number and a name below which is given (usually) a
General rule or Description which summarizes the rule in that section.
This is usually followed by numbered paragraphs called Cases, which
give the specifics of the rules. Note that the numbering of the Cases
is a decimal form of the major Section number. Players should examine
the map and counters and then quickly read the rules (without trying
to memorize them). Then the game should be set up to play and a
``trial run'' made.



\section{Introduction}
\label{sec:delta-vee-intro}



\emph{DeltaVee} is a tactical simulation of combat among spaceships
in the 24th Century. The game is intended for two players, although
more may participate in scenarios devised by the players. Each hex on
the \emph{DeltaVee} game maps represents a volume 20,000 kilometers
in diameter. The game uses a two-dimensional movement system; the
``plane'' of the playing area represents the ecliptic of the star
system in which each battle occurs. Each Game-Turn represents 15
minutes.

During play, each player moves one or more spaceships about the
game-map using a maneuver system that abstractly simulates the laws of
Newtonian motion. Each Spaceship is composed of a hull with a varying
number of attached pods. Each pod represents a system that improves
the ship's capabilities in combat, movement, and/or some other aspect
of Spaceship operations.

Each player controls his Spaceship by issuing Maneuver Commands (so
that the ship may alter its velocity or direction of movement) and
Battle Commands (so that weapons and other systems aboard the ship may
be prepared for use). A Spaceship's combat abilities include laser
fire, particle fire, four types of missiles, and battlecraft (two-man
fighter craft launched from larger ships). The attributes of each
Spaceship are recorded on a Spaceship Log before beginning play.
During the game, energy expended, missiles launched, and damage
incurred by the ship are recorded on the ship's log.

\emph{DeltaVee} is the tactical space combat system for SPI's science
fiction role-playing game, \emph{Universe}. Although there is little
open warfare in the vast human interstellar empire of the 24th Century
(contact with another space-faring race has yet to be established),
there are many types of illegal ship traffic and disputes among
self-governing worlds. These types of small conflicts form the back-
ground for the scenarios in \emph{DeltaVee}. Interstellar travel in
\emph{Universe} is accomplished by hyperjumping with the aid of a
Psionic navigator. In \emph{DeltaVee}, hyperjumping is very rare,
since all its battles occur within the confines of a star system,
where hyperjumping is impossible.

\textbf{Note:} One 20-sided die is necessary for playing
\emph{DeltaVee}. When using the die, always treat a result of
\textbf{0} as \textbf{10}.


\section{Game Components}
\label{sec:game-components}

\noindent\textbf{GENERAL RULE:}

The game components consist of these rules, including charts, tables,
and logs; four identical game maps; and 200 playing pieces.  One
20-sided die and pencils with erasers are also required in order to
play the game.

\medskip

\noindent\textbf{CASES:}

\subsection[Game Map]{The perforated game mapsheet is separated into
  four game maps, which 
  are placed end-to-end or side-to-side to form the playing area.}
\label{sec:game-map}

All maps are identical, and each consists of a hexagonal grid to
regularize the positions and movement of the playing pieces. Each
hexagon (``hex'') on each map has its own four-digit identity number.
The set-up instructions for each scenario describe how many maps are
initially used and how they are placed in relation to each other. Each
map used at the start of a scenario is assigned a letter (A through
D), to aid in deploying the playing pieces. Once play begins,
additional maps may be added to the playing area or maps may be moved
to accommodate the movement of the playing pieces (see
\ref{sec:map-placement}).



\subsection[Charts And Tables]{The charts and tables are used to
  summarize and resolve 
  certain game functions.}
\label{sec:charts-tables}



These charts and tables include the Spaceship Attribute Chart, Pod
Attribute Chart, Command Summary, Relative Velocity Chart, Fire
Results Table, Hit Table, Missile Attribute Chart, and Missile
Interception Table.



\subsection[Spaceship Logs]{Players use the Spaceship Logs to record
  the status of their Spaceship, battlecraft, and missiles during  
  the course of play.}
\label{sec:spacecraft-logs}


Each player uses one Spaceship Log for each of his spaceships in
play. There are two types of Spaceship Logs. Type 1 is used for small
ships, and Type 2 is used for large ships. The log used for a
particular Spaceship is stated in the scenario instructions.
Photocopies of the Spaceship Logs must be made for repeated play.


\subsection[Playing Pieces]{The playing pieces represent the
  spaceships, battlecraft, 
  and missiles that may be used by the players.}
\label{sec:playing-pieces}


The game also includes Velocity markers, Planet markers, Asteroid
Field markers, and other game markers.


\subsection[Unit Assignment]{The spaceships, battlecraft, and missiles
  are assigned to 
  the players by the scenario instructions.}
\label{sec:unit-assignment}

These three types of counters are collectively called \textbf{units}. 


\subsubsection*{SAMPLE SPACESHIP COUNTER}

\includegraphics{sample-spaceship}

\medskip

All spaceship counters are identical, except for an identifying
letter. The specific attributes of each spaceship are detailed in the
scenario instructions, the Spaceship Attribute Chart, and the Pod
Attribute Chart. The status of each spaceship during play is recorded
on its Spaceship Log. The arrowhead on each spaceship counter
indicates the direction in which the spaceship is moving. Each
spaceship counter must have a Velocity marker under it at all times.
The back of each spaceship counter is used when the spaceship's
force field is active (see \ref{sec:battle-commands}).


\subsubsection*{SAMPLE BATTLECRAFT COUNTER}

\includegraphics{sample-battlecraft}

\medskip

A battlecraft is a small, manned fighter spaceship that can be
launched from a spaceship. All battlecraft counters are identical,
except for a letter-number that identifies each battlecraft with the
spaceship to which it belongs. For example, battlecraft \textbf{A1}
is the first battlecraft of spaceship \textbf{A}. The attributes of each
battlecraft are detailed on the Spaceship Attribute Chart. The status
of each battlecraft during play is recorded on the appropriate
Spaceship Log. Once a battlecraft has been launched, a Velocity
marker must be under it at all times. Until detected, a battlecraft is
kept face-down. The owning player may always inspect his unrevealed
battlecraft; the enemy player may not.


A missile is a self-propelled warhead that may be launched from a
Spacecraft with the requisite capabilities. There are four types of
missiles: \emph{unguided}, \emph{guided}, \emph{intelligent}, and
\emph{MIMS} (Multi-Intelligent Missile System). All missile counters
of a single type are identical except for an identity number (each
guided missile has a letter-number that identifies it with the
Spacecraft from which it is launched). The attributes of each missile
are detailed on the Missile Attribute Chart and explanations follow.
The status of a launched missile is recorded on the appropriate
Spaceship Log. Once a missile has been launched, a Velocity marker
must be kept under it until the missile explodes or is moved off the
playing area. The back of each missile counter is kept face-up until
the missile is detected by the enemy player.

\bigskip

\includegraphics{unguided}
\parbox{0.5\textwidth}{\textbf{Unguided Missile.} Once launched, an
  unguided missile may receive no Maneuver Commands. Its velocity is
  automatically increased by one each friendly Command Phase.}

\bigskip

\includegraphics{guided}
\parbox{0.5\textwidth}{\textbf{Guided Missile.} A launched guided
  missile may be issued Maneuver Commands during each Command Phase in
  which the ship it was launched from is issued a Command to control
  that particular guided missile.}


\bigskip

\includegraphics{intelligent}
\parbox{0.5\textwidth}{\textbf{Intelligent Missile.} An intelligent
  missile may be issued Maneuver Commands during each friendly Command
  Phase.}


\bigskip

\includegraphics{mims}
\parbox{0.5\textwidth}{\textbf{MIMS.} A MIMS is identical to an
  intelligent missile except that it may launch four unguided missiles
  itself during any one friendly Fire Phase (see \ref{sec:mims-launch}).
  After doing so, it is considered an intelligent missile.}


\subsection[Velocity Marker]{A Velocity marker is placed under each
  unit in play to 
  show its current velocity.}
\label{sec:velocity-marker}




\subsubsection*{SAMPLE VELOCITY MARKER}

\indent\includegraphics{velocity}
\parbox{0.5\textwidth}{The values of the Velocity markers range from
  \textbf{0} to \textbf{9} and are presented in five denominations.
  The players place and adjust the Velocity markers under their units
  to show each unit's current velocity. No more than one Velocity
  marker is placed under a single unit at a time. A missile or
  battlecraft that has been \emph{prepared} is not assigned a Velocity
  marker until launched. Both players may always inspect the Velocity
  markers under all enemy and friendly units.}



\subsection[Game Markers]{The game markers are used on the game maps
  and the Spacecraft Logs to show the status of various  
  units.}
\label{sec:game-markers}

\indent ~

\includegraphics{planet}
\parbox{0.5\textwidth}{\textbf{Planet.} Placed on the game map in
  accordance with certain 
  scenarios.  Spacecraft may sometimes land on a planet or use its
  gravity well to alter the ship's velocity (see
  \ref{sec:planets-maneuvering}).}

\includegraphics{asteroid}
\parbox{0.5\textwidth}{\textbf{Asteroid Field.} Placed on the game
  map in accordance with certain scenarios.  An asteroid field
  presents a hazard to any units entering the hex (see
  \ref{sec:asteroid-collision}).}

\bigskip

\includegraphics{energy}
\parbox{0.5\textwidth}{\textbf{Energy Units.} Used on the Energy Unit
  Track of each Spacecraft Log to record the expenditure of the ship's
  Energy Units during play (see \ref{sec:energy-unit-track}).}

\includegraphics{jump}
\parbox{0.5\textwidth}{\textbf{Prepare Jump/Jump.} Placed atop a
  Spacecraft that has been issued a Prepare Jump or Jump Command
  (see \ref{sec:maneuver-commands}).}


\includegraphics{direction}
\parbox{0.5\textwidth}{\textbf{Direction Reminder.} Placed adjacent to
  a unit that has completed a zigzag move in the hex that the unit
  would enter next while maintaining its intended direction (see
  \ref{sec:direction}).}

\bigskip

\textbf{Randomizer Chits.} A 20-sided die is required to play
\emph{DeltaVee}. If one is not available, these 10 chits can be placed
in an opaque, wide-mouthed container (such as a coffee mug). Whenever
a roll of the die is called for, blindly draw a chit to obtain a die
result. Always return a drawn chit to the container after noting its
result, so that all 10 chits are always available to be drawn from.

\section{Sequence of Play}
\label{sec:sequence-play}


\emph{DeltaVee} is played in \emph{Game-Turns}. Each Game-Turn is
divided into six distinct \emph{Phases}, three for each player. The
player whose Phase is in progress is called the Phasing player. All
actions undertaken by the players in a Game-Turn must proceed strictly
according to the following sequence outline:

\renewcommand{\theenumii}{\alph{enumii}}
\begin{enumerate}
\item FIRST PLAYER MOVEMENT PHASE\label{item:phase-1}
  
  The first player (as assigned by the scenario instructions) must
  move all his Spacecraft, battlecraft, and missiles currently in
  play.
  
  Each unit is moved a number of hexes equal to its current velocity
  in the direction the unit is pointing, in accordance with the
  restrictions of \ref{sec:movement-direction}. If a unit is moved
  into a hex occupied by a planet, the Phasing player may alter the
  unit's velocity and/or direction (see
  \ref{sec:planets-maneuvering}). If a unit is moved into a hex
  occupied by asteroids, the Phasing player must check for possible
  collision (see \ref{sec:asteroid-collision}). If a missile is moved
  into a hex occupied by an enemy unit, or if a friendly unit is moved
  into a hex occupied by an enemy missile, the Interception Routine
  must be conducted (see \ref{sec:interception-routine}).

\item SECOND PLAYER COMMAND PHASE\label{item:phase-2}

  \begin{enumerate}
  \item Detection Segment 
    
    The second player flips over every unrevealed enemy unit within
    three hexes of each of his Spacecraft. Once an enemy unit is
    revealed, it remains revealed for the rest of the game.

  \item Command Segment 
    
    The second player issues Commands to each of his units.
    
    All of his eligible Spacecraft, battlecraft, intelligent missiles,
    and MIMS may be issued Maneuver Commands. Each of his guided
    missiles may be issued Maneuver Commands if the appropriate Battle
    Command is issued to the Spacecraft controlling the missile.
    
    An unguided missile may be issued no Commands; however, the
    current velocity of each of his unguided missiles must now be
    increased by one.
  \end{enumerate}
  
\item FIRST PLAYER FIRE PHASE\label{item:phase-3}
  
  The first player may conduct laser and particle fire and/or launch
  missiles from each of his eligible Spacecraft and battlecraft. The
  result of each laser and particle fire is determined immediately, as
  each is declared. Each launched missile is placed in a hexadjacent
  to the Spacecraft from which it is launched. Each friendly
  battlecraft may be used to conduct one laser fire (only). Each
  friendly Spacecraft may be used to conduct a variable number of
  laser and particle fires and to launch missiles, depending on the
  capabilities of its pods.

\item SECOND PLAYER MOVEMENT PHASE 

  The second player conducts the activities listed in Phase
  \ref{item:phase-1}.

\item FIRST PLAYER COMMAND PHASE 

The first player conducts the activities listed in Phase \ref{item:phase-2}.

\item SECOND PLAYER FIRE PHASE 

The second player conducts the activities listed in Phase \ref{item:phase-3}. 
\end{enumerate}

One Game-Turn is now completed and another is begun. The players
continue this sequence until one player has fulfilled his victory
conditions.

\section{Spacecraft}
\label{sec:spacecraft}

\noindent\textbf{GENERAL RULE:}

The 12 Spacecraft Classes from which the players are assigned ships in
\emph{DeltaVee} vary widely in size and quality. Each Spacecraft is
actually a hull with one to 12 attached pods. (\textbf{Note:} The two
\emph{Terwillicker} ship Classes are considered battlecraft and do not
carry pods.) In addition to the information listed for each Spacecraft
Class on the Spacecraft Attribute Chart, each ship possesses a
sub-light engine, a bridge with navigation equipment, and living
quarters for a crew necessary to keep the craft running. Four
industrial concerns produce the Spacecraft:

\textbf{Terwillicker Spaceworks, Inc.} manufactures the
\emph{Terwillicker-5000}, a high-quality two-person craft; and the
\emph{Terwillicker-X fighter}, an innovative adaptation of the 5000
designed for military use.

\textbf{Blades Research Institute} produces military craft under
long-term contract. The \emph{Dagger}, \emph{Sword}, and \emph{Spear}
Class ships are their most successful models.

\textbf{Harmonics, Inc.} specializes in finely crafted ships for
government and high-level corporate use. The \emph{Piccolo},
\emph{Flute}, and \emph{Clarinet} represent the top of their line.

The \textbf{Corco Group} manufactures a large line of commercial
vessels, often sacrificing performance for economy. The \emph{Gamma},
\emph{Zeta}, and \emph{Mu} Classes are well suited for transport in
safe regions. The \emph{Iota} is designed to appeal to merchants
working in dangerous areas.

\medskip

\noindent\textbf{CASES:}

\subsection[Velocity Rating]{The Velocity Rating represents the
  maximum change in 
  velocity a Spacecraft may make at once.}
\label{sec:velocity-rating}



Thus, a \emph{Sword} Class ship may increase or decrease its current
velocity up to three levels in a single Command Phase, while a
\emph{Spear} Class ship may increase or decrease its current velocity
by only one level in a single Command Phase.

\subsection[Maneuver Rating]{The Maneuver Rating is the maximum number
  of Maneuver Commands that may be issued to a  
Spacecraft in a single Command Phase.}
\label{sec:maneuver-rating}



The actual number of Maneuver Commands that may be issued to a ship
equals its Maneuver Rating minus its current velocity.

\subsection[Energy Capacity And Burn Rate]{The Energy Capacity and the
  Energy Burn Rate are used to 
  measure a Spacecraft's expenditure of energy.}
\label{sec:energy-capacity-burn}



The total number of \emph{Energy Units} a ship begins the game with is
represented by its Energy Capacity. Each time a ship is required to
expend an \emph{Energy Block} (see \ref{sec:commands-energy}), a number of
Energy Units equal to its Energy Burn Rate are expended. A ship that
possesses an \emph{energy pod} has 144 extra Energy Units at the start
of play (see \ref{sec:special-attributes}).

\subsection[Burster]{Each Spacecraft possesses a laser weapon, called a
  burster.}
\label{sec:burster}



A Class 1 burster may be used to conduct laser burst \emph{only}. A
Class 2 burster may be used to conduct laser bursts or laser barrages.


\subsection[Armor]{Nine of the Spacecraft Classes are armored, as a defense
  against enemy laser and particle fire.}
\label{sec:armor}

Class 2 armor provides more protection than Class 1 armor (see
\ref{sec:recording-hits}). Three ship Classes possess no armor at all.


\subsection[Force Field]{Five of the Spacecraft Classes possess a
  force field generator, as a defense against enemy missile  
  explosions.}
\label{sec:force-field}


A Class 2 force field provides more protection than a Class 1 force
field (see \ref{sec:missile-force-field}). Seven ship Classes possess
no force field generator at all.


\subsection[Civ Level]{The Civ Level of a Spacecraft Class ranges from
  6 to 8.}
\label{sec:civ-level-spacecraft}



A ship's Civ (Civilization) Level may affect the performance of
certain pods attached to it. Civ Levels represent the sophistication
of the materials and equipment that make up the ship. As a comparison,
current technology (1980--2000) is just under Civ Level 5.


\subsection[Targeting Program]{The Targeting Program represents the
  ability of the Spacecraft's tracking systems to target enemy ships  
  for laser and particle fire.}
\label{sec:targeting-program}



The effectiveness of the Targeting Program is expressed as a modifier
applied to the \emph{relative velocity} of the target ship and the
firing ship (see \ref{sec:targeting-program-rel-vel}).


\subsection[Spacecraft Attribute Chart]{The Spacecraft Attribute Chart
  describes the specific 
  characteristics of each Spacecraft Class.}
\label{sec:spacecraft-attribute-chart}

See page \pageref{tab:spacecraft-attribute}.

\newlength{\smallcolsep}\setlength{\smallcolsep}{0.5\tabcolsep}
\newcolumntype{L}{%^^A
   >{\color{black}\columncolor{grey}%^^A
      \raggedright}l}
\begin{table}[htbp]
  \setlength{\tabcolsep}{\smallcolsep}
  \footnotesize
  \centering
  \fbox{%
    \begin{minipage}{0.95\textwidth}
      \centering
      \caption{Spacecraft Attribute Chart}
      \label{tab:spacecraft-attribute}
      
      \medskip
      
      \begin{tabular}{lrrrrrcrrrrrcrrrrrr}
        CLASS &
        \rotate{\parbox{1in}{NUMBER OF\\PODS}} & 
        \rotate{\parbox{1in}{VELOCITY\\RATING}} &
        \rotate{\parbox{1in}{MANEUVER\\RATING}} &
        \rotate{\parbox{1in}{ENERGY\\CAPACITY}} &
        \rotate{\parbox{1in}{ENERGY\\BURN RATE}} &
        \rotate{STREAMLINED} &
        \rotate{\parbox{1in}{BURSTER\\CLASS}} &
        \rotate{ARMOR CLASS} &
        \rotate{\parbox{1in}{FORCEFIELD\\CLASS}} &
        \rotate{CIV LEVEL} &
        \rotate{\parbox{1in}{TARGET\\PROGRAM}} &
        \rotate{AVAILABILITY} &
        \rotate{\parbox{1in}{CREW\\REQUIRED}} &
        \rotate{\parbox{1in}{PASSENGER\\CAPACITY}} &
        \rotate{\parbox{1in}{CARGO\\CAPACITY}} &
        \rotate{\parbox{1in}{PERFORMANCE\\MODIFIER}} &
        \rotate{\parbox{1in}{BASE REPAIR\\TIME (HOURS)}} &
        \rotate{COST (TRANS)}\\

        \rowcolor{grey}
        \multicolumn{19}{L}{\textbf{BATTLECRAFT}}\\
        \textbf{Terwillicker 5000} & 0 & 2 & 7 & 15 & 1 & Y & 1 & 1 &
        0 & 7 & -2 & O & 1 & 2 & 2 & 0 & 24 & 3100\\ 
        \textbf{Terwillicker-X} & 0 & 3 & 9 & 15 & 1 & Y & 2 & 2 & 0 &
        8 & -4 & R & 1 & 2 & 0.1 & +25 & 24 & 6900\\ 
        \textbf{Lander} & 0 & 1 & 4 & 15 & 1 & Y & 0 & 0 & 0 & 8 & 0 &
        O & 1 & 4 & 0.5 & -5 & 24 & 1500\\ 
        \textbf{Corco Omega} & 0 & 1 & 3 & 10 & 1 & N & 0 & 0 & 0 & 7
        & 0 & O & 1 & 4 & 0.2 & -15 & 24 & 1100\\
        \rowcolor{grey}
        \multicolumn{19}{L}{\textbf{SPACESHIPS}}\\
        \rowcolor{grey}
        \multicolumn{19}{L}{\textbf{BLADES RI}}\\ 
        \textbf{Dagger} & 2 & 2 & 6 & 48 & 4 & Y & 2 & 2 & 1 & 8 & -4
        & M & 2 & 4 & 0.5 & +15 & 24 & 12200\\ 
        \textbf{Sword} & 5 & 3 & 8 & 78 & 6 & N & 2 & 2 & 2 & 8 & -4 &
        M & 5 & 10 & 3 & +25 & 24 & 22100\\ 
        \textbf{Spear} & 8 & 1 & 4 & 144 & 12 & N & 2 & 2 & 2 & 8 & -4
        & M & 10 & 20 & 7 & +10 & 24 & 27900\\ 
        \rowcolor{grey}
        \multicolumn{19}{L}{\textbf{HARMONICS INC}}\\
        \textbf{Piccolo} & 1 & 3 & 8 & 30 & 3 & Y & 1 & 1 & 0 & 7 & -2
        & O & 1 & 6 & 1 & +5 & 24 & 5400\\ 
        \textbf{Flute} & 4 & 3 & 6 & 66 & 6 & Y & 1 & 2 & 1 & 8 & -4 &
        R & 3 & 12 & 3 & +20 & 24 & 20700\\ 
        \textbf{Clarinet} & 7 & 2 & 7 & 104 & 8 & N & 1 & 1 & 0 & 8 &
        -4 & O & 4 & 20 & 6 & +10 & 24 & 14100\\ 
        \rowcolor{grey}
        \multicolumn{19}{L}{\textbf{CORCO GROUP}}\\
        \textbf{Corco Gamma} & 3 & 1 & 4 & 54 & 6 & Y & 1 & 0 & 0 & 7
        & -2 & O & 2 & 8 & 2 & -10 & 24 & 6700\\ 
        \textbf{Corco Zeta} & 6 & 1 & 3 & 80 & 8 & N & 1 & 0 & 0 & 6 &
        0 & O & 4 & 20 & 5 & -20 & 24 & 6400\\ 
        \textbf{Corco Iota} & 9 & 2 & 5 & 120 & 12 & N & 1 & 1 & 1 & 7
        & -4 & R & 4 & 25 & 10 & 0 & 24 & 17500\\ 
        \textbf{Corco Mu} & 12 & 1 & 4 & 176 & 16 & N & 1 & 0 & 0 & 7
        & -2 & O & 5 & 30 & 15 & -10 & 24 & 14500\\ 
      \end{tabular}
      \parbox{\textwidth}{See \ref{sec:spacecraft} for explanation of use.}
    \end{minipage}}
\end{table}

\section{Pods}
\label{sec:pods-dv}

\noindent\textbf{GENERAL RULE:}

A pod is a compartment serving a specific function that is attached to
or enclosed in a Spacecraft. Each Spacecraft is assigned a variety of
pods, in accordance with the scenario being played. The number of pods
a ship possesses and the nature of those pods make each ship in
\emph{DeltaVee} distinct. All the major attributes of each pod are
listed on the Pod Attribute Chart. Additional properties of certain
pods are listed in \ref{sec:special-attributes}.

\medskip

\noindent\textbf{CASES:}

\subsection[Weapon Pods]{Hunter, light weapon, heavy weapon, and
  arsenal pods may fire laser and particle weapons and launch  
missiles.}
\label{sec:weapon-pods}



All four of these pods may fire laser and particle bursts and barrages
(see \ref{sec:laser-particle-fire}). The number of missiles of the
four types (unguided, guided, intelligent, and MIMS) each pod carries
is listed on the Pod Attribute Chart. Certain missiles require a
Battle Command in order to be launched (see
\ref{sec:number-of-missiles}). No other pods may be used to either
fire weapons or launch missiles.


\subsection[Battle Commands]{The number of Battle Commands a player
  may issue to a ship in a single Command Phase is equal to the  
  sum of the Battle Commands provided by each eligible pod.}
\label{sec:number-battle-commands}



The light weapon, heavy weapon, and arsenal pods each contribute one
Battle Command to the ship's total. The battle communications pod
contributes two Battle Commands to the ship's total. No other pods
contribute Battle Commands.


\subsection[Civ Level]{The Civ Level of a pod may affect the functions
  it 
  performs.}
\label{sec:civ-level-pod}



The Civ Level of a pod is reduced by one if it is greater than the Civ
Level of the Spacecraft to which it is attached. Also refer to
\ref{sec:pods-victory}.


\subsection[Targeting Program]{The Targeting Program affects laser and
  particle fire 
  conducted from the pod.}
\label{sec:targeting-program-pod}



See \ref{sec:pod-attribute-chart} and
\ref{sec:targeting-program-rel-vel}. The Targeting Program modifier
for the battle communications pod is applied to fire from anywhere on
the ship.  Targeting Program modifiers in other pods apply to fire
from that pod only.



\subsection[Missile Launch]{The hunter, light weapon, and heavy weapon
  pods may be used to fire or launch one of its weapons or  
  missiles during the friendly Fire Phase.}
\label{sec:missile-launch-pod}



The arsenal pod may be used to fire or launch two of its weapons or
missiles during the friendly Fire Phase. The battle communications pod
allows one additional fire or launch (see \ref{sec:special-attributes}).


\subsection[Special Attributes]{The following pods possess special
  attributes not listed 
  on the Pod Attribute Chart:}
\label{sec:special-attributes}

\begin{description}
\item[Battle Communications.] Allows one extra fire from any
\emph{one} pod or burster on the Spacecraft during the friendly Fire
Phase. The player may conduct Active Search more effectively from the
pod (see \ref{sec:battle-commands}). The pod's Targeting Program allows a
modifier of \textbf{-6} for any laser or particle fire conducted from
anywhere on the ship.

\item[Tractor Beam.] Allows the player to issue Maneuver Commands to
another friendly or enemy Spacecraft or battlecraft during his Command
Phase, as if he controlled the unit. The player must issue a Battle
Command to use the tractor beam. If he does so, a Civ Level 7 tractor
beam may be used to issue \emph{one} Maneuver Command to any one unit
\emph{within four hexes} of the ship with the tractor beam. A Civ
Level 8 tractor beam may be used to issue \emph{two} Maneuver Commands
to any one ship \emph{within six hexes} of the ship with the tractor
beam. A tractor beam may not be used to issue Maneuver Commands to an
enemy or friendly \emph{missile}. Each Maneuver Command issued by
using a tractor beam requires the expenditure of a number of Energy
Units equal to \emph{twice} the Energy Burn Rate of the target unit.

\item[Battlecraft.] Contains one \emph{Terwillicker-5000} or one
\emph{Terwillicker-X} (as specified by the scenario) that may be
launched from the Spacecraft. To launch a battlecraft, Battle Commands
must be issued in two friendly Command Phases (see \ref{sec:battle-commands}).
Once a battlecraft has been launched from its pod, it is treated as
any other Spacecraft. However, a separate Battle Log is not used; the
requisite information for each battlecraft is listed on the ``mother''
ship's Battle Log. A battlecraft may be returned to the ship from
which it was launched (only) during any Command Phase in which the two
units occupy the \emph{same} hex, have \emph{identical velocities},
and are pointing in the \emph{same direction}. If these requirements
are met, the battlecraft may be docked in its pod by issuing a
Rendezvous Command. Each battlecraft begins play with 15 Energy Units.
When in its pod, a battlecraft may replace expended Energy Units by
drawing from the supply of Energy Units aboard the ship; no Command is
required to do so (see \ref{sec:battlecraft}).

\item[Standard Jump, Augmented Jump, and Hunter.] In certain
scenarios, one or both players may remove a ship with a jump pod
entirely from play (which is better than being destroyed). Otherwise,
a jump pod has no effect on play. See \ref{sec:maneuver-commands} for
details. A Hunter pod contains a standard jump engine.

\item[Energy.] Contains \textbf{144} additional Energy Units. A ship
with an energy pod expends all the Energy Units in the pod before
expending Energy Units in its hull.
\end{description}

\subsection[Pods and Victory Conditions]{The following pods have no
  effect on play except that damaging or destroying any of them on an
  enemy  
  Spacecraft may aid a player in fulfilling his Victory Conditions.}
\label{sec:pods-victory}



Luxury Cabin, Standard Cabin, Crew, Advanced Medical, Bio-Research,
Standard Cargo, Buffered Cargo, Living cargo, Lander, Survey, Robot
and Equipment, Explorer, Escape/EVA. Each of these pods may have an
Armor Rating ranging from \textbf{0} to \textbf{2}, as specified by
the scenario.


\subsection[Pod Attribute Chart]{The Pod Attribute Chart summarizes
  the properties of all 
  the pods that may be used during the game.}
\label{sec:pod-attribute-chart}

See charts and tables. 


\begin{table}[htbp]
  \centering
  \fbox{%
    \begin{minipage}{6in}
      \centering
      \caption{Pod Attribute Chart}
      \label{tab:pod-attribute}
      
      \medskip
      
      \begin{tabular}{lcccccccccccc}
        POD TYPE & 
        \rotate{\parbox{1.25in}{LASER/PARTICLE\\WEAPONS}} &
        \rotate{\parbox{1.25in}{UNGUIDED\\MISSILE}} &
        \rotate{GUIDED MISSILE} &
        \rotate{\parbox{1.25in}{INTELLIGENT\\MISSILE}} &
        \rotate{MIMS} &
        \rotate{\parbox{1.25in}{BATTLE\\COMMANDS}} &
        \rotate{CIV LEVEL} &
        \rotate{\parbox{1.25in}{TARGET\\PROGRAM}} &
        \rotate{JUMP} &
        \rotate{SEE \ref{sec:special-attributes}} &
        \rotate{\parbox{1.25in}{NUMBER OF\\FIRES}} &
        \rotate{ARMOR}\\ 
        \rowcolor{grey}
        Hunter & Y & 2 & 0 & 1 & 0 & 0 & 8 & -4 & Y & X & 1 & 2\\
        Light Weapon & Y & 5* & 3* & 0 & 0 & 1 & 6 & -2 & N & - & 1 & 1\\
        \rowcolor{grey}
        Heavy Weapon & Y & 6 & 5* & 3* & 1* & 1 & 7 & -4 & N & - & 1 & 2\\
        Arsenal & Y & 8 & 7 & 5* & 2* & 1 & 8 & -4 & N & - & 2 & 2\\
        \rowcolor{grey}
        Battle Comm & N & 0 & 0 & 0 & 0 & 2 & 8 & -6 & N & X & 1 & 2\\
        Tractor Beam & N & 0 & 0 & 0 & 0 & 0 & 7 & - & N & X & 0 & 0--2\\
        \rowcolor{grey}
        Tractor Beam & N & 0 & 0 & 0 & 0 & 0 & 8 & - & N & X & 0 & 0--2\\
        Battlecraft & N & 0 & 0 & 0 & 0 & 0 & 7 & - & N & X & 0 & 0--2\\
        \rowcolor{grey}
        Standard Jump & N & 0 & 0 & 0 & 0 & 0 & 7 & - & Y & X & 0 & 0--2\\
        Augmented Jump & N & 0 & 0 & 0 & 0 & 0 & 8 & - & Y & X & 0 & 0--2\\
        \rowcolor{grey}
        Energy & N & 0 & 0 & 0 & 0 & 0 & 7 & - & N & X & 0 & 0--2\\
        All others & N & 0 & 0 & 0 & 0 & 0 & V & - & N & - & 0 & 0--2\\
      \end{tabular}

      \medskip

      \parbox{\textwidth}{* Launch of missiles requires Prepare
        Missile Command in previous Command  
        Phase. See \ref{sec:pods-dv} for detailed explanation of use.}
      
      \parbox{\textwidth}{\textbf{Note:} A \emph{non-streamlined}
        spaceship hull may carry \emph{two} Energy Pods using only one
        point of pod capacity.\label{sec:add-two-energy}}
    \end{minipage}}
\end{table}

\section{Movement and Direction}
\label{sec:movement-direction}

\noindent\textbf{GENERAL RULE:}

During a player's Movement Phase, he must move each and every one of
his ships, battlecraft, and missiles currently in play. The number of
hexes each unit must be moved is determined by its Velocity marker.
The direction each unit must be moved is determined by the direction
in which the unit is pointing. The player has no choice in the
movement of his units during the Movement Phase (\textbf{Exception:}
See \ref{sec:planets-maneuvering}).

\medskip

\noindent\textbf{PROCEDURE:}

The player moves his units one at a time, in any order he desires. He
moves each unit a number of hexes equal to its current velocity.  Each
unit is moved in a straight line, in the direction in which it is
pointing. When the move is completed, the unit should point in the
same direction in its destination hex.

\medskip

\noindent\textbf{CASES:}


\subsection[Direction]{A unit may point in one of 12 directions.}
\label{sec:direction}

\begin{center}
  \includegraphics{12-directions}
\end{center}


This is shown by orienting the unit marker's arrow toward a hex side
or a hex corner. These directions may be equated to the numbers on a
clock face.


\begin{center}
  \includegraphics[height=1.25in]{movement-side}
\end{center}


A unit that is pointing toward a hex side is moved along the hex row
extending from that hex side.


\begin{center}
  \includegraphics[height=1.25in]{movement-corner}
\end{center}

A unit that is pointing toward a hex corner is moved along a line
extending from that corner. However, the unit is moved in a zigzag
pattern; first to the left, then to the right, then to the left, etc.

\medskip

If a unit that is pointing towards a hex corner is moved an odd number
of hexes, a Direction Reminder marker should be placed in the hex
immediately ahead of the unit's final position in the move. (i.e. in
the hex the unit would occupy if the length of its move were one hex
more). This reminds the players which zigzag hex row the unit should
be moved through in its next move, so that ``slippage'' of the unit's
direction to either side will not occur. A Direction Reminder marker
has no effect on play (except to remind the player of the unit's
proper direction) and is removed when the player changes direction.

The players must make sure that the orientation of each unit is always
clearly evident. When more than one unit occupies a single hex,
special care must be taken to show the orientation of each unit. The
direction a unit points may be changed only during the Command Phase
(\textbf{Exception:} See \ref{sec:planets-maneuvering}).

\subsection[Map Placement]{When a ship or battlecraft is directed to
  move off the maps currently in use, an unused map should be  
  placed to abut the map edge from which the unit will exit.}
\label{sec:map-placement}

 

This may be done whenever necessary, as long as the relative positions
of all units and markers in the game remains the same. When placing a
new map, make sure that the hex grid pattern is properly aligned with
the other maps. A missile that is directed to move off the map is
removed from play; a map is not specially positioned for it.

\subsection[Zero Velocity]{A unit with a zero Velocity marker is not moved.}
\label{sec:zero-velocity}



A unit without a Velocity marker that is stacked with a ship (such as
an unlaunched missile or battlecraft) is moved with the ship and has
no effect on the ship's movement.


\subsection[Occupied Hexes]{A unit may be moved into and through hexes
  occupied by 
  enemy or friendly units.}
\label{sec:occupied-hexes}



The Interception Routine (see \ref{sec:interception-routine}) is
conducted when a missile is moved into a hex occupied by an enemy unit
at any point during its move, or if any unit is moved into a hex
occupied by an enemy missile at any point during its move. There is no
limit to the number of units that may occupy a single hex at any given
time.


\subsection[Planets And Maneuvering]{The instant a Spacecraft or
  battlecraft is moved into a planet hex, the Phasing player may issue
  the unit  
  Maneuver Commands.}
\label{sec:planets-maneuvering}

 

The number of Maneuver Commands the unit may receive is determined as
in \ref{sec:number-maneuver-commands}. Such a unit may immediately
receive the following Maneuver Commands only: Accelerate, Decelerate,
and Direction Change, within the restrictions of
\ref{sec:maneuver-commands}. However, the unit's current velocity may
not be reduced below \textbf{1} in this manner (but may be during the
Command Phase). A unit expends no energy for Maneuver Commands
received as a result of entering a planet hex.

If a unit's current velocity is altered upon entering a planet hex,
the number of hexes the unit has already traversed in its move is
subtracted from the unit's new velocity to determine the number of
hexes the unit must now be moved (in its new direction, if also
altered). If this number is \textbf{0} or less, the unit is moved no
further (it remains in the planet hex).

A unit with a current velocity of \textbf{1} that occupies a planet
hex is considered to be orbiting that planet, and need not be moved
during the Movement Phase.

If the current velocity of a streamlined Spacecraft or battlecraft in
a planet hex is reduced to \textbf{0} during the Command Phase, the
unit is considered to land on the planet during the immediately
following friendly Movement Phase. When this occurs, the unit's
Velocity marker is removed and the unit remains in the planet hex for
the remainder of the game. The unit may not be used for any game
functions but is not considered destroyed. A unit that is not
streamlined may not land on a planet.

A missile is automatically destroyed upon entering a planet hex. 


\subsection[Asteroids]{When a unit is moved into a hex occupied by
  asteroids, the 
  owning player must check for collision.}
\label{sec:asteroid-collision}



When an asteroid hex is entered, the unit's movement is interrupted
while the player rolls a die. If the die result is less than or equal
to the current velocity of the unit, it is hit by an asteroid. The
player must then use the Hit Table as if the unit had just been hit by
enemy fire (see \ref{sec:hit-table-used}). However, if a \emph{critical hit}
result is obtained from the table, it is considered a \emph{no effect}
result.


\subsection[Energy In Movement Phase]{No Energy Units or Energy Blocks
  are expended during the 
  Movement Phase.}
\label{sec:energy-movement}



Energy is expended during the Command Phase and the Fire Phase. 


\subsection[Hyperjump]{Under certain conditions, a ship may conduct a
  hyperjump 
  during the Movement Phase.}
\label{sec:hyperjump}



When a ship does so, it is immediately removed from play. See
\ref{sec:maneuver-commands} for details.

\section{Commands}
\label{sec:commands}

\noindent\textbf{GENERAL RULE:}

Each player issues Commands to his units during his Command Phase. A
player may issue \emph{Maneuver Commands} to all his Spacecraft,
battlecraft, and missiles (except unguided missiles) in play. A player
may issue \emph{Battle Commands} to all his Spacecraft (only) that
possess the requisite pods. The number of Maneuver Commands that may
be issued to a unit in a single Command Phase equals the unit's
Maneuver Rating minus its current velocity. The number of Battle
Commands that may be issued to a Spacecraft in a single Command Phase
equals the sum of the Battle Commands provided by the ship's eligible
pods.

\medskip

\noindent\textbf{PROCEDURE:}

The Phasing player issues Commands to each of his units individually,
in any order he desires. For each unit, he calculates the number of
Maneuver Commands it may receive and then issues those Commands to the
unit by performing the appropriate function listed in
\ref{sec:maneuver-commands}. If the unit is a Spacecraft, he
calculates the number of Battle Commands it may receive and issues
those commands to the ship by performing the appropriate functions
listed in \ref{sec:battle-commands}. He then records the requisite
expenditure of Energy Blocks (if the unit is a Spacecraft) or Energy
Units (if a battlecraft or a missile).

\medskip

\noindent\textbf{CASES:}


\subsection[Number Of Maneuver Commands]{The number of Maneuver
  Commands issued to a unit in a single Command Phase may never exceed
  the  
  unit's Maneuver Rating.}
\label{sec:number-maneuver-commands}



The number of Maneuver Commands a unit may receive is further reduced
by its current velocity. Thus, if a unit with a Maneuver Rating of 7
had a current velocity of \textbf{4}, it could only receive three
Maneuver Commands. Exception: If a unit's current velocity equals or
exceeds its Maneuver Rating, the unit may be issued \emph{one}
Decelerate or Accelerate Command \emph{only}.

No Maneuver Command may be issued to a Spacecraft that possesses a
Prepare Jump or Jump marker, or that has an operating force field. No
Maneuver Commands may be issued to a \emph{guided} missile unless the
appropriate Battle Command is issued to the Spacecraft controlling the
missile (see \ref{sec:battle-commands}). An \emph{unguided} missile has no
Maneuver Rating and may not be issued Maneuver Commands. However,
during each Command Phase, the current velocity of each of the Phasing
player's unguided missiles must be increased by one.

A player is not required to issue a ship its maximum number of
Maneuver Commands. However, Maneuver Commands may not be transferred
from one unit to another or accumulated from Game-Turn to Game-Turn.
These restrictions apply to Battle Commands as well.

\subsection[Maneuver Commands]{An eligible unit may be issued
  following Maneuver 
  Commands:}
\label{sec:maneuver-commands}

\begin{description}
\item[Accelerate/Decelerate.] The Velocity marker of the unit is
changed for a marker one greater or less in value. Thus, if a unit
with a current velocity of 3 is issued an Accelerate Command, its
Velocity marker is exchanged for a 4 Velocity marker. If the unit were
issued a Decelerate Command instead, it would receive a 2 Velocity
marker. Assuming a unit has the requisite Maneuver Commands, it may be
issued any number of Accelerate or Decelerate Commands in a single
Command Phase, up to a number \emph{equal to} its Velocity Rating.

\item[Direction Change.] The direction that the unit is pointing is
altered by one position (from a hex side to an adjacent hex corner, or
from a hex corner to an adjacent hex side). Assuming a unit has the
requisite Maneuver Commands, it may be issued any number of Direction
Change Commands in a single Command Phase.

\item[Weave.] This Command may be issued to Spacecraft and
battlecraft only (not to missiles). The unit is immediately moved to
any adjacent hex. The unit's velocity and direction are not changed
(unless additional Maneuver Commands are issued). Only one Weave
Command may be issued to a given unit in a single Command Phase. A
unit may not weave into an asteroid hex or planet hex.

\item[Prepare Jump/Abort Jump/Jump.] In certain scenarios, a
Spacecraft with a standard jump pod may prepare for a hyperjump away
from the playing area. A Prepare Jump marker is placed atop the ship.
In the following friendly Command Phase, the player must issue a Jump
Command to the ship (the Prepare Jump marker is flipped over) or issue
an Abort Jump Command to the ship (the Prepare Jump marker is
removed). If a Jump Command is issued to the ship, it must be removed
from play in the following friendly Movement Phase. A ship with an
augmented jump pod need not be issued a Prepare Jump Command; it
requires only a Jump Command (place a Jump marker atop the ship). A
Jump Command may not be issued to a ship that has an active force
field (a Prepare Jump Command may be issued to such a ship).
\end{description}

\subsection[Number Of Battle Commands]{The number of Battle Commands
  issued to a Spacecraft may not exceed the allotment provided by its  
  eligible pods.}
\label{sec:number-battle-commands-1}



A light weapon, heavy weapon, or arsenal pod each allow a ship to
receive one Battle Command. A battle communications pod allows a ship
to receive two Battle Commands. Thus, a Spacecraft with two heavy
weapon pods and a battle communications pod could receive four Battle
Commands in a single Command Phase. The number of Battle Commands a
Spacecraft may receive has no effect on the number of Maneuver
Commands it may receive, and vice versa.



\subsection[Battle Commands]{An eligible Spacecraft may be issued the
  following Battle 
  Commands:}
\label{sec:battle-commands}


\begin{description}
\item[Prepare Missile.] If a Spacecraft has a light weapon, heavy
weapon, or arsenal pod, the Phasing player may prepare a missile for
launch by placing the appropriate missile counter (without a Velocity
marker) \emph{face down} atop the ship. Consult the Pod Attribute
Chart to find which pods may launch missiles and which of those
missiles require a Prepare Missile Command. A prepared missile may be
launched in any subsequent friendly Fire Phase (see \ref{sec:missile-launch}).
Assuming a Spacecraft has the requisite Battle Commands, it may be
issued any number of Prepare Missile Commands in a single Command
Phase. However, the maximum number of prepared missiles that a ship
may carry at one time is limited to the number of missile-carrying
pods the ship possesses. Thus, a ship with two heavy weapon pods may
carry no more than two prepared missiles at a time. Until a prepared
missile is launched, it is moved with its ship and has no effect on
play.

\item[Control Guided Missile.] The player may issue Maneuver
Commands to a guided missile previously launched from the Spacecraft.
By issuing the Spacecraft one such Battle Command, the player may
immediately issue any number of Maneuver Commands (within the
restrictions of \ref{sec:number-maneuver-commands}) to one of the
ship's guided missiles currently in play.

\item[Active Search.] The player may flip over every enemy unit that
is currently unrevealed within \emph{six} hexes of the Spacecraft to
which he is issuing this Command. This range is counted by including
the enemy unit's hex but not the searching Spacecraft's hex. Once a
unit is flipped over, it remains revealed for the rest of the game. If
the Spacecraft to which an Active Search Command is issued possesses a
battle communications pod, every inverted enemy unit within 10 hexes
is flipped over. Note: Active Search should not be confused with
\emph{detection}, which occurs automatically at the beginning of the
Command Phase and does not require a command.

\item[Prepare Battlecraft.] If a Spacecraft has a battlecraft pod
containing a battlecraft, the player may prepare the battlecraft for
launch by placing the appropriate battlecraft counter (without a
Velocity marker) face down atop the Spacecraft. The battlecraft
remains stacked with the Spacecraft (and is moved with the ship) until
the player issues a Launch Battlecraft Command to the ship in any
subsequent friendly Command Phase.

\item[Launch Battlecraft.] The player may launch a prepared
battlecraft (that is, a battlecraft placed atop a Spacecraft in a
previous friendly Command Phase) by placing the battlecraft in any hex
adjacent to the Spacecraft. The battlecraft must be assigned a
Velocity marker equal to, one less than, or one greater than the
current velocity of the Spacecraft. \textbf{Exception:} A battlecraft
must be launched with a minimum velocity of \textbf{1}. The
battlecraft must be pointing in the same direction that the ship is
pointing when launched, or one of the two adjacent directions on
either side (thus, a battlecraft may be pointing in one of five
directions when launched). Launching a battlecraft does not require
the expenditure of energy from the involved Spacecraft or battlecraft.

\item[Rendezvous.] If a friendly Spacecraft or battlecraft occupies
the same hex as an enemy or friendly Spacecraft or battlecraft, the
two may be docked together. However, the two units must have identical
velocities and must point in the same direction. This Command is used
when a player wishes to dock a battlecraft in the ship from which it
was launched (see \ref{sec:special-attributes}) or when a player
wishes to dock with an enemy ship to fulfil a requirement listed in a
scenario. Two Spacecraft that are docked together use \emph{one}
Velocity marker only. During the Command Phase, the Phasing player may
issue Maneuver Commands to both ships as if they were one. If the
expenditure of an Energy Block is required, a number of \emph{Energy
  Units} equal to the Energy Burn Rate of both Spacecraft combined is
expended. In this way one ship may ``tow'' another.

\item[Tractor Beam.] If a Spacecraft has a tractor beam pod, the
player may activate its tractor beam. The player then issues Maneuver
Commands to one other Spacecraft or battlecraft, as explained in
\ref{sec:special-attributes}. A single tractor beam pod may only be
issued one Command per Command Phase and does not remain active from
Game-Turn to Game-Turn.

\item[Activate/Deactivate Force Field.] If a Spacecraft possess a
force field (Class 1 or 2), it may be activated by flipping the
Spacecraft over to its force field side. When activated, the force
field provides protection against enemy missiles, but not against
enemy laser or particle fire. Furthermore, the only commands that may
be issued to a ship with an active force field are Prepare Jump, Abort
Jump, Prepare Missile, Active Search, and Prepare Battlecraft. A
missile may not be launched (but laser and particle fire may be
conducted) from a Spacecraft with an active force field. An active
force field may be deactivated by flipping the ship counter back to
its normal side.  A player may \emph{attempt} to activate the force
field of a ship that has been intercepted by a missile at the moment
of interception (see \ref{sec:missile-force-field}).
\end{description}

\subsection[Commands And Energy]{A unit must expend Energy Units or
  Blocks when issued 
  certain Commands, depending on the type of unit.}
\label{sec:commands-energy}



The expenditure of Energy Units and Blocks is recorded on the
appropriate Spacecraft Log (see \ref{sec:energy-unit-track}).

\begin{itemize}
\item A \textbf{Spacecraft} must expend one Energy \emph{Block} and a
  \emph{battlecraft} must expend one Energy \emph{Unit} when it is
  issued \emph{more than one} \textbf{Accelerate}, \textbf{Decelerate}
  and/or \textbf{Direction Change} Command in a single Command Phase.
  Regardless of how many of these Commands (beyond one) a ship or
  battlecraft receives in a Command Phase, only one Energy Block or
  Unit is expended. A Spacecraft or battlecraft that receives only one
  of the above Commands in a single Command Phase expends no energy
  (although it may expend energy as a result of other Commands it
  receives).
\item A \textbf{Spacecraft} must expend one Energy \emph{Block} and a
  battlecraft must expend one Energy \emph{Unit} when issued a
  \textbf{Weave} Command. This expenditure is in addition to any that
  may be required for other Maneuver Commands.
\item A \textbf{missile} must expend one Energy \emph{Unit} each time
  it is issued a Maneuver Command. Thus, if a missile is issued two
  Accelerate and one Direction Change Commands, three Energy Units are
  expended.
\item A \textbf{Spacecraft} must expend one Energy \emph{Block} when
  issued an \textbf{Activate Force Field} Command.
\item A \textbf{Spacecraft} that uses its \textbf{tractor beam} must
  expend a number of Energy Units equal to\emph{ twice} the Energy
  Burn Rate of the ship or battlecraft to which it is issuing a
  Maneuver Command. This expenditure must be made for \emph{each}
  Maneuver Command issued to the target unit. A unit that is issued a
  Maneuver Command through a tractor beam does not expend energy for
  that Command.
\item During the Fire Phase, a\textbf{Spacecraft} or
  \textbf{battlecraft} must expend Energy Units when conducting a
  laser barrage, a particle burst, or a particle barrage (see
  \ref{sec:fire-energy}).
\end{itemize}

\subsection[No Energy Left]{A unit that has expended all its available
  energy may be issued no Command that requires the expenditure  
  of energy.}
\label{sec:zero-energy}

 

\subsection[Energy Expenditure Summary]{The Energy Expenditure Summary
  lists the names of every 
  Command that a player may possibly issue to a  
  unit.}
\label{sec:energy-expenditure-summary}


See charts and tables. 


\begin{table}[htbp]
  \centering
  \fbox{%
    \begin{minipage}{0.95\textwidth}
      \caption{Energy Expenditure Summary}
      \label{tab:energy-expenditure}
      
      \medskip
      
      {\large\textbf{Action or Situation:} \emph{Energy Expenditure}}\\ 
      \begin{multicols}{2}
        \textbf{Issuing more than 1 Acceleration, Deceleration, or Direction 
          Change Maneuver Command per Phase:}\\
        \textbf{1} \emph{Energy Block if spaceship; 1 Energy Unit if
        battlecraft.}\\
      \textbf{Missile Maneuver:}\\
      \textbf{1} \emph{Energy Unit per Maneuver Command.} \\
      \textbf{Weave Command:}\\
      \textbf{1} E\emph{nergy Block if spaceship;} \textbf{1}
        \emph{Energy Unit if battlecraft.}  \\
      \textbf{Activate Spaceship Force Field:}\\
      \textbf{1} \emph{Energy Block.} \\
      \textbf{Use Tractor Beam:}\\
      \emph{Energy Units equal to twice the Energy Burn Rate of the target unit 
      per each Maneuver Command.} \\
      \textbf{Replenish battlecraft energy Levels:}\\
      \emph{Number of Energy Units needed or desired, up to a maximum of} \textbf{15}. \\
      \textbf{Maneuver Docked Spaceships:}\\
      \emph{Energy Units equal to sum of both ships Energy Burn Rate.}\\
      \textbf{Particle Burst:}\\
      \textbf{1} \emph{Energy Unit.}\\ 
      \textbf{Laser Barrage:}\\
      \textbf{2} \emph{Energy Units.}\\ 
      \textbf{Particle Barrage:}\\
      \textbf{3} \emph{Energy Units.}\\ 
      \textbf{If Engine is damaged:}\\
      \textbf{1} \emph{Energy Block per each and every Maneuver Command.}\\
      \textbf{If Energy Pod is Damaged:}\\
      \textbf{-10} \emph{Energy Units each Command Phase.}\\
      \textbf{If Energy Pod is Destroyed:}\\
      \emph{Total Energy Units expended immediately increased to}
        \textbf{144}.
      \end{multicols}
    \end{minipage}}
\end{table}

%%% Local Variables: 
%%% mode: latex
%%% TeX-master: "universe_plus"
%%% End: 

\section{Laser and Particle Fire }
\label{sec:laser-particle-fire}



\noindent\textbf{GENERAL RULE:}

During a player's Fire Phase, he may conduct laser and/or particle
fire against revealed and unrevealed enemy units with all his eligible
spacecraft and battlecraft. There are four types of fire: a laser
burst, a laser barrage, a particle burst, and a particle barrage.
Successful fire may result in a pod or other part of an enemy unit
being damaged or destroyed. Fire may be conducted in any direction.

\medskip

\noindent\textbf{PROCEDURE:} 

The Phasing player declares and resolves each fire one at a time. All
fires conducted from one Spacecraft or battlecraft must be resolved
before conducting fires from another Spacecraft or battlecraft. For
each fire conducted, the Phasing player undertakes the following
steps, in order.

\begin{enumerate}
\item Declare what type of fire is being conducted, from where the
  fire is coming, and which enemy Spacecraft, battlecraft or missile
  is the target of the fire. If necessary, note the expenditure of
  Energy Units on the appropriate Spacecraft Log.
\item Determine the \emph{range} in hexes from the firing unit to the
  target unit. Range is counted by including the target unit's hex and
  all hexes lying between the firing unit and the target unit, but not
  the firing unit's hex.
\item Determine the \emph{relative velocity} of the two units, using
  the Relative Velocity Chart if necessary. Subtract the
  \emph{Targeting Program modifier} of the firing unit from the
  relative velocity and then add the modified relative velocity to the
  range to determine the \emph{Target Value}.
\item Refer to the Fire Results Table, cross-referencing the proper
  Target Value column with the row matching the type of fire declared
  to find the \emph{Hit Chance}. Roll the die; if the die result is
  less than or equal to the Hit Chance, the target may have been hit.
  Proceed to Step \ref{item:fire-5}. If the die result is greater than
  the Hit Chance, the fire has missed the target and this procedure is
  concluded.
\item Roll the die again and refer to the Hit Table to determine which
  part (if any) of the enemy unit has been hit. The non-Phasing player
  must immediately apply the effects of the hit to the target
  unit.\label{item:fire-5}
\end{enumerate}

\medskip

\noindent\textbf{CASES:} 

\subsection[Number And Types Of Fires]{The number and types of fires a
  Spacecraft or battlecraft may conduct in a single Fire Phase depends
  on  
  the Class of its burster and the attributes of its pods.}
\label{sec:number-types-fires}

\begin{itemize}
\item A \textbf{Class 1 burster} on a Spacecraft or battlecraft allows
  one laser \textbf{burst} each Fire Phase.
\item A \textbf{Class 2 burster} on a Spacecraft or battlecraft allows
  one laser \textbf{burst} or one laser \textbf{barrage} each Fire
  Phase.
\item A \textbf{Hunter}, \textbf{Light Weapon} and \textbf{Heavy
    Weapon Pod} each allow \emph{one fire of any type} (or one missile
  launch, see \ref{sec:number-of-fires}) each Fire Phase.
\item An \textbf{Arsenal Pod} allows \emph{two fires of any type} (or
  one fire and one missile launch, or two missile launches) each Fire
  Phase.
\item A \textbf{Battle Communications Pod} allows \emph{one additional
    fire of any type} (or one additional missile launch) from any of
  the above eligible items each Fire Phase.
\end{itemize}

All these items are cumulative. Thus, a Spacecraft with two light
weapon pods may be used to conduct three fires each Fire Phase (one
from its burster and one from each weapon pod). If the ship also
possesses a battle communications pod, it could conduct one additional
fire from its burster or either weapon pod (for a total of four fires
per Fire Phase).

The number of fires a Spacecraft may conduct in a Fire Phase should
not be confused with the number of Battle Commands the ship may
receive in a Command Phase. Fires may not be conducted in the Command
Phase, and Commands may not be issued in the Fire Phase.


\subsection[Fire And Energy Expenditure]{A unit that conducts any type
  of fire except a laser burst 
  must expend one or more Energy Units.}
\label{sec:fire-energy}



A \textbf{particle burst} costs \textbf{1} Energy Unit, a
\textbf{laser barrage} costs \textbf{2} Energy Units, and a
\textbf{particle barrage} costs \textbf{3} Energy Units. The
expenditure of Energy Units is recorded on the appropriate Spacecraft
Log (see \ref{sec:energy-unit-track}).


\subsection[Relative Velocity]{The relative velocity of the firing unit
  and the target unit is determined by comparing the direction and  
  current velocity of each unit.}
\label{sec:relative-velocity}

\begin{minipage}[c]{0.8\textwidth}
  \begin{center}
    \hfill
    \includegraphics[height=2in]{rel-vel-side}
    \hfill
    \includegraphics[height=2in]{rel-vel-corner}
    \hfill
  \end{center}
\end{minipage}

\begin{bfseries}
  \small Imagine the two units are in the same hex and compare their
  directions on one of the following diagrams. Use the first if the
  firing unit points toward a hex side and the second if the firing
  unit points toward a hex corner. The direction of the target unit is
  matched to one of the 12 arrows radiating from the hexes.
\end{bfseries}

Depending on the unit's relative directions, one of the following
statements will apply:

\begin{enumerate}
\item If the target unit is pointing in the \emph{same} direction as
  the firing unit, or an \emph{adjacent} direction, their relative
  velocity equals the \emph{difference} between their current
  velocities.\label{item:relvel-1}
\item If the target unit is pointing in the \emph{opposite} direction
  as the firing unit, or a direction \emph{adjacent to the opposite
    direction}, their relative velocity equals the sum of their
  current velocities.\label{item:relvel-2}
\item If the target unit is pointing in any of the six directions not
  covered in the above two statements, refer to the Relative Velocity
  Chart and cross-reference the current velocity of each unit on the
  chart to find their relative velocity.\label{item:relvel-3}
\end{enumerate}

\textbf{Example:} The firing unit has a velocity of \textbf{4} and the target
unit has a velocity of \textbf{3}. If their directions apply to statement 1,
their relative velocity is \textbf{1}. If their directions apply to statement
2, their relative velocity is \textbf{7}. If their directions apply to
statement 3, the Relative Velocity Chart is used to determine that
their relative velocity is \textbf{5}.


\begin{table}[htbp]
  \centering
  \fbox{%
    \begin{minipage}{4in}
      \centering
      \caption{Relative Velocity Chart}
      \label{tab:relative-velocity}
      
      \medskip
      
      \begin{tabular}{cccccccccc}
        CURRENT &\\
        VELOCITY OF & \multicolumn{9}{c}{CURRENT VELOCITY OF FIRING
        UNIT} \\
        TARGET UNIT & 0,1 & 2 & 3 & 4 & 5 & 6 & 7 & 8 & 9\\
        \rowcolor{grey}
        0,1 & 1* & 2 & 3 & 4 & 5 & 6 & 7 & 8 & 9\\
        2 & 2 & 3 & 4 & 4 & 5 & 6 & 7 & 8 & 9\\
        \rowcolor{grey}
        3 & 3 & 4 & 4 & 5 & 6 & 7 & 8 & 9 & 9\\
        4 & 4 & 4 & 5 & 6 & 6 & 7 & 8 & 9 & 10\\
        \rowcolor{grey}
        5 & 5 & 5 & 6 & 6 & 7 & 8 & 9 & 9 & 10\\
        6 & 6 & 6 & 7 & 7 & 8 & 8 & 9 & 10 & 11\\
        \rowcolor{grey}
        7 & 7 & 7 & 8 & 8 & 9 & 9 & 10 & 11 & 11\\
        8 & 8 & 8 & 9 & 9 & 9 & 10 & 11 & 11 & 12\\
        \rowcolor{grey}
        9 & 9 & 9 & 9 & 10 & 10 & 11 & 11 & 12 & 13
      \end{tabular}

      \medskip

      \parbox{\textwidth}{* If the velocity of both units is
      \textbf{0}, the relative velocity is \textbf{0}. 
      
      If the compared directions of the target unit and the firing
      unit do not fulfil the conditions of statements
      \ref{item:relvel-1} or \ref{item:relvel-2} in
      \ref{sec:relative-velocity}, use this chart to determine the
      relative velocity of the units. The current velocity of each
      unit is cross-referenced to yield their relative velocity.}
    \end{minipage}}
\end{table}

\subsection[Relative Velocity And Range Reductions]{The relative
  velocity and the range between the firing unit and the target unit
  may be reduced if the  
  positions of the two units fulfil either of the following
  conditions.} 
\label{sec:relative-velocity-range}

\begin{enumerate}
\item A straight line may be drawn between the two units and their
  indicated directions. [\textbf{See \figurename\ \ref{fig:rel-vel-a}}] 
  
  If this applies, the relative velocity is determined as described in
  \ref{sec:relative-velocity} and is then \emph{halved}, rounding
  fractions up. The range between the two units is \emph{not}
  affected.
\item The two units are pointing in the \emph{same} direction and
  their current velocities are \emph{identical}. [\textbf{See \figurename\ 
  \ref{fig:rel-vel-b}}]
  
  If this applies, the relative velocity is \emph{automatically zero}
  and the \emph{range} between the two units is \emph{halved}, rounding
  fractions up. If conditions 1 and 2 apply in a single situation,
  condition 2 takes precedence.
\end{enumerate}

\begin{figure}[htbp]
  \centering
  \fbox{%
    \begin{minipage}{0.9\textwidth}
      \centering
      \label{fig:rel-vel-a}
      
      \medskip
      
      \includegraphics[width=0.8\textwidth]{rel-vel-line}
    \end{minipage}}
\end{figure}


\begin{figure}[htbp]
  \centering
  \fbox{%
    \begin{minipage}{0.9\textwidth}
      \centering
      \label{fig:rel-vel-b}
      
      \medskip
      
    \includegraphics[width=0.8\textwidth]{rel-vel-same-dir}
  \end{minipage}}
\end{figure}


\subsection[Targeting Program And Relative Velocity]{Each Spacecraft
  and battlecraft possesses a Targeting Program, which modifies the
  determined relative  
  velocity.}
\label{sec:targeting-program-rel-vel}



The modifier is listed on the Spacecraft Attribute Chart. If a fire is
being conducted from a hunter, light weapon, heavy weapon, or arsenal
pod, the Phasing player may use the Target Program modifier of either
the pod or the Spacecraft. If a Spacecraft possesses a battle
communications pod, a Targeting Program modifier of -6 is applied to
\emph{all} fire conducted from the ship.

If, after applying the Targeting Program modifier, the relative
velocity is less than zero, it is treated as zero. The Targeting
Program modifier is \emph{never} used to reduce the \emph{range}
between the firing unit and the target unit. After calculating the
modified relative velocity, it is added to the range to determine the
Target Value used with the Fire Results Table.

\subsection[Fire Results Table]{The Fire Results Table is used to
  determine if a fire has 
  hit its target.}
\label{sec:fire-results-table}



The Target Value [Range + (Relative Velocity - Targeting Program)] is
cross-referenced with the declared type of fire to determine the Hit
Chance. The Phasing player then rolls the die; if the die result is
equal to or less than the Hit Chance, he proceeds to the Hit Table.


\begin{table}[htbp]
  \centering
  \fbox{%
    \begin{minipage}{5in}
      \centering
      \caption{Fire Results Table}
      \label{tab:fire-results}
      
      \medskip
      
      \begin{tabular}{clcccccccc}
        ENERGY & & \multicolumn{8}{c}{TARGET VALUE}\\ 
        UNIT COST & TYPE OF FIRE & 0 & 1 & 2,3 & 4,5 & 6,7 & 8,9 &
        10,11 & 12--14 \\
        0 & Laser Burst & 7 & 6 & 5 & 4 & 3 & 2 & 1 & -\\
        \rowcolor{grey}
        2 & Laser Barrage & 9 & 8 & 7 & 6 & 5 & 4 & 3 & 1\\
        1 & Particle Burst & A & 9 & 7 & 4 & 1 & - & - & -\\
        \rowcolor{grey}
        3 & Particle Barrage & A & A & 9 & 7 & 4 & 1 & - & -\\
      \end{tabular}
    
      \medskip

      \parbox{\textwidth}{\textbf{A:} Hit is automatic; no dice roll is 
        conducted. Proceed to the Hit Table.

        \textbf{(-):} Hit is impossible; no die roll is 
        conducted. If the Target Value is greater 
        than 14, a hit with any type of fire is 
        impossible. See \ref{sec:fire-results-table} for detailed 
        explanation of use.}
    \end{minipage}}
\end{table}

\subsection[Hit Results Table]{The Hit Table is used to determine
        which pod or other part 
  of the target unit has been hit.}
\label{sec:hit-table-used}



The Phasing player rolls the die and locates the die result on the
table. With the exception of die result \textbf{1}, each result lists
two parts of the target unit. If the target unit possesses
\emph{neither} of the listed parts, the hit is a glancing blow that
has no effect. If the target unit possesses \emph{only one} of the
listed parts, that part has been hit. If the target unit possesses
\emph{both} of the listed parts, the \emph{Phasing} player
\emph{chooses} which of the two parts has been hit. He can inspect the
opposing player's applicable Spacecraft Log before choosing.

If a \textbf{1} is rolled when using the Hit Table, a \emph{critical
  hit} has occurred; the \emph{Phasing} player chooses one part of the
target unit listed on the Hit Table to receive the hit. He can inspect
the opposing player's applicable Spacecraft Log before choosing.

\textbf{Exception:} If the target unit is \emph{unrevealed}, a critical hit
is treated as \emph{no hit}.

If the target unit is a \emph{revealed missile}, it is destroyed on
the result of \textbf{1} or \textbf{2}. If the missile is
\emph{unrevealed}, it is destroyed on a result of \textbf{2} only. No
other result on the Hit Table affects a missile.


\begin{table}[htbp]
  \centering
  \fbox{%
    \begin{minipage}{3in}
      \centering
      \caption{Hit Table}
      \label{tab:hit}
      
      \medskip
      
      \begin{tabular}{cp{2in}}
        DIE & Part of Target Hit\\
        \rowcolor{grey}
        1 & Critical Hit. If the unit is a 
        revealed missile, it is 
        destroyed. If the unit is 
        unrevealed (of any type), treat 
        as ``no hit''. \\
        2 & 
        Bridge, Engine. If the unit is a 
        missile (revealed or 
        unrevealed), it is destroyed. \\
        \rowcolor{grey}
        3 & Force Field, Pod 8 \\
        4 & Pod 1, Pod 9 \\
        \rowcolor{grey}
        5 & Pod 2, Pod 10 \\
        6 & Pod 3, Pod 11 \\
        \rowcolor{grey}
        7 & Pod 4, Pod 12 \\
        8 & Pod 5, Pod 13 \\
        \rowcolor{grey}
        9 & Pod 6, Pod 14 \\
        10 & Pod 7, Pod 15 \\
      \end{tabular}

      \medskip

      \parbox{\textwidth}{See \ref{sec:hit-table-used} for detailed
      explanation of use.}
    \end{minipage}}
\end{table}


\subsection[Recording Hits]{When a unit receives a hit, the owning
  player must record it on the appropriate Spacecraft Log. The effects  
  of a hit depend on the Armor Rating of the part hit. }
\label{sec:recording-hits}

\begin{itemize}
\item A part with a \textbf{0} Armor Rating is \emph{destroyed} when
  first hit. An \textbf{\textsf{X}} is placed in the Status Box for that part
  on the Spacecraft Log. Any further hits on that part have no
  additional effect.
\item A part with a \textbf{1} Armor Rating is \emph{damaged} when
  first hit. A \textbf{D} is placed in the Status Box for that part on
  the Spacecraft Log. The part is destroyed when it receives a second
  hit.
\item A part with a \textbf{2} Armor Rating is made \emph{vulnerable}
  when first hit. A \textbf{V} is placed in the Status Box for that
  part on the Spacecraft Log.  The part is damaged when it receives a
  second hit and destroyed when it receives a third hit.
\item A missile is always destroyed when first hit. Draw a line
  through all the boxes for that missile on the Spacecraft Log and
  remove the missile from play.
\end{itemize}

The bridge, engine and force field (if any) of a Spacecraft are
located in the main hull and are considered to have the Armor Rating
of the Spacecraft.

\subsection[Damage And Capabilities]{When a pod or other part of a
  Spacecraft or battlecraft is damaged or destroyed, the capabilities
  of that  
  part are immediately impaired.}
\label{sec:damage-capabilities}



The following list summarizes all the effects of damage and destruction. 

\begin{description}
\item[Bridge.] \emph{Damaged:} The Maneuver Rating of the unit is
  reduced by two and the unit may no longer receive Weave Commands.
  \emph{Destroyed:} The Maneuver Rating of the unit is reduced to
  \textbf{1} and the unit may no longer receive Weave Commands.
\item[Engine.] \emph{Damaged:} Each and every Maneuver Command issued
  to the unit requires the expenditure of one Energy Block.
  \emph{Destroyed:} The unit may receive no Maneuver Commands at all.
\item[Class 1 Force Field.] \emph{Damaged} or\emph{Destroyed:} The force
  field may not be used at all.
\item[Class 2 Force Field.] \emph{Damaged:} The force field is
  considered to have the protective ability of a Class 1 force field
  and may not be activated at the moment of missile interception (see
  \ref{sec:missile-force-field}). \emph{Destroyed:} The force field
  may not be used at all.
\item[Hunter Pod.] \emph{Damaged:} All missiles in the pod are lost,
  including any currently prepared for launch (cross them off the
  appropriate Spacecraft Log); laser and particle barrages may not be
  conducted from the pod (laser and particle bursts may be conducted);
  the pod may not be used to hyperjump. \emph{Destroyed:} The pod is
  totally eliminated.
\item[Light Weapon or Heavy Weapon Pod.] \emph{Damaged:} All missiles
  in the pod are lost, including any currently prepared for launch;
  any guided missiles previously launched from the pod may not be
  issued Maneuver Commands; laser and particle barrages may not be
  conducted from the pod (laser and particle bursts may be conducted).
  \emph{Destroyed:} The pod is totally eliminated.
\item[Arsenal Pod.] \emph{Damaged:} Same as damage to a light weapon
  or heavy weapon pod; in addition, the pod only allows one fire per
  Fire Phase (instead of two). \emph{Destroyed:} The pod is totally
  eliminated.
\item[Battle Communications Pod.] \emph{Damaged:} The pod allows only
  one additional Battle Command per Command Phase (instead of two);
  the pod's Targeting Program modifier is eliminated (the modifier of
  the Spacecraft or firing pod being used instead); the pod does not
  allow an additional fire; the pod does not increase the range of an
  Active Search. \emph{Destroyed:} The pod is totally eliminated.
\item[Tractor Pod.] \emph{Damaged} or \emph{Destroyed:} The tractor beam may
  not be used at all.
\item[Battlecraft Pod.] \emph{Damaged} or \emph{Destroyed:} A battlecraft may
  not be launched from or dock with the pod. A battlecraft inside the
  pod when damaged or destroyed may not be used at all.
\item[Standard or Augmented Jump Pod.] \emph{Damaged} or \emph{Destroyed:}
  The pod may not be used to hyperjump.
\item[Energy Pod.] \emph{Damaged:} Ten Energy Units must be expended
  each friendly Command Phase (in addition to any other expenditures
  of energy) until a total of 144 Energy Units have been expended
  (including previously expended energy). \emph{Destroyed:} The total
  expenditure of energy for the Spacecraft must be immediately
  brought up to 144 Energy Units; the pod is considered empty.
\end{description}

Damage and destruction of any other pod has no effect on play (but may
affect a victory in a scenario). The capabilities of a pod or other
part are not affected when made vulnerable.

\section{Missile Launch and Interception}
\label{sec:miss-launch-interc}

\noindent\textbf{GENERAL RULE:}

During a player's Fire Phase, he may launch missiles from any of his
Spacecraft that possess missile-carrying pods. Certain missiles must
be prepared before launch, depending on the type of missile and the
pod from which it is being launched. Once launched, each missile is
moved in accordance with \ref{sec:movement-direction}, and is issued
Commands in accordance 
with \ref{sec:commands} and the restrictions of the following cases.  The
Interception Routine is undertaken each time any missile is in a hex
occupied by an enemy unit. If interception occurs, the missile
explodes, destroying itself and the enemy unit (unless the enemy unit
is a Spacecraft with an active force field).

\medskip

\noindent\textbf{CASES:}

\subsection[Number Of Missiles]{The number of missiles of each type a
  pod possesses at the beginning of play is listed on the Pod  
  Attribute Chart.}
\label{sec:number-of-missiles}



The chart also states whether or not the missile must be prepared
before it may be launched, by issuing a Prepare Missile Command to the
Spacecraft in a previous Command Phase (see
\ref{sec:battle-commands}). In order to launch 
a missile, it must be atop the launching Spacecraft at the beginning
of the friendly Fire Phase, or must be a type of missile that need not
be prepared.

\subsection[Number Of Fires]{The launch of a missile counts as one
  fire towards the total number of fires that may be conducted from a  
  Spacecraft in a single Fire Phase.}
\label{sec:number-of-fires}



Thus, if a Spacecraft with two light weapon pods launches two missiles
in a Fire Phase, it may conduct only one additional fire (from its
burster). Also see \ref{sec:number-types-fires}. The launch of a
missile does not require the 
expenditure of energy by the missile or by the launching Spacecraft.

\subsection[Launch Procedure]{When the Phasing player wishes to launch
  a missile, he chooses a missile counter and marks his  
  Spacecraft Log.}
\label{sec:launch-procedure}



He chooses the counter that matches the chosen missile type and, if a
guided missile, whose identity letter matches that of the Spacecraft
from which it is being launched. He then writes the number of the pod
from which it is being launched and the identity number of the missile
in the first unused \textbf{Pod/\#} box for that missile type on the
appropriate Spacecraft Log. For example, if guided missile A-03 were
launched from a heavy weapon pod (assigned pod \#2), the Phasing player
would write \textbf{2/3} in the first unused Pod/\# box of the guided
missile section of Spacecraft A's Log.

\subsection[Missile Launch]{A missile is launched by assigning it a
  Velocity marker and placing it face-down in a hex adjacent to the  
  launching Spacecraft.}
\label{sec:missile-launch}



A missile must be assigned a Velocity marker \emph{equal to},
\emph{one greater than}, or \emph{one less than} the current velocity
of the Spacecraft from which it is launched. \textbf{Exception:} The
initial velocity of a missile must be at least 1.

The hex in which a missile may be placed and the direction in which
the missile may point are restricted. The following diagrams show all
possible missile placements. \figurename\ \ref{fig:launch-hex-side} is
used if the launching ship points toward a hex side. \figurename\ 
\ref{fig:launch-hex-corner} is used if it points toward a hex corner.
A missile may be placed in any hex shown and, within a hex, may point
in any direction indicated by an arrow radiating from the hex.

\begin{figure}[htbp]
  \begin{minipage}[t]{0.45\textwidth}
    \centering
    \caption[Missile launch from hex side]{~}
    \label{fig:launch-hex-side}
    \includegraphics[width=0.4\textwidth]{missile-launch-side}
  \end{minipage}
  \begin{minipage}[t]{0.45\textwidth}
    \centering
    \caption[Missile launch from hex corner]{~}
    \label{fig:launch-hex-corner}
    \includegraphics[width=0.4\textwidth]{missile-launch-corner}
  \end{minipage}
\end{figure}


A launched missile may not be initially placed in the hex occupied by
the launching Spacecraft. More than one missile may be launched into
the same hex. Such missiles may be assigned identical or different
directions and velocities.


\subsection[Missile Chart]{The Velocity Rating, Maneuver Rating, and
  Energy Unit 
  Allowance of each type of missile are listed on the  
  Missile Chart.}
\label{sec:missile-chart}



The Civ Level of each missile is equal to the Civ Level of the pod
from which it is launched. Unguided missiles may not receive Maneuver
Commands and are thus not listed on the chart. Other types of missiles
may be issued Maneuver Commands in accordance with
\ref{sec:unit-assignment} and \ref{sec:number-maneuver-commands}. Note
that a missile must expend one Energy Unit for each and every Maneuver
Command that it receives (see \ref{sec:commands-energy}). A missile
must be removed from play at the conclusion of the friendly Movement
Phase following the Command Phase in which it expended its last Energy
Unit.


\newcolumntype{G}{%^^A
   >{\color{black}\columncolor{grey}%^^A
      \raggedright}c}
\newcolumntype{W}{%^^A
   >{\color{black}\columncolor{white}%^^A
      \raggedright}c}
\begin{table}[htbp]
  \centering
  \fbox{%
    \begin{minipage}{3in}
      \centering
      \caption{Missile Chart}
      \label{tab:missile}
      
      \medskip
      
      \begin{tabular}{ccccc}
        MISSILE\\
        TYPE & 
        \rotate{CIV LEVEL} &
        \rotate{\parbox{1in}{VELOCITY\\RATING}} &
        \rotate{\parbox{1in}{MANEUVER\\RATING}} &
        \rotate{\parbox{1in}{ENERGY\\UNITS}}\\ 
        & 6 & 1 & 5 & 7\\
        \rowcolor{grey}
        \multicolumn{1}{W}{GUIDED} & 7 & 2 & 5 & 9\\
        & 8 & 2 & 6 & 10\\
        \rowcolor{grey}
        & 6 & 2 & 5 & 6\\
        \multicolumn{1}{G}{UNGUIDED} & 7 & 2 & 6 & 7\\
        \rowcolor{grey}
        & 8 & 2 & 7 & 9\\
        & 7 & 2 & 6 & 6\\
        \rowcolor{grey}
        \multicolumn{1}{W}{\raisebox{1.5ex}[0pt]{MIMS}} & 8 & 2 & 7 & 7\\
      \end{tabular}
    
      \medskip
      
      \parbox{\textwidth}{See \ref{sec:missile-chart}
        for detailed 
        explanation.}
    \end{minipage}}
\end{table}


\subsection[Interception Routine]{The Interception Routine is
  performed whenever a friendly 
  missile enters a hex occupied by an enemy unit,  
  or when any enemy unit enters a hex occupied by a friendly missile,
  regardless of the Phase in progress.}
\label{sec:interception-routine}



The player owning the missile undertakes the following steps: 

\begin{enumerate}
\item Determine the relative velocity of the two units as described in
  \ref{sec:relative-velocity}. The conditions of
  \ref{sec:relative-velocity-range} may also apply, but the conditions
  of \ref{sec:targeting-program-rel-vel} do not. Since the range
  during interception will always be zero, it has no effect.
\item Cross-reference the determined relative velocity with the Civ
  Level of the intercepting missile on the Missile Interception Table
  to determine the Interception Chance.
\item Roll the die. If the die result is equal to or less than the
  Interception Chance, interception has occurred; the missile
  explodes, destroying itself and the enemy unit (\textbf{Exception:}
  See \ref{sec:missile-force-field}). If the die result is greater than the
  Interception Chance, interception does not occur; the missile and
  the enemy unit are not affected, and interception is not attempted
  between the two units again as long as they occupy the same hex.
\end{enumerate}

The Interception Routine must be conducted whenever possible.
\textbf{Exception:} A player may decline to conduct the Interception
Routine if his involved missile has a Civ Level of 8 (but may not
decline if being intercepted by an enemy missile). The Interception
Routine is not conducted between friendly units; interception between
friendly units is impossible.

If an enemy and a friendly missile occupy the same hex, the Phasing
player, and then the non-Phasing player, conduct the Interception
Routine.

If a friendly missile is in a hex occupied by more than one enemy
unit, the Interception Routine is conducted with the enemy unit with
the lowest relative velocity only. If more than one such unit presents
the same relative velocity, the player owning the missile may choose
the unit to intercept.

\subsection[Missile Interception Table]{The Missile Interception Table
  is used during the Interception Routine to determine if a missile
  intercepts  
  an enemy unit.}
\label{sec:missile-interception-table}

See Table \ref{tab:missile-interception}.


\begin{table}[htbp]
  \centering
  \fbox{%
    \begin{minipage}{4.5in}
      \centering
      \caption{Missile Interception Table}
      \label{tab:missile-interception}
      
      \medskip
      
      \begin{tabular}{ccccccc}
        & \multicolumn{6}{c}{RELATIVE VELOCITY}\\ 
        MISSILE CIV LEVEL & 0 & 1,2 & 3,4 & 5--7 & 8--10 &
        \parbox{2cm}{\centering 11\\OR MORE}\\  
        \rowcolor{grey}
        6 & 8 & 6 & 4 & 2 & 1 & 1\\
        7 & 9 & 7 & 5 & 3 & 2 & 1\\
        \rowcolor{grey}
        8 & A & 8 & 6 & 4 & 3 & 2\\
      \end{tabular}
      \parbox{\textwidth}{\textbf{A:} Interception is automatic; no
        dice roll is conducted.

        See \ref{sec:interception-routine} for detailed explanation of use.}
    \end{minipage}}
\end{table}

\subsection[Missiles And Force Fields]{A Spacecraft with an active
  force field is not destroyed 
  when intercepted by an enemy missile.}
\label{sec:missile-force-field}



Instead, the player owning the missile rolls the die and refers to the
Hit Table, in accordance with \ref{sec:hit-table-used}. If the
Spacecraft has a Class 1 active force field, the Hit Table is used
\emph{three} times when interception occurs. If the Spacecraft has a
Class 2 active force field, the Hit Table is used \emph{once} when
interception takes place.

A player owning a Spacecraft that possesses an inactive force field
may attempt to activate the force field at the moment of interception.
When the enemy player has determined that interception occurs, the
player owning the Spacecraft rolls the die. If the die result is
\emph{more than one less} than the Civ Level of the Spacecraft, the
force field is activated; flip the Spacecraft counter over.  On any
other die result, the Spacecraft is destroyed. No Command is required
to activate a force field in this manner. However, an Energy Block
must be expended (see \ref{sec:commands-energy}) and a Battle Command
is required to deactivate the force field (see
\ref{sec:battle-commands}).

\subsection[MIMS Launching Of Missiles]{Four unguided missiles may be
  launched from a MIMS that is currently in play during any one
  friendly Fire  
  Phase.}
\label{sec:mims-launch}



The player owning the MIMS declares this action and places four
unguided missile counters in hexes adjacent to the MIMS, in accordance
with \ref{sec:missile-launch} (as if the MIMS were a Spacecraft). He
may use any of his unused missile counters of the appropriate type for
this purpose. The launch of these missiles is not recorded on the
Spacecraft Log, but a single MIMS may only conduct this special launch
once. No Command is required for a MIMS to launch its missiles, and
the MIMS remains in play after doing so, as an intelligent missile.


\section{How to Use the Spacecraft Logs}
\label{sec:how-use-spacecraft-logs}

\noindent\textbf{GENERAL RULE:}

Before beginning play, each player fills out a Spacecraft Log for each
Spacecraft assigned to him by the scenario instructions. During the
game, energy expenditure by each ship and the current status of the
ship's equipment is updated on the Log. The status of the ship's
missiles and battlecraft is also kept track of on the Log.

\medskip

\noindent\textbf{CASES:} 

\subsection[Compartments Section]{The Compartment section of
  the Spacecraft 
  Log is used to 
  assign pods specific locations on the  
  Spacecraft and to record hits incurred by the pods, the bridge, the 
  engine, and the force field.}
\label{sec:compartment-section}



To prepare the Compartment section for play, complete the following steps: 

\begin{enumerate}
\item If the Spacecraft does \emph{not} have a force field, put an
  \textbf{\textsf{X}} in the Force Field Status box.
\item Consult the Spacecraft Attribute Chart to find how many pods the
  Spacecraft possesses. Then cross out all boxes for pods beyond the
  number available to the ship.\label{item:log-2}
\item Consult the scenario instructions to find which types of pods
  the ship possesses. Write the names of these pods in the available
  numbered Pod Type boxes. The pods may be assigned to the boxes in
  any order the player desires, as long as the boxes crossed out in
  accordance with Step \ref{item:log-2} are not used.
\item Note the Armor Rating for the bridge, engine, and force field
  (that of the Spacecraft) and for each pod in the appropriate boxes.
\end{enumerate}

During play, the Status box for the bridge, engine, force field, and
each pod is used to record hits incurred, by marking a \textbf{V},
\textbf{D}, or \textbf{\textsf{X}} in each box (see \ref{sec:recording-hits}).


\subsection[Missile Section]{Each Missile section of the Spacecraft
  Log is used to note 
  how many missiles are available on the  
  Spacecraft and to record the expenditure of energy by each missile
  after launch.}
\label{sec:missile-section}



To prepare each Missile section for play, count the total number of
missiles of that type available (the total of the amounts listed on
the Pod Attribute Chart for the ship's missile-carrying pods). If this
total is less than the total number of missiles shown in the section,
cross out lines in the section (from the bottom up) until the totals
match. Unless the Unguided Missile section is being filled out,
consult the Missile Chart to find how many Energy Units each missile
possesses (see \ref{sec:missile-chart}). If this number is less than
the number of Energy Unit boxes for each missile, cross out boxes for
each missile (starting from the right) until the numbers match.

When a missile is prepared for launch or is launched (if preparation
is not necessary), the owning player notes the number of the pod and
the identity number of the missile counter in the first available
\textbf{Pod/\#} box in the appropriate Missile section. A pod that has
launched a number of missiles equal to the amount of missiles shown
for the pod on the Pod Attribute Chart may launch no more missiles of
that type.

Each time a missile receives a Maneuver Command, the owning player
must put an \textbf{\textsf{X}} through one of the missile's Energy Unit boxes.
When all the boxes for a missile are marked, the missile is removed
from play (see \ref{sec:missile-chart}). Unguided missiles do not
expend Energy Units, and thus have no Energy Unit boxes.



\subsection[Energy Unit Track And Energy Block Section]{The Energy
  Unit Track and Energy Block section of the Spacecraft Log is used to
  note how much energy  
the Spacecraft possesses at the start of play and to record the
expenditure of energy during play.}
\label{sec:energy-unit-track}


An \emph{Energy Unit} is a measure of energy common to all units in
the game. An \emph{Energy Block} is a variable measure of energy used
by Spacecraft only. The size of an Energy Block for a particular
Spacecraft equals the Energy Burn Rate of the Spacecraft (see the
Spacecraft Attribute Chart) and is expressed in terms of Energy Units.
Thus, an Energy Block for a \emph{Flute} Spacecraft equals six Energy
Units.

To calculate the number of Energy Blocks possessed by a Spacecraft at
the start of play, divide the Energy Capacity of the ship by its
Energy Burn Rate. If the ship possesses an energy pod, add 144 to the
Energy Capacity before dividing. This number is noted on the Energy
Block section of the log by crossing out boxes in excess of the number
(from the bottom up).  Before beginning play, cross out all the boxes
on the Energy Unit Track in excess of the Spacecraft's Energy Burn
Rate, and place an Energy Unit marker in the 0 space of the track.

Each time a Spacecraft expends an Energy Block during play (see
\ref{sec:commands-energy}) an Energy Block is marked. When all the
available boxes are marked, the Spacecraft has no more energy (see
\ref{sec:zero-energy}).

Each time a Spacecraft expends one or more Energy Units (for
conducting fire or operating a tractor beam) the Energy Unit marker is
moved the appropriate number of spaces along the Energy Unit Track.
Each time the marker is moved into the space matching the Energy Burn
Rate of the Spacecraft, the marker is returned to the 0 space, and the
expenditure of one Energy \emph{Block} is marked.  Movement of the
marker is then continued (if necessary).

\subsection[Battlecraft Section]{The Battlecraft section of the
  Spacecraft Log is used to 
  record the status of a launched battlecraft.}
\label{sec:battlecraft}



The status of the battlecraft's bridge and engine (in terms of hits
received) is recorded in the Bridge and Engine boxes. The expendi-
ture of Energy Units by the battlecraft is recorded by marking the
Energy Unit Boxes (see \ref{sec:commands-energy}). When all the Energy
Unit Boxes for a battlecraft are marked, it has no more energy. A
docked battlecraft may receive Energy Units from its Spacecraft; erase
marks from any number of the battlecraft's Energy Units Boxes and
record the expenditure of an equal number of Energy Units by the ship.
A battlecraft may never possess more than \textbf{15} Energy Units.


\subsection*{SPACECRAFT LOG EXAMPLE}
\label{sec:spacecraft-log-example}



See below. 

A Spacecraft Log for a \emph{Flute} with a heavy weapon pod, an energy
pod, a battlecraft pod (containing a \emph{Terwillicker-5000}) and a
standard jump pod (all armor Class 2) has been filled out.

After crossing out the box for Pod 5, the player assigned the four
pods to the remaining boxes in the Compartment section and noted the
Armor Ratings of all the compartments. He then consulted the Pod
Attribute Chart to see how many missiles the heavy weapon pod carries
and crossed out four unguided missile boxes, two guided missile lines,
two intelligent missile lines, and one MIMS line. The heavy weapon pod
has a Civ Level of 7, which means that the guided missiles possess
nine Energy Units each, the intelligent missiles possess seven Energy
Units each, and the MIMS six Energy Units (as noted on the Missile
Chart); so the player crossed out the rightmost columns of boxes in
each Missile section to indicate these reductions.

The ship possesses \textbf{35} Energy Blocks (\textbf{66} Energy
Capacity plus \textbf{144} for the pod, divided by the Energy Burn
Rate of \textbf{6}). The player crossed out all but the top
\textbf{35} boxes in the Energy Block section. Since the Energy Burn
rate is \textbf{6}, he crossed out the 7 and 8 spaces of the Energy
Unit Track. He then placed an Energy Unit marker on the 0 space of the
Track. Finally, the player noted the Armor Class of the battlecraft's
bridge and engine in the Battlecraft section.


\begin{figure}[htbp]
  \centering
  \fbox{%
    \begin{minipage}{0.95\textwidth}
      \centering
      \label{fig:spacecraft-log}
      
      \medskip
      
%    \includegraphics{blank-counter}
  \end{minipage}}
\end{figure}


\section{Scenarios}
\label{sec:scenarios}

\noindent\textbf{GENERAL RULE:}

Before beginning the game, the players choose which of the following
five scenarios they will play. Each scenario provides a brief
description of the situation, how the maps are arranged, the forces
that each player receives, how those forces are set up, the deployment
of planets and asteroid fields (if any), and how each player may
achieve victory. Scenario \thescenario\ is recommended for those
playing \emph{DeltaVee} for the first time.

In all scenarios, a Spacecraft or battlecraft may be destroyed for
purposes of victory. A Spacecraft or battlecraft is considered
destroyed if it does not possess an active force field when
intercepted by an enemy missile; or if its bridge, engine and more
than half of its pods are destroyed (remove the unit from play).
Unless specifically stated otherwise in a scenario, hyperjumping may
not be conducted.

\subsection*{SCENARIO \thescenario: The Showdown}
\label{sec:scenario-1:-showdown}



A gang of cut-throats flying a long range pursuit craft stolen from a
federal installation on a nearby planet are intercepted by a similar
ship manned by the local guard. Enraged by the theft, the military
authorities order the complete destruction of the criminals.

\textbf{Map Deployment:}\\
\includegraphics{map-ab}
% A B

\textbf{Player 1 Deployment:} One Piccolo (Spacecraft counter E) with
one hunter pod, in hex A0207 pointing towards 3 o'clock with a
velocity of 1. Use Spacecraft Log Nr1.

\textbf{Player 2 Deployment:} One Piccolo (Spacecraft counter D) with
one hunter pod, in hex B1511 pointing towards 9 o'clock with a
velocity of 1. Use Spacecraft Log Nr1.

\textbf{Victory Conditions:} The instant one player's Spacecraft is
destroyed, the opposing player is declared the winner. Neither player
may conduct a jump.

\stepcounter{scenario}


\subsection*{SCENARIO 2: The First Shot}
\label{sec:scenario-2:-first}

Tensions were high between the opposed governments of Venable and
Laidley, two planets in the Eridani system. When a Venable light
cruiser ventured into Laidley space to ``test the waters,'' it
encountered two Laidley patrol craft. The smaller ships opened fire
and the brief Eridani War began.

\textbf{Map Deployment:}\\
\includegraphics{map-a}
% A

\textbf{Player 1 Deployment:} One \emph{Sword} (Spacecraft counter A)
with two heavy weapon pods, one battle communications pod, one
Battlecraft pod (with a \emph{Terwillicker-X}) and one energy pod; in
hex A1112, pointing towards 9 o'clock with a velocity of 3. All pods
are armor Class 2. Use Spacecraft Log Nr.2.

\textbf{Player 2 Deployment:} Two \emph{Daggers} (counters A and B)
each with a heavy weapon pod and an energy pod (armor Class 2); in
hexes A0706 and A0705, pointing towards 3 o'clock with a velocity of
3. Use two copies of Spacecraft Log Nr.~1.

\textbf{Victory Conditions:} Player 1 wins if both \emph{Daggers} are
destroyed. Player 2 wins if the \emph{Sword} is destroyed. If neither
player has fulfilled his victory conditions and all opposing
Spacecraft and battlecraft are more than 25 hexes apart at any time,
the game is declared a draw.

\stepcounter{scenario}


\subsection*{SCENARIO \thescenario: Escape from Tau Ceti}
\label{sec:scenario-3:-escape}

As four smuggler ships head out of the Tau Ceti system with a cargo of
deadly drugs and escaped convicts, a federal heavy cruiser gives
chase. The naval vessel's orders are to prevent the criminal ships
from hyperjumping at any cost.

\textbf{Map Deployment:}\\
\includegraphics{map-abcde}
% A B C D E 

\textbf{Player 1 Deployment:}

One Corco \emph{Iota} (Spacecraft counter A) with two heavy weapon pods
(neither pod has any intelligent missiles or MIMS), one energy pod,
one standard jump pod, one crew pod, and four buffered cargo pods. All
pods are armor Class 2. Use Spacecraft Log Nr.~2.

Two Corco \emph{Gammas} (counters B and C), each with one light weapon
pod, one standard jump pod, and one standard cargo pod. All pods are
armor Class 1. Use two copies of Spacecraft Log Nr.~1.

One Corco \emph{Gamma} (counter D) with one standard jump pod, one
crew pod, and one standard cargo pod. All pods are armor Class 0.  Use
Spacecraft Log Nr.~1.

All four ships must be placed within one hex of A0909. All must be
placed in different hexes and must point toward 3 o'clock with a
velocity of 2. Each Spacecraft has already expended 3 Energy Blocks.

\textbf{Player 2 Deployment:} One \emph{Sword} (counter A) with two
arsenal pods, one battle communications pod and two Battlecraft pods
(each with a \emph{Terwillicker-X}); in any hex in the 0100 hex row of
map A, pointing in any direction with a velocity of 3. All pods are
armor Class 2. Use Spacecraft Log Nr.~2.

\textbf{Victory Conditions:} Player 1 wins if the Corco \emph{Iota} or
two Corco \emph{Gammas} are able to jump (see 7.2). A ship may not
jump until it enters map E (to be placed during play as shown in the
diagram) or enters a map placed above or below map E (in the direction
of the arrows). Player 1 also wins if the \emph{Sword} is destroyed.
Player 2 wins if three enemy ships are destroyed (including the Corco
\emph{Iota}).

\stepcounter{scenario}


\subsection*{SCENARIO \thescenario: Pirates!}
\label{sec:scenario-4:-pirates}



A Corco \emph{Mu} loaded with passengers and valuable cargo is
approaching the planet Esata after hyperjumping into the system. As it
nears the dense Bicker's Asteroid Belt, it is set upon by a pair of
ruthless pirate ships looking for booty. A distress call is sent to
Esata in the hopes that aid will come to the Corco \emph{Mu}.

\textbf{Map Deployment:}\\ 
\includegraphics{map-abc}
% A B C 

\textbf{Player 1 Deployment:}

One Corco \emph{Mu} (Spacecraft counter B) with one light weapon pod,
one Battlecraft pod (with a \emph{Terwillicker-5000}), one standard
jump pod, one energy pod, one standard support pod, three standard
cabin pods, one crew pod, and three buffered cargo pods; in hex C1406,
pointing at 9 o'clock with a velocity of 2. All pods are armor Class
1. Use Spacecraft Log Nr.~2.

One \emph{Dagger} (Spacecraft counter A) with one heavy weapon pod and
one energy pod (both armor Class 2); in hex A0409 pointing at 3
o'clock with a velocity of 1. Use Spacecraft Log Nr.~1. The
\emph{Dagger} may not move, fire or be fired at until alerted. During
each friendly Command Phase, Player 1 rolls the die; if the result is
a 1 or 2 , the \emph{Dagger} has been alerted and may be used normally
(beginning with that Command Phase).

\textbf{Player 2 Deployment:}

One \emph{Flute} (counter A) with one arsenal pod, one energy pod, one
tractor pod, and one buffered cargo pod; in hex B1612, pointing in any
direction with a velocity of 0. All pods are armor Class 2. Use
Spacecraft Log Nr.~1.

One \emph{Flute} (counter B) with one heavy weapon pod, one
Battlecraft pod (with a \emph{Terwillicker-X}), one energy pod and one
buffered cargo pod; in hex B1611, pointing in any direction with a
velocity of 0. All pods are armor Class 2. Use Spacecraft Log Nr.~1.

\textbf{Planet:} In hex A0409

\textbf{Asteroid Fields:} In hexes C0902, C0904, C0907, C0909, C0912
and C0915. An asteroid field is considered to exist in all six hexes
adjacent to each Asteroid Field marker, as well as in the hex occupied
by each marker.

\textbf{Victory Conditions:} Player 1 wins the moment the Corco
\emph{Mu} is put into orbit around Esata, or if both \emph{Flutes} are
destroyed. If the Corco \emph{Mu} is destroyed, the game is
immediately declared a draw. Player 2 wins if either \emph{Flute} is
able to dock with the Corco \emph{Mu} (see \ref{sec:battle-commands}).

\stepcounter{scenario}


\subsection*{SCENARIO \thescenario: Attack on Convoy Red}
\label{sec:scenario-5:-attack}


A vital convoy of arms and ammunition hurriedly organised by the
Imperial fleet and establishment merchant heads for the planet
Zaraznov, after hyperjumping from a nearby system. A successful
revolutionary uprising on the planet has gained control of small
well-equipped fleet. A task force from the insurgents is patrolling
Zaraznov space, awaiting the expected convoy.

\textbf{Map Deployment:}\\
\includegraphics{map-abc}
% A B C 

\textbf{Player 1 Deployment:}

One \emph{Spear} (Spacecraft counter A) with two arsenal pods, one
battle communications pod, one Battlecraft pod (with a
\emph{Terwillicker-X} Battlecraft), one tractor pod (Civ Level 8), one
standard jump pod, one energy pod, and one crew pod. All pods are
armor Class 2.  Use Spacecraft Log Nr.~2.

Three Corco \emph{Zetas} (counters B, C and D) each with one light
weapon pod, one energy pod, one standard jump pod, and three standard
cargo pods. All pods are armor Class 1. Use three copies of Spacecraft
Log Nr.~1.

One \emph{Dagger} (counter E) with a hunter pod and an energy pod.
Both pods are armor Class 2. Use Spacecraft Log Nr.~2.

All five ships must be placed within one hex of A0407. All must be
placed in different hexes. Player 1 may choose any one direction and
any one velocity (from 0 to 4) for the ships, but all must point in
the same relative direction and have the same velocity. Each
Spacecraft has already expended 10 Energy Blocks. Spacecraft D has no
guided missiles remaining.

\textbf{Player 2 Deployment:}

One \emph{Clarinet} (counter A) with two heavy pods, one battle
communications pod, two Battlecraft pods (each with a
\emph{Terwillicker-X} Battlecraft), one energy pod and one crew pod;
in hex B0717. All pods are armor Class 2. Use Spacecraft Log Nr.~2.

One \emph{Flute} (counter B) with one heavy weapon pod, one
Battlecraft pod (with a \emph{Terwillicker-X}), one energy pod and one
equipment pod; in hex B0617. All pods are Armor Class 2. Use
Spacecraft Log Nr.~1.

One \emph{Flute} (counter C) with one light weapon pod, one tractor
pod (Civ Level 8), one energy pod and one crew pod; in hex B0516. All
pods are armor Class 2. Use Spacecraft Log Nr.1.

Player 2 may choose any one direction and any one velocity (from 0 to
6) for the ships, but all must point in the same relative direction
and have the same velocity.

\textbf{Planet:} In hex C1110. 

\textbf{Victory Conditions:}

Player 1 receives one Victory Point for each of his Spacecraft placed
in orbit around the planet. He receives an additional VP for each
undestroyed cargo pod aboard such a ship. Once a Spacecraft is placed
in orbit, it is removed from play. When player 1 has ac- cumulated
eight VP's, he wins the game.

Player 2 receives one VP for each enemy ship destroyed and each cargo
pod destroyed (thus, the destruction of a Corco \emph{Zeta} is worth
four VP's). Player 2 loses two VP's for each of his \emph{Flutes} that
is destroyed. When player 2 has accumulated eight VP's, he wins the
game.  If his \emph{Clarinet} is destroyed, Player 2 loses the game
(regardless of how many VP's he has earned).

\bigskip

\fbox{\parbox[top]{0.4\textwidth}{Abbreviated Sequence of Play

  \begin{enumerate}
  \item First Player Movement Phase 
  \item Second Player Command Phase 
    \begin{enumerate}
    \item Detection Segment 
    \item Command Segment 
    \end{enumerate}
  \item First Player Fire Phase 
  \item Second Player Movement Phase 
  \item First Player Command Phase 
    \begin{enumerate}
    \item Detection Segment 
    \item Command Segment 
    \end{enumerate}
  \item Second Player Fire Phase 
  \end{enumerate}}} 



%%% Local Variables: 
%%% mode: latex
%%% TeX-master: "delta_vee"
%%% End: 
